
\documentclass{article}
\usepackage[a4paper, total={6in, 8in}]{geometry}
% \usepackage[utf8]{inputenc}
\usepackage{abstract}
\title{4.1 - Properties of Matter}
\author{PJ Gibson - Peace Corps Tanzania}
\date{May 2020}

\begin{document}

\maketitle


\section{Properties of Matter}

\subsection{Surface Tension}
\begin{itemize}
\item (2018)  Mention any two factors which affect the surface tension of the liquid and in each case explain two typical examples. 
\item (2018)  Why molecules on the surface of a liquid have more potential energy than those within the liquid? Briefly explain. 
\item (2018)  Derive an expression for excess pressure inside a soap bubble of radius $ R$ and surface tension $ \gamma $ when the pressures inside and outside the bubble are $ P_{2}$ and $ P_{1}$ respectively. 
\item (2018)  A soap bubble has a diameter of $ 5$ mm. Calculate the pressure inside it if the atmospheric pressure is $ 10^{5}$ Pa and the surface tension of a soap solution is $ 2.8 \times 10^{-2}$ N$/$m.
\item (2018)  Water rises up in a glass capillary tube up to a height of $ 9.0$ cm while mercury falls down by $ 3.4$ cm in the same capillary. Assume angles of contact for water-glass and . mercury-glass as $ 0^{\circ}$ and $ 135^{\circ}$ respectively. Determine the ratio of surface tensions of mercury and water. 
\end{itemize}

\end{document}
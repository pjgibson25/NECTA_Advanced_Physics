
\documentclass{article}
\usepackage[a4paper, total={6in, 8in}]{geometry}
% \usepackage[utf8]{inputenc}
\usepackage{abstract}
\title{\textbf{4.1 - Surface Tension}}
\author{PJ Gibson - Peace Corps Tanzania}
\date{May 2020}

\begin{document}

\maketitle

\begin{itemize}
\item (1999)  Explain in terms of surface energy, what is meant by the surface tension, ​ $ \gamma $ ​ of a liquid. 
\item (1999)  What energy is required to form a soap bubble of radius $ 1.00$ mm if the surface tension of the soap solution is $ 2.5 \times 10$ ​$ E-4$ ​ N$/$m$ ^{2}$ ​ ?
\item (2000)  Find the work done required to break up a drop of water of radius $ 0.5$ cm into drops of water each having radius of $ 1.0$ mm, assuming isothermal condition.
\item (2010)  State surface tension In terms of energy. 
\item (2010)  The Surface tension of water at $ 20^{\circ}$C is $ 7.28 \times 10^{-2}N/m^{2}$ . The vapor pressure of water at this temperature is $ 2.33 \times 10^{3}$ Pa Determine the radius of smallest spherical water droplet which it can form without evaporating
\item (2010)  A circular ring of thin wire $ 3$ cm in radius is suspended with its plane horizontal by a thread passing through the $ 10$ cm mark of a metre rule pivoted at its centre and is balanced by $ 8$ g weight suspended at the $ 80$ cm mark. When the ring is just brought in contact with the surface of a liquid, the $ 8$ g weight has to be moved to the $ 90$ cm mark to just detach the ring from the liquid. Find the surface tension of the liquid (assume zero angle of contact.)
\item (2013)  Using the method of dimensions, indicate which of the following equations are dimensionally correct and which are not, given that, $ f=$ frequency, $ \gamma =$ surface tension, $ \rho =$ density, $ r=$ radius and $ k=$ dimensionless constant.
 \begin{itemize}
\item  $ \rho^{2}=k\sqrt{r^{3}f/\gamma }$
\item  $ f=(kr^{3}\sqrt{\gamma })/(\rho^{1/2})$
\item  $ f=(k\gamma^{1/2})/(\sqrt{\rho}r^{3/2})$
\end{itemize}
\item (2013)  Distinguish surface tension from surface energy.
\item (2013)  Explain the phenomenon of surface tension in terms of the molecular theory.
\item (2013)  A clean open ended glass U-tube has vertical limbs one of which has a uniform internal diameter of $ 4.0$ mm and the other of $ 20.0$ mm. Mercury is poured into the tube; and observed that the height of mercury column in the two limbs ts different.
 \begin{itemize}
\item Explain this observation
\item Calculate the difference in levels
\end{itemize}
\item (2016)  Define the following terms:
 \begin{itemize}
\item Free surface energy
\item Capillary action
\item Angle of contact
\end{itemize}
\item (2016)  Briefly explain the following observations:
 \begin{itemize}
\item Soap solution is a better cleansing agent than ordinary water.
\item When a piece of chalk is put into water, it emits bubbles in all directions.
\end{itemize}
\item (2016)  Two spherical soap bubbles are combined.  If v is the change in volume of the contained air, $ A$ is the change in total surface area, show that $ 3P_{A}V+4A T=0$ . Where $ T$ is the surface tension and $ P_{A}$ is the atmospheric pressure.
\item (2016)  There is a soap bubble of radius $ 3.6 \times 10^{-4}$ m in air cylinder which is originally at a pressure of $ 10^{5}$ N$/$m$ ^{2}$ . The air in the cylinder is now compressed isothermally until the radius of the bubble is halved. Calculate the pressure of air in the cylinder.
\item (2017)  Define free surface energy in relation to the quid surface.
 \begin{itemize}
\item Explain what will happen if two bubbles of unequal radii are joined by a tube without bursting. 
\end{itemize}
\item (2017)  A spherical drop of mercury of radius $ 5$ mm falls on the ground and breaks into $ 1000$ droplets. Calculate the work done in breaking the drop. 
\item (2018)  Mention any two factors which affect the surface tension of the liquid and in each case explain two typical examples. 
\item (2018)  Why molecules on the surface of a liquid have more potential energy than those within the liquid? Briefly explain. 
\item (2018)  Derive an expression for excess pressure inside a soap bubble of radius $ R$ and surface tension $ \gamma $ when the pressures inside and outside the bubble are $ P_{2}$ and $ P_{1}$ respectively. 
\item (2018)  A soap bubble has a diameter of $ 5$ mm. Calculate the pressure inside it if the atmospheric pressure is $ 10^{5}$ Pa and the surface tension of a soap solution is $ 2.8 \times 10^{-2}$ N$/$m.
\item (2018)  Water rises up in a glass capillary tube up to a height of $ 9.0$ cm while mercury falls down by $ 3.4$ cm in the same capillary. Assume angles of contact for water-glass and . mercury-glass as $ 0^{\circ}$ and $ 135^{\circ}$ respectively. Determine the ratio of surface tensions of mercury and water. 
\end{itemize}

\end{document}

\documentclass{article}
\usepackage[a4paper, total={6in, 8in}]{geometry}
% \usepackage[utf8]{inputenc}
\usepackage{abstract}
\title{\textbf{1.2 - Errors}}
\author{PJ Gibson - Peace Corps Tanzania}
\date{May 2020}

\begin{document}

\maketitle

\begin{itemize}
\item (2000)  What is an error? Mention two causes of systematic and two causes of random errors.
\item (2000)  The pressure $ P$ is calculated from the relation $ P=F/( \pi R^{2})$ where $ F$ is the force and $ R$ the radius. If the percentage possible errors are $ +2\%$ for $ F$ and $ +1\%$ for $ R$ . Calculate the possible percentage error for $ P$ .
\item (2007)  What is systematic error?
\item (2007)  The smallest divisions for the voltmeter and ammeter are $ 0.1$ V and $ 0.01$ A respectively.  If $ V=IR$ , find the relative error in the resistance $ R$ , when $ V=2$ V and $ I=0.1$ A.
\item (2010)  Define an error.
\item (2010)  In an experiment to determine the acceleration due to gravity $ g$ , a small ball bearing is timed while falling freely from rest through a measured vertical height. The following data were obtained: vertical height $ h=(600\pm 1)$ mm, time taken $ t=(350\pm 1)$ ms. Calculate the numerical value of $ g$ from the experimental data, clearly specify the errors. 
\item (2013)  What is the difference between degree of accuracy and precision.
\item (2013)  In an experiment to determine Young's modulus of a wooden material the following measurements were recorded:
 \begin{itemize}
\item length $ l=80.0\pm 0.05$ cm 
\item breadth $ b=28.65\pm 0.03$ mm
\item thickness $ t=6.40\pm 0.03$ mm and
\item slope $ G=0.035\pm 0.001$ cm/gm
\item Given that the Young’s modulus $ Y$ is given by:
\item $ Y=(4/Gb)(l/t)^{3}$
\item Calculate the maximum percentage error in the value of $ Y$ .
\end{itemize}
\item (2014)  Distinguish random error from systematic error.
 \begin{itemize}
\item Give a practical example of random error and systematic error and briefly explain how they can be reduced or eliminated.
\end{itemize}
\item (2014)  Define the terms error and mistake.
\item (2014)  An experiment was done to find the acceleration due to gravity by using the formula: $ T=2\pi\sqrt{l/g}$ , where all symbols carry their usual meaning.  If the clock losses $ 3$ seconds in $ 5$ minutes, determine the error in measuring ‘$ g$ ’ given that, $ T=2.22$ sec, $ l=121.6$ cm, $ \Delta T_{1}=0.1$ sec, and  $ \Delta l=\pm 0.05$ .
\item (2014)  The following measurements were taken by a student fort he length of a piece of rod: $ 21.02$ , $ 20.99$ , $ 20.92$ , $ 21.11$ and $ 20.69$ . Basing on error analysis find the true value at the length of a piece of rod and its associated error.
\item (2015)  What is meant by random errors?
 \begin{itemize}
\item Briefly explain two causes of random errors in measurements. 
\end{itemize}
\item (2015)  The period $ T$ of oscillation of a body is said to be $ 1.5\pm 0.002$ s while its amplitude A is $ 0.3\pm 0.005$ m and the radius of gyration $ k$ is $ 0.28+0.004$ m. If the acceleration due
 \begin{itemize}
\item to gravity $ g$ was found to be related to $ T$ , $ A$ and $ k$ by the equation $ (gA)/(4\pi^{2})=( A^{2}+k^{2})/T^{2}$ , find the:
\item Numerical value of $ g$ in four decimal places
\item Percentage error in $ g$ .
\end{itemize}
\item (2016)  The period of oscillation of a simple pendulum is given by $ T=2\pi\sqrt{l/g}$ where by $ 100$ vibrations were taken to measure $ 200$ seconds. If the least count for the time and length of a pendulum of $ 1$ m are $ 0.1$ sec and $ 1$ mm respectively, calculate the maximum percentage error in the measurement of $ g$ .
\item (2017)  Give the meaning of the following terms as used in error analysis:
 \begin{itemize}
\item Absolute error. 
\item Relative error. 
\end{itemize}
\item (2017)  The force ‘$ F$ ’ acting on an object of mass ‘$ m$ ’, travelling at velocity ‘v’ in a circle of radius ‘$ r$ ’ is given by: $ F= \frac{mv^{2}}{r}$ If the measurements are recorded as: $ m=(3.5 \pm 0.1)$ kg, $ V=(20\pm 1)$ m$/$s, $ r=(12.5\pm 0.5)m$ ; find the maximum possible
 \begin{itemize}
\item Fractional error. 
\item Percentage error in the measurement of force.
\item Show how you will record the reading of force, ‘$ F$ ’ in the question above. 
\end{itemize}
\item (2018)  How can random and Systematic errors be minimized during an experiment?
\item (2018)  Estimate the precision to which the Young’s modulus, $ \gamma $ of the wire can be determined from the formula $ \gamma =(4Fl)/(\pi d^{2} e)$ , given that the applied tension, $ F=500$ N, the length of the loaded wire,  $ l=3$ m, the diameter of the wire, $ d=1$ mm, the extension of the wire, $ e=5$ mm and the errors associated with these quantities are $ 0.5$ N, $ 2$ mm, $ 0.01$ mm and $ 0.1$ mm respectively. 
\item (2019)  What causes systematic errors in an experiment? Give four points. 
\item (2019)  Estimate the numerical value of drag force $ D= 1/2 C \rho  A V^{2}$ with its associated error given that the measurements of the quantities $ C$ , $ A$ , $ \rho $ and $ v$ were recorded as $ (10\pm 0.00)$ unit less $ (5\pm 0.2) $ cm$ ^{2}$ , $ (15\pm 0.15)$ g$/$cm$ ^{3}$ and $ (3\pm 0.5)$ cm$/$sec$ ^{2}$ respectively. 
\end{itemize}

\end{document}

\documentclass{article}
\usepackage[a4paper, total={6in, 8in}]{geometry}
% \usepackage[utf8]{inputenc}
\usepackage{abstract}
\title{\textbf{3.1 - Streamline Flow and Continuity}}
\author{PJ Gibson - Peace Corps Tanzania}
\date{May 2020}

\begin{document}

\maketitle

\begin{itemize}
\item (1998)  What is terminal velocity?
\item (1998)  Briefly explain an experiment designed to measure terminal velocity.
\item (1998)  A small sphere of radius $ r$ and density $ \sigma $ is released from the bottom of a column of liquid of density $ \rho $ which is slightly higher than $ \sigma $ . Deduce expressions for;
 \begin{itemize}
\item the initial acceleration of the sphere.
\item the terminal velocity of the sphere.
\end{itemize}
\item (1998)  Explain why a length of horse pipe which is lying in a curve on a smooth horizontal surface, straightens out when a fast flowing stream of water passes through it.
\item (1999)  Write down the equation of continuity of a fluid defining all your symbols.
\item (2000)  At two points on a horizontal tube of varying circular cross-section carrying water, the radii are 1cm and $ 0.4$ cm and the pressure difference between these points is $ 4.9$ cm of water. How much liquid flows through the tube per second?
\item (2007)  Write the Continuity and Bernoullis’ equations as applied to fluid dynamics. 
\item (2007)  Develop an equation to determine the velocity of a fluid in a venture meter pipe.
 \begin{itemize}
\item What amount of fluid passes through a section at any given time? 
\end{itemize}
\item (2013)  What is meant by Newtonian fluid? 
\item (2015)  Name the principle on which the continuity equation is based.
\item (2015)  Air is moving fast horizontally past an air-plane.  The speed over the top surface is $ 60$ m$/$s and under the bottom surface is $ 45$ m$/$s.  Calculate the difference in pressure.
\item (2016)   A jet of of water from a fire hose is capable of reaching a height of $ 20$ m.  If the cross sectional area of the hose outlet is $ 4.0	\times 10^{-4}$ m$ ^{2}$ , calculate the:
 \begin{itemize}
\item Minimum speed of water from the hose.
\item Mass of water leaving the hose each second.
\item Force on the hose due to the water jet.
\end{itemize}
\item (2017)  What is the terminal velocity?
\item (2018)  Compute the mass of water striking the wall per second when a jet of water with a velocity of $ 5$ m$/$s and cross-sectional area of $ 3 \times 10^{-2}$ m$ ^{2}$ strikes the wall at right angle losing its velocity to zero. 
\item (2018)  Define the following terms when applied to fluid flow:
 \begin{itemize}
\item Non-viscous fluid 
\item Steady flow 
\item Line of flow 
\item Turbulent flow
\end{itemize}
\end{itemize}

\end{document}
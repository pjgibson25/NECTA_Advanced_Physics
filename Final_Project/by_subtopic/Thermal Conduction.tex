
\documentclass{article}
\usepackage[a4paper, total={6in, 8in}]{geometry}
% \usepackage[utf8]{inputenc}
\usepackage{abstract}
\title{\textbf{5.2 - Thermal Conduction}}
\author{PJ Gibson - Peace Corps Tanzania}
\date{May 2020}

\begin{document}

\maketitle

\begin{itemize}
\item (1999)  What is the coefficient of thermal conductivity of a material?
\item (1999)  The temperature difference between the inside and outside of a room is $ 25​^{\circ}$C . The room has a window of an area $ 2$ m$ ​^{2}$ and the thickness of the window material is $ 2$ mm. Calculate the heat flow through the window if the coefficient of thermal conductivity of the window material is $ 0.5$ SI units.
\item (2000)  Define the thermal conductivity of a material
\item (2000)  Give one major similarity and one major difference between heat conduction and wave propagation.
\item (2000)  Deep bore holes into the earth show that the temperature increases about $ 1^{\circ}$C for each $ 30$ m depth. How much heat flows out from the core of the earth each second for each square metre of surface area.
\item (2007)  Explain why in cold climates, windows of modern buildings are double glazed, ie: There are two pieces of glass with a small air space between them.
\item (2010)  A cylindrical element of $ 1$ kW electric fire $ 1$ s $ 30$ cm long and $ 1.0$ cm in  diameter. If the temperature of the surroundings is $ 20^{\circ}$C , estimate the working temperature of the element.
\item (2013)  Compare the law governing the conduction of heat and electricity pointing out the corresponding quantities in each case.
\item (2013)  A Lagged copper rod is uniformly heated by a passage of an electric current. Show by considering a small section dx that the temperature $ \theta $ varies with distance $ x$ along a rod in a way that, $ k\frac{d^{2}T}{dx^{2}}=-H$ , where $ k$ is a thermal conductivity and $ H$ is the rate of heat generation per unit volume.
\item (2015)  Define coefficient of thermal conductivity.
\item (2015)  Write down two characteristics of a perfectly lagged bar.
\item (2015)  A thin copper wall of a hot water tank having a total surface area of $ 5.0$ m$ ^{2}$ contains $ 0.8$ cm$ ^{3}$ of water at $ 350$ K and is lagged with a $ 50$ mm thick layer of a material of thermal conductivity $ 4.0\times10^{-2}$ W$/$mK. If the thickness of copper wall is neglected and the temperature of the outside surface is $ 290$ K,
 \begin{itemize}
\item Calculate the electrical power supplied to an immersion heater.
\item If the heater were switched off, how long would it take for the temperature of hot water to fall by $ 1$ K?
\end{itemize}
\item (2016)  Identify two factors on which the coefficient of thermal conductivity of a material depend.
\item (2016)  A brass boiler of base area $ 1.50\times 10^{-1}$ and thickness $ 1.0$ cm boils water at a rate of $ 6.0$ kg$/$min when placed on a gas Stove. Estimate the temperature of the part of the flame in contact with the boiler.
\item (2019)  A closed metal vessel containing water at $ 75^{\circ}$C , has a surface area of $ 0.5$ m$ ^{2}$ and uniform thickness of $ 4.0$ mm.  If its outside temperature is $ 15^{\circ}$C , calculate the head loss per minute by conduction.
\end{itemize}

\end{document}

\documentclass{article}
\usepackage[a4paper, total={6in, 8in}]{geometry}
% \usepackage[utf8]{inputenc}
\usepackage{abstract}
\title{\textbf{8.3 - Magnetic Forces}}
\author{PJ Gibson - Peace Corps Tanzania}
\date{May 2020}

\begin{document}

\maketitle

\begin{itemize}
\item (1998)  An electron is projected horizontally with a velocity of $ 2.0 \times 10^{6}$ ms$ ^{-1}$ into a large evacuated enclosure. A magnetic field which has a flux density of $ 15 \times 10^{-4}$ tesla is directed vertically downwards throughout the enclosure. Find
 \begin{itemize}
\item the radius of curvature of the electron's path.
\item how many complete loops must the electron describe before it falls by $ 1.0$ cm under the influence of gravity?
\item What would be the effect of changing the direction of the magnetic field to upwards?
\end{itemize}
\item (2000)  An electron with charge $ e$ and mass $ m_{e}$ is initially projected with a speed v at right angles to a uniform magnetic field of flux density $ B$ .
 \begin{itemize}
\item Explain why the path of the electron $ 1$ s circular.
\item Show also that the time to describe one complete circle is independent of the speed of the electron.
\end{itemize}
\item (2000)  Calculate the radius of the path traversed by an electron of energy $ 450$ eV moving at right angles to a uniform magnetic field of flux density $ 1.5\times 10^{-3}$ T.
\item (2009)  Develop an equation for the torque acting on a current carrying coil of dimensions lxb placed in a magnetic field.  How is this effect applied in a moving coil galvanometer?
\item (2009)  A galvanometer coil has $ 50$ turns, each with an area of $ 1.0 $ cm$ ^{2}$ .  If the coil is in a radian field of $ 10^{-2}$ T and suspended by a suspension of torsion constant $ 2 \times 10^{-9}$ Nm per degree, what current is needed to give a deflection of $ 30^{\circ}$ ?
\item (2009)  Give a general form expressing the force exerted on the wire carrying current i if its length $ l$ is inclined at angle angle $ \theta $ to the magnetic field $ B$ .  
\item (2009)  A wire carrying a current of $ 2$ A has a length of $ 100$ mm in a uniform magnetic field of $ 0.8$ Wb$/$m$ ^{2}$ .  Find the force acting on the wire when the field is at $ 60^{\circ}$ to the wire.
\item (2009)  A wire carrying a current of $ 25$ A and $ 8$ m long is placed in a magnetic field of flux density $ 0.42$ T . What is the force on the wire if it is placed:
 \begin{itemize}
\item At right angles to the field?
\item At $ 45^{\circ}$ to the field?
\item Along the field?
\end{itemize}
\item (2013)  Derive the formula for the torque acting of the rectangular current-carrying coil in a magnetic field
\item (2013)  Give comment on the statement that, an electron suffers no force when it moves parallel to the magnetic field, $ B$ .
\item (2015)  A horizontal straight wire $ 0.05$ m long weighing $ 2.4$ g$/$m is placed perpendicular to a uniform horizontal magnetic field of flux density $ 0.8$ T.  If the resistance of the wire is $ 7.6\Omega /$m, calculate the potential difference that has to be applied between the ends of the wire to make it just self-supporting.
\item (2015)  Two very long wires made of copper and of equal lengths are placed parallel to each other in such a way that they are $ 10$ cm apart.  If the total power dissipated in the two wires is $ 75$ W, find the force between them if the resistivity of the copper wire is $ 1.69	imes 10^{-8}\Omega m$ and of diameter $ 2$ mm.
\item (2017)  State the law of force acting on a conductor of length $ l$ carrying an electric current in a magnetic field. 
\item (2019)  Determine the magnitude of force experienced by a stationary charge in a uniform magnetic field. 
\end{itemize}

\end{document}
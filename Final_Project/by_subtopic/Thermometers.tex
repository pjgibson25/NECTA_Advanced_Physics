
\documentclass{article}
\usepackage[a4paper, total={6in, 8in}]{geometry}
% \usepackage[utf8]{inputenc}
\usepackage{abstract}
\title{\textbf{5.1 - Thermometers}}
\author{PJ Gibson - Peace Corps Tanzania}
\date{May 2020}

\begin{document}

\maketitle

\begin{itemize}
\item (1999)  What do you understand by the term:   Triple point of water
\item (1999)  The resistance of a platinum wire at a temperature T​$ ^{\circ}$C measured on a gas scale is given by $ R(T)=R​_{0​}(1+ a T+bT​^{2}​)$ .
 \begin{itemize}
\item What temperature will the platinum thermometer indicate when the temperature on the gas scale is $ 200​^{\circ}$C ? (take a $ =3.8 \times 10^{-3}$ ​ and $ b=-5.6 \times 10^{-7}$ ​)
\end{itemize}
\item (2000)  What does one require in order to establish a scale of temperature?
\item (2000)  A copper-constantan thermocouple with its cold junction at $ 0^{\circ}$C had an emf of $ 4.28$ mV when its other hot junction was at $ 100^{\circ}$C. The emf became $ 9.29$ mV when the temperature of the hot junction was $ 200^{\circ}$C. If the emf $ E$ is related to the temperature difference $ 8$ between hot and cold junctions by the equation $ E= A(\theta )+B(\theta ^{2})$ , calculate:
 \begin{itemize}
\item The values of $ A$ and $ B$ .
\item The range of temperature for which $ E$ may be assumed proportional to $ 8$ without incurring an error of more than $ 1\%$ .
\end{itemize}
\item (2000)  The resistance $ R$ , of a platinum varies with temperature $ t$ according to the equation $ R_{t}=R_{o}(1+8000bt -b t^{2})$ where $ b$ is a constant. Calculate the temperature on platinum scale corresponding to $ 400^{\circ}$C on the gas scale. 
\item (2000)  Heat is supplied at a rate of $ 80$ W to one end of a well lagged copper bar of uniform cross section area $ 10$ cm? having a total length of $ 20$ cm. The heat is removed by water cooling at the other end of the bar. Temperature recorded by two thermometers $ T_{1}$ and $ T_{2}$ at distances $ 5$ cm and $ 15$ cm from the hot end are $ 48^{\circ}$C and $ 28^{\circ}$C respectively.
 \begin{itemize}
\item Calculate the thermal conductivity of copper.
\item Estimate the rate of flow (in g$/$min) of cooling water sufficient for the water temperature to rise $ 5$ K. 
\item What is the temperature at the cold end of the bar? 
\end{itemize}
\item (2007)  What is meant by a thermometric property of a substance?
\item (2007)  What qualities make a particular property suitable for use in practical thermometers?
\item (2007)  Explain why at least two $ (2)$ fixed points are required to define a temperature scale.
\item (2007)  Mention the type of thermometer which is most suitable for calibration of thermometers.
\item (2010)  In a special type thermometer a fixed mass of a gas has a volume of $ 100$ cm? at a pressure of $ 81.6$ cmHg at the ice point and volume of $ 124$ cm$ ^{3}$ and pressure of $ 90$ cmHg at steam point. Determine the temperature if its volume is $ 120$ cm$ ^{3}$ and pressure of $ 85$ cmHg.
 \begin{itemize}
\item What value does the scale of this thermometer give for absolute
\item zero? 
\end{itemize}
\item (2013)  Name the temperature of a thermocouple at which the thermo,
 \begin{itemize}
\item e.m.f. changes its sign.
\item electric power becomes zero.
\end{itemize}
\item (2013)  A Nichrome-coustantan thermocouple gives about $ 70$ $\mu$V for each $ 1^{\circ}$C difference in temperature between the junctions. If $ 100$ such thermocouples are made into a thermopile, what voltage is produced when the junctions are at $ 20^{\circ}$C and $ 240^{\circ}$C? 
\item (2014)  What is meant by temperature of inversion?
\item (2014)  A thermometer was wrongly calibrated as mt reads the melting point of ice as $ -10^{\circ}$C and reading a temperature of $ 60^{\circ}$C in place of $ 50^{\circ}$C What would be the temperature of boiling point of water on this scale? 
\item (2015)  What is meant by a thermometric property?
\item (2015)  Mention three qualities that make a particular property suitable for use in a practical thermometer.
\item (2016)  Briefly describe the working principle of a thermocouple. 
\item (2016)  In a certain thermocouple thermometer the e.m.f. is given by $ E= a \theta + 1/2 b\theta^{2}$ where $ \theta $ is the temperature of hot junction. If a$ =10 $ mV$ ^{\circ}C^{-2}$ , $ b=-1/20 $ mV$ ^{\circ}C^{-2}$ and the cold junction is at $ 0^{\circ}$C, calculate the neutral temperature. 
\item (2017)  The value of the property $ X$ of a certain substance Is given by $ X_{\theta}=X_{0}+0.5\theta +2\times 10^{-4}\theta ^{2}$  , Where $ \theta $ is the temperature in degree Celsius. What would be the Celsius temperature defined by the property $ X$ which corresponds to a temperature of $ 50^{\circ}$C on this gas thermometer scale? 
\item (2018)  Which type of thermometer is most suitable for calibration of other thermometers? 
\item (2018)  Why at least two fixed points are required to define a temperature scale?
\item (2018)  List two qualities which makes a particular property suitable for use in practical thermometers. 
\item (2018)  Describe how mercury in glass thermometer could be made sensitive.
\item (2018)  What is meant by triple point of water? 
\item (2018)  Evaluate the temperature in Kelvin if the pressure recorded by a constant volume gas thermometer is $ 6.8 \times 10^{4}$ Nm$ ^{-2}$ given that the pressure at triple point $ 273.16$ K is $ 4.6 \times 10^{4}$ Nm$ ^{-2}$ .
\item (2019)  A thermometer has wrong calibration as it reads the melting point of ice as $ -10^{\circ}$C . If it reads $ 40^{\circ}$C in a place where the temperature reads $ 30^{\circ}$C ,  determine the boiling point of water on this scale.
\end{itemize}

\end{document}
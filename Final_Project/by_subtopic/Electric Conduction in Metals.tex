
\documentclass{article}
\usepackage[a4paper, total={6in, 8in}]{geometry}
% \usepackage[utf8]{inputenc}
\usepackage{abstract}
\title{\textbf{9.1 - Electric Conduction in Metals}}
\author{PJ Gibson - Peace Corps Tanzania}
\date{May 2020}

\begin{document}

\maketitle

\begin{itemize}
\item (1999)  State Kirchhoff’s laws of circuit analysis
\item (2000)  State Kirchhoff’s laws of electric circuits.
\item (2000)  What do you understand by the term “drift velocity” as applied to any current carriers in a wire?
\item (2000)  Determine the drift velocity of electrons in a silver wire of a cross—sectional area $ 4.5 \times 10^{-6}$ m$ ^{2}$ when a current of $ 15$ A flows through it. Given: The density of silver $ =1.05 \times 10^{4}$ kg$/$m$ ^{3}$ . The atomic weight of silver $ =108$ .
\item (2000)  An unknown wire of $ 1$ mm diameter is found to carry and passes a total charge of $ 90$ C in $ 1$ hour and $ 15$ min. If the wire has $ 5.8 \times 10^{28}$ free electrons per $ m^{3}$ , find
 \begin{itemize}
\item  the current in the wire.
\item the drift velocity of the electrons in m s$ ^{-1}$
\end{itemize}
\item (2000)  The current of $ 12$ A is made to pass through an aluminium wire of radius $ 1.5$ mm which is joined in series with a copper wire of radius $ 0.8$ mm. Determine.
 \begin{itemize}
\item the current density in an aluminium wire.
\item the drift velocity of the electron tn the copper wire, given that the number of free electrons per unit volume in a copper wire is $ 10^{29}$ .
\end{itemize}
\item (2007)  Define the internal resistance (r) of a cell and the terminal potential difference.
\item (2007)  The e.m.f. of a cell is a special terminal potential difference.  Comment.
\item (2007)  State Kirchhoff's laws of electrical network.
\item (2007)  Discuss two $ (2)$ harmful effects of electrolysis. 
\item (2009)  Explain the mechanism of electric conduction in:
 \begin{itemize}
\item Gases
\item Electrolytes
\end{itemize}
\item (2010)  Define the temperature coefficient of resistance
\item (2013)  What is meant by “power rating" as regards to a resistor?
 \begin{itemize}
\item Mention two distinct velocities of an electron in a wire.
\end{itemize}
\item (2013)  Explain why it is better to use a small current for a long time to plate a metal with a given thickness of silver than using a larger current for a short time? 
\item (2013)  Give four difference between the passage of electricity through metals and  ionized solution.
\item (2014)  Define the following terms:
 \begin{itemize}
\item Current density
\item Conductivity 
\end{itemize}
\item (2014)  Under what condition is $ \Omega $ ’s law true?
\item (2014)  Why does the voltage across the terminals of a cell or battery fall when it is delivering a current? 
\item (2014)  Define temperature coefficient of resistance.
 \begin{itemize}
\item A heating coil of Nichrome wire with cross sectional area of $ 0.1 $ mm$ ^{2}$ operates on a $ 12$ V supply, and has a power of $ 36$ W when immersed in water at $ 373$ K. Calculate the length of the wire.
\end{itemize}
\item (2015)  What is meant by the following terms:
 \begin{itemize}
\item  Internal resistance of a cell. 
\item  Drift velocity. 
\end{itemize}
\item (2015)  What is a potentiometer. 
 \begin{itemize}
\item Mention two advantages and two disadvantages of potentiometer.
\end{itemize}
\item (2015)  Distinguish between ohmic and non-ohmic conductor. Give one example in each
\item (2016)  What ts the physical significance of Kirchhoff’s first law.
\item (2016)  Why is Kirchhoff’s second law sometimes referred to as the voltage law?
\item (2016)  List down five points to be considered when applying Kirchhoff’s second law in formulating analytical problems or equations.
\item (2017)  What is the advantage of using a greater length of potentiometer wire?
\item (2017)  Why is Wheatstone bridge not suitable for measuring very high resistance?
\item (2017)  List two factors on which the resistivity of a material depends. 
\item (2017)  A wire of resistivity, $ \rho $ , is stretched to double its length. What will be its new resistivity? Give reason for your answer. 
\item (2017)  Why a high voltage supply should have high internal resistance?
\item (2017)  Justify the statement that ‘it is not possible to verify Ohm's law by using a filament lamp’.
\item (2017)  A potential difference of $ 4$ V is connected to $ 4$ uniform resistance wire of length $ 3.0$ m and cross-sectional area $ 9\times 10^{-9}$ , when a current of $ 0.2$ A is flowing in the wire. Find the:
 \begin{itemize}
\item Resistivity of the wire.
\item Conductivity of the wire. 
\end{itemize}
\item (2018)  Outline three important points which are usually referred as sign convection in  solving Kirchhoff’s second law problems. 
\item (2018)  How is ohmic conductor differ from non-ohmic conductor? Give one example in each case. 
\item (2018)  State a condition that could be employed to make an insulator conduct some electricity. 
\item (2018)  What is meant by the term Ballistic galvanometer? 
\item (2018)  State two conditions to be fulfilled for a galvanometer to be used as a ballistic galvanometer. 
\item (2019)  A researcher has $ 2$ g of gold and wishes to form it into a wire having a resistance of $ 80\Omega $ at $ 0^{\circ}$C . How long should the wire be? 
\end{itemize}

\end{document}
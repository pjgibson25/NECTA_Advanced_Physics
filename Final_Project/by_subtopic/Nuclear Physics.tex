
\documentclass{article}
\usepackage[a4paper, total={6in, 8in}]{geometry}
% \usepackage[utf8]{inputenc}
\usepackage{abstract}
\title{\textbf{11.4 - Nuclear Physics}}
\author{PJ Gibson - Peace Corps Tanzania}
\date{May 2020}

\begin{document}

\maketitle

\begin{itemize}
\item (1998)  Explain the terms: atomic mass unit, mass defect, packing fraction and binding energy.
\item (1998)  Discuss carbon dating.
\item (1998)  Find the age at death of an organism, if the ratio of amount of C$ 14$ at death to that of the present time is $ 10^{8}$ and that the half life of Cl$ 4$ is $ 5600$ years.
\item (1999)  What is nuclear fusion 
\item (1999)  What is nuclear fission?
\item (1999)  Define the term “binding energy” of a nuclide.
\item (1999)  Distinguish between:
 \begin{itemize}
\item $ \beta -$ decay and $ \beta +$ decay.
\item nuclear fission and nuclear fusion
\item activity and half-life of a radioactive material.
\item Taking the half-life of Radium $ -226$ to be $ 1600$ years, what fraction of a given sample remains after $ 4800$ years?
\end{itemize}
\item (2000)  A sample of soil from Olduvai Gorge cave was examined. It was found to contain, among other things, pieces of charcoal. Further investigation on the charcoal revealed that $ 1$ kg of C$ 14$ nuclei decayed each second. It is assumed that this charcoal has resulted from decomposition of the stone-age people who died there (i.e. at the cave) long time ago. Calculate the number of years that have elapsed since these people died.
\item (2007)  It is not possible to separate the different isotopes of an element by chemical means.  Explain.
\item (2007)  Define a mass spectrometer. 
\item (2007)  Ion A of mass $ 24$ and charge $ +e$ and ion $ B$ of mass $ 22$ and charge $ +2e$ both enter the magnetic field of a mass spectrometer with the same speed. If the radius of A is $ 2.5 \times 10^{-1}m$ , calculate the radius of the circular path of $ B$ . 
\item (2007)  If the ratio of mass of lead – $ 206$  to mass of uranium – $ 238$ in a certain rock was found to be $ 0.45$ and that the rock originally contained no lead – $ 206$ , estimate the age of the rock given that the half life of uranium – $ 238$ is $ 4.5 \times 10^{9}$ years.
\item (2007)  Define the following terms:
 \begin{itemize}
\item Atomic mass unit
\item Binding energy
\item Mass defect.
\end{itemize}
\item (2009)  Explain the following observations:
 \begin{itemize}
\item A radioactive source is placed in front of a detector which can detect all forms of radioactive emissions. It is found that the activity registered as noticeably reduced when a thin sheet of paper is placed between the source and detector.
\item When a brass plate with a narrow vertical shit is placed in front of the radioactive source (above) and a horizontal: magnetic field normal to the line joining the source and the detector is applied, its found that the activity is further reduced.
\item The magnetic field (above) is removed and a sheet of aluminum is placed in front of the source. The activity recorded is similarly reduced.
\end{itemize}
\item (2009)  A $ 2.71$ g sample of Kcl from the chemistry stock is found to be radioactive and decays at a constant rate of $ 4490$ disintegrations per second.  The decays are traced to the element potassium and in particular to the isotope $ ^{40}$ K which constitutes $ 1.17\%$ of normal potassium.  Calculate the half life of the nuclide.
\item (2013)  Distinguish between white spectrum and line spectrum. 
\item (2013)  What is the significance of the binding energy per nucleon? 
\item (2013)  Briefly explain why the $ \beta$  — particles emitted from a radioactive source differ from the electrons obtained by thermionic emission? 
\item (2013)  The mass of a particular radioisotope in « sample is initially $ 6.4 \times 10^{-3}$ kg, After $ 42$ days the isotope was separated from the sample and found to have a mass of $ 1.0 \times 10^{-4}$ kg. Calculate the half- life of the isotope.
\item (2015)  Define activity and half-life.
\item (2015)  The half-life of radioactive substance is $ 1$ hour.  How long will it take for $ 60\%$ of the substance to decay?
\item (2015)  What is a nuclear reactor?
 \begin{itemize}
\item Briefly explain any three main components in a nuclear reactor.
\end{itemize}
\item (2015)  Sketch the binding energy curve.
 \begin{itemize}
\item State any two conclusions that can be drawn from the curve above.
\end{itemize}
\item (2015)  If the mass of deuterium nucleus is $ 2.015$ a.m.u, that of one isotope of helium is $ 3.017$ a.m.u. and that of neutron is $ 1.009$ a.m.u., calculate the energy released by the fusion of $ 1$ kg of deuterium. 
 \begin{itemize}
\item Suppose $ 50\%$ of this energy was used to produce $ 1$ MW of electricity, for how many days would be able to function.
\end{itemize}
\item (2016)  The number of particles $ n$ crossing a unit area perpendicular to $ x-$ axis in a unit time is given as $ n=-D(n_{2}-n_{1})/(x_{2}-x_{1})$ where $ n_{1}$ and $ n_{2}$ are the number of particles per unit volume for the values of $ x_{1}$ and $ x_{2}$ respectively.  What are the dimensions of diffusion constant $ D$ ?
\item (2016)  Differentiate natural radioactivity from artificial radioactivity.
\item (2016)  Name three applications of radioisotopes in medicine.
\item (2016)  State two conditions for stability of nuclides referring to light nuclides and heavy nuclides.
\item (2016)  Derive an expression for the half-life using the radioactive decay law.
\item (2016)  What is carbon $ -14$ ?  Explain its production and how it is used in the dating process.
\item (2016)  Living wood has an activity of $ 16.0$ counts per minute per gram of carbon.  A certain sample of dead wood is found to have an activity of $ 18.4$ counts per minute for $ 4.0$ grams.  Calculate the age of the sample of dead wood.  Assume the half-life of carbon $ -14$ is $ 5568$ years.
\item (2017)  What is meant by the following?
 \begin{itemize}
\item Atomic Mass Unit (a.m.u.)
\item Binding energy. 
\item Mass defect
\end{itemize}
\item (2017)  Write down the equation for the disintegration.
\item (2018)  Use the concept of radioactive decay and nuclear reactions to define the following terms:
 \begin{itemize}
\item $ \alpha $ decay
\item $ \beta$ decay
\item $ \gamma $ decay
\item Fission
\item Fusion.
\item For each of the terms above, give one suitable reaction equation. 
\end{itemize}
\item (2018)  A freshly prepared sample of a radioactive isotope $ Y$ contains $ 10^{12}$ atoms. The half-life of the isotope is $ 15$ hours. Calculate;
 \begin{itemize}
\item the initial activity. 
\item the number of radioactive atoms of $ Y$ remaining after $ 2$ hours, 
\end{itemize}
\item (2018)  Mention any four important features in the design of a nuclear reactor.
\item (2018)  Differentiate binding energy from mass defect.
\item (2018)  Calculate the binding energy per nucleon, in MeV and the packing fraction of an alpha particle.
\item (2018)  A freshly prepared sample of a radioactive isotope $ Y$ contains $ 10^{12}$ atoms. The half-life of the isotope is $ 15$ hours. Calculate;
 \begin{itemize}
\item the initial activity. 
\item the number of radioactive atoms of $ Y$ remaining after $ 2$ hours, 
\end{itemize}
\item (2018)  Mention any four important features in the design of a nuclear reactor.
\item (2018)  Differentiate binding energy from mass defect.
\item (2018)  Calculate the binding energy per nucleon, in MeV and the packing fraction of an alpha particle.
 \begin{itemize}
\item Given: Mass of proton $ =1.0080$ u, Mass of neutron $ =1.0087$ u and Mass of alpha particle $ =4.0026$ u.
\end{itemize}
\item (2019)  What is meant by the following terms as used in nuclear Physics?
 \begin{itemize}
\item Mass defect 
\item Binding energy. 
\end{itemize}
\item (2019)  Elaborate two aspects on which fission reactions differs from fusion reactions.
\item (2019)  Why is high temperature required to cause nuclear fusion? 
\end{itemize}

\end{document}

\documentclass{article}
\usepackage[a4paper, total={6in, 8in}]{geometry}
% \usepackage[utf8]{inputenc}
\usepackage{abstract}
\title{\textbf{7.2 - Electric Potential}}
\author{PJ Gibson - Peace Corps Tanzania}
\date{May 2020}

\begin{document}

\maketitle

\begin{itemize}
\item (1998)  Describe and explain briefly a method for measuring the specific charge. Mention the errors expected in this method.
\item (1999)  Write down an expression for the forces on an electron when moving perpendicular to: an electric field
 \begin{itemize}
\item Write down an expression for the forces on an electron when moving perpendicular to: a magnetic field.
\end{itemize}
\item (1999)  An electron is moving in a uniform electric field of intensity $ 1.2 \times 10^5$ Vm​ $ -1$ .  Find the acceleration of the electron.
\item (2000)  What is electric potential at a point in an electrostatic field? 
\item (2000)  A proton of mass $ 1.673 \times 10^{-27}$ kg falls through a distance of $ 1.5$ cm in a uniform electric field of magnitude $ 2.0 \times 10^{4}$ N C$ ^{-1}$ . Determine the time of fall [Neglect $ g$ and air resistance.]
\item (2007)  What is the potential at the centre of the square of side $ 1.0$ m, due to charges:
 \begin{itemize}
\item $ q_{1}=+1.0\times10^{-8}$ C , $ q_{2}=-2.0\times10^{-8}$ C , $ q_{3}=+3.0\times10^{-8}$ C , $ q_{1}= +2.0\times10^{-8}$ C
\item situated at the corners of the square?
\end{itemize}
\item (2007)  A charge $ Q$ is distributed over the concentric hollow spheres of radii $ r$ and $ R$ $ (R>r)$ such that the surface densities are the same.  Calculate the potential at the common centre of the two spheres.
\item (2010)  Show that the path of an electron moving In an electric field is a parabola.
\item (2013)  Define electric potential.
\item (2013)  A radioactive source in the form of metallic sphere of radius $ 1.0$ cm emits Beta particles at the rate of $ 5.0 \times 10^{10}$ particles per second.  If the source is electrically insulated, how long will it take for its electric potential to be raised by $ 2.0$ V? (assuming that $ 40\%$ of the emitted Beta-particles escape the source).
\item (2013)  A silver and copper voltammeter are connected in parallel across a $ 6$ V battery of negligible internal resistance. In half an hour $ 1.0$ g of copper and $ 2.0$ g of silver are deposited. Calculate the rate at which the energy is supplied by the battery. 
\item (2015)  Differentiate electric potential from electric potential difference.
\item (2015)  Sketch a graph of variation of electrical potential from the centre of a hollow charged conducting sphere of radius, $ r$ , up to infinity.  Explain the shape of the graph.
\item (2015)  A square ABCD has each side of $ 100$ cm.  Four points charges of $ +0.04$ $\mu$C, $ -0.05$ $\mu$C, $ +0.06$ $\mu$C, and $ +0.05$ $\mu$C are placed at $ A$ , $ B$ , $ C$ , and $ D$ respectively.  Calculate the electric potential at the centre of the square.
\item (2016)  A proton of mass $ 16.7 \times 10^{-28}$ kg falls through a distance of $ 2.5$ cm in a uniform electric field of magnitude $ 2.65 \times 10^{4}$ V$/$m.  Determine the time of fall if the air resistance and the acceleration due to gravity, $ g$ , are neglected.
\item (2017)  Define the terms capacitance and electric potential. 
\item (2018)  Why the emf of a cell is sometimes called a special terminal potential difference? 
\item (2019)  What is the potential difference between two points if $ 5$ Joules of work are required to move $ 10$ Coulombs from one point to another? 
\item (2019)  Define the terms electric potential and electric field-strength $ E$ at a point in the electrostatic field.
 \begin{itemize}
\item How the two quantities above related? 
\end{itemize}
\item (2019)  Can there be a potential difference between two adjacent conductors carrying the same positive charge? Give a reason. 
\end{itemize}

\end{document}
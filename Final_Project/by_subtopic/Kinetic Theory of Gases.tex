
\documentclass{article}
\usepackage[a4paper, total={6in, 8in}]{geometry}
% \usepackage[utf8]{inputenc}
\usepackage{abstract}
\title{\textbf{4.3 - Kinetic Theory of Gases}}
\author{PJ Gibson - Peace Corps Tanzania}
\date{May 2020}

\begin{document}

\maketitle

\begin{itemize}
\item (1999)  Write down the equation of state of an ideal gas defining all the symbols used.
\item (1999)  If the root-mean-square velocity of a hydrogen molecule at $ 0​ ^{\circ}$C is $ 1840$ m$/$s, find the root-mean-square velocity of the molecule at $ 100​ ^{\circ}$ ​ $ C$ .
\item (1999)  State the main assumptions of the “kinetic theory" of gases.
\item (1999)  Derive an expression for the pressure exerted by an ideal gas on the walls of its container.
\item (1999)  How does the average translational kinetic energy of a molecule of an ideal gas change if
 \begin{itemize}
\item the pressure is doubled while the volume is kept constant?
\item the volume is doubled while the pressure is kept constant?
\end{itemize}
\item (1999)  Calculate the value of the root mean-square speed of molecules of helium at $ 0^{\circ}$C .
\item (2000)  What factors lead the real gas to obey the ideal gas equation $ PV = RT$ ?
\item (2000)  Define the root-mean-square (r.m.s.) speed of the gas molecules. Hence find the r.m.s. speed of oxygen gas molecules at $ 10^{5}$ Pa pressure when the density is $ 1.43$ kg$/$m$ ^{3}$ .
\item (2000)  Derive an expression for the work done per mole in an isothermal expansion of Vander Waal’s gas from volume $ V_{1}$ to volume $ V_{2}$ .
\item (2007)  Define an ideal gas.
\item (2007)  State the four $ (4)$ assumptions necessary for an ideal gas that are used to develop the expression $ p=$ ½ $ \rho C^{2}$ .
\item (2007)  How is pressure explained in terms of the kinetic theory? 
\item (2007)  Without a detailed mathematical analysis argue the steps to follow in deriving the relation $ p=$ ½ $ \rho C^{2}$ .
\item (2007)  Define the temperature of an ideal gas as a consequence of the kinetic theory.
\item (2007)  A mole of an ideal gas at $ 300K$ is subjected to a pressure of $ 10^{5}N/m^{2}$ and its volume is $ 2.5 \times 10^{-2}m^{3}$ .  Calculate the:
 \begin{itemize}
\item molar gas constant $ R$
\item Boltzmann constant $ k$
\item average transnational kinetic energy of a molecule of the gas.
\end{itemize}
\item (2013)  Define comprehensibility of a gas in terms of the elasticity of gases. 
\item (2013)  Helium gas occupies a volume of $ 4 \times 10^{-2}$ m$ ^{3}$ at a pressure of $ 2 \times 10^{5}$ Pa and temperature of $ 300$ K. Calculate the mass of helium and the r.m.s speed of its molecules.
\item (2014)  One mole of a gas expands from volume, $ V_{1}$ , to a volume $ V_{2}$ . If the gas obeys the Van-der-Waal’s equation, $ (p+ a/v^{2})(v – b)=$ RT, derive the formula for work done in this process.
\item (2019)  Based on the kinetic theory of gases determine:
 \begin{itemize}
\item The average translational kinetic energy of air at a temperature of $ 290$ K.
\item The root mean square seed (r.m.s) of air at the same temperature (above).
\end{itemize}
\end{itemize}

\end{document}

\documentclass{article}
\usepackage[a4paper, total={6in, 8in}]{geometry}
% \usepackage[utf8]{inputenc}
\usepackage{abstract}
\title{\textbf{9.2 - Electric Conduction in Gases}}
\author{PJ Gibson - Peace Corps Tanzania}
\date{May 2020}

\begin{document}

\maketitle

\begin{itemize}
\item (1998)  What is thermionic emission?
\item (2013)  Explain the following observation:
 \begin{itemize}
\item Light in the bulb comes on once the switch is kept on despite the drift velocity of electrons being very low.
\item The potentiometer is said to be a better device for measuring the potential difference (p.d) than a moving coil voltmeter.
\end{itemize}
\item (2013)  A milliameter connected in series with a hydrogen discharge tube indicates a current of $ 1.0 \times 10^{-3}$ A. If the number of electrons passing the cross section of the tube at a particular point is $ 4.0 \times 10^{15}$ per second, find the number of protons that pass the same cross section per second. 
\item (2015)  Sketch the diagram showing the variation of current with potential difference across the following:
 \begin{itemize}
\item  Filament electric bulb. 
\item Gas-filled diode. 
\end{itemize}
\item (2018)  Distinguish between ionization energy and excitation energy.
\end{itemize}

\end{document}

\documentclass{article}
\usepackage[a4paper, total={6in, 8in}]{geometry}
% \usepackage[utf8]{inputenc}
\usepackage{abstract}
\title{\textbf{6.3 - Sound}}
\author{PJ Gibson - Peace Corps Tanzania}
\date{May 2020}

\begin{document}

\maketitle

\begin{itemize}
\item (2000)  Two open organ pipes of length $ 50$ cm and $ 51$ cm respectively give beat frequency of $ 6.0$ Hz when sounding their fundamental notes together, neglecting end corrections. What value does this give for the velocity of sound in air?
\item (2007)  A metre-long tube at one end, with a movable piston at the other end, shows resonance with a fixed frequency source (a tuning fork) of frequency $ 340$ Hz when the tube length is $ 25.5$ cm or $ 79.3$ cm.  Estimate the speed of sound in air at the temperature of the experiment (ignore edge effects).
\item (2007)  The shortest length of the resonance tube closed at one end which resounds to a fork of frequency $ 256$ Hz is $ 32.0$ cm.  The corresponding length for a fork of frequency $ 384$ Hz is $ 20.8$ cm.  Determine the end correction for the tube and the velocity of sound in air.
\item (2013)  A small speaker emitting $ 4$ note of frequency $ 250$ Hz is placed over the open upper end of a vertical tube which is full of water. When the water is gradually run out of the tube the air column resonates. If the initial and final position of the water surface below the top are $ 0.31$ m and $ 0.998$ m respectively, calculate the speed of sound in air and the end-correction of the tube. 
\item (2015)  A source of sound emits waves of frequency, $ f$ , and is moving with a speed of $ u_{s}$ towards the listener and away from the listener.  Derive an expression for apparent frequency $ f_{A}$ of sound in each case if the velocity of sound wave in air is v.  
\item (2016)  Define the following terms:
 \begin{itemize}
\item Intensity of sound
\item Beats
\item Ultrasonic
\item Overtones
\end{itemize}
\item (2016)  Give any two applications of ultrasonic as applied to sound waves.
\item (2017)  Briefly explain why diffraction is common in sound but not in light.
\item (2018)  The shortest length of the resonance tube closed at one end which resounds to fork of frequency $ 256$ Hz is $ 31.6$ cm, The corresponding length for a fork of frequency $ 384$ Hz is $ 20.5$ cm. Determine the end correction for the tube and the velocity of sound in air. 
\item (2019)  Provide one evidence which proves that sound is a wave.
\item (2019)  Why thunder of lightning is heard some moments after seeing the flash?
\end{itemize}

\end{document}
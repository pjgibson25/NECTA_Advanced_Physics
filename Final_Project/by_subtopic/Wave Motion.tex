
\documentclass{article}
\usepackage[a4paper, total={6in, 8in}]{geometry}
% \usepackage[utf8]{inputenc}
\usepackage{abstract}
\title{\textbf{6.2 - Wave Motion}}
\author{PJ Gibson - Peace Corps Tanzania}
\date{May 2020}

\begin{document}

\maketitle

\begin{itemize}
\item (2000)  What vibrates in the following types of wave motion?
 \begin{itemize}
\item Light waves
\item Sound waves
\item X-rays
\item Water waves
\end{itemize}
\item (2000)  A plane progressive wave on a water surface is given by the equation $ y=2 \sin 2x(100t -x/30)$ ; where $ x$ is the distance covered in a time $ t$ . $ x$ , $ y$ and $ t$ are in cm and seconds respectively.  Find:
 \begin{itemize}
\item the wavelength, and frequency of the wave motion.
\item the phase difference between two points on the water surface that are $ 60$ cm apart.
\end{itemize}
\item (2007)  Give two $ (2)$ differences between progressive and standing waves.
\item (2007)  Two progressive waves travelling along the same line in a medium are represented by $ Y_{1}=10 \sin(\omega t +\pi/2)$ and $ Y_{2}=10 \sin(\omega t +\pi/6)$
 \begin{itemize}
\item If the two progressive waves form a standing wave, determine the resultant amplitude and phase angle of the wave formed.
\end{itemize}
\item (2010)  Distinguish between stationary waves and progressive waves.
\item (2010)  A wave is represented by the equation $ y=10 \sin(0.42\pi(60$ t-x)), where the distance parameters are measured in metres and the time in seconds.
 \begin{itemize}
\item State whether the wave is stationary or progressive.
\item Determine the wavelength and frequency of the wave.
\item What will be the phase difference between two points which are $ 40$ cm apart? 
\item Calculate the period and amplitude of the wave. 
\end{itemize}
\item (2013)  Define the term standing wave.
\item (2013)  State the position in a stationary wave where a man can hear a louder sound.
\item (2016)  State the principle of:
 \begin{itemize}
\item Superposition of waves
\item Huygens construction of wave fronts.
\end{itemize}
\item (2017)  The equation $ y= a  \sin(\omega t – kx)$ represents a plane wave traveling in a medium along the $ x$ - direction, $ y$ being the displacement at the point $ x$ at time $ t$ . Deduce whether the wave is traveling in the positive $ x$ – direction or in the negative $ x$ – direction.
 \begin{itemize}
\item If $ z=1.1 \times 10^{-7}$ m , $ \omega = 6.5 \times 10^{3}$ s$ ^{-1}$ , $ k=19$ m$ ^{-1}$ ; determine the speed of the wave.
\end{itemize}
\item (2018)  What do you understand by the terms:
 \begin{itemize}
\item Progressive wave 
\item Refraction of waves 
\item Diffraction of waves 
\item Standing wave. 
\end{itemize}
\item (2018)  Two progressive waves traveling in the opposite direction in the medium are represented by $ Y_{1}=5 \sin(\omega t+\pi/3)$ and  $ Y_{2}=5 \sin(\omega t- \pi/3)$ . If the two progressive waves form a standing wave, determine the resultant amplitude and the phase angle formed. 
\item (2019)  Give the meaning of the terms wave function, longitudinal wave and transverse waves.
\item (2019)  The equation of a Progressive wave traveling in the $ +x$ direction is given by $ y= a \sin(\omega t-kx)$ .  Show that the maximum velocity, $ V_{max}=2\pi a /T$ . 
\end{itemize}

\end{document}
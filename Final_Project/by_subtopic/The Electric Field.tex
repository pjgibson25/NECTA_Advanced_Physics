
\documentclass{article}
\usepackage[a4paper, total={6in, 8in}]{geometry}
% \usepackage[utf8]{inputenc}
\usepackage{abstract}
\title{\textbf{7.1 - The Electric Field}}
\author{PJ Gibson - Peace Corps Tanzania}
\date{May 2020}

\begin{document}

\maketitle

\begin{itemize}
\item (1998)  The distance between the electron and proton in the hydrogen atom is about $ 5.3 \times 10^{11}$ m. Calculate the electrical and gravitational forces between these particles. How do they compare?
\item (1999)  Explain why an uncharged metal is attracted by a charged one?
\item (1999)  Charges $ Q​_{1}=1.2 \times 10$ ​$ E-12$ C and $ Q​_{2}=-4 \times 10$ ​ $ -12$ C are placed $ 5.0$ m apart in air. A third charge $ Q​_{3}=1 \times 10^{-14}$ C is introduced midway between them. Find the resultant force on the third charge.
\item (2000)  Derive an expression for an electric potential at a point a distance a from a positive point charge $ Q$ .
\item (2000)  Positive charge is distributed over a solid spherical volume of radius $ R$ and the charge per unit volume is $ \sigma $
 \begin{itemize}
\item Show that the electric field inside the volume at a distance $ r<R$ from the centre is given by $ E=(\sigma r/3e_{0})$
\item What is the electric field at a point $ r>R$ (i.e. outside the spherical volume).
\end{itemize}
\item (2000)  A proton is placed in a uniform electric field $ E$ . What must be the magnitude and direction of the field if the electrostatic force acting on the proton $ 1$ s just to balance its weight?
\item (2000)  Give the statement of Coulomb’s law.
\item (2000)  A $ 100$ V battery terminals are connected to two large and parallel plates which are $ 2$ cm apart. The field in the region between the plates is nearly uniform. 
 \begin{itemize}
\item If electric field intensity $ E$ is $ 10^{6}$ N C$ ^{-1}$ and points vertically upwards, determine the force of an electron in this field and compare it with the weight of an electron. An electron is released from rest from the upper plate inside the field above.
\item At what velocity will it hit the lower plate?
\item Determine its kinetic energy and the time it takes for the whole journey.
\end{itemize}
\item (2007)  Two similar balls of mass $ m$ are hung from silk thread of length "a" and carry a similar charge $ q$ .  Assume $ \theta $ is small enough that $ X = (\frac{q^2 a}{2 \pi \epsilon_0 m g})^{1/3}$
 \begin{itemize}
\item where $ X$ is the distance of separation.
\end{itemize}
\item (2010)  State Coulomb’s law for charged particles.
\item (2010)  Does the coulomb force that one charge exert on another charge change when a third charge is brought nearby? Explain.
\item (2010)  The electric field intensity inside a capacitor is $ E$ . What is the work done in displacing a charge $ q$ over a closed rectangular surface?
\item (2010)  Explain the following observations:
 \begin{itemize}
\item A dressing table mirror becomes dusty when wiped with a dry cloth on a warm day.
\item A charged metal ball comes into contact with an uncharged identical ball.  (Illustrate your answer by using diagrams).
\end{itemize}
\item (2010)  Without giving any experimental or theoretical detail explain how the results of Millikan’s experiment led to the idea that charge comes in ‘packets’, the size of the smallest packet being carried by an electron. 
\item (2013)  Describe Coulomb’s law and give the dimensions of each quantity.
\item (2015)  Two bodies $ A$ and $ B$ are $ 0.1$ m apart.  A point charge of $ 3\times 10^{-3}$ $\mu$C is placed at A and a point charge of $ 1\times 10^{-9}\mu C$ is placed at $ B$ .  $ C$ is the point on the straight line between $ A$ and $ B$ , where the electric potential is zero.  Calculate the distance between $ A$ and $ C$ .
\item (2016)  State coulomb’s law of electrostatics.
\item (2016)  Define electric field strength, $ E$ at any point.
\item (2016)  Mention two common properties of electric field lines.
\item (2016)  By using the coulomb’s law of electrostatics, derive an expression for the electric field strength $ E$ , due to a point charge if the material is surrounded by a material of permittivity $ \epsilon $ , and hence show how it relates with charge density $ \rho $ .
\item (2018)  Two point charges of equal mass $ m$ and charge $ Q$ are suspended at a common point by two threads of negligible mass and length $ L$ . If the two point charges are at equilibrium,  show that;
 \begin{itemize}
\item The distance of separation $ x=({Q^{2}L}/{2\pi\epsilon _{0}mg})^{1/3}$
\item The angle of inclination $ \beta = ^{3}\sqrt{(Q^{2})/(16\pi\epsilon _{0}mgL^{2})}$ 
\end{itemize}
\item (2018)  Two point charges, $ q_{A}=+3$ $\mu$C and $ q_{b}=-3$ $\mu$C, are located $ 0.2$ m apart in vacuum. Find; 
 \begin{itemize}
\item the electric field at the midpoint of the line joining two charges. 
\item the force experienced by the negative test charge of magnitude $ 1.5 \times 10^{-9}$ C placed at this point.
\end{itemize}
\item (2019)  State Coulomb’s law of force between two electrically charged bodies. 
\end{itemize}

\end{document}

\documentclass{article}
\usepackage[a4paper, total={6in, 8in}]{geometry}
% \usepackage[utf8]{inputenc}
\usepackage{abstract}
\title{\textbf{3.2 - Bernoulli's Principle}}
\author{PJ Gibson - Peace Corps Tanzania}
\date{May 2020}

\begin{document}

\maketitle

\begin{itemize}
\item (1999)  The velocity at a certain point in a flow pipe is $ 1.0$ ms$ ^{-1}$ and the gauge pressure there is $ 3 \times 10^5 $ N$/$m$ ^{2}$ ​ . The cross-sectional area at a point $ 10$ m above the first is half that at the first point. If the flowing fluid is pure water, calculate the gauge pressure at the second point.
\item (2000)  Write down the Bernoulli's equations for fluid flow in a pipe and indicate the term which will disappear when the flow of fluid is stopped.
\item (2000)  Water flows into a tank of large cross-section area at a rate of $ 10^{-4}$ m$ ^{3}/$s but flows out from a  hole of area 1cm$ ^{2}$ which has been punched through the base. How high does the water rise in the tank?
\item (2007)  Under what conditions is the Bernoullis’ equation applicable?
\item (2007)  Discuss two $ (2)$ applications of the Bernoullis equation. 
\item (2013)  A submarine model is situated in a part of a tube with diameter $ 5.1$ cm where water moves at $ 2.4$ m$/$s.  Determine the:
 \begin{itemize}
\item velocity of flow in the water supply pipe of diameter $ 25.4$ cm. 
\item pressure difference between the narrow and the wide tube. 
\end{itemize}
\item (2015)  Write down the Bernoulli’s equation for fluid flow in a pipe and indicate the term which will disappear when the fluid is stopped.
\item (2015)  Basing on the applications of Bernoulli’s principle, briefly explain why two ships which are moving parallel and close to each other experience an attractive force.
\item (2015)  Water is flowing through a horizontal pipe having different cross-sections at two points $ A$ and $ B$ .  The diameters of the ippe at $ A$ and $ B$ are $ 0.6$ m and $ 0.2$ m respectively.   The pressure difference between points $ A$ and $ B$ is $ 1$ m column of water.  Calculate the volume of water flowing per second.
\item (2016)  Distinguish between static pressure, dynamic pressure and total pressure when applied to streamline or laminar fluid flow and write down expression at a point in the fluid in terms of the fluid velocity v, the fluid density $ \rho $ , pressure $ P$ and the height $ h$ , of the point with respect to a datum.  
\item (2016)  The static pressure in a horizontal pipeline is $ 4.3 \times 10^{4}$ Pa, the total pressure is $ 4.7 \times 10^{4}$ Pa and the area of cross-section is $ 20 $ cm$ ^{2}$ . The fluid may be considered to be incompressible and non-viscous and has a density of $ 1000$ kg$/$m$ ^{3}$ .  Calculate the flow velocity and the volume flow rate in the pipeline.
\item (2016)  Briefly explain the carburetor of a car as applied to Bernoulli’s theorem.
\item (2016)  Three capillaries of the same length but with internal radii $ 3R$ , $ 4R$ , and $ 5R$ are connected in series and a liquid flows through them under streamline conditions.  If the pressure across the third capillary is $ 8.1$ mm of liquid, find the pressure across the first capillary.
\item (2017)  State Bernoulli's theorem for the horizontal flow. 
\item (2017)  On which principle does the Bernoulli's theorem based. 
\item (2017)  A pipe is running full of water. At a certain point $ A$ , it tapers from $ 30$ cm diameter to $ 10$ cm diameter at $ B$ , the pressure difference between point $ A$ and $ B$ is $ 100$ cm of water column. Find the rate of flow of water through the pipe. 
\item (2017)  Two capillaries of the same length and radii in the ratio of $ 1$:$ 2$ are connected in series and the liquid flow through the system under stream line conditions. If the pressure across the two extreme ends of the combination is  $ 1$ m of water, what is the pressure difference across the first capillary?
\item (2018)  Given the Bernoulli’s equation: $ p+\rho gh+\rho v^{2}=$ constant where all the symbols carry their usual meaning.
 \begin{itemize}
\item What quantity does each expression on the left hand side of the equation represent? 
\item Mention any three conditions which make the equation to be valid. 
\end{itemize}
\item (2018)  Water is supplied to a house at ground level through a pipe of inner diameter $ 1.5$ cm at an absolute pressure of $ 6.5 \times 10^{5}$ Pa and velocity of $ 5$ m$/$s. The pipe line leading to the second floor bath room $ 8$ m above has an inner diameter of $ 0.75$ cm. Find the flow velocity and pressure at the pipe outlet in the second floor bathroom. 
\item (2018)  A horizontal pipeline increases uniformly from $ 0.080$ m diameter to $ 0.160$ m diameter in the direction of flow of water. When $ 96$ litres of water is flowing per second, a pressure gauge at the $ 0.080$ m diameter section reads $ 3.5 \times 10^{5}$ Pa. What should be the reading of the gauge at the $ 0.160$ m diameter section neglecting any loss? 
\item (2019)  A horizontal pipe of cross - sectional area $ 10 $ cm$ ^{2}$ has one section of cross sectional area $ 5 $ cm$ ^{2}$ . If water flows through the pipe, and the pressure difference between the two sections is $ 300$ Pa, how many cubic meters of water will flow out of the pipe in $ 1$ minute?
\end{itemize}

\end{document}
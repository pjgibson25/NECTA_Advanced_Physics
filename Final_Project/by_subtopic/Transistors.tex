
\documentclass{article}
\usepackage[a4paper, total={6in, 8in}]{geometry}
% \usepackage[utf8]{inputenc}
\usepackage{abstract}
\title{\textbf{10.3 - Transistors}}
\author{PJ Gibson - Peace Corps Tanzania}
\date{May 2020}

\begin{document}

\maketitle

\begin{itemize}
\item (1999)  Draw the symbol of $ n-p-n$ transistor.
\item (1999)  With the help of illustrative diagrams explain the action of a choke in a circuit.
\item (1999)  Explain the term “thermal run away” as regards a transistor amplifier.
\item (2000)  Briefly discuss the formation of the potential difference barrier (depletion layer) of a $ p-n$ junction diode.
\item (2000)  Using $ p-n$ junction diodes, draw the arrangement of a full-wave rectifier and briefly explain how it works.
 \begin{itemize}
\item Define the electron – volt.
\end{itemize}
\item (2000)  Mention any three uses of a transistor
\item (2000)  A certain transistor has a current gain $  \beta =55$ . If it is used in a circuit with common-base configuration, how much change occurs in the collector current if an emitter current is changed by $ 100$ micro A? (Assume the collector potential to be constant and neglect the small collector — current due to the minority current carriers).
\item (2010)  Briefly explain why a $ P-N$ junction is referred as a junction diode.
\item (2013)  What is meant by transistor action?
\item (2013)  Briefly explain why the collector of a transistor is made wider than the emitter and base?
\item (2013)  Derive the closed – loop gain A of an inverting amplifier.  If the input resistor is equal to the feedback resistor, what would be the value of the gain A?
\item (2014)  Mention two types of transistors.
 \begin{itemize}
\item Which among the transistors mentioned above responds quickly to electrical signal? Give reason for your answer.
\end{itemize}
\item (2015)  A wire of diameter $ 0.1$ mm and resistivity $ 1.69\times10^{-8}\Omega$ m with temperature coefficient
 \begin{itemize}
\item of resistance of $ 4.3\times10^{-3}$ K$ ^{-1}$ was required to make a resistance,
\item  What length of the wire is required to make a coil with a resistance of $ 0.5\Omega $ ?
\item If on passing a Current of $ 2$ A the temperature of the coil above rises  by $ 10^{\circ}$C, what error would arise in taking the potential drop as $ 1.0$ V 
\end{itemize}
\item (2015)  Why a $ p-n$ junction diode when connected in a circuit and then reversed gives a very small leakage current across the junction? 
 \begin{itemize}
\item How is the size of the current stated in above dependent on the temperature of the diode?
\end{itemize}
\item (2015)  Define closed loop gain. 
\item (2016)  Define the term junction as applied in electrical network.
\item (2016)  Why are transistors mostly used in common emitter arrangement?
\item (2016)  When does a transistor amplifier work as an oscillator?
\item (2017)  List three types of transistor configurations.
\item (2017)  Why is collector of a transistor made wider than emitter and base? 
\item (2018)  What do you understand by the term node as applied to electric circuit?
\item (2018)  Mention four types of energy losses suffered by a transformer.  
\item (2018)  What is meant by depletion layer as used in pn -junction device? 
\item (2018)  Describe the effect of applying a reverse bias to the junction diode. 
\item (2018)  Sketch the graph of transfer characteristic of a transistor. 
 \begin{itemize}
\item State the significance of the slope from the graph above.
\end{itemize}
\item (2018)  What is the basic condition for a transistor to operate properly as an amplifier? 
\item (2018)  Briefly explain how a junction transistor can be connected to act as a current operated device. 
\item (2018)  Why the magnitude of output frequency of a full wave rectifier is twice the input frequency? 
\item (2018)  Draw a simple basic transistor switching circuit diagram. 
\item (2019)  Why transistors can not be used as rectifiers? 
\item (2019)  In NPN transistor circuit the collector current is $ 5$ mA. If $ 95\%$ of the emitted electrons reach the collector region, calculate the base current. 
\item (2019)  What causes damage to transistors? 
\end{itemize}

\end{document}
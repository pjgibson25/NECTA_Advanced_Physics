
\documentclass{article}
\usepackage[a4paper, total={6in, 8in}]{geometry}
% \usepackage[utf8]{inputenc}
\usepackage{abstract}
\title{\textbf{7.3 - Capacitance}}
\author{PJ Gibson - Peace Corps Tanzania}
\date{May 2020}

\begin{document}

\maketitle

\begin{itemize}
\item (1998)  A girl is holding a metal rod in her hand and rubs its surface with fur. Explain what happens to the rod.
\item (1998)  Can charge be conserved? Give at least two examples to support your answer.
\item (1998)  A capacitor of capacitance $ 3$ micro$ -F$ is charged until a potential difference of $ 200$ V is developed across its plates. Another capacitor of capacitance $ 2$ micro$ -F$ developed a p.d. of $ 100$ V across its plates on being charged.
 \begin{itemize}
\item What is the energy stored in each capacitor?
\item The capacitors are then connected by wires of negligible resistance, so that the plates carrying like charges are connected together. What is the total energy stored in the combined capacitors?
\item What would the time constant of the circuit be, if the resistance of each wire connecting the plates was $ 10\Omega $ ?
\end{itemize}
\item (1999)  What is "capacitance"?
\item (1999)  List three factors that govern the capacitance of a parallel plate capacitor.
\item (1999)  Show that the energy per unit volume stored in a parallel plate capacitor is given by: $ U=1/2\epsilon E^{2}$ and define all the symbols in this equation.
\item (1999)  Given that the distance of separation between the parallel plates of a capacitor is $ 5$ mm, and the plates have an area of $ 5$ m$ ^{2}$ . A potential difference of $ 10$ kV is applied across the capacitor which is
 \begin{itemize}
\item parallel in vacuum. Compute:
\item the capacitance
\item the electric intensity in the space between the plates
\item the change in the stored energy if the separation of the plates is increased from $ 5$ mm to $ 5.5$ mm.
\end{itemize}
\item (1999)  When an impedance consisting of an inductance $ L$ and a resistance $ R$ in series is connected across a $ 12$ V, $ 50$ Hz power supply, a current of $ 0.050$ A flows, which differs in phase from that of the applied potential difference by $ 60^{\circ}$ .
 \begin{itemize}
\item Find the value of $ R$ and $ L$ .
\item Find the capacitance of the capacitor which, when connected in series in the above circuit, has the effect of bringing the current into phase with the applied voltage.
\end{itemize}
\item (1999)  (i) Show that the possible energy levels (in Joules) for the hydrogen atom are given by the formula:
 \begin{itemize}
\item $ E_{n}=-me^{4}/(8h^{2}\epsilon _{0}^{2}$ * $ 1/n^{2}$
\item where $ m=$ mass of the electron
\item $ e=$ electronic charge
\item $ h=$ Planck's constant
\item $ \epsilon _{0}=$ permittivity constant of vacuum
\item What does the negative sign signify in the formula for $ E$ , in above?
\end{itemize}
\item (2000)  Electrons in a certain television tube are accelerated through a potential difference of $ 2.0$ kV
 \begin{itemize}
\item Calculate the velocity acquired by the electrons.
\item If these electrons lose all their energy on impact and given that $ 10^{12}$ electrons pass per second in the TV tube, calculate the power dissipated.
\end{itemize}
\item (2000)  A coil and a capacitor in parallel are used to make a tuning circuit for a radio receiver. Sketch the resonance curve for the circuit. State two ways of changing the circuit to increase the resonant frequency.
\item (2007)  What do you understand by an electrostatic generator?
\item (2007)  The belt of a Van de Graaf generator carries a charge of $ 100$ $\mu$C per metre.  If the diameter of the lower pulley is $ 10$ cm and its angular velocity is $ 5$ rad$/$s, what p.d. will the upper conductor attain in $ 5$ minutes if its capacitance to ground is $ 5x10^{-12}$ F and if there is no leakage of charge?
\item (2010)  Describe the action of dielectric in a capacitor.
\item (2010)  A capacitor of $ 12$ $\mu$F is connected in series with a resistor of $ 0.7$ M$ \Omega $ across a $ 250$ V d.c supply. Calculate the current and p.d across the capacitor after $ 4.2$ seconds.
\item (2010)  Show that the unit of CR (time constant) is seconds and prove that for a discharging capacitor it is the time taken for the charge to fall by $ 37\%$ . 
\item (2010)  The variable radio capacitor can be charged from $ 50$ pF to $ 950$ pF by turning the dial from $ 0$ degrees to $ 180$ degrees. With the dial at $ 180$ degrees, the capacitor is connected to a $ 400$ V battery. After charging the capacitor is disconnected from the battery and the dial is turned to $ 0$ degrees. What is the charge on the capacitor? What is the p.d across the capacitor when the dial reads $ 0$ degrees and the work done required to turn the dial to $ 0$ degrees? (Neglect frictional effects).
\item (2013)  Define electric discharge and give one example.
\item (2013)  An alternating current (a.c) of $ 0.2$ A r.m.s and frequency of $ 110/2\pi$ Hz flow in a circuit containing a series arrangement of a resistor $ R$ of resistance $ 20\Omega$ , an inductor $ L$ of $ 0.15$ H and a capacitor $ C$ of capacitance $ 500$ $\mu$F . Calculate the potential difference (p.d) and the impedence of the circuit.
\item (2015)  What do you understand by dielectric constant?
\item (2015)  When are the capacitors said to be connected in parallel?
\item (2015)  The parallel plate capacitor consisting of two metal plates each of area $ 20$ cm$ ^{2}$ placed at $ 1$ cm apart are connected to the terminals of an electrostatic voltmeter.  The system is charged to give a reading of $ 120$ V on the voltmeter scale.  When the space between the plates is filled with a glass of dielectric constant of $ 5$ , the voltmeter reading falls to $ 50$ V.  What is the capacitance of the voltmeter?  You may assume that volutage recorded by a voltmeter is directly proportional to the scale reading.
\item (2015)  A $ 4.0$ $\mu$F capacitor is charged by $ 12$ V supply and is then discharged through $ 1.5M\Omega $ resistor.  
 \begin{itemize}
\item Obtain the time constant.
\item Calculate the charge on the capacitor at the start of the discharge.
\item What will the value of the charge on the capacitor, the potential difference across the capacitor and the current in the circuit be $ 2$ seconds after the discharge starts?
\end{itemize}
\item (2016)  A $ 25$ $\mu$F capacitor, a $ 0.10$ H inductor and a $ 25\Omega $ resistor are connected in series with an a.c. source whose e.m.f. is given by $ E=310 \sin(314t)$ .  Determine the;
 \begin{itemize}
\item Frequency of the e.m.f.
\item Net reactance of the circuit.
\end{itemize}
\item (2016)  Two capacitors $ C_{1}$ and $ C_{2}$ each of area $ 36$ cm$ ^{2}$ separated by $ 4$ cm have capacities of $ 6$ $\mu$C and $ 8$ $\mu$C respectively.  The capacitor $ C_{1}$ is charged to a potential difference of $ 110$ V whereas the capacitor $ C_{2}$ is charged to a potential difference of $ 140$ V.  The capacitors are now joined with plates of like charges connected together.
 \begin{itemize}
\item What will be the loss of energy transferred to heat in the connecting wires?
\item What will be the loss of energy per unit volume transferred to heat in the connecting wires?
\end{itemize}
\item (2016)  Define the following terms:
 \begin{itemize}
\item Capacitance
\item Charge density
\item Equipotential surface
\end{itemize}
\item (2016)  Identify any three factors on which the capacitance of parallel plate capacitor depends.
\item (2016)  A parallel plate capacitor is made of a paper $ 40$ mm wider and $ 3.0 \times 10^{-2}$ mm thick.  Determine the length of the paper sheet required to construct a capacitance of $ 15$ $\mu$F , if its relative permitting is $ 2.5$ .
\item (2016)  Show that the possible energy levels (in joules) for the hydrogen atom are given by the formula: $ E_{n}=-k^{2}(2\pi^{2}me^{4}/h^{2})(1/n^{2})$ .  Where $ m$ is the mass of electron, $ e$ is the electronic charge, $ h$ is the Planck’s constant, $ k=1/4\pi\epsilon _{0}$ and $ \epsilon _{0}$ is the permittivity constant of vacuum.  
 \begin{itemize}
\item What does the negative sign signify in the formula above?
\end{itemize}
\item (2017)  A parallel plate capacitor has plates each of area $ 0.24$ m$ ^{2}$ separated by a small distance
 \begin{itemize}
\item $ 0.50$ mm. If the capacitor is full charged by a battery of electromotive force of $ 24$ V, calculate:
\item the capacitance of the capacitor. 
\item the energy stared tn the capacitor. 
\end{itemize}
\item (2017)  Comment on the assertion that, the safest way of protecting yourself from lightning is to be inside a car. 
\item (2018)  A series LCR circuit with inductance, $ L=0.12H$ , capacitance, $ C=480$ nF and resistance, $ R=23\Omega $ is connected to a $ 230V$ variable frequency supply. Determine the:
 \begin{itemize}
\item Maximum current flowing in the circuit. 
\item Source frequency for which the current is maximum. 
\end{itemize}
\item (2018)  Briefly explain the effect of the dielectric material on the capacitance of a capacitor when the capacitor is:
 \begin{itemize}
\item Isolated. 
\item Connected to the battery.
\end{itemize}
\item (2018)  How are the electrolytic capacitors made? 
\item (2019)  Elaborate three significance of dielectric material in a capacitor. 
\item (2019)  Give the reason behind a loss of electrical energy when two capacitors are joined either in series or parallel. 
\item (2019)  Why does a room light turn on at once when the switch is closed? Give comment.
\item (2019)  Outside the sphere, a charged sphere behaves like its charges were concentrated at the centre. If the electric field strength inside the sphere is zero and one sphere of radius $ 5.0$ cm carries a positive charge of $ 6.7$ nC, calculate; 
 \begin{itemize}
\item the potential at the surface of the sphere. 
\item the capacitance of the sphere. 
\end{itemize}
\item (2019)  What is meant by dielectric constant? 
\item (2019)  A parallel plate capacitor with air as a dielectric has plates of area $ 4.0 \times 10^{-2}$ m$ ^{2}$ which are $ 2.0$ mm apart. The capacitor is charged to $ 100$ V battery and connected in parallel with a similar unchanged capacitor with plates of half the area and twice the distance apart. If the edge effect is neglected, calculate the final charge on each plate. 
\item (2019)  Derive an expression for the total capacitance of two capacitors $ C_{1}$ and $ C_{2}$ connected in series. 
\item (2019)  Two capacitor of $ 15$ $\mu$F and $ 20$ $\mu$F are connected in series with a $ 600$ V supply.  Calculate the charge and Potential difference across each capacitor. 
\end{itemize}

\end{document}
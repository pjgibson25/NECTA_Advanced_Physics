
\documentclass{article}
\usepackage[a4paper, total={6in, 8in}]{geometry}
% \usepackage[utf8]{inputenc}
\usepackage{abstract}
\title{\textbf{2.3 - Uniform Circular Motion}}
\author{PJ Gibson - Peace Corps Tanzania}
\date{May 2020}

\begin{document}

\maketitle

\begin{itemize}
\item (2000)  Show that the period of a body of mass $ m$ revolving in a horizontal circle with constant velocity v at the end of a string of length $ l$ is independent of the mass of the object.
\item (2000)  A ball of mass $ 100$ g is attached to the end of a string and is swung in a circle of radius $ 100$ cm at a constant velocity of $ 200$ cm$/$s. While in motion the string is shortened to $ 50$ cm. Calculate:
 \begin{itemize}
\item The new velocity of the motion.
\item The new period of the motion.
\end{itemize}
\item (2000)  A car travels over a humpback bridge of radius of curvature $ 45$ m. Calculate the maximum speed of the car if the wheels are to remain in contact with the bridge.
\item (2007)  What is meant by centripetal force?
\item (2007)  Derive the expression $ a =(v^{2}/r)$ where a, v, and $ r$ stands for the centripetal acceleration, linear velocity and radius of a circular path respectively.  
\item (2007)  A ball of mass $ 0.5$ kg attached to a light inextensible string rotates in a vertical circle of radius $ 0.75$ m such that it has a speed of $ 5$ m$/$s when the string is horizontal.  Calculate:
 \begin{itemize}
\item  The speed of the ball and the tension in the string at the lowest point of its circular path.
\end{itemize}
\item (2010)  What is the origin of centripetal force for:
 \begin{itemize}
\item A satellite orbiting around the Earth. 
\item An electron in the hydrogen atom?
\end{itemize}
\item (2010)  A small mass of $ 0.15$ kg is suspended from a fixed point by a thread of a fixed length. The mass is given a push so that it moves along a circular path of radius $ 1.82$ m in a horizontal plane at a Steady speed, taking $ 18.0$ s to make $ 10$ complete revolutions. Calculate:
 \begin{itemize}
\item The speed of the small mass.
\item The centripetal acceleration. 
\item The tension in the thread. 
\end{itemize}
\item (2013)  Why is it technically advised to bank a road at corners?
\item (2013)  A wheel rotates at a constant rate of $ 10$ revolutions per second. Calculate the centripetal acceleration at a distance of $ 0.80$ m from the centre of the wheel.
\item (2014)  Define the term ‘radial acceleration’. 
\item (2015)  Mention three effects of looping the loop.
 \begin{itemize}
\item Why there must be a force acting on a particle moving with uniform speed in a circular path? Write down an expression for its magnitude. 
\end{itemize}
\item (2015)  A driver negotiating a sharp bend usually tend to reduce the speed of the car.
 \begin{itemize}
\item  What provides the centripetal force on the car?
\item Why is it necessary to reduce its speed?
\end{itemize}
\item (2015)  A ball of mass $ 0.5$ kg is attached to the end of a cord whose length is $ 1.5$ m then whirled in horizontal circle. If the cord can withstand a maximum tension of $ 50$ N calculate the:
 \begin{itemize}
\item Maximum speed the ball can have before the cord breaks. 
\item Tension in the cord if the ball speed is $ 5$ m$/$s
\end{itemize}
\item (2015)  Define the term tangential velocity.
\item (2016)  A boy ties a string around a stone of mass $ 0.15$ kg and then whirls it in a horizontal circle at constant speed. If the period of rotation of the stone is $ 0.4$ sec and the length between the stone and boy’s hand is $ 0.50$ m ;
 \begin{itemize}
\item Calculate the tension in the string. 
\item State one assumption taken to reach the answer above.
\end{itemize}
\item (2017)  A car is moving with a speed of $ 30$ m$/$s on a circular track of radius $ 500$ m. If its speed is increasing at the rate of $ 2$ m$/$s, find its resultant linear acceleration.
\item (2017)  An object of mass $ 1$ kg is attached to the lower end of a string $ 1$ m long whose upper end is fixed and made to rotate in a horizontal circle of radius $ 0.6$ m. If the circular speed of the mass is constant, find the:
 \begin{itemize}
\item Tension in the string. 
\item Period of motion. 
\end{itemize}
\item (2019)  In which aspect does circular motion differ from linear motion? 
\item (2019)  Why there must be a force acting on a particle moving with uniform speed in a circular path? 
\item (2019)  A stone tied to the end of string $ 80$ cm long, is whirled in a horizontal circle with a constant speed making $ 25$ revolutions in $ 14$ seconds. Determine the magnitude of its acceleration. 
\end{itemize}

\end{document}

\documentclass{article}
\usepackage[a4paper, total={6in, 8in}]{geometry}
% \usepackage[utf8]{inputenc}
\usepackage{abstract}
\title{\textbf{2.4 - Simple Harmonic Motion}}
\author{PJ Gibson - Peace Corps Tanzania}
\date{May 2020}

\begin{document}

\maketitle

\begin{itemize}
\item (1998)  Define simple harmonic motion.
\item (1998)  Prove that, the velocity v of a particle moving in simple harmonic motion is given by: $ v=w(A^{2}-y^{2})^{0.5}$ , where A is the amplitude of oscillation, $ w$ the angular frequency and $ y$ the displacement from the mean position.
\item (1998)  A simple pendulum has a period of $ 2.8$ seconds. When its length is shortened by $ 1.0$ metre, the period becomes $ 2.0$ seconds. From this information, determine the acceleration $ g$ , of gravity and the original length of the pendulum.
\item (1998)  A particle rests on a horizontal platform which is moving vertically in simple harmonic motion with an amplitude of $ 50$ mm. Above a certain frequency the particle ceases to remain in contact with the platform throughout the motion. With a help of a diagram and illustrative equations, find;
 \begin{itemize}
\item the lowest frequency at which this situation occurs.
\item the position at which contact ceases.
\end{itemize}
\item (1999)  Give two similarities between simple harmonic motion and circular motion.
\item (1999)  On the same set of axes, sketch how energy exchange (kinetic to potential) takes place in an oscillator placed in a damping medium.
\item (2000)  Define simple harmonic motion.
\item (2000)  Two simple pendulums of length $ 0.4$ m and $ 0.6$ m respectively are set oscillating in step. 
 \begin{itemize}
\item After what further time will the two pendulums be in step again? 
\item Find the number of oscillations made by each pendulum during the time found above.
\end{itemize}
\item (2000)  Cite two examples of SHM which are of importance to everyday life experience.
\item (2000)  Explain, giving reasons, whether either transverse or longitudinal waves could exist, if the vibratory motion causing them were not simple harmonic motion.
\item (2014)  State where the magnitude of acceleration is greatest in simple harmonic motion.
\item (2014)  Sketch a graph of acceleration against displacement for a simple harmonic motion.
\item (2014)  The displacement of a particle from the equilibrium position moving with simple harmonic motion is given by $ x=0.05 \sin(6t)$ , where $ t$ is the time in seconds measured at an instant when $ x=0$ .  Calculate the:
 \begin{itemize}
\item Amplitude of oscillations.
\item Period of oscillations. 
\item  Maximum acceleration of the particle. 
\end{itemize}
\item (2015)  Briefly explain why the motion of a simple pendulum is not strictly simple harmonic? 
 \begin{itemize}
\item Why is the velocity and acceleration of a body executing simple harmonic motion (S.H.M.) out of phase? 
\end{itemize}
\item (2015)  A body of mass $ 0.30$ kg executes simple harmonic motion with a period of $ 2.5$ s and amplitude of $ 4.0\times10^{-2}$ m. Determine the:
 \begin{itemize}
\item Maximum velocity of the body. 
\item Maximum acceleration of the body. 
\item Energy associated with the motion.
\end{itemize}
\item (2015)  A particle of mass $ 0.25$ kg vibrates with a period of $ 2.0$ s. If its greatest displacement is $ 0.4$ m what is its maximum kinetic energy?
\item (2016)  Show that the total energy of a body executing S.H.M. is independent of time.
\item (2016)  A mass of $ 05$ kg connected to a light spring of force constant $ 20$ N$/$m oscillates on a  horizontal frictionless surface. If the amplitude of the motion $ 1$ s $ 3.0$ cm , calculate the;
 \begin{itemize}
\item Maximum speed of the mass.
\item  Kinetic energy of the system when the displacement is $ 2.0$ cm.
\end{itemize}
\item (2017)  The equation of simple harmonic motion is given as $ x=6 \sin(10\pi t)+8 \sin(10\pi t)$ , where $ x$ is in centimeters and $ t$ in seconds. Determine the:
 \begin{itemize}
\item Amplitude 
\item Initial phase of motion. 
\end{itemize}
\item (2017)  Show that the total energy of a body executing simple harmonic motion is independent of time. 
\item (2017)  Find the periodic time of a cubical body of side $ 0.2$ m and mass $ 0.004$ kg floating in water then pressed and released such that it oscillates vertically. 
\item (2018)  What is meant by the following terms as used in simple harmonic motion (S.H.M)?
 \begin{itemize}
\item Periodic motion. 
\item Oscillatory motion. 
\end{itemize}
\item (2018)  List four important properties of a particle executing simple harmonic motion (S.H.M). 
\item (2018)  Sketch a labeled graph that represents the total energy of a particle executing simple harmonic motion (S.H.M). 
\item (2018)  The periodic time of a body executing S.H.M is $ 4$ seconds. How much time interval from time, $ t=0$ will its displacement be half its amplitude? 
\item (2018)  Giving reasons, explain whether either transverse or longitudinal waves could exist, if the vibratory motion causing them were not simple harmonic motion. 
\item (2019)  Provide two typical examples of simple harmonic motion (S.H.M). 
\item (2019)  Why the velocity and acceleration of a body executing simple harmonic motion are out of phase? 
\item (2019)  The period of a particle executing simple harmonic motion (S.H.M) is $ 3$ seconds. If its amplitude is $ 25$ cm, calculate the time taken by the particle to move a distance of $ 12.5$ cm on either side from the mean position.
\item (2019)  A person weighing $ 50$ kg stands on a platform which oscillates with a frequency of $ 2$ Hz and of amplitude $ 0.05$ m. Find his/her minimum weight as recorded by a machine of the platform. 
\end{itemize}

\end{document}
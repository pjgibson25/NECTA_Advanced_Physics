
\documentclass{article}
\usepackage[a4paper, total={6in, 8in}]{geometry}
% \usepackage[utf8]{inputenc}
\usepackage{abstract}
\title{\textbf{11.1 - Structure of the Atom}}
\author{PJ Gibson - Peace Corps Tanzania}
\date{May 2020}

\begin{document}

\maketitle

\begin{itemize}
\item (1999)  State Bohr’s postulates of the atomic model.
\item (1999)  Show that for an electron in a hydrogen atom, the possible radii of an electron orbit are given by:
 \begin{itemize}
\item $ r_{n}=a_{0}n^{2}$ , $ n=1$ , $ 2$ , $ 3$ , ...
\end{itemize}
\item (2000)  In the Bohr model of the hydrogen atom, an electron circles the nucleus in an orbit of radius $ r$
 \begin{itemize}
\item Explain what keeps the electron in the orbit and why it does not spiral towards the nucleus.
\item What are the assumptions put forward by Bohr about the orbits of the electron in the hydrogen atom?
\end{itemize}
\item (2007)  Develop an expression for electrical energy spent in the decomposition of water. 
\item (2007)  In a hydrogen atom model an electron of mass $ m$ and charge $ e$ rotates about a heavy nucleus of charge $ e$ in a circular orbit of radius $ r$ . Develop an expression for the angular momentum of the electron in terms of $ m$ , $ e$ , $ r$ , $ \pi$ and $ \epsilon  _{0}-$ the permitting of free space.
\item (2007)  The four lowest energy levels in a mercury atom are $ -10.4$ eV, $ -5.5$ eV, $ -3.7$ eV and $ -1.6$ eV.
 \begin{itemize}
\item Determine the ionization energy of mercury in joules. 
\item Calculate the wavelength of the radiation emitted when an electron jumps from $ -1.6$ eV to $ -5.5$ eV energy levels. 
\item What will happen if a mercury atom in its excited state is bombarded with electrons having an energy of $ 11$ eV. 
\end{itemize}
\item (2013)  Given that Rydberg’s constant is approximately $ 1.1 \times 10^{7}$ m$ ^{-1}$ Calculate the corresponding range of frequency for emitted radiation in the:
 \begin{itemize}
\item Lyman series. 
\item Balmer series. 
\end{itemize}
\item (2015)  Why are the energy levels labelled with negative energies?
\item (2016)  The first member of the Balmer series of hydrogen spectrum has wavelength of $ 6563 \times 10^{-10}$ m. Calculate the wavelength of its second member.
\item (2017)  Use the Rydberg constant, $ R_{H}=1.0974 \times 10^{7}$ m$ ^{-1}$ to calculate the shortest wavelength of the Balmer series. 
\item (2017)  Use the Bohr's theory for hydrogen atom to determine the:
 \begin{itemize}
\item Radius of the first orbit of the hydrogen atom in A units. 
\item Velocity of the electron in the first orbit. 
\end{itemize}
\item (2017)  What is ionization potential of an atom?
\item (2017)  Show that the ionization potential of hydrogen is $ 13.6$ eV. 
\item (2017)  How can you account for the chemical behavior of atoms on the basis of the atomic electrons and shells? 
\item (2017)  How can you account for the chemical behavior of atoms on the basis of the atomic electrons and shells? 
\item (2018)  Given: Mass of proton $ =1.0080$ u, Mass of neutron $ =1.0087$ u and Mass of alpha particle $ =4.0026$ u.
 \begin{itemize}
\item State any three limitations of Bohr’s model of the hydrogen atom.
\end{itemize}
\item (2018)  Why hydrogen spectrum contains a larger number of spectral lines although its  atom has only one electron? 
\item (2018)  State any three limitations of Bohr’s model of the hydrogen atom.
\item (2018)  Distinguish between ionization energy and excitation energy.
\item (2018)  Why hydrogen spectrum contains a larger number of spectral lines although its  atom has only one electron? 
\item (2019)  Based on Balmer series of hydrogen spectra determine the wavelength of the series limit of Paschen series. 
\item (2019)  Why hydrogen atom is stable in the ground state? 
\item (2019)  According to Bohr’s theory, the angular momentum of an electron is an integral multiple of $ h/2\pi$ .  Express this statement. by using a mathematical equation in which angular momentum is represented by the letter Land orbit by the letter $ n$ , 
\end{itemize}

\end{document}
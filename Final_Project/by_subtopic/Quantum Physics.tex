
\documentclass{article}
\usepackage[a4paper, total={6in, 8in}]{geometry}
% \usepackage[utf8]{inputenc}
\usepackage{abstract}
\title{\textbf{11.2 - Quantum Physics}}
\author{PJ Gibson - Peace Corps Tanzania}
\date{May 2020}

\begin{document}

\maketitle

\begin{itemize}
\item (1999)  What is the “work function” of a metal?
\item (1999)  The work function of a metal is $ 2.0$ eV. Calculate the stopping potential when the metal is illuminated by light of frequency of $ 6.0 \times 10^{14}$ Hz.
\item (2000)  What is the de Broglie wave equation?
\item (2000)  An electron is accelerated through a potential of $ 400$ V. Determine the de Broglie wavelength of this electron.
\item (2000)  Determine the de Broglie wavelength for the beam of electron whose total energy is
 \begin{itemize}
\item $ 250$ eV.
\end{itemize}
\item (2000)  What is a photoelectric cell?
\item (2000)  The emission of electrons from the surface of a cathode of a certain phototube when irradiated with a light of wavelength $ 3500 \times 10^{-10}$ m is found to stop when the plate potential is $ 1.2$ V with respect to the cathode. Determine the work function of the cathode.
\item (2007)  A certain diatomic gas is contained in a vessel whose inner surface is a small absorber which retains any atoms or molecules of gas which strike it.  Show that if doubling the absolute temperature causes one half of the molecules to dissociate into atoms then the rate at which the absorber is gaining mass increases by a factor $ 1+1/\sqrt{2}$ .
\item (2007)  What is a line spectrum? 
\item (2009)  Write down Bragg’s equation for the study of the atomic structure of the crystals by $ X-$ rays.
\item (2009)  The radiation from an $ X$ — ray tube which operates at $ 50$ kV is diffracted by is diffracted by a cubic KCl crystal of molecular mass $ 74.6$ and density $ 1.99 \times 10^{3}$ kg$/$m$ ^{3}$ .  Calculate:
 \begin{itemize}
\item The shortest wavelength limit of the spectrum from the tube.
\item The glancing angle for first order reflection from the planes of the crystal for that wavelength and angle of deviation of a diffracted beam.
\end{itemize}
\item (2009)  The radiation emitted by an $ X$ — ray tube consists of continuous spectrum with a line spectrum superimposed on it. Explain how the continuous spectrum and the line spectrum are produced.
 \begin{itemize}
\item Draw the graph of the spectra stated. ‘
\end{itemize}
\item (2013)  If the energy necessary to cause the ejection of an electron by photoelectric effect from the $ N$ — shell and $ K-$ shell of an atom is $ 10$ eV and $ 20$ eV respectively, calculate the maximum wavelength of radiation for each level.
\item (2015)  Show that the de Broglie hypothesis of matter wave are in agreement with Bohr’s theory.
\item (2015)  Ultraviolet light of wavelength $ 3600 \times 10^{-10}$ m is made to fall on a smooth surface of potassium. Determine:
 \begin{itemize}
\item The maximum energy of emitted photoelectrons
\item The stopping potential.
\item The velocity of the most energetic photoelectrons given that work function for potassium is $ 2$ eV.
\end{itemize}
\item (2016)  Briefly explain the production of X-rays.
\item (2016)  List down any three uses of X-rays.
\item (2016)  How are the intensity and penetrating power of an X-ray beam controlled?
\item (2018)  Briefly explain what led de-Broglie to think that the material particles may also show wave nature and why the wave nature of matter not noticeable in our daily observations? 
\item (2018)  Prove that de-Broglie wavelength $ \lambda $ , of electrons of kinetic energy $ E$ is given by $ \lambda = h/ \sqrt{2}$ meV  where $ m$ is the mass of the electron, $ e$ is the charge of the electron, $ h$ is the Planck’s constant and v is the accelerating potential difference. 
\item (2018)  Light of wavelength $ 488$ nm is produced by an argon laser which is used in the photoelectric effect. When light from this spectral line is incident on the emitter, the stopping (cut-off) potential of photoelectrons is $ 0.38$ V. Find the work function of the material from which the emitter is made. 
\item (2018)  In a hydrogen atom model, an electron of mass $ m$ and charge $ e$ revolves around the nucleus in a circular orbit of radius $ r$ . Develop an expression for the radius $ 3$ m of the orbit in terms of $ m$ , $ e$ , $ x$ , the quantum number $ n$ , Planck constant $ h$ and the permitting of free space $ \epsilon _{0}$ , and hence, use their values to find the Bohr’s radius. 
\item (2018)  In a hydrogen atom model, an electron of mass $ m$ and charge $ e$ revolves around the nucleus in a circular orbit of radius $ r$ . Develop an expression for the radius $ 3$ m of the orbit in terms of $ m$ , $ e$ , $ x$ , the quantum number $ n$ , Planck constant $ h$ and the permitting of free space $ \epsilon _{0}$ , and hence, use their values to find the Bohr’s radius. 
\item (2019)  Why electrons do not fall into the nucleus due to electrostatic force of attraction?
\item (2019)  Determine the angular momentum of the electron in the orbit of energy level $ -3.4$ eV given that $ E_{n}=-13.6/n^{2}$ eV, where $ E$ is the energy of an electron and $ n$ is the principal quantum number of hydrogen atom. 
\end{itemize}

\end{document}
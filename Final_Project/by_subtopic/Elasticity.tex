
\documentclass{article}
\usepackage[a4paper, total={6in, 8in}]{geometry}
% \usepackage[utf8]{inputenc}
\usepackage{abstract}
\title{\textbf{4.2 - Elasticity}}
\author{PJ Gibson - Peace Corps Tanzania}
\date{May 2020}

\begin{document}

\maketitle

\begin{itemize}
\item (1999)  Define "Young's Modulus" of a material and give its SI units.
\item (1999)  With the aid of a sketch graph, explain what happens when a steel wire is stretched gradually by an increasing load until it breaks. 
\item (1999)  A force $ F$ is applied to a long steel wire of length $ L$ and cross-sectional area A.
 \begin{itemize}
\item Show that if the wire is considered to be a spring, the force constant $ k$ is given by: $ k= AY/L$ , where $ Y$ is Young's Modulus of the wire.
\item Show that the energy stored in the wire is $ U=1/2F \Delta L$ where $ \Delta{L}$ is the extension of the wire
\end{itemize}
\item (2000)  Define the “bulk modulus” of a gas
\item (2000)  Find the ratio of the adiabatic bulk modulus of a gas to that of its isothermal bulk modulus in terms of the specific heat capacities of the gas.
\item (2000)  Explain Young’s Modulus of rigidity
\item (2000)  Find the work done in stretching a steel wire of $ 1.0$ mm$ ^{2}$ cross-sectional area and $ 2.0$ m in length through $ 0.1$ mm.
\item (2007)  With the aid of a diagram describe a simple laboratory experiment to measure Young’s modulus of a wooden bar acting as a loaded cantilever from its period of vibration given that the depression s is given by $ S=(WL^{3})/(3IE)$ . 
\item (2007)  Differentiate between tensile and shear stress. 
\item (2007)  A lift is designed to hold a maximum of $ 12$ people. The lift cage has a mass of $ 500$ kg and the distance from the top floor of the building to the ground floor is $ 50$ m.
 \begin{itemize}
\item What minimum cross-sectional area should the cable have in order to support the lift and the people in it?
\item Why should the cable have to be thicker than the minimum cross-sectional area above in practice? 
\item How much will the lift cable above stretch if $ 10$ people get into the lift at the ground floor, assuming that the lift cable has a cross section of $ 1.36$ cm? 
\item Note: Mass of an average person $ =70$ kg . $ E_{steel}=2 \times 10^{11}$ N$/$m$^{2}$ , Tensile strength of steel $ =4 \times 10^{11}$ N$/$m$^{2}$ .
\end{itemize}
\item (2009)  Define the following terms:
 \begin{itemize}
\item Tensile stress
\item Tensile strain
\item Young’s modulus
\end{itemize}
\item (2009)  Derive the expression for the work done in stretching a wire of length $ L$ by a load $ W$ through an extension $ X$ .
\item (2009)  A vertical wire made of steel of length $ 2.0$ m and $ 1.0$ mm diameter has a load of $ 5.0$ kg applied to its lower end.  What is the energy stored in the wire?
\item (2009)  A copper wire $ 2.0$ m long and $ 1.22 \times 10^{-3}$ m diameter is fixed horizontally to two rigid supports $ 2.0$ m apart.  Find the mass in kg of the load, which when suspended at the mid point of the wire, produces a sag of $ 2.0 \times 10^{-2}$ m at the point.
\item (2013)  The bulk modulus of elasticity for lead is $ 8 \times 10^{9}$ N$/$m$ ^{2}$ . Find the density of lead if the pressure applied is $ 2 \times 10^{8}$ N$/$m$ ^{2}$ . 
\item (2013)  Define the terms: proportional limit, elastic limit, yield point and elasticity.
\item (2013)  Use a sketch graph to show how the extension of the wire varies with the applied force and mark the elastic limit and yield point on it. Explain how the magnitude of the Young's modulus is obtained from the graph.
 \begin{itemize}
\item A block of metal weighing $ 20$ N with a volume of $ 8 \times 10^{-4}$ m? is completely immersed
\item in oil of density $ 700$ kg$/$m$ ^{3}$ then attached to one end of a vertical wire of length $ 4.0$ m and diameter of $ 0.6$ mm whose other end is fixed. If the length of the wire is increased by $ 1.0$ mm. find the:
\item young’s modulus of the wire. 
\item energy stored in the wire. 
\end{itemize}
\item (2015)  Define the following materials as classified on the basis of elastic properties:
 \begin{itemize}
\item  Ductile materials 
\item Brittle materials
\item Elastomers
\end{itemize}
\item (2015)  Briefly explain why the stretching of a coil spring is determined by its shear modulus.
\item (2015)  A copper wire of negligible mass, $ 1$ m long and cross-sectional area $ 10^{-5}$ m$ ^{2}$ is kept on a smooth horizontal table with one end fixed.  A ball of $ 1$ kg is attached to the other end.  The wire and the ball are rotating with an angular velocity of $ 35$ rad$/$s.  If the elongation of the wire is $ 10^{-3}$ m, find Young’s modulus of wire.  If on increasing the angular velocity to $ 100$ rad$/$s, the wire breaks down, find the breaking stress.
\item (2015)  Differentiate bulk modulus from shear modulus.
\item (2015)  Two wires, one of steel and one of phosphor bronze each $ 1.5$ m long and $ 2$ mm diameter are joined end to end as a composite wire of length $ 3$ cm.  What tension in the composite wire will produce total extension of $ 0.064$ cm?
\item (2016)  What is strain energy?
\item (2017)  A steel rod of length $ 0.60$ m and cross-sectional area $ 2.5 \times 10^{-5}$ m$ ^{2}$ at a temperature of $ 100^{\circ}$C is clamped so that when it cools was unable to contract. Find the tension in the rod when it has cooled to $ 20^{\circ}$C. 
\item (2017)  A spring $ 60$ cm long is stretched by $ 2$ cm for the application of load of $ 200$ g. What will be the length when a load of $ 500$ g is applied? 
\item (2017)  Calculate the percentage increase in length of a wire of diameter $ 2.2$ mm stretched by a load of $ 100$ kg. ( Young's modulus of wire is $ 12.5 \times 10^{10}$ N$/$m$ ^{2})$
\item (2017)  Define tensile stress and tensile strain. 
\item (2017)  Calculate the work done in a stretching copper wire of $ 100$ cm long and $ 0.03 $ cm$ ^{2}$ cross — sectional area when a load of $ 120$ N is applied. 
\item (2017)  Mention any two factors on which modulus of elasticity of a material depends.
\item (2018)  Briefly explain the following observations as applied to strengths of materials:
 \begin{itemize}
\item Bridges are declared unsafe after long use. 
\item Iron is more elastic than rubber. 
\end{itemize}
\item (2018)  A composite wire of diameter $ 1$ cm consists of copper and steel wires of lengths $ 2.2$ m and $ 2$ m respectively. Total extension of the wire when stretched by a force is $ 1.2$ mm. Calculate the force, given that Young’s modulus for copper is $ 1.1 \times 10^{11}$ Pa and for steel is $ 2 \times 10^{11}$ Pa. 
\item (2018)  What do you understand by the following terms?
 \begin{itemize}
\item A perfectly plastic material 
\item The ultimate tensile strength 
\item An elastic limit 
\item Poisson’s ratio. 
\end{itemize}
\item (2018)  Two rods of different materials but of equal cross-sections and lengths $ 1.0$ m each are joined to make a rod of length $ 2.0$ m. The metal of one rod has coefficient of linear thermal expansion of $ 10^{-5}^{\circ}$C$ ^{-1}$ and Young’s Modulus $ 3 \times 10^{10}$ N$/$m$ ^{2}$ . The other metal has the values $ 2 \times 10^{-5}^{\circ}$C$ ^{-1}$ and $ 10^{10}$ N$/$m$ ^{2}$ respectively. How much pressure must be applied to the ends of the composite rod to prevent its expansion when the temperature is raised by $ 100^{\circ}$C? 
\item (2019)  Define Young’s Modulus of a material. 
\item (2019)  Why work is said to be done in stretching a wire? 
\item (2019)  A steel wire AB of the length $ 60$ cm and cross-sectional area $ 1.5 \times 10^{-6}$ m$ ^{2}$ is attached at $ B$ to copper wire BC of length $ 39$ cm and cross sectional area $ 3.0 \times 10^{-6}$ m$ ^{2}$ . If the combination of the two wires is suspended vertically from a fixed point at A, and supports a weight of $ 250$ N at $ C$ ; find the extension (in millimeter) of the:
 \begin{itemize}
\item steel wire. 
\item copper wire. 
\end{itemize}
\end{itemize}

\end{document}
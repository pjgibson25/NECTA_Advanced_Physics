
\documentclass{article}
\usepackage[a4paper, total={6in, 8in}]{geometry}
% \usepackage[utf8]{inputenc}
\usepackage{abstract}
\title{\textbf{12.3 - Earthquakes}}
\author{PJ Gibson - Peace Corps Tanzania}
\date{May 2020}

\begin{document}

\maketitle

\begin{itemize}
\item (1998)  Explain the following terms: Earthquake, Earthquake focus, Epicentre and Body waves.
\item (1998)  List down three $ (3)$ sources of earthquakes.
\item (2000)  With reference to an earthquake on a certain point of the earth explain the terms ‘Focus’ and ‘Epicentre’.
\item (2000)  Describe two ways by which seismic waves may be produced.
 \begin{itemize}
\item Describe briefly the meaning and application of “seismic prospecting”. 
\end{itemize}
\item (2007)  What are the difference between $ P$ and s waves?
\item (2007)  Explain how the two terms of waves ($ P$ and $ S$ ) can be used in studying the internal structure of the earth. 
\item (2007)  What is geomagnetic micropulsation.
\item (2010)  Explain the following terms Earthquake, Earthquake focus and Epicenter.
\item (2010)  Describe clearly how $ P$ and s waves are used to ascertain that the outer core of the Earth is in liquid form. 
\item (2013)  The main interior of the earth (core) is believed to be in molten form. What seismic evidence supports this belief?
\item (2015)  What is the origin of earthquake?
\item (2015)  A large explosion at the earth's surface creates compressional (P) and shear (S) waves moving with a speed of $ 6.0$ km$/$s and $ 3.5$ km$/$s respectively. If both waves arrive at seismological station with $ 30$ s interval, calculate the distance measured between seismological station and the site of explosion. 
\item (2019)  What 's meant by epicentre and wind belt as used in Geophysics? 
\item (2019)  Identify three types of seismic waves.
 \begin{itemize}
\item Outline two characteristics of each type of wave described above.
\end{itemize}
\end{itemize}

\end{document}
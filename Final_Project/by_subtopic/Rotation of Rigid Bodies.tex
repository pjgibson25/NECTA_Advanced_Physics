
\documentclass{article}
\usepackage[a4paper, total={6in, 8in}]{geometry}
% \usepackage[utf8]{inputenc}
\usepackage{abstract}
\title{\textbf{2.6 - Rotation of Rigid Bodies}}
\author{PJ Gibson - Peace Corps Tanzania}
\date{May 2020}

\begin{document}

\maketitle

\begin{itemize}
\item (1999)  State the parallel axis theorem.
\item (1999)  Show that the Kinetic energy (K.E.) of rotation of a rigid body about an axis with a constant angular velocity $ w$ is given by $ KE =1/2Iw^{2}$ where i is the moment of inertia of the rigid body about the given axis.
\item (1999)  What do you understand by the term "moments of inertia" of a rigid body?
\item (1999)  State the perpendicular axes theorem of moments of inertia for a body in the form of a lamina
\item (1999)  Calculate the moments of inertia of a thin circular disc of radius $ 50$ cm and mass $ 2$ kg about an axis along a diameter of the disc.
\item (1999)  A wheel mounted on an axle that is not frictionless is initially at rest. A constant external torque of $ 50$ Nm is applied to the wheel for $ 20$ s. At the end of the $ 20$ s, the wheel has an angular velocity of
 \begin{itemize}
\item $ 600$ rev/min. The external torque is the removed, and the wheel comes to rest after $ 120$ s more.
\item Determine the moments of inertia of the wheel.
\item Calculate the frictional torque which is assumed to be constant. 
\end{itemize}
\item (2007)  The $ T$ is then suspended from the free end of rod $ Y$ and the pendulum swings in the plane of $ T$ about the axis Of rotation.
 \begin{itemize}
\item Calculate the moment of inertia i of the $ T$ about the axis of rotation. 
\item Obtain the expression for the k.e. and p.e. in terms of the angle $ \theta $ of inclination to the vertical oscillation of the pendulum. 
\item Show that the period of oscillation is $ 2\pi\sqrt{17L/18g}$ . 
\item ( Moment of inertia of a thin rod about its centre $ I_{C}=mL^{2}/12$ . )
\end{itemize}
\item (2009)  Define angular momentum and give its dimensions.
\item (2009)  A grinding wheel in a form of solid cylinder of $ 0.2$ m diameter and $ 3$ kg mass is rotated at $ 3600$ rev/minute.
 \begin{itemize}
\item What is its kinetic energy?
\item Find how far it would have to fall to acquire the same kinetic energy as in the question above.
\end{itemize}
\item (2014)  A disc of moment of inertia $ 2.5\times10^{-4}$ kg$/$m$ ^{2}$ is rotating freely about an axis through its centre at $ 20$ rev/min. If some wax of mass $ 0.04$ kg is dropped gently on to the disc $ 0.05$ m from its axis, what will be the new revolution per minute of the disc? 
\item (2014)  Explain briefly why a:
 \begin{itemize}
\item high diver can turn more somersaults before striking the water?
\item dancer on skates can spin faster by folding her arms?
\end{itemize}
\item (2014)  A heavy flywheel of moment of inertia $ 0.4$ kg$/$m$ ^{2}$ is mounted on a horizontal axle of radius $ 0.01$ m. If a force of $ 60$ N is applied tangentially to the axle:
 \begin{itemize}
\item  Calculate the angular velocity of the flywheel after $ 5$ seconds from rest.
\item List down two assumptions taken to arrive at your answer in above.
\end{itemize}
\item (2015)   Define moment of inertia of a body.
 \begin{itemize}
\item Briefly explain why there is no unique value for the moment of inertia of a given body?
\end{itemize}
\item (2015)  State the principle of conservation of angular momentum. 
 \begin{itemize}
\item A horizontal disc rotating freely about a vertical axis makes $ 45$ revolutions per minute. A small piece of putty of mass $ 2.0\times10^{-2}$ kg falls vertically onto the disc and sticks to it at a distance of $ 5.0\times10^{-2}$ m from the axis. If the number of revolutions per minute is thereby reduced to $ 36$ , calculate the moment of inertia of the disc. 
\end{itemize}
\item (2015)  What would be the length of a day if the rate of rotation of the Earth were such that the acceleration due to gravity $ g=0$ at the equator?
\item (2016)  Why is Newton’s first law of motion called the law of inertia?
\item (2016)  What is meant by moment of inertia of a body?
\item (2016)  List two factors on which the moment of inertia of a body depends. 
\item (2016)  A thin sheet of aluminum of mass $ 0.032$ kg has the length of $ 0.25$ m and width of $ 0.1$ m. Find its moment of inertia on the plane about an axis parallel to the:
 \begin{itemize}
\item Length and passing through its centre of mass, $ m$ .
\item Width and passing through the centre of mass, $ m$ , in its own plane.
\end{itemize}
\item (2016)  Define the term angular momentum.
\item (2016)  A thin circular ring of mass, $ M$ , and radius, $ r$ , is rotating about its axis with constant angular velocity, $ w_{1}$ .  If two objects each of mass, $ m$ , are attached gently at the ring, what will be the angular velocity of the rotating wheel?
\item (2016)  Why are space rockets usually launched from west to east?
\item (2017)  Justify the statement that ‘If no external torque acts on a body, its angular velocity will not conserved.
\item (2018)  Why is flywheel designed such that most of its mass is concentrated at the rim? Briefly explain. 
\item (2018)  Estimate the couple that will bring the wheel to rest in $ 10$ seconds when a grinding wheel of radius $ 40$ cm and mass $ 3$ kg is rotating at $ 3600$ revolutions per minute. 
\item (2018)  Why an ice skater rotates at relatively low speed when stretches her arms and a leg outward? 
\item (2018)  Calculate the moment of inertia of a sphere about an axis which is a tangent to its surface given that the mass and radius of the sphere are $ 10$ kg and $ 0.2$ m respectively. 
\end{itemize}

\end{document}
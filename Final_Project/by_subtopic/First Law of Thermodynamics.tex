
\documentclass{article}
\usepackage[a4paper, total={6in, 8in}]{geometry}
% \usepackage[utf8]{inputenc}
\usepackage{abstract}
\title{\textbf{5.5 - First Law of Thermodynamics}}
\author{PJ Gibson - Peace Corps Tanzania}
\date{May 2020}

\begin{document}

\maketitle

\begin{itemize}
\item (1999)  What do you understand by the term: Thermodynamic temperature scale
\item (2000)  The longitudinal wave speed in gases is given by $ v=\sqrt{\gamma p/ \rho }$ ; where $ \gamma =C_{p}/C_{v}$ , $ P$ is the pressure and $ \rho $ the density of gas. If $ v_{1,}$ and $ v_{2,}$ are the speeds of sound in air at temperature $ T_{1}$ and $ T_{2}$ respectively, show that $ v_{1/v_2}=\sqrt{T_{1}/T_{2}}$
 \begin{itemize}
\item NOTE: $ C_{p}$ and $ C_{v}$ are the specific heats of the gas at constant pressure and constant volume respectively.
\end{itemize}
\item (2000)  A number of $ 16$ moles of an ideal gas which is kept at constant temperature of $ 320$ K is compressed isothermally from its initial volume of $ 18$ litres to the final volume of $ 4$ litres.
 \begin{itemize}
\item Calculate the total work done in the whole process.
\item Comment on the sign of numerical answer you've obtained.
\end{itemize}
\item (2000)  A cylinder fitted with a frictionless piston contains $ 1.0$ g of oxygen at a pressure of $ 760$ mmHg and at a temperature of $ 27^{\circ}$C. the following operations are performed in stages: $ (1)$ The oxygen is heated at a constant pressure to $ 127^{\circ}$C and then $ (2)$ it is compressed isothermally to its original volume and finally $ (3)$ it is cooled at a constant volume to its original temperature.
 \begin{itemize}
\item Illustrate these changes in a sketch $ P-V$ diagram.
\item What is the input of heat to the cylinder in stage $ (1)$ above?
\item How much work does the oxygen do in pushing back the piston during stage $ (1)$ ?
\item How much work is done on the oxygen in stage $ (2)$ ?
\item How much heat must be extracted from the oxygen in stage $ (3)$ ? 
\item (For oxygen: density $ =1.43$ kg$/$m$ ^{3}$ (at stp), $ C_v =670$ J kg$ ^{-1}$ K$ ^{-1}$ and molecular mass $ =32$ )
\end{itemize}
\item (2000)  What is the difference between an “isothermal” process and an “adiabatic” process?
\item (2000)  How much work is required to compress $ 5$ mol of air at $ 20^{\circ}$C and $ l$ atmosphere to $ 1/10$ th of the original volume by
 \begin{itemize}
\item an isothermal process
\item an adiabatic process?
\item What are the final pressures for the cases and above?
\end{itemize}
\item (2000)  Explain the fact that the temperature of the ocean at great depths is very nearly constant the year round, at a temperature of about $ 4^{\circ}$C .
\item (2000)  In a diesel engine, the cylinder compresses air from approximately standard temperature and pressure to about one-sixteenth the original volume and a pressure of about $ 50$ atmospheres. What is the temperature of the compressed air?
\item (2007)  When a metal cylinder of mass $ 2.0x10^{-2}$ kg and specific heat capacity $ 500$ J$/$kgK is heated at constant power, the initial rate of rise of temperature is $ 3.0$ K$/$min.  After a time the heater is switched off and the initial rate of fall of temperature is $ 0.3$ K$/$min.  What is the rate at which the cylinder gains heat energy immediately before the heater is switched off?
\item (2007)  State the expression for the $ 1$ st law of thermodynamics.
\item (2007)  What do you understand by the terms:
 \begin{itemize}
\item critical temperature? 
\item adiabatic change?
\end{itemize}
\item (2007)  Find the number of molecules and their mean kinetic energy for a cylinder of volume $ 5 \times 10^{-4}m^{3}$ containing oxygen at a pressure of $ 2 \times 10^{5}$ Pa and a temperature of $ 300K$ . 
 \begin{itemize}
\item When the gas is compressed adiabatically to a volume of $ 2 \times 10^{-4}m^{3}$ , the temperature rises to $ 434K$ . Determine $ \gamma $ , the ratio of the principal heat capacities.
\item [ Molar gas constant $ R=8.31$ J$/$mol$/$K ,$ N  =6 \times 10^{32}$ mol$^{-1}$ ] 
\end{itemize}
\item (2009)  What is the difference between isothermal and adiabatic processes?
 \begin{itemize}
\item Write down the equation of state obeyed by each process in the question above.
\end{itemize}
\item (2009)  Using the same graph and under the same conditions sketch the isotherms and the adiabatics.
\item (2009)  Derive the expression for the work done by the gas when it expands from volume $ V_{1}$ to volume $ V_{2}$ during an:
 \begin{itemize}
\item Isothermal process
\item Adiabatic process
\end{itemize}
\item (2009)  When water is boiled under a pressure of $ 2$ atmospheres the boiling point is $ 120^{\circ}$C. At this pressure $ 1$ kg of water has a volume of $ 10^{-3}$ m$ ^{3}$ and $ 2$ kg of steam have a volume of $ 1.648$ m$ ^{3}$ . Compute the work done when $ 1$ kg of steam is formed at this temperature increase in the internal energy. 
\item (2010)  Briefly describe an experiment to measure temperature coefficient of a wire.
\item (2010)  A heating coil is made of a nichrome wire which will operate on a $ 12$ V supply and will have a power of $ 36$ W when immersed in water at $ 373$ K. The wire available has a cross-sectional area of $ 0.10$ mm$ ^{2}$ . What length of the wire will be required? 
\item (2013)  Briefly give comments on the following observations:
 \begin{itemize}
\item Polyatomic and diatomic gases have larger molar heat capacities than monatomic gases. 
\item  Cubical container is used for the derivation of pressure of an ideal gas.
\end{itemize}
\item (2013)  What is meant by a gas constant. 
\item (2013)  When a gas expand adiabatically it does work on its surroundings although there is no heat input to the gas. Explain where this energy is coming from.
\item (2013)  An ideal gas at $ 17^{\circ}$C and $ 750$ mmHg is compressed isothermally Until its volume is reached to ¾ of its initial value If it then allowed to expand adiabatically to a volume of $ 20\%$ greater than its original value. calculate the final temperature and pressure of the gas. 
\item (2013)  How does the first law of thermodynamics change under isothermal and adiabatic processes? 
\item (2013)  Show that the specific heat capacities of an ideal gas are related by the relation $ C_{p}=C_{v}+nR$ .
 \begin{itemize}
\item Explain the meaning of all the symbols used in the equation above.
\end{itemize}
\item (2013)  One mole of an ideal monatomic gas is heated at constant volume from the temperature of $ 300$ K to $ 600$ K. Calculate the:
 \begin{itemize}
\item amount of heat added 
\item work done by the gas 
\item change in its infernal energy
\end{itemize}
\item (2013)  The piston of a bicycle pump at room temperature of $ 290$ K is slowly moved in until the volume of air enclosed is one — fifth of the total volume of the pump. The outlet is then sealed and the piston suddenly drawn out to full extension. If no air passes the piston, find the temperature of the air in the pump immediately after withdrawing the piston, assuming that air ts an ideal gas with cryoscopic constant, $ \gamma =1.4$ .
\item (2014)  List down two simple applications of the First law of thermodynamics in our daily life.
\item (2014)  A heat engine works at two temperatures of $ 27^{\circ}$C and $ 227^{\circ}$C. Calculate the:
 \begin{itemize}
\item Efficiency of the engine. 
\item Temperature which will increase the efficiency by $ 10\%$ if the room temperature is kept at $ 27^{\circ}$C. 
\end{itemize}
\item (2017)  Give a common example of adiabatic process. 
\item (2017)  What happens to the internal energy of a gas during adiabatic expansion?
\item (2017)  A mass of an ideal gas of volume $ 400$ cm$ ^{3}$ at $ 288$ K expands adiabatically. If its temperature falls to $ 273$ K;
 \begin{itemize}
\item Find the new volume of the gas. 
\item Calculate the final volume of the gas if it is then compressed isothermally until the pressure returns to its original value.
\end{itemize}
\item (2017)  Briefly explain why:
 \begin{itemize}
\item Steam pipes are wrapped with insulating materials?
\item Stainless steel cooking pans fitted with extra copper at the bottom are more preferred?
\end{itemize}
\item (2017)  The capacitance $ C$ of a capacitor ts full charged by a $ 200$ V battery. It is then discharged through a small coil of resistance wire embedded in a thermally insulated block of specific heat capacity $ 2.5 \times 10^{2}$ J$/$kgK and of mass of $ 0.1$ kg.  If the temperature of the block rises by $ 0.4$ K. what is the value of $ C$ ?
\item (2018)  One gram of water becomes $ 1671 $ cm$ ^{3}$ of steam at a pressure of $ 1$ atmosphere. If the latent heat of vaporization at this pressure is $ 2256$ J$/$g, determine the:
 \begin{itemize}
\item external work done. 
\item increase in internal energy 
\end{itemize}
\item (2019)  Why water is preferred as a cooling agent in many automobiles?
\item (2019)  Analyze  three practical applications of thermal expansion of solids in daily life situations.
\item (2019)  Why stainless steel cooking pans are made with extra copper at the bottom?
\end{itemize}

\end{document}
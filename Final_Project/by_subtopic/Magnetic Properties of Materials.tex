
\documentclass{article}
\usepackage[a4paper, total={6in, 8in}]{geometry}
% \usepackage[utf8]{inputenc}
\usepackage{abstract}
\title{\textbf{8.2 - Magnetic Properties of Materials}}
\author{PJ Gibson - Peace Corps Tanzania}
\date{May 2020}

\begin{document}

\maketitle

\begin{itemize}
\item (1999)  With the help of clear diagrams, explain briefly how you would convert a sensitive galvanometer into:
 \begin{itemize}
\item an ammeter
\item a voltmeter
\end{itemize}
\item (2007)  List three $ (3)$ classes of magnetic materials on the basis of magnetic susceptibility and give one example for each class.
\item (2007)  How are the magnetic susceptibility and relative permeability of a magnetic material related to each other?
\item (2007)  State the main differences between.
 \begin{itemize}
\item diamagnetism and paramagnetism. 
\item ferromagnetism and auntiferromagnetism. 
\item ferromagnetism and ferrielectricity. 
\end{itemize}
\item (2007)  Draw hysteresis loops diagrams for soft iron and hard steel and use them to discuss:
 \begin{itemize}
\item permanent magnets.
\item electromagnets.
\item transformer cores. 
\end{itemize}
\item (2016)  Draw hysteresis loops diagram for soft iron and hard steel and use them to discuss permanent magnets.
\item (2016)  Define permeability constant.
\item (2018)  Mention the three magnetic materials and briefly explain each one. 
 \begin{itemize}
\item Give the differences between the magnetic materials mentioned above in terms of their magnetic susceptibility. 
\end{itemize}
\item (2018)  Define the following terms:
 \begin{itemize}
\item Ampere 
\item Hysteresis 
\end{itemize}
\item (2019)  Distinguish the terms magnetically soft and magnetically hard materials.
\end{itemize}

\end{document}
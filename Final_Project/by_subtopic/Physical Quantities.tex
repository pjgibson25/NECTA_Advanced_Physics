
\documentclass{article}
\usepackage[a4paper, total={6in, 8in}]{geometry}
% \usepackage[utf8]{inputenc}
\usepackage{abstract}
\title{\textbf{1.1 - Physical Quantities}}
\author{PJ Gibson - Peace Corps Tanzania}
\date{May 2020}

\begin{document}

\maketitle

\begin{itemize}
\item (1999)  Mention two applications and two limitations of dimensional analysis.
\item (1999)  The frequency $ f$ of a note produced by a taut wire stretched between two supports depends on the distance ​ $ l$ ​ between the supports, the mass per unit length of the wire,$ \mu $ , and the tension $ T$ . Use dimensional analysis to find how $ f$ is related to ​ $ l$ ​ , $\mu$, and $ T$ .
\item (1999)  The period $ T$ of vibrations of a tuning fork may be expected to depend on the density $ D$ , Young's Modulus $ Y$ of the material of which it is made and the length a of its prongs. Using dimensional analysis deduce an expression for $ T$ in terms of $ D$ , $ Y$ and a.
\item (2000)  The speed v of a wave is found to depend on the tension $ T$ in the string and the mass per unit length $ u$ (linear mass density). Using dimensional analysis derive the relationship between v, $ T$ and $ u$ .
\item (2007)  Mention two$ (2)$ uses of dimensional analysis.
\item (2007)  The frequency $ f$ of a note given by an organ pipe depends on the length, $ l$ , the air pressure $ P$ and the air density $ D$ .  Use the method of dimensions to find a formula for the frequency.
 \begin{itemize}
\item What will be the new frequency of a pipe whos original frequency was $ 256$ Hz if the air density falls by $ 2\%$ and the pressure increases by $ 1\%$ ?
\end{itemize}
\item (2010)  Mention two uses of dimensional analysis.
\item (2010)  The critical velocity of a liquid flowing in a certain pipe is $ 3$ m$/$s, assuming that the critical velocity v depends on the density $ \rho $ of the liquid. its viscosity mu, and the diameter $ d$ . of the pipe. 
 \begin{itemize}
\item Use the method of dimensional analysis to derive the equation of the critical velocity of the liquid in a pipe of half the diameter.
\item Calculate the value of critical velocity.
\end{itemize}
\item (2014)  What is the importance of dimensional analysis inspite of its drawbacks?
\item (2015)  State the law of dimensional analysis.
\item (2015)  The largest mass, $ m$ of a stone that can be moved by the flowing river depends on the velocity of flow v, the density $ \rho $ of water, and the acceleration due to gravity $ g$ . Show that the mass, $ m$ varies to the sixth power of the velocity of flow.
\item (2016)  Define the term dimension of a physical quantity.
\item (2016)  Give two basic rules of dimensional analysis. 
\item (2016)  The frequency, $ f$ of a vibrating string depends upon the force applied, $ F$ , the length, $ l$ , of the string and the mass per unit length,$ \mu $ . Using dimension show how $ f$ is related to $ F$ , $ l$ and $ \mu $ .
\item (2016)  What is meant by least count of a measurement?
\item (2017)  Define the term dimensions of a physical quantity. 
\item (2017)  Identify two uses of dimensional equations.
\item (2017)  What is the basic requirement for a physical relation to be correct? 
\item (2017)  List two quantities whose dimension is [ $ ML^{2}T^{-1}$ ]
\item (2017)  The frequency ‘$ f$ ’ of vibration of a stretched string depends on the tension ‘$ F$ ’, the length ‘$ l$ ’ and the mass per unit length $ $ $\mu$of the string. Derive the formula relating the physical quantities by the method of dimensions. 
\item (2017)  Use dimensional analysis to prove the correctness of the relation, $ \rho = \frac{3g}{4 \pi RG}$ where by $ \rho =$ density of the earth, $ g=$ acceleration due to gravity, $ R=$ radius of the earth and $ G=$ gravitational constant.
\item (2018)  State the law of dimensional analysis. 
\item (2018)  If the speed v of the transverse wave along a wire of tension, $ T$ and mass, $ m$ is given by, $ V=\sqrt{T/m}$ .  Apply dimensional analysis to check whether the given expression is correct or not.  
\item (2019)  Identify two basic rules of dimensional analysis.
\item (2019)  The frequency $ n$ of vibration of a stretched string is a function of its tension $ F$ , the length, $ l$ and mass per unit length $ m$ . Use the method of dimensions to derive the formula relating the stated physical quantities.
\end{itemize}

\end{document}
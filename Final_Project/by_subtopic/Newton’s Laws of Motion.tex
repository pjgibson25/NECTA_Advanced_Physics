
\documentclass{article}
\usepackage[a4paper, total={6in, 8in}]{geometry}
% \usepackage[utf8]{inputenc}
\usepackage{abstract}
\title{\textbf{2.1 - Newton’s Laws of Motion}}
\author{PJ Gibson - Peace Corps Tanzania}
\date{May 2020}

\begin{document}

\maketitle

\begin{itemize}
\item (1998)  State Newton's laws of motion.
\item (1998)  A ball of mass $ 0.4$ kg is dropped vertically from a height of $ 2.5$ m on to a horizontal table and bounces to a height of $ 1.5$ m.
 \begin{itemize}
\item Find the kinetic energy of the ball just before striking the table.
\item Find the kinetic energy just after impact.
\item Suggest reasons for the difference between these two values of kinetic energy.
\item What height would you expect the ball to reach after its next bounce from the table?
\end{itemize}
\item (1998)  A jet of water flowing with a velocity of $ 20$ ms$ ^{-1}$ from a pipe of cross-sectional area, $ 5.0 \times 10^{-3}$ m$ ^{2}$ , strikes a wall at right angles and loses all its velocity.
 \begin{itemize}
\item What is the mass of water striking the wall per second?
\item What is the change in momentum per second of the water hitting the wall?
\item What is the force exerted on the wall?
\end{itemize}
\item (1999)  Define momentum
\item (1999)  Define impulse of a force
\item (1999)  A jet of water emerges from a hose pipe of a cross-sectional area $ 5.0\times 10^{-3}$ m​$ ^{2}$ with a velocity of $ 3.0$ m$/$s and strikes a wall at right angle. Assuming the water to be brought to rest by the wall and does not rebound, calculate the force on the wall.
\item (1999)  Distinguish between static and dynamic friction.
\item (2007)  A ball is thrown towards a vertical wall from a point $ 2$ m above the ground and $ 3$ m from the wall.  The initial velocity of the ball is $ 20$ m$/$s at an angle of $ 30$ deg above the horizontal.  If the collision of the ball with the wall is perfectly elastic, how far behind the thrower does the ball hit the ground?
\item (2007)  Explain why when catching a fast moving ball, the hands are drawn back will the ball is being brought to rest.
\item (2007)  Rockets are propelled by the ejection of the products of the combustion of fuel.  Consider a rocket of total mass $ M$ travelling at a speed v in a region of space where the gravitational forces are negligible.  
\item (2007)  Supposing the combustion products are ejected at a constant speed v, relative to the rocket, show that a fuel "burn" which reduces the total mass $ M$ of the rocket to $ m$ results in an increase in the speed of the rocket to v such that $ v-V=V_{f} \ln (M/m)$ .
\item (2007)  Supposing that $ 2.1\times10^{6}$ kg of fuel are consumed during a "burn" lasting $ 1.5\times10^{2}$ seconds and given that there is a constant force on the rocket of $ 3.4\times 10^{7}$ N during this burn, calculate v, and increase in speed resulting from the burn if $ M=2.8\times10^{6}$ kg.  
\item (2007)  What is the initial vertical acceleration that can be imparted to this rocket when it is launched from the Earth if the initital mass is $ 2.8\times 10^{6}$ kg?
\item (2007)  State and define Newton’s 2nd law of motion with respect to angular motion. 
\item (2013)  A man stands in a lift which is being accelerated upwards at $ 3.2$ m$/$s$ ^{2}$ . If the man has a mass of $ 65$ kg, what is the net force exerted on the man by the floor of the lift?
\item (2013)  A rubber cord of a $ Y-$ shaped object has a cross sectional area of $ 4 \times 10^{-6}$ m$ ^{2}$ ? And relaxation length of $ 100$ mm. If the arms of the catapult are $ 70$ mm apart, calculate the: 
 \begin{itemize}
\item tension in the rubber. 
\item force required to stretch it when the rubber cord is pulled back until its length doubles. 
\end{itemize}
\item (2014)  State the principle of conservation of linear momentum. 
 \begin{itemize}
\item Give two examples of the principle of conservation of linear momentum. 
\end{itemize}
\item (2014)  An insect is released from rest at the top of the smooth bowling ball such that it slides over the ball. Prove that it will loose its footing with the ball at an angle of about $ 48^{\circ}$ with the vertical.
\item (2014)  A vertical spring fixed at one end has a mass of $ 0.2$ kg and is attached at the other end.
 \begin{itemize}
\item Determine the:
\item Extension of the spring.
\item Energy stored in the spring.
\end{itemize}
\item (2014)  Define torque and give its S.I. unit.
\item (2014)  Give two ways in which the internal energy of the system can be changed.
\item (2016)  State the principles on which the rocket propulsion is based. 
\item (2016)  A jet engine on a test bed takes in $ 40$ kg of air per second at a velocity of $ 100$ m$/$s  and burns $ 0.80$ kg of fuel per second. After compression and heating the exhaust gases are ejected at $ 600$ m$/$s relative to the air craft. Calculate the thrust of the engine.
\item (2016)  An object of mass $ 2$ kg is attached to the hook of a spring balance which is suspended vertically to the roof of a lift.  What is the reading on the spring balance when the lift is:
 \begin{itemize}
\item going up with the rate of $ 0.2$ m$/$s$ ^{2}$
\item going down with an acceleration of $ 0.1$ m$/$s$ ^{2}$
\item ascending with uniform velocity of $ 0.15$ m$/$s
\end{itemize}
\item (2016)  Define the term inertia.
\item (2017)  A $ 75$ kg hunter fires a bullet of mass $ 10$ g with a velocity of $ 400$ m$/$s from a gun of mass $ 5$ kg. Calculate the:
 \begin{itemize}
\item Recoil velocity of the gun. 
\item Velocity acquired by the hunter during firing.
\end{itemize}
\item (2017)  A traffic light is suspended with two steel wires of equal lengths and radii of $ 0.5$ cm. If the wires make an angle of $ 15^{\circ}$ with the horizontal, what is the fractional increase in their length due to the weight of the light? 
\item (2018)  Under what condition a passenger in a lift feels weightless? 
\item (2018)  Calculate the tension in the supporting cable of an elevator of mass $ 500$ kg which was originally moving downwards at $ 4$ m$/$s and brought to rest with constant acceleration at a distance of $ 20$ m. 
\item (2018)  The rotating blades of a hovering helicopter swept out an area of radius $ 2$ m imparting a downward velocity of $ 8$ m$/$s of the air displaced. Find the mass of a helicopter. 
\item (2019)  A rocket of mass $ 20$ kg has $ 180$ kg of fuel. If the exhaust velocity of the fuel is $ 1.6$ km/sec, calculate;
 \begin{itemize}
\item The minimum rate of fuel consumption that enable the rocket to rise from the ground. 
\item The ultimate vertical speed gained by the rocket when the rate of fuel consumption ts $ 2$ kg/sec. 
\end{itemize}
\item (2019)  Determine the least number of pieces required to stop the bullet if a rifle bullet loses $ 1/20$ of its velocity when passing through them.
\item (2019)  A man of $ 100$ kg jumps into a swimming pool from a height of $ 5$ m. If it takes $ 0.4$ seconds for the water in a pool to reduce its velocity to zero, what average force did  the water exert on the man? 
\end{itemize}

\end{document}
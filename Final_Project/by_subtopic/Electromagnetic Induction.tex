
\documentclass{article}
\usepackage[a4paper, total={6in, 8in}]{geometry}
% \usepackage[utf8]{inputenc}
\usepackage{abstract}
\title{\textbf{8.4 - Electromagnetic Induction}}
\author{PJ Gibson - Peace Corps Tanzania}
\date{May 2020}

\begin{document}

\maketitle

\begin{itemize}
\item (1998)  Define the term self inductance for a coil.
\item (1998)  Give the S.I units of self inductance.
\item (1998)  Derive an expression for the coefficient of self induction of a uniformly wound solenoid; of length $ 1$ , cross-sectional area A having $ N$ turns in air.
\item (1998)  Two coils $ A$ and $ B$ have $ 200$ and $ 800$ turns respectively. A current of $ 2$ amperes in A produces a magnetic flux of $ 1.8 \times 10^{-4}$ Wb in each turn of $ B$ . Compute:
 \begin{itemize}
\item the mutual inductance.
\item the magnetic flux through A when there is a current of $ 4.0$ amperes in $ B$ and
\item the emf induced in $ B$ when the current in A changes from $ 3$ amperes to $ 1$ ampere in $ 0.2$ seconds.
\end{itemize}
\item (1999)  State the laws of electromagnetic induction and describe briefly experiments (one in each case) which can be used to demonstrate them.
\item (2007)  State Faraday’s two $ (2)$ laws of electrolysis and calculate the value of Faradays constant given that the e.c.e. of copper is $ 3.30 \times 10^{-7}$ kg$/C$ and the copper is a divalent element. 
\item (2007)  A piece of metal weighing $ 200g$ is to be electroplated with $ 5\%$ of its weight in gold. If the strength of the available current is $ 2$ A, how long would it take to deposit the required amount of gold?
\item (2007)  State Faraday’s law of electromagnetic induction. 
\item (2007)  A coil of cross section area A rotates with an angular velocity $ \omega $ in a uniform. magnetic field, $ B$ . Derive the equation for induced e.m.f. of the system.
\item (2009)  State the laws of electromagnetic induction.
\item (2009)  A coil of $ 100$ turns is rotated at $ 1500$ revolutions per minute in a magnetic field of uniform density $ 0.05$ T.  If the axis of rotation is at right angles to the direction of the flux and the area per turn is $ 4000 $ mm$ ^{2}$ .  Calculate the:
 \begin{itemize}
\item Frequency
\item Period
\item Maximum induced e.m.f.
\item Maximum value of the induced e.m.f. when the coil has rotated through $ 30^{\circ}$ from the position of zero e.m.f.
\end{itemize}
\item (2013)  State the laws of electromagnetic induction.
\item (2013)  State Lenz’s Jaw of electromagnetic induction.
\item (2015)  Distinguish between self-inductance and mutual inductance.
\item (2018)  Consider a small flat coil which has $ N$ turns of area A and whose plane is perpendicular to a magnetic field of flux density $ B$ . If the search coil is connected to the ballistic galvanometer and the total resistance of the circuit is $ R$ , use the laws of electromagnetic induction to show that the charge delivered to the galvanometer does not depend on how long it takes to remove the search coil from the field. 
\item (2019)  At which position of the rotating coil in the magnetic field, the induced e.m.f. is zero? Give a reason. 
\end{itemize}

\end{document}
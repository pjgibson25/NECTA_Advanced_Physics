
\documentclass{article}
\usepackage[a4paper, total={6in, 8in}]{geometry}
% \usepackage[utf8]{inputenc}
\usepackage{abstract}
\title{6.5 - Vibrations and Waves}
\author{PJ Gibson - Peace Corps Tanzania}
\date{May 2020}

\begin{document}

\maketitle


\section{Vibrations and Waves}

\subsection{Physical Optics (interferance/diffraction/polarization)}
\begin{itemize}
\item (2019)  What is meant by diffraction grating?
\item (2019)  A diffraction grating has $ 500$ lines per millimetre when used with monochromatic light of wavelength $ 6 \times 10^{-7}$ m at normal incidence. Determine the angle at which the bright diffraction images will be observed. 
 \begin{itemize}
\item Why other orders of image above can not be observed? 
\end{itemize}
\item (2019)  State Huygens’s principle of wave construction.
\item (2019)  A lens was placed with a convex surface of radius of curvature $ 50.0$ cm in contact with the plane surface such that Newton’s rings were observed when the lens was illuminated with monochromatic light. If the radius of the $ 15$ th ring was $ 2.13$ mm determine the wavelength. 
\end{itemize}

\end{document}
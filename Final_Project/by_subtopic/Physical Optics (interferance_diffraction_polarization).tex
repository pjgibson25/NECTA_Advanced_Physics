
\documentclass{article}
\usepackage[a4paper, total={6in, 8in}]{geometry}
% \usepackage[utf8]{inputenc}
\usepackage{abstract}
\title{\textbf{6.5 - Physical Optics (interferance/diffraction/polarization)}}
\author{PJ Gibson - Peace Corps Tanzania}
\date{May 2020}

\begin{document}

\maketitle

\begin{itemize}
\item (1998)  What is a diffraction grating?
\item (1998)  A diffraction grating has $ 5000$ lines per centimetre. At what angles will bright diffraction images be observed, if it is used with monochromatic light of wavelength $ 6.0 \times 10^{-7}$ m at normal incidence?
\item (1998)  A lamp emits two wavelengths, $ 4.2 \times 10^{-7}$ m and $ 6.0 \times 10^{-7}$ m. Find the angular separation of these two waves in the third order diffraction pattern produced by a diffraction grating having $ 4000$ lines per centimetre, when light is at normal incidence on the grating?
\item (1999)  What is the difference between refraction and diffraction as applied to waves?
\item (1999)  A parallel beam containing two wavelengths $ 600$ nm and $ 602$ nm is incident on a diffraction grating with $ 400$ lines per mm. Calculate the angular separation of the first order spectrum of the two wavelengths. ($ 1$ nm $ =10^{-9}$ m)
\item (2000)  Explain briefiy the necessary conditions for the effects of interference in optics to be observed
\item (2000)  Interference patterns are formed when using Young’s double slit arrangement. Mention other three methods that can be used to form interference patterns.
\item (2000)  A beam of monochromatic light of wavelength $ 600$ nm in air passes into glass. Calculate:
 \begin{itemize}
\item the speed of light in glass.
\item the frequency of light.
\item the wavelength of light in glass.
\end{itemize}
\item (2000)  What is meant by “diffraction grating”?
\item (2000)  A monochromatic light of wavelength $ 5.2 \times 10^{-7}$ m falls normally on a grating which has $ 4 \times 10^{3}$ lines per cm.
 \begin{itemize}
\item What is the largest order of spectrum that can be visible?
\item Find the angular separation between the third and fourth order image.
\end{itemize}
\item (2007)  Using the notation of energy bands, explain the following optical properties of solids.
 \begin{itemize}
\item  All metals are opaque to light of all wavelengths.
\item  Semi-conductors are transparent to infrared light although opaque to visible light.
\item  Most insulators are transparent to visible light.
\end{itemize}
\item (2007)  Describe briefly the formation of Newton rings. How would you measure the wavelength of yellow light by use of Newton’s rings? 
\item (2007)  What would happen to the central spot when air rests between the lens and the plate of the apparatus for Newton’s rings? 
\item (2007)  State Rayleigh’s criterion for the resolution of two objects. 
\item (2007)  The diameter of the pupil of the human eye is $ 2$ mm in bright light.
 \begin{itemize}
\item What is its resolving power with light of wavelength lamda $ =5 \times 10^{-7}m$ ? 
\item Would it be possible to resolve two large birds $ 30$ cm apart sitting on a wire$ 1.5 \times 10^{3}m$ away at daytime? 
\item What would the situation be at night when the pupil dilates to $ 4$ mm? 
\end{itemize}
\item (2007)  What is meant by the back e.m.f. (polarization potential) in a water voltameter? 
\item (2009)  What is interference?  Explain the term path difference with reference to the interference of two wave-trains.
\item (2009)  Why is it not possible to see interference when the light beams from head lamps of a car overlap?
\item (2009)  Discuss whether it is possible to observe an interference pattern when white light is shone on a Young’s double slit experiment.
\item (2009)  A grating has $ 500$ lines per millimetre and is illuminated normally with monochromatic light of wavelength $ 5.89 \times 10^{-7}$ m.
 \begin{itemize}
\item How many diffraction maxima may be observed?
\item Calculate the angular separation.
\end{itemize}
\item (2013)  What is an electron microscope? 
\item (2013)  Outline three disadvantages of electron microscope.
\item (2013)  Draw a schematic diagram of an electron microscope showing its main parts.
 \begin{itemize}
\item Give the order of resolution of electron microscope in the question above.
\end{itemize}
\item (2013)  What is meant by crossed polaroids? 
\item (2013)  Briefly describe the appearance of fringes produced by monochromatic fight.
\item (2013)  Give two difference between diffracting grating spectra and prism spectra.
\item (2013)  A diffraction grating used at normal incidence gives a yellow line. $ \lambda =5750$ A in a certain spectral order: superimposed on a blue line, $ \lambda =4600$ A of the next higher order, If the angle of diffraction is $ 30^{\circ}$ , what is the spacing between the grating lines? 
\item (2013)  State Huygens principle of wave construction. 
\item (2013)  A thin wedge of air of small angle ts enclosed by two thin glass plates. When the plates are illuminated by a parallel beam of monochromatic light of wavelength $ 589$ nm, the distance apart of the fringes is $ 0.8$ mm. Calculate the angle of the wedge. 
\item (2015)  What is meant by the statement that light is plane polarized.
\item (2015)  State Brewster’s law.
\item (2015)  Sunlight is reflected from a calm lake.  The reflected sunlight is totally polarized.  What is the angle between the sun and the horizon.
\item (2015)  State four conditions for sustained interference of light.
\item (2015)  In a Young’s double slit experiment the interval between the slits is $ 0.2$ mm.  For the light of wavelength $ 6.0\times 10^{-7}$ m, Find the distance of the second dark fringe from the central fringe.
\item (2015)  Distinguish between diffraction and diffraction grating.
\item (2015)  A parallel beam of the monochromatic light is incident normally on a diffraction grating.  The angle between the two first-order spectra on either side of the normal is $ 30^{\circ}$ .  Assume that the wavelength of the light is $ 5893\times 10^{14}$ m. Find the number of ruling per mm on the grating and the greatest number of bright images obtained. 
\item (2016)  The incident parallel light is a monochromatic beam of wavelength $ 450$ nm.  The two slits $ A$ and $ B$ have their centres, a distance of $ 0.3$ mm apart.  The screen is situated a distance of $ 2.0$ m from the slits.
 \begin{itemize}
\item Calculate the spacing between fringes observed on the screen.
\item How would you expect the pattern to change when the slits $ A$ and $ B$ are each made wider?
\end{itemize}
\item (2016)  Describe the formation of interference patterns by using Newton’s rings experiment.
 \begin{itemize}
\item Calculate the radius of curvature of a Plano-convex lens used to produce Newton’s rings with a flat glass plate if the diameter of the tenth dark ring is $ 4.48$ mm, viewed by normally reflected light of wavelength $ 5.0 \times 10^{-7}$ m.  What is the diameter of the twentieth bright ring?
\end{itemize}
\item (2017)  Explain the advantage of using optical fibre systems instead of coaxial cable systems in telecommunication processes.
\item (2017)  In a Young's double - slit experiment a total of $ 23$ bright fringes occupying $ 4$ total distance of $ 3.9$ mm were visible in traveling microscope, which was focused on a plane being at a distance of $ 31$ cm from the double slit. If the wavelength of light being used was $ 5.5 \times 10^{-7}$ m; determine the separation of the double slit.
\item (2017)  When a grating with $ 300$ lines per millimeters is illuminated normally with parallel beam of monochromatic light a second order principal maximum is observed at $ 18.9^{\circ}$ to the straight through direction. Find the wavelength of the light.
\item (2017)  A white light fall on a slit of width ‘a’: for what value of 'a' will be the first minimum of light falling at the angle of $ 30^{\circ}$ when the wavelength of light is $ 6500$ nm? 
\item (2018)  What do you understand by the term interference of waves?
\item (2018)  A viewing screen is separated from a double-slit source by $ 1.2$ m. The distance between the two slits is $ 0.030$ mm. The second order bright fringe $ (m=2)$ is $ 4.5$ cm from the centre line. Determine the wavelength of the light and the distance between adjacent bright fringes. 
\item (2018)  Define the term coherent sources of light. 
\item (2018)  Interference patterns are formed when using Young’s double slit experiment. Mention other three methods that can be used to form interference patterns. 
\item (2018)  A beam of monochromatic light of wavelength $ 680$ nm in air passes into glass.  Calculate: 
 \begin{itemize}
\item The speed of light in glass
\item The frequency of light
\item The wavelength of light in glass
\end{itemize}
\item (2018)  Light of wavelength $ 644$ nm is incident on a grating with a spacing of $ 2.00 \times 10^{-6}$ m. 
 \begin{itemize}
\item What is the angle to the normal of a second order maximum? 
\item What is the largest number of orders that can be visible? 
\item Find the angular separation between the third and fourth order image.
\end{itemize}
\item (2018)  State any four laws of photoelectric emission. 
\item (2019)  Two sheets of a Polaroid are lined up so that their polarization directions are initially parallel. When one sheet is rotated:
 \begin{itemize}
\item How does the transmitted light intensity vary with the angle between the polarization directions of the polaroid? 
\item What angle must the polaroid be rotated to reduce the light Intensity by $ 50\%$ ?
\end{itemize}
\item (2019)  What is meant by diffraction grating?
\item (2019)  A diffraction grating has $ 500$ lines per millimetre when used with monochromatic light of wavelength $ 6 \times 10^{-7}$ m at normal incidence. Determine the angle at which the bright diffraction images will be observed. 
 \begin{itemize}
\item Why other orders of image above can not be observed? 
\end{itemize}
\item (2019)  State Huygens’s principle of wave construction.
\item (2019)  A lens was placed with a convex surface of radius of curvature $ 50.0$ cm in contact with the plane surface such that Newton’s rings were observed when the lens was illuminated with monochromatic light. If the radius of the $ 15$ th ring was $ 2.13$ mm determine the wavelength. 
\end{itemize}

\end{document}

\documentclass{article}
\usepackage[a4paper, total={6in, 8in}]{geometry}
% \usepackage[utf8]{inputenc}
\usepackage{abstract}
\title{\textbf{9.3 - Alternating Current (ac)}}
\author{PJ Gibson - Peace Corps Tanzania}
\date{May 2020}

\begin{document}

\maketitle

\begin{itemize}
\item (1999)  What is a resonant frequency of an oscillator?
\item (1999)  An inductance of $ 4$ mH is connected in series with a resistance of $ 20\Omega $ together with a battery:
 \begin{itemize}
\item Determine how the current will vary with time in this circuit.
\item Sketch the current of above against time
\item Calculate the inductive time constant
\end{itemize}
\item (2000)  What is meant by the terms electrical resistivity and ohmic conductor.
\item (2000)  A $ 4$ m long resistance wire has a cross-sectional area of $ 0.8$ mm? and has a resistance of $ 2.80\Omega $ .  Determine:
 \begin{itemize}
\item The resistivity of the wire.
\item The length of a similar wire which when joined in parallel will give a total resistance of $ 2.0\Omega $ .
\end{itemize}
\item (2000)  Two cells of emf $ 1.5$ V and $ 2.0$ V and internal resistances of $ 1\Omega $ and $ 2.0\Omega $ respectively are connected in parallel and across them an external resistance of $ 5.0$ Q. Calculate the currents in each of the three branches of the network. 
\item (2000)  What is a rectifier?
\item (2007)  An a.c. generator consists of a coil of $ 50$ turns and an area of $ 2.5$ m$ ^{2}$ , rotates at an angular speed of $ 60$ rad$/$s in a uniform magnetic field of $ 0.30$ T between two fixed pole pieces.  The resistance of the circuit including that of the coil is $ 500\Omega $ .  
 \begin{itemize}
\item  What is the maximum current that can be drawn from the generator?
\item  What is the magnetic flux through the coil if the current is maximum?
\end{itemize}
\item (2013)  A $ 20$ k$ \Omega$ resistor is to be connected across a potential difference of $ 300$ V Calculate the required power rating.
\item (2013)  Derive an expression for impedance of a series $ R-C$ circuit. 
\item (2013)  Write down two advantages of digital circuits over the analogue circuits.
\item (2014)  What is meant by the following terms:
 \begin{itemize}
\item Alternating current (a.c.)
\item Effective value of A.C. 
\end{itemize}
\item (2014)  A $ 60$ V, $ 10$ W lamp is to be run on $ 100$ V, $ 60$ Hz A.C mains.
 \begin{itemize}
\item Calculate the inductance of a choke coil required.
\item If a resistor is used in above instead of choke, what will be value of its resistance.
\end{itemize}
\item (2014)  An LCR circuit with $ R=70\Omega$ in series with a parallel combination of $ L=1.5$ H and
 \begin{itemize}
\item $ C=30$ $\mu$F is driven by a $ 230$ V supply with angular frequency of $ 300$ rad$/$s.
\item $ (1)$ Find the power in put to the circuit. 
\item  At the frequency $ \omega_{o}=1/(\sqrt{LC})$ , how does the circuit respond?
\end{itemize}
\item (2015)  Explain the statement that, a sinusoidal current, of peak value $ 5$ A passed through an a.c. ammeter reads $ 5/\sqrt{2}$ A.  
\item (2015)  Show that the average power transferred to an a.c. circuit is, in general, given by $ EIR/Z$ , where $ R$ is the resistance in the circuit defined to be the real part of complex impedance and $ Z$ is its impedance.
\item (2015)  A coil which has an inductance of $ 0.2$ H and negligible resistance is in series in a resistor, whose resistance is $ 60\Omega $ . The pair is connected across a $ 50$ V supply alternating at $ 100/\pi$ Hz.  Calculate the toal impedance of the circuit and its power factor.
\item (2016)  An a.c. circuit consists of a pure resistance of $ 10\Omega $ is connected across an a.c. supply of $ 230$ V , $ 50$ Hz.  Calculate the;
 \begin{itemize}
\item Current flowing in the circuit.
\item Power dissipated
\end{itemize}
\item (2016)  An X-ray tube, operated at a d.c. potential difference of $ 60$ kV , produces heat at the target at the rate of $ 840$ W .  Assuming $ 0.65\%$ of the energy of the incident electrons is converted into X-radiation, calculate:
 \begin{itemize}
\item The number of electrons per second striking the target.
\item The velocity of the incident electrons.
\item The energy of incident electrons
\end{itemize}
\item (2018)  Calculate the current flowing in the circuit when three similar cells each of emf $ 1.5$ V and internal resistance $ 0.3\Omega $ are connected in parallel across a $ 2\Omega $ resistor. 
\item (2018)  Why choke coil is preferred over resistance to control alternating current?
\item (2018)  Explain what could be done to light a $ 30$ V bulb from a $ 220$ volt A.C. supply?
\item (2019)  A current of $ 3.0$ mA flows in a Television resistor $ R$ when a potential difference of $ 6.0$ V is connected across its terminals. Determine the value of conductance.
\end{itemize}

\end{document}
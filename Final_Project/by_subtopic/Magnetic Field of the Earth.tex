
\documentclass{article}
\usepackage[a4paper, total={6in, 8in}]{geometry}
% \usepackage[utf8]{inputenc}
\usepackage{abstract}
\title{\textbf{8.5 - Magnetic Field of the Earth}}
\author{PJ Gibson - Peace Corps Tanzania}
\date{May 2020}

\begin{document}

\maketitle

\begin{itemize}
\item (1999)  A flat coil of $ 100$ turns and mean radius $ 5.0$ cm is tying on a horizontal surface and is turned over in $ 0.20$ sec. against the vertical component of the Earth's magnetic field. Calculate the average e.m.f. induced.
\item (2007)  Write short notes on the following terms in relation to changes in the Earth's magnetic field:  long-term (secular) changes, short-period (regular) changes and short-term (irregular) changes.
\item (2013)  An aircraft is flying horizontally at $ 200$ m$/$s through the region where the vertical component of the earth magnetic field is $ 4.0 \times 10^{-5}$ T. If the air craft has a wing span of $ 40$ m, what will be the potential difference (p.d.) produced between the wing tips? 
\item (2015)  List down three sources of earth's magnetism. 
\item (2016)  State any three magnetic components of the earth’s magnetic field.
\item (2016)  The horizontal and vertical components of the Earth’s magnetic field at a certain location are $ 2.7 \times 10^{-5}$ T and $ 2.0 \times 10^{-5}$ T respectively.  Determine the Earth’s magnetic field at the location and its angle of inclination i.
\end{itemize}

\end{document}
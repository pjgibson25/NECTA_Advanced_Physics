
\documentclass{article}
\usepackage[a4paper, total={6in, 8in}]{geometry}
% \usepackage[utf8]{inputenc}
\usepackage{abstract}
\title{\textbf{2.5 - Gravitation}}
\author{PJ Gibson - Peace Corps Tanzania}
\date{May 2020}

\begin{document}

\maketitle

\begin{itemize}
\item (1999)  What do you understand by the term escape velocity?
\item (1999)  Calculate the escape velocity from the moon’s surface given that a man on the moon has $ 1/6$ his weight on earth. The mean radius of the moon is $ 1.75 \times 10^6$ m.
\item (1999)  Explain the meaning of the following terms:
 \begin{itemize}
\item Gravitational Potential of the Earth.
\item Gravitational Field Strength of the Earth.
\item How are the above quantities in and related?
\end{itemize}
\item (1999)  Show that the total energy of a satellite in a circular orbit equals half its potential energy.
\item (1999)  Calculate the height above the Earth's surface for a satellite in a parking orbit.
\item (1999)  What would be the length of a day if the rate of rotation of the Earth were such that the acceleration of gravity $ g=0$ at the equator?
\item (2007)  Evaluate the work done by the Earth's gravitational force and by the tension in the string as the ball moves from its highest to its lowest point.
\item (2007)  Two small spheres each of mass $ 10g$ are attached to a light rod $ 50$ cm long. The system Is set into oscillation and the period of torsional oscillation is found to be $ 770$ seconds. To produce maximum torsion to the system two large spheres each of mass $ 10$ kg are placed near each suspended sphere, if the angular deflection of the suspended rod Is $ 3.96 \times 10^{-3}$ rad. and the distance between the centres of the large spheres and small spheres is $ 10$ cm, determine the value of the universal constant of gravitation, $ G$ , from the given information. 
\item (2007)  On the basis of Newton’s universal law of gravitation, derive Kepler’s third law of planetary motion. 
\item (2007)  A planet has half the density of earth but twice its radius. What will be the speed of a satellite moving fast past the surface of the planet which has on no atmosphere?
 \begin{itemize}
\item ( Radius of earth $ R_{E}=6.4 \times 10^{3}$ km and gravitational potential energy $ g_{E}=9.81$ N$/$kg )
\end{itemize}
\item (2009)  State Kepler's laws of planetary motion.
\item (2009)  Explain the variation of acceleration due to gravity, $ g$ . inside and outside the earth.
\item (2009)  Derive the formula for mass and density of the earth.
\item (2009)  What do you understand by the term satellite?
\item (2009)  A satellite of mass $ 100$ kg moves in a circular orbit of radius $ 7000$ km around the earth, assumed to be a sphere of radius $ 6400$ km.  Calculate the total energy needed to place the satellite in orbit from the earth assuming $ g=10$ N$/$kg at the earth’s surface.
\item (2013)  With the aid of a labeled diagram, sketch the possible orbits for a satellite launched from the earth.
 \begin{itemize}
\item From the diagram above, write down an expression for the velocity of a satellite corresponding to each orbit.
\end{itemize}
\item (2014)  Define the universal gravitational constant.
\item (2014)  How is the gravitational potential related to gravitational field strength?
\item (2014)  Write down an expression for the acceleration due to gravity (g) of a body of mass (m) which is at a  distance (r) from the centre of the earth. 
 \begin{itemize}
\item If the Earth were made of lead of relative density of $ 11.3$ kg$/$m$ ^{3}$ , what would he the value of acceleration due to gravity on the surface of the earth?
\end{itemize}
\item (2014)  Why the value of acceleration due to gravity (g) changes due to the change in latitude? Give two reasons.
\item (2014)  A rocket is fired from the earth towards the sun. At what point on its path is the gravitational force on the rocket zero?
\item (2015)  Explain why the astronaut appears to be weightless when traveling in the space vehicle.
\item (2015)  State Newton's law of gravitation. 
 \begin{itemize}
\item Use Newton’s law of gravitation to derive Kepler’s third law.
\end{itemize}
\item (2015)  Briefly explain why Newton’s equation of universal gravitation does not hold for bodies falling near the surface of the earth? 
\item (2015)  Show that the total energy of a satellite in a circular orbit equals half its potential energy.
\item (2015)  Calculate the height above the Earth’s surface for a satellite in a parking orbit.
\item (2015)  A $ 10$ kg satellite circles the Earth once every $ 2$ hours in an orbit having a radius of $ 8000$ km.  Assuming Bohr’s angular momentum postulate applies to the satellite just as it does to an electron in the hydrogen atom,  find the quantum number of the orbit of the satellite.
\item (2016)  Mention one application of parking orbit.
\item (2016)  Briefly explain how parking orbit of a satellite is achieved.
\item (2016)  The earth satellite revolves in a circular orbit at a height of $ 300$ km above the earth’s surface.  Find the; 
 \begin{itemize}
\item Velocity of the satellite
\item Period of the satellite.
\end{itemize}
\item (2016)  A spaceship is launched into a circular orbit close to the earth’s surface.  What additional velocity has to be imparted on the spaceship if order to overcome the gravitational pull?
\item (2017)  Why does the kinetic energy of an earth satellite change in the elliptical orbit?
\item (2017)  A space craft is launched from the earth to the moon, If the mass of the earth is $ 81$ times that of the moon and the distance from the centre of the earth to that of the moon is about $ 4.0 \times 10^{5}$ km;
 \begin{itemize}
\item Draw a sketch showing how the gravitational force on the spacecraft varies during its journey. 
\item Calculate the distance from the centre of the earth where the resultant gravitational force becomes zero. 
\end{itemize}
\item (2018)  A satellite of mass $ 600$ kg is in a circular orbit at a height $ 2 \times 10^{6}$ km above the earth’s surface. Determine the:
 \begin{itemize}
\item Orbital speed. 
\item Gravitational potential energy. 
\end{itemize}
\item (2018)  What would happen if gravity suddenly disappears?  
\item (2018)  Two base of a mountain are at sea level where the gravitational field strength is $ 9.81$ N$/$kg . If the value of gravitational field at the top of the mountain is $ 9.7$ N$/$kg, calculate the height of the mountain above the sea level. 
\item (2019)  Why the weight of a body becomes zero at the centre of the earth? 
\item (2019)  How far above the earth surface does the value of acceleration due to gravity becomes $ 36\%$ of its value on the surface? 
\item (2019)  Compute the period of revolution of a satellite revolving in a circular orbit at a height of $ 3400$ km above the Earth’s surface. 
\item (2019)  Prove that the angular momentum fora satellite of mass $ M_{s}$ revolving round the
 \begin{itemize}
\item earth of mass $ M_{e}$ in an orbit of radius $ r$ is equal to $ (G M_{e}$  $ M_{s}^{2}r)^{1/2}$ .
\end{itemize}
\end{itemize}

\end{document}
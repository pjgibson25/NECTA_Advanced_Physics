
\documentclass{article}
\usepackage[a4paper, total={6in, 8in}]{geometry}
% \usepackage[utf8]{inputenc}
\usepackage{abstract}
\title{\textbf{3.3 - Viscosity and Turbulent Flow}}
\author{PJ Gibson - Peace Corps Tanzania}
\date{May 2020}

\begin{document}

\maketitle

\begin{itemize}
\item (1998)  Two equal drops of water are falling through air with a steady velocity of $ 0.15$ ms$ ^{-1}$ , If the drops coalesce, find their new terminal velocity.
\item (1999)  With the help of a well labelled diagram briefly explain how you will determine the coefficient of viscosity of a liquid by a constant pressure head apparatus in the laboratory.
\item (2010)  In the form of Millikan’s experiment, an oil drop was observe fall with a constant velocity of $ 2.5	\times 10^{-4}m/s$ in the absence of an electric field. When a p.d of $ 1000$ V was applied between the plates $ 10$ mm apart, the drop remained stationary between them. i the density of oil is $ 9 \times 10^{2}$ kg$/$m$ ^{3}$ , density of air is $ 1.2$ kg$/$m$ ^{3}$ and viscosity of air is $ 1.8\times 10^{-5}$ Ns$/$m$ ^{2}$ , Calculate the radius of the oil drop and the number of electric charges it carries.
\item (2013)  Write down the Poiscuille’s equation for a viscous fluid flowing through a tube defining all the symbols.
 \begin{itemize}
\item What assumptions are used to develop the equation above. 
\end{itemize}
\item (2015)  A sphere is dropped under gravity through a fluid of viscosity, $ \eta $ .  Taking average acceleration as half of the initial acceleration, show that the time taken to attain terminal velocity is independent of fluid density.
\item (2015)  The flow rate of water from a tap of diameter $ 1.25$ cm is $ 3$ litres per minute.  The coefficient of viscosity of water is $ 10^{-3}$ Ns/m$ ^{2}$ .  Determine the Reynolds’ number and then state the type of flow of water.
\item (2016)  State Newton’s law of viscosity and hence deduce the dimensions of the coefficient of viscosity.
\item (2016)  In an experiment to determine the coefficient of viscosity of motor oil, the following measurements are made:
 \begin{itemize}
\item Mass of glass sphere $ =1.2 \times 10^{-4}$ kg.
\item Diameter of sphere $ =4.0 \times 10^{-3}$ m.
\item Terminal velocity of sphere $ =5.4 \times 10^{-5}$ m$/$s.
\item Density of oil $ =860$ kg$/$m$ ^{3}$
\item Calculate the coefficient of viscosity of the oil.
\end{itemize}
\item (2016)  Give reasons for the following observations as applied in fluid dynamics.
 \begin{itemize}
\item A flag flutter when strong winds are blowing on a certain day.
\item A parachute is used while jumping from an airplane.
\item Hotter liquids flow faster than cold ones.
\end{itemize}
\item (2017)  Derive an expression for the terminal velocity of a spherical body falling  from rest through a viscous fluid. 
\item (2019)  Give the meaning of the terms velocity gradient, tangential stress and coefficient of viscosity as used in fluid dynamics.
\item (2019)  Write Stokes’ equation defining clearly the meaning of all symbols used.
 \begin{itemize}
\item State two assumptions used to develop the equation above
\end{itemize}
\item (2019)  Calculate the terminal velocity of the rain drops falling in air assuming that the flow is laminar, the rain drops are spheres of diameter $ 1$ mm and the coefficient of viscosity, $ \eta =1.8 \times 10^{-5}$ Ns$/$m$ ^{2}$ . 
\item (2019)  Water flows past a horizontal plate of area $ 1.2$ m$ ^{2}$ . If its velocity gradient and coefficient of viscosity adjacent to the plate are $ 10$ s$ ^{-1}$ and $ 1.3 \times 10^{-5}$ Ns$/$m$ ^{2}$ respectively, calculate the force acting on the plate.  
\end{itemize}

\end{document}
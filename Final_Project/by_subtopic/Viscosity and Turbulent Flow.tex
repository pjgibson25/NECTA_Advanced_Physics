
\documentclass{article}
\usepackage[a4paper, total={6in, 8in}]{geometry}
% \usepackage[utf8]{inputenc}
\usepackage{abstract}
\title{3.3 - Fluid Dynamics}
\author{PJ Gibson - Peace Corps Tanzania}
\date{May 2020}

\begin{document}

\maketitle


\section{Fluid Dynamics}

\subsection{Viscosity and Turbulent Flow}
\begin{itemize}
\item (2019)  Give the meaning of the terms velocity gradient, tangential stress and coefficient of viscosity as used in fluid dynamics.
\item (2019)  Write Stokes’ equation defining clearly the meaning of all symbols used.
 \begin{itemize}
\item State two assumptions used to develop the equation above
\end{itemize}
\item (2019)  Calculate the terminal velocity of the rain drops falling in air assuming that the flow is laminar, the rain drops are spheres of diameter $ 1$ mm and the coefficient of viscosity, $ \eta =1.8 \times 10^{-5}$ Ns$/$m$ ^{2}$ . 
\item (2019)  Water flows past a horizontal plate of area $ 1.2$ m$ ^{2}$ . If its velocity gradient and coefficient of viscosity adjacent to the plate are $ 10$ s$ ^{-1}$ and $ 1.3 \times 10^{-5}$ Ns$/$m$ ^{2}$ respectively, calculate the force acting on the plate.  
\end{itemize}

\end{document}

\documentclass{article}
\usepackage[a4paper, total={6in, 8in}]{geometry}
% \usepackage[utf8]{inputenc}
\usepackage{abstract}
\title{\textbf{2.2 - Projectile Motion}}
\author{PJ Gibson - Peace Corps Tanzania}
\date{May 2020}

\begin{document}

\maketitle

\begin{itemize}
\item (2000)  Mention two motions that add up to make projectile motion.
\item (2000)  In long jumps does it matter how high you jump? State the factors which determine the span of the jump. 
\item (2000)  Derive an expression that relates the span of the jump and the factors you have mentioned.
\item (2000)  A bullet is fired from a gun on the top of a cliff $ 140$ m high with a velocity of $ 150$ m$/$s at an elevation of $ 30^{\circ}$ to the horizontal. Find the horizontal distance from the foot of a cliff to the point where the bullet lands on the ground.
\item (2007)  What is meant by the term "projectile" as applied to projectile motion?
\item (2007)  Give two $ (2)$ practical applications of projectile motion at your locality.
\item (2007)  The ceiling of a long hall is $ 25$ m high.  Determine the maximum horizontal distance that a ball thrown with a speed of $ 40$ m$/$s can go without hitting the ceiling of the wall.
\item (2010)  Mention two examples of projectile motion. 
\item (2010)  Define the trajectory. 
\item (2010)  Mention two uses of projectile motion.
\item (2010)  Find the velocity and angle of projection of a particle which passes in a horizontal direction Just over the top of a wall which is $ 12$ m high and $ 32$ m away. 
\item (2013)  List down two main assumptions in deriving the equation of projectile motion.
\item (2013)  Why the horizontal motion of a projectile constant? 
\item (2013)  A ball is thrown horizontally with a speed of $ 14.0$ m$/$s from a point $ 6.4$ m above the ground, calculate:
 \begin{itemize}
\item The horizontal distance traveled in that time.
\item Its velocity when it reaches the ground.
\end{itemize}
\item (2014)  Outline the motions that add up to make projectile motion. 
\item (2014)  In the first second of its flight, a rocket ejects $ 1/60$ of  its mass with a relative velocity of $ 2400$ m$/$s.
 \begin{itemize}
\item Find its acceleration.
\item What is the final velocity if the ratio of initial to final mass of the rocket is $ 4$ at a time of $ 60$ seconds? 
\end{itemize}
\item (2014)  A ball is thrown upwards with an initial velocity of $ 33$ m$/$s from a point $ 65^{\circ}$ on the side of a hill which slopes upward uniformly at an angle of $ 28^{\circ}$ .
 \begin{itemize}
\item At what distance up the slope does the ball strike?
\item Calculate the time of flight of the ball. 
\end{itemize}
\item (2014)  A cannon of mass $ 1300$ kg fires a $ 72$ kg ball in a horizontal direction with a nuzzle speed of $ 55$ m$/$s, If the cannon is mounted so that it can recoil freely calculate the:
 \begin{itemize}
\item  recoil velocity of the cannon relative to the earth. 
\item horizontal velocity of the ball relative to the earth. 
\end{itemize}
\item (2015)  Define the term trajectory.
\item (2015)  Briefly explain why the horizontal component of the initial] velocity of a projectile always remains constant.
\item (2015)  List down two limitations of projectile motion. 
\item (2015)  A body projected from the ground at the angle of $ 60^{\circ}$ is required to pass just above the two vertical walls each of height $ 7$ m. If the velocity of projection is $ 100$ m$/$s, calculate the distance between the two walls. 
\item (2015)  A fireman standing at a horizontal distance of $ 34$ m from the edge of the burning story building aimed to raise streams of water at an angle of $ 60^{\circ}$ into the first floor through an open window which is at $ 20$ m high from the ground level. If water strikes on this floor $ 2$ m away from the outer edge, 
 \begin{itemize}
\item  Sketch a diagram of the trajectory.
\item What speed will the water leave the nozzle of the fire hose?
\end{itemize}
\item (2016)  Mention two characteristics of projectile motion.
\item (2016)  If the range of the projectile is $ 120$ m and its time of flight is $ 4$ sec , determine the angle of projection and its initial velocity of projection assuming that the acceleration due to gravity $ g=10$ m$/$s. 
\item (2017)  A jumbo jet traveling horizontally at $ 50$ m$/$s at a height of $ 500$ m from sea level drops a luggage of food to a disaster area.
 \begin{itemize}
\item At what horizontal distance from the target should the luggage be dropped?
\item Find the velocity of the luggage as it hit the ground. 
\end{itemize}
\item (2018)  How does projectile motion differ from uniform circular motion? 
\item (2018)  A rifle shoots a bullet with a muzzle velocity of $ 1000$ m$/$s at a small target $ 200$ m away. How high above the target must the rifle be aimed so that the bullet will hit the target? 
 \begin{itemize}
\item Where does the object strike the ground when thrown horizontally with a velocity of $ 15$ m$/$s from the top of a $ 40$ m high building? 
\item Find the speed of travel when a man jumps a maximum horizontal distance of $ 1$ m spending a minimum time on the ground.
\end{itemize}
\item (2019)  Justify the statement that projectile motion is two dimensional motion.
\item (2019)  A rocket was launched with a velocity of $ 50$ m$/$s from the surface of the moon at an angle of $ 40^{\circ}$ to the horizontal, Calculate the horizontal distance covered  after half time of flight.
\item (2019)  Show that the angle of projection $ \theta ^{\circ}$ for a projectile launched from the origin is given by $ \theta ^{\circ}= tan^{-1}(4h_{m}/R)$ , where $ R$ stand for horizontal range and $ h_{m}$ is the maximum vertical height.
\item (2019)  Determine the angle of projection for which the horizontal range of a projectile is $ 4\sqrt{3}$ times its maximum height. 
\end{itemize}

\end{document}
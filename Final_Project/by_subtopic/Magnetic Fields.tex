
\documentclass{article}
\usepackage[a4paper, total={6in, 8in}]{geometry}
% \usepackage[utf8]{inputenc}
\usepackage{abstract}
\title{\textbf{8.1 - Magnetic Fields}}
\author{PJ Gibson - Peace Corps Tanzania}
\date{May 2020}

\begin{document}

\maketitle

\begin{itemize}
\item (2000)  A proton is moving in a uniform magnetic field $ B$ . Draw the diagram representing $ B$ and the path of the proton if its initial direction makes an oblique angle to the direction of the field $ B$ . 
\item (2007)  Define the magnetic field intensity.
\item (2007)  A long solenoid has $ 10$ turns per cm and carries a current of $ 2.0$ A.  Calculate the magnetic field intensity at its centre.
\item (2007)  An electron having $ 450$ eV of energy enters at right angles to a uniform magnetic field of strength $ 1.50x10^{-3}$ T.  Show that the path traced by the electron in a uniform magnetic field is circular and estimate its radius.
\item (2007)  A charged oil drop of mass $ 6.0x10^{-15}$ kg falls vertically in air with a steady velocity between two long parallel vertical plates $ 5.0$ mm apart.  When a potential difference of $ 3000$ V is applied between the plates the drop falls with a steady velocity at an angle of $ 58^{\circ}$ to the vertical.
 \begin{itemize}
\item Determine the charge $ Q$ , on the oil drop.
\end{itemize}
\item (2007)  A coil having $ 475$ turns and cross sectional area $ 20 cm^{2}$ , rotates at $ 600r.p.m$ . in a uniform magnetic field of $ 0.01$ T. Find:
 \begin{itemize}
\item the peak e.m.f and the r.m.s. e.m.f induced in the coil. 
\item show these values on a graph of $ E$ vs time. 
\end{itemize}
\item (2009)  Outline four applications of eddy currents.
\item (2010)  Distinguish between magnetic flux density and magnetic induction.
\item (2010)  Describe using a sketch graph how magnetic flux density varies with the axis (both inside and at the ends) of a long solenoid carrying current. 
\item (2010)  A solenoid $ 80$ m long has a cross-sectional area of $ 16$ cm$ ^{2}$ and a total of $ 3500$ turns closely wound. If the coil is filled with air and carries a current of $ 3$ A, Calculate:
 \begin{itemize}
\item Magnetic field density $ B$ at the middle of the coil.
\item Magnetic flux inside the coil. 
\item Magnetic force $ H$ at the centre of the coil. 
\item Magnetic induction at the end of the coil.
\item $ (v$ ) Magnetic field intensity at the middle of the coil. 
\end{itemize}
\item (2013)  Mention the factors which determine the magnitude and direction of the force experienced by a current-carrying conductor in a magnetic field.
\item (2013)  What is the maximum torque on a $ 400-$ turns circular coil of radius $ 0.75$ cm that carrying a current of $ 1.6$ mA and resides in a uniform magnetic field of $ 0.25$ T?
\item (2013)  Brielfly explain how you can demonstrate that there are two types of charges in nature.
\item (2013)  A $ 10$ eV proton is circulating in a plane at right angles to a uniform magnetic field of magnetic flux density of $ 1.0 \times 10^{-4}$ Wb$/$m$ ^{2}$ Calculate the cyclotron frequency of a proton.
\item (2013)  A toroid of inner radius $ 25$ cm and an outer radius of $ 28$ cm has $ 4500$ turns of wound around it which passes a Current of $ 12$ A. What will be the induction of the magnetic flux;
 \begin{itemize}
\item Outside the toroid. 
\item inside the core of the toroid, 
\item in an empty space surrounding the toroid. 
\end{itemize}
\item (2016)  What is meant by the following terms:
 \begin{itemize}
\item  Phase of alternating e.m.f.
\item  Root mean square (r.m.s.) value of alternating e.m.f.
\end{itemize}
\item (2016)  State the following laws or theorems as applied in magnetism.
 \begin{itemize}
\item Biot-Savart law
\item Ampere’s theorem
\end{itemize}
\item (2016)  Derive an expression for the magnetic flux density $ B$ at the centre of the circular coil of radius $ r$ and $ N$ turns placed in air carrying a current i.
\item (2016)  The diameter of a $ 40$ turn circular coil is $ 16$ cm and it has a current of $ 5$ A.  Calculate:
 \begin{itemize}
\item The magnetic induction at the centre of the coil
\item The magnetic moment of the coil.
\item The torque action on the coil if it is suspended in a uniform magnetic field of $ 0.76$ T such that its plane is parallel to the field.
\end{itemize}
\item (2017)  Draw the diagram of the solenoid with certain number of tums placed in the magnetic field and indicate any suitable directions of the flow of current in it.
\item (2017)  Write down the formula for the magnetic field induced at the centre of solenoid. 
\item (2017)  It is desired to design a solenoid that produces a magnetic field of $ 0.1$ T at the centre. If the radius of solenoid is $ 5$ cm, its length is $ 50$ cm and carries a current of $ 10$ A; Calculate:
 \begin{itemize}
\item The number of turns per unit length of the solenoid. 
\item The total length of a wire required. 
\end{itemize}
\item (2017)  State the Biot-Savart law. 
\item (2017)  In a hydrogen atom, an electron keeps moving around its nucleus with a constant speed of $ 2.18 \times 10^{6}$ m$/$s. Assuming that the orbit is a circular of radius $ 5.3 \times 10^{-11}$ m. determine the magnetic flux density produced at the site of the proton in the nucleus. 
\item (2018)  A circular coil of $ 300$ turns has a radius of $ 10$ cm and carries a current of $ 7.5$ A. Calculate the magnetic field at:
 \begin{itemize}
\item the centre of the coil. 
\item a point which is at a distance of $ 5$ cm from the centre of the coil. 
\end{itemize}
\item (2019)  Identify four factors that affect the force experienced by a current-carrying conductor in a magnetic field. 
\item (2019)  Write the mathematical expression which define magnetic flux density and use it to deduce its S.I. units. 
 \begin{itemize}
\item Apply an expression obtained above to develop the formula for the force on a conductor carrying current i if the conductor and the magnetic fields are not at night angles.
\end{itemize}
\item (2019)  State the condition which makes the magnetic force on a moving charge in a magnetic field to be maximum. 
\item (2019)  Use mathematical expression to justify the statement that there will be no change in the kinetic energy of a charged particle which enters a uniform magnetic field when its initial velocity is directed parallel to the field.
\end{itemize}

\end{document}

\documentclass{article}
\usepackage[a4paper, total={6in, 8in}]{geometry}
% \usepackage[utf8]{inputenc}
\usepackage{abstract}
\title{\textbf{10.5 - Operational Amplifiers}}
\author{PJ Gibson - Peace Corps Tanzania}
\date{May 2020}

\begin{document}

\maketitle

\begin{itemize}
\item (1998)  Sketch the traces seen on the screen of a cathode ray oscilloscope when two sinusoidal potential differences of the same frequency — and amplitude are applied simultaneously to $ X$ and $ Y$ plates of  a cathode ray oscilloscope, when the phase difference between them is:
 \begin{itemize}
\item $ 0^{\circ}$ $ 45^{\circ}$ $ 90^{\circ}$ .
\end{itemize}
\item (1999)  Briefly describe the major factors that you would consider when designing a voltage amplifier.
\item (1999)  With the help of clear diagrams, explain how you would overcome thermal run away in a voltage amplifier.
\item (2000)  Mention any three uses of a CRO.
\item (2000)  What is an operational amplifier 
\item (2000)  List three desirable features of an operational amplifier.
\item (2000)  In almost all cases, where an operation amplifier is used as a linear voltage amplifier, negative feedback is employed. State the advantage of negative feedback.
\item (2007)  Make well labelled diagram of the cathode ray oscilloscope and explain briefly how a sinusoidal voltage signal is displayed on its screen.
\item (2007)  Mention three $ (3)$ practical applications of the cathode ray oscilloscope.
\item (2007)  Explain the terms output saturation and negative feedback as applied to op-amplifiers. 
\item (2007)  For an ideal operational amplifier, what are the values of the:
 \begin{itemize}
\item current into both inputs of the op-amp? 
\item voltage between the inputs if the output is not saturated? 
\end{itemize}
\item (2007)  What is a non-inverting amplifier? 
\item (2009)  Explain the following terms:
 \begin{itemize}
\item Forward bias.
\item Reverse bias.
\item Inverting and non-inverting amplifier. 
\end{itemize}
\item (2009)  An operational amplifier is to have a voltage gain of $ 100$ .  Calculate the required values for the external resistances $ R_{1}$ and $ R_{2}$ when the following gains are required:
 \begin{itemize}
\item non-inverting.
\item Inverting.
\end{itemize}
\item (2013)  Briefly explain why Cathode Ray Oscilloscope (C.R.O.) is said to be an excellent instrument for measuring the emf 
\item (2013)  Draw a well labeled circuit diagram of an inverting amplifier.
\item (2014)  What is the purpose of amplifiers in a phone link? 
\item (2015)  List three properties of operational amplifiers.
\item (2015)  What is meant by the term negative feedback? Give four advantages of using it in an op-amp or any type of voltage amplifier.
\item (2015)  Derive an expression of the closed loop gain for an inverting op-amp voltage amplifier with an input resistor $ R$ , and a feedback resistor.
\item (2016)  Explain the use of an op-amp as a summing amplifier.
\item (2016)  Name three electronic circuits in which multivibrators can be constructed.
 \begin{itemize}
\item List down three types of multivibrators.
\item Briefly explain the applications of multivibrators listed above.
\end{itemize}
\item (2016)  Mention two characteristics of op-amps.
\item (2016)  Briefly explain why op-amps are sometimes called differential amplifiers?
\item (2016)  Describe the structure and the mode of action of a simplified version of the Van de Graaff generator.
\item (2017)  Briefly explain the function of the following:
 \begin{itemize}
\item Oscilloscope
\item Op-amps
\end{itemize}
\item (2017)  A change of $ 100$ A in the base current produces a change of $ 3$ mA in the collector current. Calculate:
 \begin{itemize}
\item The current amplification factor, $ \beta$
\item The current gain, $ \alpha $
\end{itemize}
\item (2019)  Distinguish between inverting OP-AMP and non-inverting OP-AMP. 
 \begin{itemize}
\item Give one application of each type of OP-AMP described above.
\end{itemize}
\end{itemize}

\end{document}

\documentclass{article}
\usepackage[a4paper, total={6in, 8in}]{geometry}
% \usepackage[utf8]{inputenc}
\usepackage{abstract}
\title{\textbf{6.1 - Mechanical Vibrations}}
\author{PJ Gibson - Peace Corps Tanzania}
\date{May 2020}

\begin{document}

\maketitle

\begin{itemize}
\item (2000)  Show how wavelength and frequency of a wave are related.
\item (2007)  State the modes of vibrations in closed and open pipes.  
\item (2013)  What is meant by dispersion of waves? 
\item (2013)  Briefly explain if it is possible for dispersion to take place on a wave whose frequency lies in the audible range.
\item (2015)  Define the following terms:
 \begin{itemize}
\item Damped oscillations
\item Forced oscillations
\item Resonance
\end{itemize}
\item (2016)  What do you understand by the following terms: 
 \begin{itemize}
\item Damped oscillations. 
\item Undamped oscillations.
\end{itemize}
\item (2016)  Sketch the waveform diagrams to represent the terms: damped oscillations & undamped oscillations
\item (2016)  A steel wire hangs vertically from a fixed point, supporting a weight of $ 80$ N as its lower end.  The length of the wire from the fixed point to the weight is $ 1.5$ m.  Calculate the fundamental frequency emitted by the wire when it is plucked if its diameter is $ 0.5$ mm. 
\item (2017)  A $ 40$ cm long wire is in unison with a tuning fork of frequency $ 256$ Hz, when stretched by a load of density $ 9$ gm$ ^{-3}$ hanging vertically. The load is then immersed in water. By how much the length of the wire should be reduced to bring it again in unison with the same tuning fork,
\end{itemize}

\end{document}
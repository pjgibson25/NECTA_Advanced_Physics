
\documentclass{article}
\usepackage[a4paper, total={6in, 8in}]{geometry}
% \usepackage[utf8]{inputenc}
\usepackage{abstract}
\title{THE UNITED REPUBLIC OF TANZANIA

NATIONAL EXAMINATIONS COUNCIL

ADVANCED CERTIFICATE OF SECONDARY EDUCATION EXAMINATION

\textbf{2017 PHYSICS 2}}
\author{Transcribed by:  PJ Gibson}

\begin{document}

\maketitle

\begin{itemize}
\item State Bernoulli's theorem for the horizontal flow. 
\item On which principle does the Bernoulli's theorem based. 
\item A pipe is running full of water. At a certain point $ A$ , it tapers from $ 30$ cm diameter to $ 10$ cm diameter at $ B$ , the pressure difference between point $ A$ and $ B$ is $ 100$ cm of water column. Find the rate of flow of water through the pipe. 
\item What is the terminal velocity?
\item Derive an expression for the terminal velocity of a spherical body falling  from rest through a viscous fluid. 
\item Two capillaries of the same length and radii in the ratio of $ 1$:$ 2$ are connected in series and the liquid flow through the system under stream line conditions. If the pressure across the two extreme ends of the combination is  $ 1$ m of water, what is the pressure difference across the first capillary?
\item A cyclist and a railway train are approaching each other with a speed of $ 10$ m$/$s and $ 20$ m$/$s respectively. If the engine driver sounds a warning siren at a frequency of  $ 480$ Hz, calculate the frequency of the noise heard by the cyclist:
 \begin{itemize}
\item Before the train has passed.
\item After the tram has passed. 
\end{itemize}
\item The equation $ y= a  \sin(\omega t – kx)$ represents a plane wave traveling in a medium along the $ x$ - direction, $ y$ being the displacement at the point $ x$ at time $ t$ . Deduce whether the wave is traveling in the positive $ x$ – direction or in the negative $ x$ – direction.
 \begin{itemize}
\item If $ z=1.1 \times 10^{-7}$ m , $ \omega = 6.5 \times 10^{3}$ s$ ^{-1}$ , $ k=19$ m$ ^{-1}$ ; determine the speed of the wave.
\end{itemize}
\item Briefly explain why diffraction is common in sound but not in light.
\item A $ 40$ cm long wire is in unison with a tuning fork of frequency $ 256$ Hz, when stretched by a load of density $ 9$ gm$ ^{-3}$ hanging vertically. The load is then immersed in water. By how much the length of the wire should be reduced to bring it again in unison with the same tuning fork,
\item In a Young's double - slit experiment a total of $ 23$ bright fringes occupying $ 4$ total distance of $ 3.9$ mm were visible in traveling microscope, which was focused on a plane being at a distance of $ 31$ cm from the double slit. If the wavelength of light being used was $ 5.5 \times 10^{-7}$ m; determine the separation of the double slit.
\item When a grating with $ 300$ lines per millimeters is illuminated normally with parallel beam of monochromatic light a second order principal maximum is observed at $ 18.9^{\circ}$ to the straight through direction. Find the wavelength of the light.
\item A white light fall on a slit of width ‘a’: for what value of 'a' will be the first minimum of light falling at the angle of $ 30^{\circ}$ when the wavelength of light is $ 6500$ nm? 
\item A steel rod of length $ 0.60$ m and cross-sectional area $ 2.5 \times 10^{-5}$ m$ ^{2}$ at a temperature of $ 100^{\circ}$C is clamped so that when it cools was unable to contract. Find the tension in the rod when it has cooled to $ 20^{\circ}$C. 
\item A spring $ 60$ cm long is stretched by $ 2$ cm for the application of load of $ 200$ g. What will be the length when a load of $ 500$ g is applied? 
\item Calculate the percentage increase in length of a wire of diameter $ 2.2$ mm stretched by a load of $ 100$ kg. ( Young's modulus of wire is $ 12.5 \times 10^{10}$ N$/$m$ ^{2})$
\item Define the terms capacitance and electric potential. 
\item The capacitance $ C$ of a capacitor ts full charged by a $ 200$ V battery. It is then discharged through a small coil of resistance wire embedded in a thermally insulated block of specific heat capacity $ 2.5 \times 10^{2}$ J$/$kgK and of mass of $ 0.1$ kg.  If the temperature of the block rises by $ 0.4$ K. what is the value of $ C$ ?
\item A parallel plate capacitor has plates each of area $ 0.24$ m$ ^{2}$ separated by a small distance
 \begin{itemize}
\item $ 0.50$ mm. If the capacitor is full charged by a battery of electromotive force of $ 24$ V, calculate:
\item the capacitance of the capacitor. 
\item the energy stared tn the capacitor. 
\end{itemize}
\item Comment on the assertion that, the safest way of protecting yourself from lightning is to be inside a car. 
\item Define tensile stress and tensile strain. 
\item Calculate the work done in a stretching copper wire of $ 100$ cm long and $ 0.03 $ cm$ ^{2}$ cross — sectional area when a load of $ 120$ N is applied. 
\item Mention any two factors on which modulus of elasticity of a material depends.
\item A traffic light is suspended with two steel wires of equal lengths and radii of $ 0.5$ cm. If the wires make an angle of $ 15^{\circ}$ with the horizontal, what is the fractional increase in their length due to the weight of the light? 
\item Define free surface energy in relation to the quid surface.
 \begin{itemize}
\item Explain what will happen if two bubbles of unequal radii are joined by a tube without bursting. 
\end{itemize}
\item A spherical drop of mercury of radius $ 5$ mm falls on the ground and breaks into $ 1000$ droplets. Calculate the work done in breaking the drop. 
\item What is meant by the following?
 \begin{itemize}
\item Atomic Mass Unit (a.m.u.)
\item Binding energy. 
\item Mass defect
\end{itemize}
\item Write down the equation for the disintegration.
\item State the law of force acting on a conductor of length $ l$ carrying an electric current in a magnetic field. 
\item Draw the diagram of the solenoid with certain number of tums placed in the magnetic field and indicate any suitable directions of the flow of current in it.
\item Write down the formula for the magnetic field induced at the centre of solenoid. 
\item It is desired to design a solenoid that produces a magnetic field of $ 0.1$ T at the centre. If the radius of solenoid is $ 5$ cm, its length is $ 50$ cm and carries a current of $ 10$ A; Calculate:
 \begin{itemize}
\item The number of turns per unit length of the solenoid. 
\item The total length of a wire required. 
\end{itemize}
\item State the Biot-Savart law. 
\item In a hydrogen atom, an electron keeps moving around its nucleus with a constant speed of $ 2.18 \times 10^{6}$ m$/$s. Assuming that the orbit is a circular of radius $ 5.3 \times 10^{-11}$ m. determine the magnetic flux density produced at the site of the proton in the nucleus. 
\item Use the Rydberg constant, $ R_{H}=1.0974 \times 10^{7}$ m$ ^{-1}$ to calculate the shortest wavelength of the Balmer series. 
\item Use the Bohr's theory for hydrogen atom to determine the:
 \begin{itemize}
\item Radius of the first orbit of the hydrogen atom in A units. 
\item Velocity of the electron in the first orbit. 
\end{itemize}
\item What is ionization potential of an atom?
\item Show that the ionization potential of hydrogen is $ 13.6$ eV. 
\item How can you account for the chemical behavior of atoms on the basis of the atomic electrons and shells? 
\item How can you account for the chemical behavior of atoms on the basis of the atomic electrons and shells? 
\end{itemize}

\end{document}
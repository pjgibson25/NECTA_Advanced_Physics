
\documentclass{article}
\usepackage[a4paper, total={6in, 8in}]{geometry}
% \usepackage[utf8]{inputenc}
\usepackage{abstract}
\title{THE UNITED REPUBLIC OF TANZANIA

NATIONAL EXAMINATIONS COUNCIL

ADVANCED CERTIFICATE OF SECONDARY EDUCATION EXAMINATION

\textbf{2017 PHYSICS 1}}
\author{Transcribed by:  PJ Gibson}

\begin{document}

\maketitle

\begin{itemize}
\item Give the meaning of the following terms as used in error analysis:
 \begin{itemize}
\item Absolute error. 
\item Relative error. 
\end{itemize}
\item The force ‘$ F$ ’ acting on an object of mass ‘$ m$ ’, travelling at velocity ‘v’ in a circle of radius ‘$ r$ ’ is given by: $ F= \frac{mv^{2}}{r}$ If the measurements are recorded as: $ m=(3.5 \pm 0.1)$ kg, $ V=(20\pm 1)$ m$/$s, $ r=(12.5\pm 0.5)m$ ; find the maximum possible
 \begin{itemize}
\item Fractional error. 
\item Percentage error in the measurement of force.
\item Show how you will record the reading of force, ‘$ F$ ’ in the question above. 
\end{itemize}
\item Define the term dimensions of a physical quantity. 
\item Identify two uses of dimensional equations.
\item What is the basic requirement for a physical relation to be correct? 
\item List two quantities whose dimension is [ $ ML^{2}T^{-1}$ ]
\item The frequency ‘$ f$ ’ of vibration of a stretched string depends on the tension ‘$ F$ ’, the length ‘$ l$ ’ and the mass per unit length $ $ $\mu$of the string. Derive the formula relating the physical quantities by the method of dimensions. 
\item Use dimensional analysis to prove the correctness of the relation, $ \rho = \frac{3g}{4 \pi RG}$ where by $ \rho =$ density of the earth, $ g=$ acceleration due to gravity, $ R=$ radius of the earth and $ G=$ gravitational constant.
\item Why does the kinetic energy of an earth satellite change in the elliptical orbit?
\item Give two factors which determine whether a planet has an atmosphere or not.
\item A space craft is launched from the earth to the moon, If the mass of the earth is $ 81$ times that of the moon and the distance from the centre of the earth to that of the moon is about $ 4.0 \times 10^{5}$ km;
 \begin{itemize}
\item Draw a sketch showing how the gravitational force on the spacecraft varies during its journey. 
\item Calculate the distance from the centre of the earth where the resultant gravitational force becomes zero. 
\end{itemize}
\item Justify the statement that ‘If no external torque acts on a body, its angular velocity will not conserved.
\item A car is moving with a speed of $ 30$ m$/$s on a circular track of radius $ 500$ m. If its speed is increasing at the rate of $ 2$ m$/$s, find its resultant linear acceleration.
\item An object of mass $ 1$ kg is attached to the lower end of a string $ 1$ m long whose upper end is fixed and made to rotate in a horizontal circle of radius $ 0.6$ m. If the circular speed of the mass is constant, find the:
 \begin{itemize}
\item Tension in the string. 
\item Period of motion. 
\end{itemize}
\item A $ 75$ kg hunter fires a bullet of mass $ 10$ g with a velocity of $ 400$ m$/$s from a gun of mass $ 5$ kg. Calculate the:
 \begin{itemize}
\item Recoil velocity of the gun. 
\item Velocity acquired by the hunter during firing.
\end{itemize}
\item A jumbo jet traveling horizontally at $ 50$ m$/$s at a height of $ 500$ m from sea level drops a luggage of food to a disaster area.
 \begin{itemize}
\item At what horizontal distance from the target should the luggage be dropped?
\item Find the velocity of the luggage as it hit the ground. 
\end{itemize}
\item The equation of simple harmonic motion is given as $ x=6 \sin(10\pi t)+8 \sin(10\pi t)$ , where $ x$ is in centimeters and $ t$ in seconds. Determine the:
 \begin{itemize}
\item Amplitude 
\item Initial phase of motion. 
\end{itemize}
\item Show that the total energy of a body executing simple harmonic motion is independent of time. 
\item Find the periodic time of a cubical body of side $ 0.2$ m and mass $ 0.004$ kg floating in water then pressed and released such that it oscillates vertically. 
\item Give a common example of adiabatic process. 
\item What happens to the internal energy of a gas during adiabatic expansion?
\item A mass of an ideal gas of volume $ 400$ cm$ ^{3}$ at $ 288$ K expands adiabatically. If its temperature falls to $ 273$ K;
 \begin{itemize}
\item Find the new volume of the gas. 
\item Calculate the final volume of the gas if it is then compressed isothermally until the pressure returns to its original value.
\end{itemize}
\item State the following according to heat exchange:
 \begin{itemize}
\item  Prevost’s theory. 
\item Wien's displacement law.
\end{itemize}
\item Briefly explain why:
 \begin{itemize}
\item Steam pipes are wrapped with insulating materials?
\item Stainless steel cooking pans fitted with extra copper at the bottom are more preferred?
\end{itemize}
\item The value of the property $ X$ of a certain substance Is given by $ X_{\theta}=X_{0}+0.5\theta +2\times 10^{-4}\theta ^{2}$  , Where $ \theta $ is the temperature in degree Celsius. What would be the Celsius temperature defined by the property $ X$ which corresponds to a temperature of $ 50^{\circ}$C on this gas thermometer scale? 
\item What is the advantage of using a greater length of potentiometer wire?
\item Why is Wheatstone bridge not suitable for measuring very high resistance?
\item List two factors on which the resistivity of a material depends. 
\item A wire of resistivity, $ \rho $ , is stretched to double its length. What will be its new resistivity? Give reason for your answer. 
\item Why a high voltage supply should have high internal resistance?
\item Justify the statement that ‘it is not possible to verify Ohm's law by using a filament lamp’.
\item A potential difference of $ 4$ V is connected to $ 4$ uniform resistance wire of length $ 3.0$ m and cross-sectional area $ 9\times 10^{-9}$ , when a current of $ 0.2$ A is flowing in the wire. Find the:
 \begin{itemize}
\item Resistivity of the wire.
\item Conductivity of the wire. 
\end{itemize}
\item Briefly explain the function of the following:
 \begin{itemize}
\item Oscilloscope
\item Op-amps
\end{itemize}
\item List three basic elements of communication system. 
\item Explain the advantage of using optical fibre systems instead of coaxial cable systems in telecommunication processes.
\item Define the term semiconductor.
 \begin{itemize}
\item Give three examples of semiconductor materials. 
\end{itemize}
\item Outline two factors on which electrical conductivity of a pure semiconductor depends. 
\item How does the forbidden energy gap of an intrinsic semiconductor vary with increase in temperature? 
\item Explain the meaning of the following terms:
 \begin{itemize}
\item $ P-$ type semiconductor.
\item $ N-$ type semiconductor. 
\end{itemize}
\item List three types of transistor configurations.
\item Why is collector of a transistor made wider than emitter and base? 
\item A change of $ 100$ A in the base current produces a change of $ 3$ mA in the collector current. Calculate:
 \begin{itemize}
\item The current amplification factor, $ \beta$
\item The current gain, $ \alpha $
\end{itemize}
\item State three sources of heat energy within the interior of the earth. 
\item Discuss two advantages of windbreaks to plant environment. 
\item Briefly explain the major causes of the following types of environmental pollution:
 \begin{itemize}
\item Water pollution. 
\item  Air pollution.
\end{itemize}
\end{itemize}

\end{document}
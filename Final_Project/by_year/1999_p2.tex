
\documentclass{article}
\usepackage[a4paper, total={6in, 8in}]{geometry}
% \usepackage[utf8]{inputenc}
\usepackage{abstract}
\title{THE UNITED REPUBLIC OF TANZANIA

NATIONAL EXAMINATIONS COUNCIL

ADVANCED CERTIFICATE OF SECONDARY EDUCATION EXAMINATION

\textbf{1999 PHYSICS 2}}
\author{Transcribed by:  PJ Gibson}

\begin{document}

\maketitle

\begin{itemize}
\item Define "Young's Modulus" of a material and give its SI units.
\item With the aid of a sketch graph, explain what happens when a steel wire is stretched gradually by an increasing load until it breaks. 
\item A force $ F$ is applied to a long steel wire of length $ L$ and cross-sectional area A.
 \begin{itemize}
\item Show that if the wire is considered to be a spring, the force constant $ k$ is given by: $ k= AY/L$ , where $ Y$ is Young's Modulus of the wire.
\item Show that the energy stored in the wire is $ U=1/2F \Delta L$ where $ \Delta{L}$ is the extension of the wire
\end{itemize}
\item The period $ T$ of vibrations of a tuning fork may be expected to depend on the density $ D$ , Young's Modulus $ Y$ of the material of which it is made and the length a of its prongs. Using dimensional analysis deduce an expression for $ T$ in terms of $ D$ , $ Y$ and a.
\item Explain the meaning of the following terms:
 \begin{itemize}
\item Gravitational Potential of the Earth.
\item Gravitational Field Strength of the Earth.
\item How are the above quantities in and related?
\end{itemize}
\item Show that the total energy of a satellite in a circular orbit equals half its potential energy.
\item Calculate the height above the Earth's surface for a satellite in a parking orbit.
\item What would be the length of a day if the rate of rotation of the Earth were such that the acceleration of gravity $ g=0$ at the equator?
\item What do you understand by the term "moments of inertia" of a rigid body?
\item State the perpendicular axes theorem of moments of inertia for a body in the form of a lamina
\item Calculate the moments of inertia of a thin circular disc of radius $ 50$ cm and mass $ 2$ kg about an axis along a diameter of the disc.
\item A wheel mounted on an axle that is not frictionless is initially at rest. A constant external torque of $ 50$ Nm is applied to the wheel for $ 20$ s. At the end of the $ 20$ s, the wheel has an angular velocity of
 \begin{itemize}
\item $ 600$ rev/min. The external torque is the removed, and the wheel comes to rest after $ 120$ s more.
\item Determine the moments of inertia of the wheel.
\item Calculate the frictional torque which is assumed to be constant. 
\end{itemize}
\item State the main assumptions of the “kinetic theory" of gases.
\item Derive an expression for the pressure exerted by an ideal gas on the walls of its container.
\item How does the average translational kinetic energy of a molecule of an ideal gas change if
 \begin{itemize}
\item the pressure is doubled while the volume is kept constant?
\item the volume is doubled while the pressure is kept constant?
\end{itemize}
\item Calculate the value of the root mean-square speed of molecules of helium at $ 0^{\circ}$C .
\item What is "capacitance"?
\item List three factors that govern the capacitance of a parallel plate capacitor.
\item Show that the energy per unit volume stored in a parallel plate capacitor is given by: $ U=1/2\epsilon E^{2}$ and define all the symbols in this equation.
\item Given that the distance of separation between the parallel plates of a capacitor is $ 5$ mm, and the plates have an area of $ 5$ m$ ^{2}$ . A potential difference of $ 10$ kV is applied across the capacitor which is
 \begin{itemize}
\item parallel in vacuum. Compute:
\item the capacitance
\item the electric intensity in the space between the plates
\item the change in the stored energy if the separation of the plates is increased from $ 5$ mm to $ 5.5$ mm.
\end{itemize}
\item With the help of illustrative diagrams explain the action of a choke in a circuit.
\item When an impedance consisting of an inductance $ L$ and a resistance $ R$ in series is connected across a $ 12$ V, $ 50$ Hz power supply, a current of $ 0.050$ A flows, which differs in phase from that of the applied potential difference by $ 60^{\circ}$ .
 \begin{itemize}
\item Find the value of $ R$ and $ L$ .
\item Find the capacitance of the capacitor which, when connected in series in the above circuit, has the effect of bringing the current into phase with the applied voltage.
\end{itemize}
\item An inductance of $ 4$ mH is connected in series with a resistance of $ 20\Omega $ together with a battery:
 \begin{itemize}
\item Determine how the current will vary with time in this circuit.
\item Sketch the current of above against time
\item Calculate the inductive time constant
\end{itemize}
\item State the laws of electromagnetic induction and describe briefly experiments (one in each case) which can be used to demonstrate them.
\item A flat coil of $ 100$ turns and mean radius $ 5.0$ cm is tying on a horizontal surface and is turned over in $ 0.20$ sec. against the vertical component of the Earth's magnetic field. Calculate the average e.m.f. induced.
\item With the help of clear diagrams, explain briefly how you would convert a sensitive galvanometer into:
 \begin{itemize}
\item an ammeter
\item a voltmeter
\end{itemize}
\item State Bohr’s postulates of the atomic model.
\item Show that for an electron in a hydrogen atom, the possible radii of an electron orbit are given by:
 \begin{itemize}
\item $ r_{n}=a_{0}n^{2}$ , $ n=1$ , $ 2$ , $ 3$ , ...
\end{itemize}
\item (i) Show that the possible energy levels (in Joules) for the hydrogen atom are given by the formula:
 \begin{itemize}
\item $ E_{n}=-me^{4}/(8h^{2}\epsilon _{0}^{2}$ * $ 1/n^{2}$
\item where $ m=$ mass of the electron
\item $ e=$ electronic charge
\item $ h=$ Planck's constant
\item $ \epsilon _{0}=$ permittivity constant of vacuum
\item What does the negative sign signify in the formula for $ E$ , in above?
\end{itemize}
\item Define the term “binding energy” of a nuclide.
\item Distinguish between:
 \begin{itemize}
\item $ \beta -$ decay and $ \beta +$ decay.
\item nuclear fission and nuclear fusion
\item activity and half-life of a radioactive material.
\item Taking the half-life of Radium $ -226$ to be $ 1600$ years, what fraction of a given sample remains after $ 4800$ years?
\end{itemize}
\item Briefly describe the major factors that you would consider when designing a voltage amplifier.
\item Explain the term “thermal run away” as regards a transistor amplifier.
\item With the help of clear diagrams, explain how you would overcome thermal run away in a voltage amplifier.
\end{itemize}

\end{document}
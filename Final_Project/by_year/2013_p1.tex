
\documentclass{article}
\usepackage[a4paper, total={6in, 8in}]{geometry}
% \usepackage[utf8]{inputenc}
\usepackage{abstract}
\title{THE UNITED REPUBLIC OF TANZANIA

NATIONAL EXAMINATIONS COUNCIL

ADVANCED CERTIFICATE OF SECONDARY EDUCATION EXAMINATION

\textbf{2013 PHYSICS 1}}
\author{Transcribed by:  PJ Gibson}

\begin{document}

\maketitle

\begin{itemize}
\item What is the difference between degree of accuracy and precision.
\item In an experiment to determine Young's modulus of a wooden material the following measurements were recorded:
 \begin{itemize}
\item length $ l=80.0\pm 0.05$ cm 
\item breadth $ b=28.65\pm 0.03$ mm
\item thickness $ t=6.40\pm 0.03$ mm and
\item slope $ G=0.035\pm 0.001$ cm/gm
\item Given that the Young’s modulus $ Y$ is given by:
\item $ Y=(4/Gb)(l/t)^{3}$
\item Calculate the maximum percentage error in the value of $ Y$ .
\end{itemize}
\item Using the method of dimensions, indicate which of the following equations are dimensionally correct and which are not, given that, $ f=$ frequency, $ \gamma =$ surface tension, $ \rho =$ density, $ r=$ radius and $ k=$ dimensionless constant.
 \begin{itemize}
\item  $ \rho^{2}=k\sqrt{r^{3}f/\gamma }$
\item  $ f=(kr^{3}\sqrt{\gamma })/(\rho^{1/2})$
\item  $ f=(k\gamma^{1/2})/(\sqrt{\rho}r^{3/2})$
\end{itemize}
\item List down two main assumptions in deriving the equation of projectile motion.
\item Why the horizontal motion of a projectile constant? 
\item A ball is thrown horizontally with a speed of $ 14.0$ m$/$s from a point $ 6.4$ m above the ground, calculate:
 \begin{itemize}
\item The horizontal distance traveled in that time.
\item Its velocity when it reaches the ground.
\end{itemize}
\item A man stands in a lift which is being accelerated upwards at $ 3.2$ m$/$s$ ^{2}$ . If the man has a mass of $ 65$ kg, what is the net force exerted on the man by the floor of the lift?
\item Why is it technically advised to bank a road at corners?
\item A wheel rotates at a constant rate of $ 10$ revolutions per second. Calculate the centripetal acceleration at a distance of $ 0.80$ m from the centre of the wheel.
\item With the aid of a labeled diagram, sketch the possible orbits for a satellite launched from the earth.
 \begin{itemize}
\item From the diagram above, write down an expression for the velocity of a satellite corresponding to each orbit.
\end{itemize}
\item Distinguish surface tension from surface energy.
\item Explain the phenomenon of surface tension in terms of the molecular theory.
\item A clean open ended glass U-tube has vertical limbs one of which has a uniform internal diameter of $ 4.0$ mm and the other of $ 20.0$ mm. Mercury is poured into the tube; and observed that the height of mercury column in the two limbs ts different.
 \begin{itemize}
\item Explain this observation
\item Calculate the difference in levels
\end{itemize}
\item Name the temperature of a thermocouple at which the thermo,
 \begin{itemize}
\item e.m.f. changes its sign.
\item electric power becomes zero.
\end{itemize}
\item A person sitting on a bench on a calm hot summer day is aware of a cool breeze blowing from the sea. Briefly explain why there is a natural convection?
\item A Nichrome-coustantan thermocouple gives about $ 70$ $\mu$V for each $ 1^{\circ}$C difference in temperature between the junctions. If $ 100$ such thermocouples are made into a thermopile, what voltage is produced when the junctions are at $ 20^{\circ}$C and $ 240^{\circ}$C? 
\item A black body of temperature $ \theta $ is placed in a blackened enclosure maintained at a temperature of $ 100^{\circ}$C. When its temperature rises to $ 30^{\circ}$C the net rate of loss of energy from the body was found to be $ 10$ Watts. Find the power generated by the body at $ 50^{\circ}$C if the energy exchange takes place solely by the process of forced convection.
\item Compare the law governing the conduction of heat and electricity pointing out the corresponding quantities in each case.
\item Write down three laws governing the black body radiation.
\item A cup of tea kept in a room with temperature of $ 22^{\circ}$C cools from $ 66^{\circ}$C to $ 63^{\circ}$C in $ 1$ minute. How long will the same cup of tea take to cool from the temperature of $ 43^{\circ}$C to $ 40^{\circ}$C under the same condition?
\item A Lagged copper rod is uniformly heated by a passage of an electric current. Show by considering a small section dx that the temperature $ \theta $ varies with distance $ x$ along a rod in a way that, $ k\frac{d^{2}T}{dx^{2}}=-H$ , where $ k$ is a thermal conductivity and $ H$ is the rate of heat generation per unit volume.
\item Define the term standing wave.
\item State the position in a stationary wave where a man can hear a louder sound.
\item What is meant by dispersion of waves? 
\item Briefly explain if it is possible for dispersion to take place on a wave whose frequency lies in the audible range.
\item A small speaker emitting $ 4$ note of frequency $ 250$ Hz is placed over the open upper end of a vertical tube which is full of water. When the water is gradually run out of the tube the air column resonates. If the initial and final position of the water surface below the top are $ 0.31$ m and $ 0.998$ m respectively, calculate the speed of sound in air and the end-correction of the tube. 
\item What is meant by “power rating" as regards to a resistor?
 \begin{itemize}
\item Mention two distinct velocities of an electron in a wire.
\end{itemize}
\item A $ 20$ k$ \Omega$ resistor is to be connected across a potential difference of $ 300$ V Calculate the required power rating.
\item Explain the following observation:
 \begin{itemize}
\item Light in the bulb comes on once the switch is kept on despite the drift velocity of electrons being very low.
\item The potentiometer is said to be a better device for measuring the potential difference (p.d) than a moving coil voltmeter.
\end{itemize}
\item State the laws of electromagnetic induction.
\item Mention the factors which determine the magnitude and direction of the force experienced by a current-carrying conductor in a magnetic field.
\item Derive the formula for the torque acting of the rectangular current-carrying coil in a magnetic field
\item What is the maximum torque on a $ 400-$ turns circular coil of radius $ 0.75$ cm that carrying a current of $ 1.6$ mA and resides in a uniform magnetic field of $ 0.25$ T?
\item What is band theory?
\item How does the band theory explain electrical properties of solids?
\item In an intrinsic semiconductor, the energy gap $ E_{g}=1.2$ eV, and its hole mobility is very much smaller than electron mobility which is Independent of temperature. Assuming that the temperature dependence of intrinsic carrier concentration, $ n_{i}$ is expressed as:
 \begin{itemize}
\item $ N_{i}=n_{o}$ exp$ (-E_{g}/(K_{B}T))$ , where $ n_{o}$ and $ K_{B}$ are constants, $ T$ is temperature and $ E_{g}$ is an energy equal to $ E_{q}/2$ .  
\item What is the ratio between conductivity at $ 600$ K and that at $ 300$ K?
\item Comment on the result obtained above.
\end{itemize}
\item Describe Coulomb’s law and give the dimensions of each quantity.
\item Brielfly explain how you can demonstrate that there are two types of charges in nature.
\item Define electric potential.
\item A radioactive source in the form of metallic sphere of radius $ 1.0$ cm emits Beta particles at the rate of $ 5.0 \times 10^{10}$ particles per second.  If the source is electrically insulated, how long will it take for its electric potential to be raised by $ 2.0$ V? (assuming that $ 40\%$ of the emitted Beta-particles escape the source).
\item What is an electron microscope? 
\item Outline three disadvantages of electron microscope.
\item Draw a schematic diagram of an electron microscope showing its main parts.
 \begin{itemize}
\item Give the order of resolution of electron microscope in the question above.
\end{itemize}
\item Briefly explain why Cathode Ray Oscilloscope (C.R.O.) is said to be an excellent instrument for measuring the emf 
\item An electron gun fires electrons at the screen of a TV tube. The electrons start from rest and are accelerated through a potential difference of $ 30$ kV. What is the speed of impact of electrons on the screen of the picture tube?
\item Give comment on the statement that, an electron suffers no force when it moves parallel to the magnetic field, $ B$ .
\item A $ 10$ eV proton is circulating in a plane at right angles to a uniform magnetic field of magnetic flux density of $ 1.0 \times 10^{-4}$ Wb$/$m$ ^{2}$ Calculate the cyclotron frequency of a proton.
\item The main interior of the earth (core) is believed to be in molten form. What seismic evidence supports this belief?
\item Explain why the small ozone layer on the top of the stratosphere is crucial for human survival
\item Electrical properties of the atmosphere are significantly exhibited in the ionosphere.
 \begin{itemize}
\item  What is the layer composed of and what do you think is the origin of such constituents.
\item  Mention two uses of the ionosphere.
\end{itemize}
\item Briefly explain why long distance radio broadcasts make use of short wave bands.
\item Briefly explain on the following types of environmental pollution:
 \begin{itemize}
\item  Thermal pollution.
\item  Water pollution.
\end{itemize}
\item Describe the soil temperature with regard to agricultural physics which causes lower crop growth at a particular area.
\end{itemize}

\end{document}
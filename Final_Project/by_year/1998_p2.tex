
\documentclass{article}
\usepackage[a4paper, total={6in, 8in}]{geometry}
% \usepackage[utf8]{inputenc}
\usepackage{abstract}
\title{THE UNITED REPUBLIC OF TANZANIA

NATIONAL EXAMINATIONS COUNCIL

ADVANCED CERTIFICATE OF SECONDARY EDUCATION EXAMINATION

\textbf{1998 PHYSICS 2}}
\author{Transcribed by:  PJ Gibson}

\begin{document}

\maketitle

\begin{itemize}
\item Define simple harmonic motion.
\item Prove that, the velocity v of a particle moving in simple harmonic motion is given by: $ v=w(A^{2}-y^{2})^{0.5}$ , where A is the amplitude of oscillation, $ w$ the angular frequency and $ y$ the displacement from the mean position.
\item A simple pendulum has a period of $ 2.8$ seconds. When its length is shortened by $ 1.0$ metre, the period becomes $ 2.0$ seconds. From this information, determine the acceleration $ g$ , of gravity and the original length of the pendulum.
\item A particle rests on a horizontal platform which is moving vertically in simple harmonic motion with an amplitude of $ 50$ mm. Above a certain frequency the particle ceases to remain in contact with the platform throughout the motion. With a help of a diagram and illustrative equations, find;
 \begin{itemize}
\item the lowest frequency at which this situation occurs.
\item the position at which contact ceases.
\end{itemize}
\item What is terminal velocity?
\item Briefly explain an experiment designed to measure terminal velocity.
\item A small sphere of radius $ r$ and density $ \sigma $ is released from the bottom of a column of liquid of density $ \rho $ which is slightly higher than $ \sigma $ . Deduce expressions for;
 \begin{itemize}
\item the initial acceleration of the sphere.
\item the terminal velocity of the sphere.
\end{itemize}
\item Two equal drops of water are falling through air with a steady velocity of $ 0.15$ ms$ ^{-1}$ , If the drops coalesce, find their new terminal velocity.
\item State Newton's laws of motion.
\item Explain why a length of horse pipe which is lying in a curve on a smooth horizontal surface, straightens out when a fast flowing stream of water passes through it.
\item A ball of mass $ 0.4$ kg is dropped vertically from a height of $ 2.5$ m on to a horizontal table and bounces to a height of $ 1.5$ m.
 \begin{itemize}
\item Find the kinetic energy of the ball just before striking the table.
\item Find the kinetic energy just after impact.
\item Suggest reasons for the difference between these two values of kinetic energy.
\item What height would you expect the ball to reach after its next bounce from the table?
\end{itemize}
\item A jet of water flowing with a velocity of $ 20$ ms$ ^{-1}$ from a pipe of cross-sectional area, $ 5.0 \times 10^{-3}$ m$ ^{2}$ , strikes a wall at right angles and loses all its velocity.
 \begin{itemize}
\item What is the mass of water striking the wall per second?
\item What is the change in momentum per second of the water hitting the wall?
\item What is the force exerted on the wall?
\end{itemize}
\item What is a diffraction grating?
\item A diffraction grating has $ 5000$ lines per centimetre. At what angles will bright diffraction images be observed, if it is used with monochromatic light of wavelength $ 6.0 \times 10^{-7}$ m at normal incidence?
\item A lamp emits two wavelengths, $ 4.2 \times 10^{-7}$ m and $ 6.0 \times 10^{-7}$ m. Find the angular separation of these two waves in the third order diffraction pattern produced by a diffraction grating having $ 4000$ lines per centimetre, when light is at normal incidence on the grating?
\item A girl is holding a metal rod in her hand and rubs its surface with fur. Explain what happens to the rod.
\item Can charge be conserved? Give at least two examples to support your answer.
\item The distance between the electron and proton in the hydrogen atom is about $ 5.3 \times 10^{11}$ m. Calculate the electrical and gravitational forces between these particles. How do they compare?
\item A capacitor of capacitance $ 3$ micro$ -F$ is charged until a potential difference of $ 200$ V is developed across its plates. Another capacitor of capacitance $ 2$ micro$ -F$ developed a p.d. of $ 100$ V across its plates on being charged.
 \begin{itemize}
\item What is the energy stored in each capacitor?
\item The capacitors are then connected by wires of negligible resistance, so that the plates carrying like charges are connected together. What is the total energy stored in the combined capacitors?
\item What would the time constant of the circuit be, if the resistance of each wire connecting the plates was $ 10\Omega $ ?
\end{itemize}
\item Define the term self inductance for a coil.
\item Give the S.I units of self inductance.
\item Derive an expression for the coefficient of self induction of a uniformly wound solenoid; of length $ 1$ , cross-sectional area A having $ N$ turns in air.
\item Two coils $ A$ and $ B$ have $ 200$ and $ 800$ turns respectively. A current of $ 2$ amperes in A produces a magnetic flux of $ 1.8 \times 10^{-4}$ Wb in each turn of $ B$ . Compute:
 \begin{itemize}
\item the mutual inductance.
\item the magnetic flux through A when there is a current of $ 4.0$ amperes in $ B$ and
\item the emf induced in $ B$ when the current in A changes from $ 3$ amperes to $ 1$ ampere in $ 0.2$ seconds.
\end{itemize}
\item Describe and explain briefly a method for measuring the specific charge. Mention the errors expected in this method.
\item An electron is projected horizontally with a velocity of $ 2.0 \times 10^{6}$ ms$ ^{-1}$ into a large evacuated enclosure. A magnetic field which has a flux density of $ 15 \times 10^{-4}$ tesla is directed vertically downwards throughout the enclosure. Find
 \begin{itemize}
\item the radius of curvature of the electron's path.
\item how many complete loops must the electron describe before it falls by $ 1.0$ cm under the influence of gravity?
\item What would be the effect of changing the direction of the magnetic field to upwards?
\end{itemize}
\item What is thermionic emission?
\item Describe the function of each of;
 \begin{itemize}
\item the electron gun
\item the deflection system and
\item the display system of the Cathode ray Oscilloscope.
\end{itemize}
\item Sketch the traces seen on the screen of a cathode ray oscilloscope when two sinusoidal potential differences of the same frequency — and amplitude are applied simultaneously to $ X$ and $ Y$ plates of  a cathode ray oscilloscope, when the phase difference between them is:
 \begin{itemize}
\item $ 0^{\circ}$ $ 45^{\circ}$ $ 90^{\circ}$ .
\end{itemize}
\item Explain the terms: atomic mass unit, mass defect, packing fraction and binding energy.
\item Discuss carbon dating.
\item Find the age at death of an organism, if the ratio of amount of C$ 14$ at death to that of the present time is $ 10^{8}$ and that the half life of Cl$ 4$ is $ 5600$ years.
\item Explain the following terms: Earthquake, Earthquake focus, Epicentre and Body waves.
\item List down three $ (3)$ sources of earthquakes.
\item Define ionosphere.
\item Mention the ionospheric layers that exist during the day time.
\item Give the reason for better reception of radio waves for high Frequency signals at night than during the day time.
\item Explain briefly three different types of radio waves travelling from a transmitting station to a receiving antenna.
\end{itemize}

\end{document}
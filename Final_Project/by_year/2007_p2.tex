
\documentclass{article}
\usepackage[a4paper, total={6in, 8in}]{geometry}
% \usepackage[utf8]{inputenc}
\usepackage{abstract}
\title{THE UNITED REPUBLIC OF TANZANIA

NATIONAL EXAMINATIONS COUNCIL

ADVANCED CERTIFICATE OF SECONDARY EDUCATION EXAMINATION

\textbf{2007 PHYSICS 2}}
\author{Transcribed by:  PJ Gibson}

\begin{document}

\maketitle

\begin{itemize}
\item With the aid of a diagram describe a simple laboratory experiment to measure Young’s modulus of a wooden bar acting as a loaded cantilever from its period of vibration given that the depression s is given by $ S=(WL^{3})/(3IE)$ . 
\item Two small spheres each of mass $ 10g$ are attached to a light rod $ 50$ cm long. The system Is set into oscillation and the period of torsional oscillation is found to be $ 770$ seconds. To produce maximum torsion to the system two large spheres each of mass $ 10$ kg are placed near each suspended sphere, if the angular deflection of the suspended rod Is $ 3.96 \times 10^{-3}$ rad. and the distance between the centres of the large spheres and small spheres is $ 10$ cm, determine the value of the universal constant of gravitation, $ G$ , from the given information. 
\item Write the Continuity and Bernoullis’ equations as applied to fluid dynamics. 
\item Under what conditions is the Bernoullis’ equation applicable?
\item Discuss two $ (2)$ applications of the Bernoullis equation. 
\item Develop an equation to determine the velocity of a fluid in a venture meter pipe.
 \begin{itemize}
\item What amount of fluid passes through a section at any given time? 
\end{itemize}
\item Differentiate between tensile and shear stress. 
\item A lift is designed to hold a maximum of $ 12$ people. The lift cage has a mass of $ 500$ kg and the distance from the top floor of the building to the ground floor is $ 50$ m.
 \begin{itemize}
\item What minimum cross-sectional area should the cable have in order to support the lift and the people in it?
\item Why should the cable have to be thicker than the minimum cross-sectional area above in practice? 
\item How much will the lift cable above stretch if $ 10$ people get into the lift at the ground floor, assuming that the lift cable has a cross section of $ 1.36$ cm? 
\item Note: Mass of an average person $ =70$ kg . $ E_{steel}=2 \times 10^{11}$ N$/$m$^{2}$ , Tensile strength of steel $ =4 \times 10^{11}$ N$/$m$^{2}$ .
\end{itemize}
\item State and define Newton’s 2nd law of motion with respect to angular motion. 
\item The $ T$ is then suspended from the free end of rod $ Y$ and the pendulum swings in the plane of $ T$ about the axis Of rotation.
 \begin{itemize}
\item Calculate the moment of inertia i of the $ T$ about the axis of rotation. 
\item Obtain the expression for the k.e. and p.e. in terms of the angle $ \theta $ of inclination to the vertical oscillation of the pendulum. 
\item Show that the period of oscillation is $ 2\pi\sqrt{17L/18g}$ . 
\item ( Moment of inertia of a thin rod about its centre $ I_{C}=mL^{2}/12$ . )
\end{itemize}
\item On the basis of Newton’s universal law of gravitation, derive Kepler’s third law of planetary motion. 
\item A planet has half the density of earth but twice its radius. What will be the speed of a satellite moving fast past the surface of the planet which has on no atmosphere?
 \begin{itemize}
\item ( Radius of earth $ R_{E}=6.4 \times 10^{3}$ km and gravitational potential energy $ g_{E}=9.81$ N$/$kg )
\end{itemize}
\item Define an ideal gas.
\item State the four $ (4)$ assumptions necessary for an ideal gas that are used to develop the expression $ p=$ ½ $ \rho C^{2}$ .
\item How is pressure explained in terms of the kinetic theory? 
\item Without a detailed mathematical analysis argue the steps to follow in deriving the relation $ p=$ ½ $ \rho C^{2}$ .
\item Define the temperature of an ideal gas as a consequence of the kinetic theory.
\item A certain diatomic gas is contained in a vessel whose inner surface is a small absorber which retains any atoms or molecules of gas which strike it.  Show that if doubling the absolute temperature causes one half of the molecules to dissociate into atoms then the rate at which the absorber is gaining mass increases by a factor $ 1+1/\sqrt{2}$ .
\item A mole of an ideal gas at $ 300K$ is subjected to a pressure of $ 10^{5}N/m^{2}$ and its volume is $ 2.5 \times 10^{-2}m^{3}$ .  Calculate the:
 \begin{itemize}
\item molar gas constant $ R$
\item Boltzmann constant $ k$
\item average transnational kinetic energy of a molecule of the gas.
\end{itemize}
\item State the expression for the $ 1$ st law of thermodynamics.
\item What do you understand by the terms:
 \begin{itemize}
\item critical temperature? 
\item adiabatic change?
\end{itemize}
\item Find the number of molecules and their mean kinetic energy for a cylinder of volume $ 5 \times 10^{-4}m^{3}$ containing oxygen at a pressure of $ 2 \times 10^{5}$ Pa and a temperature of $ 300K$ . 
 \begin{itemize}
\item When the gas is compressed adiabatically to a volume of $ 2 \times 10^{-4}m^{3}$ , the temperature rises to $ 434K$ . Determine $ \gamma $ , the ratio of the principal heat capacities.
\item [ Molar gas constant $ R=8.31$ J$/$mol$/$K ,$ N  =6 \times 10^{32}$ mol$^{-1}$ ] 
\end{itemize}
\item Describe briefly the formation of Newton rings. How would you measure the wavelength of yellow light by use of Newton’s rings? 
\item What would happen to the central spot when air rests between the lens and the plate of the apparatus for Newton’s rings? 
\item What is meant by Doppler effect? 
\item Mention two $ (2)$ common applications of the Doppler shift. 
\item Ultra sound of frequency $ 5 \times 10^{6}$ Hz is incident at an angle of $ 30^{\circ}$ to the blood vessel of a patient and a doppler shift of $ 4.5$ KHz is observed. If the blood vessel has a diameter $ 10^{-3}m$ and the velocity of ultrasound is $ 1.5 \times 10^{3}$  $ m/s$ . Calculate the:
 \begin{itemize}
\item blood flow velocity. 
\item volume rate of blood flow. 
\end{itemize}
\item State Rayleigh’s criterion for the resolution of two objects. 
\item The diameter of the pupil of the human eye is $ 2$ mm in bright light.
 \begin{itemize}
\item What is its resolving power with light of wavelength lamda $ =5 \times 10^{-7}m$ ? 
\item Would it be possible to resolve two large birds $ 30$ cm apart sitting on a wire$ 1.5 \times 10^{3}m$ away at daytime? 
\item What would the situation be at night when the pupil dilates to $ 4$ mm? 
\end{itemize}
\item State Faraday’s two $ (2)$ laws of electrolysis and calculate the value of Faradays constant given that the e.c.e. of copper is $ 3.30 \times 10^{-7}$ kg$/C$ and the copper is a divalent element. 
\item Discuss two $ (2)$ harmful effects of electrolysis. 
\item What is meant by the back e.m.f. (polarization potential) in a water voltameter? 
\item Develop an expression for electrical energy spent in the decomposition of water. 
\item A piece of metal weighing $ 200g$ is to be electroplated with $ 5\%$ of its weight in gold. If the strength of the available current is $ 2$ A, how long would it take to deposit the required amount of gold?
\item State the main differences between.
 \begin{itemize}
\item diamagnetism and paramagnetism. 
\item ferromagnetism and auntiferromagnetism. 
\item ferromagnetism and ferrielectricity. 
\end{itemize}
\item Draw hysteresis loops diagrams for soft iron and hard steel and use them to discuss:
 \begin{itemize}
\item permanent magnets.
\item electromagnets.
\item transformer cores. 
\end{itemize}
\item State Faraday’s law of electromagnetic induction. 
\item A coil of cross section area A rotates with an angular velocity $ \omega $ in a uniform. magnetic field, $ B$ . Derive the equation for induced e.m.f. of the system.
\item A coil having $ 475$ turns and cross sectional area $ 20 cm^{2}$ , rotates at $ 600r.p.m$ . in a uniform magnetic field of $ 0.01$ T. Find:
 \begin{itemize}
\item the peak e.m.f and the r.m.s. e.m.f induced in the coil. 
\item show these values on a graph of $ E$ vs time. 
\end{itemize}
\item Explain the terms output saturation and negative feedback as applied to op-amplifiers. 
\item For an ideal operational amplifier, what are the values of the:
 \begin{itemize}
\item current into both inputs of the op-amp? 
\item voltage between the inputs if the output is not saturated? 
\end{itemize}
\item What is a non-inverting amplifier? 
\item It is not possible to separate the different isotopes of an element by chemical means.  Explain.
\item Define a mass spectrometer. 
\item Ion A of mass $ 24$ and charge $ +e$ and ion $ B$ of mass $ 22$ and charge $ +2e$ both enter the magnetic field of a mass spectrometer with the same speed. If the radius of A is $ 2.5 \times 10^{-1}m$ , calculate the radius of the circular path of $ B$ . 
\item If the ratio of mass of lead – $ 206$  to mass of uranium – $ 238$ in a certain rock was found to be $ 0.45$ and that the rock originally contained no lead – $ 206$ , estimate the age of the rock given that the half life of uranium – $ 238$ is $ 4.5 \times 10^{9}$ years.
\item Explain breifly the action of a helium-neon laser.
\item Define the following terms:
 \begin{itemize}
\item Atomic mass unit
\item Binding energy
\item Mass defect.
\end{itemize}
\item In a hydrogen atom model an electron of mass $ m$ and charge $ e$ rotates about a heavy nucleus of charge $ e$ in a circular orbit of radius $ r$ . Develop an expression for the angular momentum of the electron in terms of $ m$ , $ e$ , $ r$ , $ \pi$ and $ \epsilon  _{0}-$ the permitting of free space.
\item What is a line spectrum? 
\item The four lowest energy levels in a mercury atom are $ -10.4$ eV, $ -5.5$ eV, $ -3.7$ eV and $ -1.6$ eV.
 \begin{itemize}
\item Determine the ionization energy of mercury in joules. 
\item Calculate the wavelength of the radiation emitted when an electron jumps from $ -1.6$ eV to $ -5.5$ eV energy levels. 
\item What will happen if a mercury atom in its excited state is bombarded with electrons having an energy of $ 11$ eV. 
\end{itemize}
\end{itemize}

\end{document}

\documentclass{article}
\usepackage[a4paper, total={6in, 8in}]{geometry}
% \usepackage[utf8]{inputenc}
\usepackage{abstract}
\title{THE UNITED REPUBLIC OF TANZANIA

NATIONAL EXAMINATIONS COUNCIL

ADVANCED CERTIFICATE OF SECONDARY EDUCATION EXAMINATION

\textbf{2009 PHYSICS 2}}
\author{Transcribed by:  PJ Gibson}

\begin{document}

\maketitle

\begin{itemize}
\item Define the following terms:
 \begin{itemize}
\item Tensile stress
\item Tensile strain
\item Young’s modulus
\end{itemize}
\item Derive the expression for the work done in stretching a wire of length $ L$ by a load $ W$ through an extension $ X$ .
\item A vertical wire made of steel of length $ 2.0$ m and $ 1.0$ mm diameter has a load of $ 5.0$ kg applied to its lower end.  What is the energy stored in the wire?
\item A copper wire $ 2.0$ m long and $ 1.22 \times 10^{-3}$ m diameter is fixed horizontally to two rigid supports $ 2.0$ m apart.  Find the mass in kg of the load, which when suspended at the mid point of the wire, produces a sag of $ 2.0 \times 10^{-2}$ m at the point.
\item Define angular momentum and give its dimensions.
\item A grinding wheel in a form of solid cylinder of $ 0.2$ m diameter and $ 3$ kg mass is rotated at $ 3600$ rev/minute.
 \begin{itemize}
\item What is its kinetic energy?
\item Find how far it would have to fall to acquire the same kinetic energy as in the question above.
\end{itemize}
\item What is the difference between isothermal and adiabatic processes?
 \begin{itemize}
\item Write down the equation of state obeyed by each process in the question above.
\end{itemize}
\item Using the same graph and under the same conditions sketch the isotherms and the adiabatics.
\item Derive the expression for the work done by the gas when it expands from volume $ V_{1}$ to volume $ V_{2}$ during an:
 \begin{itemize}
\item Isothermal process
\item Adiabatic process
\end{itemize}
\item When water is boiled under a pressure of $ 2$ atmospheres the boiling point is $ 120^{\circ}$C. At this pressure $ 1$ kg of water has a volume of $ 10^{-3}$ m$ ^{3}$ and $ 2$ kg of steam have a volume of $ 1.648$ m$ ^{3}$ . Compute the work done when $ 1$ kg of steam is formed at this temperature increase in the internal energy. 
\item State Kepler's laws of planetary motion.
\item Explain the variation of acceleration due to gravity, $ g$ . inside and outside the earth.
\item Derive the formula for mass and density of the earth.
\item What do you understand by the term satellite?
\item A satellite of mass $ 100$ kg moves in a circular orbit of radius $ 7000$ km around the earth, assumed to be a sphere of radius $ 6400$ km.  Calculate the total energy needed to place the satellite in orbit from the earth assuming $ g=10$ N$/$kg at the earth’s surface.
\item What is interference?  Explain the term path difference with reference to the interference of two wave-trains.
\item Why is it not possible to see interference when the light beams from head lamps of a car overlap?
\item Discuss whether it is possible to observe an interference pattern when white light is shone on a Young’s double slit experiment.
\item A grating has $ 500$ lines per millimetre and is illuminated normally with monochromatic light of wavelength $ 5.89 \times 10^{-7}$ m.
 \begin{itemize}
\item How many diffraction maxima may be observed?
\item Calculate the angular separation.
\end{itemize}
\item Explain the mechanism of electric conduction in:
 \begin{itemize}
\item Gases
\item Electrolytes
\end{itemize}
\item Develop an equation for the torque acting on a current carrying coil of dimensions lxb placed in a magnetic field.  How is this effect applied in a moving coil galvanometer?
\item A galvanometer coil has $ 50$ turns, each with an area of $ 1.0 $ cm$ ^{2}$ .  If the coil is in a radian field of $ 10^{-2}$ T and suspended by a suspension of torsion constant $ 2 \times 10^{-9}$ Nm per degree, what current is needed to give a deflection of $ 30^{\circ}$ ?
\item Explain the following terms:
 \begin{itemize}
\item Forward bias.
\item Reverse bias.
\item Inverting and non-inverting amplifier. 
\end{itemize}
\item Define the following:
 \begin{itemize}
\item Logic gate.
\item Integrated circuit.
\item Modulation.
\end{itemize}
\item An operational amplifier is to have a voltage gain of $ 100$ .  Calculate the required values for the external resistances $ R_{1}$ and $ R_{2}$ when the following gains are required:
 \begin{itemize}
\item non-inverting.
\item Inverting.
\end{itemize}
\item State the laws of electromagnetic induction.
\item Outline four applications of eddy currents.
\item A coil of $ 100$ turns is rotated at $ 1500$ revolutions per minute in a magnetic field of uniform density $ 0.05$ T.  If the axis of rotation is at right angles to the direction of the flux and the area per turn is $ 4000 $ mm$ ^{2}$ .  Calculate the:
 \begin{itemize}
\item Frequency
\item Period
\item Maximum induced e.m.f.
\item Maximum value of the induced e.m.f. when the coil has rotated through $ 30^{\circ}$ from the position of zero e.m.f.
\end{itemize}
\item Give a general form expressing the force exerted on the wire carrying current i if its length $ l$ is inclined at angle angle $ \theta $ to the magnetic field $ B$ .  
\item A wire carrying a current of $ 2$ A has a length of $ 100$ mm in a uniform magnetic field of $ 0.8$ Wb$/$m$ ^{2}$ .  Find the force acting on the wire when the field is at $ 60^{\circ}$ to the wire.
\item A wire carrying a current of $ 25$ A and $ 8$ m long is placed in a magnetic field of flux density $ 0.42$ T . What is the force on the wire if it is placed:
 \begin{itemize}
\item At right angles to the field?
\item At $ 45^{\circ}$ to the field?
\item Along the field?
\end{itemize}
\item Write down Bragg’s equation for the study of the atomic structure of the crystals by $ X-$ rays.
\item The radiation from an $ X$ — ray tube which operates at $ 50$ kV is diffracted by is diffracted by a cubic KCl crystal of molecular mass $ 74.6$ and density $ 1.99 \times 10^{3}$ kg$/$m$ ^{3}$ .  Calculate:
 \begin{itemize}
\item The shortest wavelength limit of the spectrum from the tube.
\item The glancing angle for first order reflection from the planes of the crystal for that wavelength and angle of deviation of a diffracted beam.
\end{itemize}
\item The radiation emitted by an $ X$ — ray tube consists of continuous spectrum with a line spectrum superimposed on it. Explain how the continuous spectrum and the line spectrum are produced.
 \begin{itemize}
\item Draw the graph of the spectra stated. ‘
\end{itemize}
\item Explain the following observations:
 \begin{itemize}
\item A radioactive source is placed in front of a detector which can detect all forms of radioactive emissions. It is found that the activity registered as noticeably reduced when a thin sheet of paper is placed between the source and detector.
\item When a brass plate with a narrow vertical shit is placed in front of the radioactive source (above) and a horizontal: magnetic field normal to the line joining the source and the detector is applied, its found that the activity is further reduced.
\item The magnetic field (above) is removed and a sheet of aluminum is placed in front of the source. The activity recorded is similarly reduced.
\end{itemize}
\item Define the terms laser and maser. 
\item Give three applications of laser. 
\item A laser beam has a power of $ 20 \times 10^{9}$ watts and a diameter of $ 2$ mm.  Calculate the peak values of electric field and magnetic fields.
\item A $ 2.71$ g sample of Kcl from the chemistry stock is found to be radioactive and decays at a constant rate of $ 4490$ disintegrations per second.  The decays are traced to the element potassium and in particular to the isotope $ ^{40}$ K which constitutes $ 1.17\%$ of normal potassium.  Calculate the half life of the nuclide.
\end{itemize}

\end{document}
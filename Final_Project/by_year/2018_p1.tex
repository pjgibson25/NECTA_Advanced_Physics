
\documentclass{article}
\usepackage[a4paper, total={6in, 8in}]{geometry}
% \usepackage[utf8]{inputenc}
\usepackage{abstract}
\title{THE UNITED REPUBLIC OF TANZANIA

NATIONAL EXAMINATIONS COUNCIL

ADVANCED CERTIFICATE OF SECONDARY EDUCATION EXAMINATION

\textbf{2018 PHYSICS 1}}
\author{Transcribed by:  PJ Gibson}

\begin{document}

\maketitle

\begin{itemize}
\item How can random and Systematic errors be minimized during an experiment?
\item Estimate the precision to which the Young’s modulus, $ \gamma $ of the wire can be determined from the formula $ \gamma =(4Fl)/(\pi d^{2} e)$ , given that the applied tension, $ F=500$ N, the length of the loaded wire,  $ l=3$ m, the diameter of the wire, $ d=1$ mm, the extension of the wire, $ e=5$ mm and the errors associated with these quantities are $ 0.5$ N, $ 2$ mm, $ 0.01$ mm and $ 0.1$ mm respectively. 
\item State the law of dimensional analysis. 
\item If the speed v of the transverse wave along a wire of tension, $ T$ and mass, $ m$ is given by, $ V=\sqrt{T/m}$ .  Apply dimensional analysis to check whether the given expression is correct or not.  
\item Under what condition a passenger in a lift feels weightless? 
\item Calculate the tension in the supporting cable of an elevator of mass $ 500$ kg which was originally moving downwards at $ 4$ m$/$s and brought to rest with constant acceleration at a distance of $ 20$ m. 
\item The rotating blades of a hovering helicopter swept out an area of radius $ 2$ m imparting a downward velocity of $ 8$ m$/$s of the air displaced. Find the mass of a helicopter. 
\item Compute the mass of water striking the wall per second when a jet of water with a velocity of $ 5$ m$/$s and cross-sectional area of $ 3 \times 10^{-2}$ m$ ^{2}$ strikes the wall at right angle losing its velocity to zero. 
\item How does projectile motion differ from uniform circular motion? 
\item A rifle shoots a bullet with a muzzle velocity of $ 1000$ m$/$s at a small target $ 200$ m away. How high above the target must the rifle be aimed so that the bullet will hit the target? 
 \begin{itemize}
\item Where does the object strike the ground when thrown horizontally with a velocity of $ 15$ m$/$s from the top of a $ 40$ m high building? 
\item Find the speed of travel when a man jumps a maximum horizontal distance of $ 1$ m spending a minimum time on the ground.
\end{itemize}
\item What is meant by the following terms as used in simple harmonic motion (S.H.M)?
 \begin{itemize}
\item Periodic motion. 
\item Oscillatory motion. 
\end{itemize}
\item List four important properties of a particle executing simple harmonic motion (S.H.M). 
\item Sketch a labeled graph that represents the total energy of a particle executing simple harmonic motion (S.H.M). 
\item The periodic time of a body executing S.H.M is $ 4$ seconds. How much time interval from time, $ t=0$ will its displacement be half its amplitude? 
\item A satellite of mass $ 600$ kg is in a circular orbit at a height $ 2 \times 10^{6}$ km above the earth’s surface. Determine the:
 \begin{itemize}
\item Orbital speed. 
\item Gravitational potential energy. 
\end{itemize}
\item What would happen if gravity suddenly disappears?  
\item Two base of a mountain are at sea level where the gravitational field strength is $ 9.81$ N$/$kg . If the value of gravitational field at the top of the mountain is $ 9.7$ N$/$kg, calculate the height of the mountain above the sea level. 
\item Why is flywheel designed such that most of its mass is concentrated at the rim? Briefly explain. 
\item Estimate the couple that will bring the wheel to rest in $ 10$ seconds when a grinding wheel of radius $ 40$ cm and mass $ 3$ kg is rotating at $ 3600$ revolutions per minute. 
\item Why an ice skater rotates at relatively low speed when stretches her arms and a leg outward? 
\item Calculate the moment of inertia of a sphere about an axis which is a tangent to its surface given that the mass and radius of the sphere are $ 10$ kg and $ 0.2$ m respectively. 
\item Which type of thermometer is most suitable for calibration of other thermometers? 
\item Why at least two fixed points are required to define a temperature scale?
\item List two qualities which makes a particular property suitable for use in practical thermometers. 
\item Describe how mercury in glass thermometer could be made sensitive.
\item What is meant by triple point of water? 
\item Evaluate the temperature in Kelvin if the pressure recorded by a constant volume gas thermometer is $ 6.8 \times 10^{4}$ Nm$ ^{-2}$ given that the pressure at triple point $ 273.16$ K is $ 4.6 \times 10^{4}$ Nm$ ^{-2}$ .
\item One gram of water becomes $ 1671 $ cm$ ^{3}$ of steam at a pressure of $ 1$ atmosphere. If the latent heat of vaporization at this pressure is $ 2256$ J$/$g, determine the:
 \begin{itemize}
\item external work done. 
\item increase in internal energy 
\end{itemize}
\item Why during emission of radiations from black body its temperature does not  reach zero Kelvin? 
\item A black ball of radius $ 1$ m is maintained at a temperature of $ 30^{\circ}$C . How much  heat is radiated by the ball in $ 4$ seconds? 
\item What do you understand by the term node as applied to electric circuit?
\item Outline three important points which are usually referred as sign convection in  solving Kirchhoff’s second law problems. 
\item How is ohmic conductor differ from non-ohmic conductor? Give one example in each case. 
\item Why the emf of a cell is sometimes called a special terminal potential difference? 
\item Calculate the current flowing in the circuit when three similar cells each of emf $ 1.5$ V and internal resistance $ 0.3\Omega $ are connected in parallel across a $ 2\Omega $ resistor. 
\item Mention four types of energy losses suffered by a transformer.  
\item Why choke coil is preferred over resistance to control alternating current?
\item Identify two difficulties which would arise when two straight wires are used to transmit electricity direct from the source to the city station. 
\item Explain what could be done to light a $ 30$ V bulb from a $ 220$ volt A.C. supply?
\item A series LCR circuit with inductance, $ L=0.12H$ , capacitance, $ C=480$ nF and resistance, $ R=23\Omega $ is connected to a $ 230V$ variable frequency supply. Determine the:
 \begin{itemize}
\item Maximum current flowing in the circuit. 
\item Source frequency for which the current is maximum. 
\end{itemize}
\item List two chief properties of semiconductors. 
\item Why is it easier to establish the current in a semiconductor than in an insulator?
\item State a condition that could be employed to make an insulator conduct some electricity. 
\item Distinguish between conductors and semiconductors on the basis of their energy band structures. 
\item What is meant by depletion layer as used in pn -junction device? 
\item Describe the effect of applying a reverse bias to the junction diode. 
\item Sketch the graph of transfer characteristic of a transistor. 
 \begin{itemize}
\item State the significance of the slope from the graph above.
\end{itemize}
\item What is the basic condition for a transistor to operate properly as an amplifier? 
\item Briefly explain how a junction transistor can be connected to act as a current operated device. 
\item Why the magnitude of output frequency of a full wave rectifier is twice the input frequency? 
\item Draw a simple basic transistor switching circuit diagram. 
\item What is meant by a logic gate? 
\item List three basic logic gates that make up all digital circuits. 
\item What is meant by solar constant? 
\item List two factors on which the solar constant depends. 
\item Give two advantages of photovoltaic system. 
\item Briefly explain how photovoltaic cells work. 
\item Estimate the maximum power available from $ 10$ m$ ^{2}$ of solar panels.  Calculate the volume of water per second which must pass through if the inlet and outlet temperature of the panels are at $ 10^{\circ}$C and $ 60^{\circ}$C respectively. (Assume the wave carries away energy at the same rate as the maximum power available)
\end{itemize}

\end{document}
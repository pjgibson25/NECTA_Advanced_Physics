
\documentclass{article}
\usepackage[a4paper, total={6in, 8in}]{geometry}
% \usepackage[utf8]{inputenc}
\usepackage{abstract}
\title{THE UNITED REPUBLIC OF TANZANIA

NATIONAL EXAMINATIONS COUNCIL

ADVANCED CERTIFICATE OF SECONDARY EDUCATION EXAMINATION

\textbf{2013 PHYSICS 2}}
\author{Transcribed by:  PJ Gibson}

\begin{document}

\maketitle

\begin{itemize}
\item Write down the Poiscuille’s equation for a viscous fluid flowing through a tube defining all the symbols.
 \begin{itemize}
\item What assumptions are used to develop the equation above. 
\end{itemize}
\item What is meant by Newtonian fluid? 
\item A submarine model is situated in a part of a tube with diameter $ 5.1$ cm where water moves at $ 2.4$ m$/$s.  Determine the:
 \begin{itemize}
\item velocity of flow in the water supply pipe of diameter $ 25.4$ cm. 
\item pressure difference between the narrow and the wide tube. 
\end{itemize}
\item Define comprehensibility of a gas in terms of the elasticity of gases. 
\item The bulk modulus of elasticity for lead is $ 8 \times 10^{9}$ N$/$m$ ^{2}$ . Find the density of lead if the pressure applied is $ 2 \times 10^{8}$ N$/$m$ ^{2}$ . 
\item Define the terms: proportional limit, elastic limit, yield point and elasticity.
\item Use a sketch graph to show how the extension of the wire varies with the applied force and mark the elastic limit and yield point on it. Explain how the magnitude of the Young's modulus is obtained from the graph.
 \begin{itemize}
\item A block of metal weighing $ 20$ N with a volume of $ 8 \times 10^{-4}$ m? is completely immersed
\item in oil of density $ 700$ kg$/$m$ ^{3}$ then attached to one end of a vertical wire of length $ 4.0$ m and diameter of $ 0.6$ mm whose other end is fixed. If the length of the wire is increased by $ 1.0$ mm. find the:
\item young’s modulus of the wire. 
\item energy stored in the wire. 
\end{itemize}
\item A rubber cord of a $ Y-$ shaped object has a cross sectional area of $ 4 \times 10^{-6}$ m$ ^{2}$ ? And relaxation length of $ 100$ mm. If the arms of the catapult are $ 70$ mm apart, calculate the: 
 \begin{itemize}
\item tension in the rubber. 
\item force required to stretch it when the rubber cord is pulled back until its length doubles. 
\end{itemize}
\item Briefly give comments on the following observations:
 \begin{itemize}
\item Polyatomic and diatomic gases have larger molar heat capacities than monatomic gases. 
\item  Cubical container is used for the derivation of pressure of an ideal gas.
\end{itemize}
\item What is meant by a gas constant. 
\item Helium gas occupies a volume of $ 4 \times 10^{-2}$ m$ ^{3}$ at a pressure of $ 2 \times 10^{5}$ Pa and temperature of $ 300$ K. Calculate the mass of helium and the r.m.s speed of its molecules.
\item When a gas expand adiabatically it does work on its surroundings although there is no heat input to the gas. Explain where this energy is coming from.
\item An ideal gas at $ 17^{\circ}$C and $ 750$ mmHg is compressed isothermally Until its volume is reached to ¾ of its initial value If it then allowed to expand adiabatically to a volume of $ 20\%$ greater than its original value. calculate the final temperature and pressure of the gas. 
\item How does the first law of thermodynamics change under isothermal and adiabatic processes? 
\item Show that the specific heat capacities of an ideal gas are related by the relation $ C_{p}=C_{v}+nR$ .
 \begin{itemize}
\item Explain the meaning of all the symbols used in the equation above.
\end{itemize}
\item One mole of an ideal monatomic gas is heated at constant volume from the temperature of $ 300$ K to $ 600$ K. Calculate the:
 \begin{itemize}
\item amount of heat added 
\item work done by the gas 
\item change in its infernal energy
\end{itemize}
\item The piston of a bicycle pump at room temperature of $ 290$ K is slowly moved in until the volume of air enclosed is one — fifth of the total volume of the pump. The outlet is then sealed and the piston suddenly drawn out to full extension. If no air passes the piston, find the temperature of the air in the pump immediately after withdrawing the piston, assuming that air ts an ideal gas with cryoscopic constant, $ \gamma =1.4$ .
\item What is meant by crossed polaroids? 
\item Briefly describe the appearance of fringes produced by monochromatic fight.
\item Give two difference between diffracting grating spectra and prism spectra.
\item A diffraction grating used at normal incidence gives a yellow line. $ \lambda =5750$ A in a certain spectral order: superimposed on a blue line, $ \lambda =4600$ A of the next higher order, If the angle of diffraction is $ 30^{\circ}$ , what is the spacing between the grating lines? 
\item State Huygens principle of wave construction. 
\item A thin wedge of air of small angle ts enclosed by two thin glass plates. When the plates are illuminated by a parallel beam of monochromatic light of wavelength $ 589$ nm, the distance apart of the fringes is $ 0.8$ mm. Calculate the angle of the wedge. 
\item Explain why it is better to use a small current for a long time to plate a metal with a given thickness of silver than using a larger current for a short time? 
\item Give four difference between the passage of electricity through metals and  ionized solution.
\item Define electric discharge and give one example.
\item A milliameter connected in series with a hydrogen discharge tube indicates a current of $ 1.0 \times 10^{-3}$ A. If the number of electrons passing the cross section of the tube at a particular point is $ 4.0 \times 10^{15}$ per second, find the number of protons that pass the same cross section per second. 
\item A silver and copper voltammeter are connected in parallel across a $ 6$ V battery of negligible internal resistance. In half an hour $ 1.0$ g of copper and $ 2.0$ g of silver are deposited. Calculate the rate at which the energy is supplied by the battery. 
\item State Lenz’s Jaw of electromagnetic induction.
\item An aircraft is flying horizontally at $ 200$ m$/$s through the region where the vertical component of the earth magnetic field is $ 4.0 \times 10^{-5}$ T. If the air craft has a wing span of $ 40$ m, what will be the potential difference (p.d.) produced between the wing tips? 
\item A toroid of inner radius $ 25$ cm and an outer radius of $ 28$ cm has $ 4500$ turns of wound around it which passes a Current of $ 12$ A. What will be the induction of the magnetic flux;
 \begin{itemize}
\item Outside the toroid. 
\item inside the core of the toroid, 
\item in an empty space surrounding the toroid. 
\end{itemize}
\item Derive an expression for impedance of a series $ R-C$ circuit. 
\item An alternating current (a.c) of $ 0.2$ A r.m.s and frequency of $ 110/2\pi$ Hz flow in a circuit containing a series arrangement of a resistor $ R$ of resistance $ 20\Omega$ , an inductor $ L$ of $ 0.15$ H and a capacitor $ C$ of capacitance $ 500$ $\mu$F . Calculate the potential difference (p.d) and the impedence of the circuit.
\item What is meant by transistor action?
\item Briefly explain why the collector of a transistor is made wider than the emitter and base?
\item Draw a well labeled circuit diagram of an inverting amplifier.
\item Derive the closed – loop gain A of an inverting amplifier.  If the input resistor is equal to the feedback resistor, what would be the value of the gain A?
\item Mention one application of LED. What type of a semiconductor is it?
\item Write down two advantages of digital circuits over the analogue circuits.
\item Distinguish between white spectrum and line spectrum. 
\item If the energy necessary to cause the ejection of an electron by photoelectric effect from the $ N$ — shell and $ K-$ shell of an atom is $ 10$ eV and $ 20$ eV respectively, calculate the maximum wavelength of radiation for each level.
\item What is the significance of the binding energy per nucleon? 
\item Given that Rydberg’s constant is approximately $ 1.1 \times 10^{7}$ m$ ^{-1}$ Calculate the corresponding range of frequency for emitted radiation in the:
 \begin{itemize}
\item Lyman series. 
\item Balmer series. 
\end{itemize}
\item Briefly explain why the $ \beta$  — particles emitted from a radioactive source differ from the electrons obtained by thermionic emission? 
\item The mass of a particular radioisotope in « sample is initially $ 6.4 \times 10^{-3}$ kg, After $ 42$ days the isotope was separated from the sample and found to have a mass of $ 1.0 \times 10^{-4}$ kg. Calculate the half- life of the isotope.
\end{itemize}

\end{document}

\documentclass{article}
\usepackage[a4paper, total={6in, 8in}]{geometry}
% \usepackage[utf8]{inputenc}
\usepackage{abstract}
\title{THE UNITED REPUBLIC OF TANZANIA

NATIONAL EXAMINATIONS COUNCIL

ADVANCED CERTIFICATE OF SECONDARY EDUCATION EXAMINATION

\textbf{2018 PHYSICS 2}}
\author{Transcribed by:  PJ Gibson}

\begin{document}

\maketitle

\begin{itemize}
\item Given the Bernoulli’s equation: $ p+\rho gh+\rho v^{2}=$ constant where all the symbols carry their usual meaning.
 \begin{itemize}
\item What quantity does each expression on the left hand side of the equation represent? 
\item Mention any three conditions which make the equation to be valid. 
\end{itemize}
\item Water is supplied to a house at ground level through a pipe of inner diameter $ 1.5$ cm at an absolute pressure of $ 6.5 \times 10^{5}$ Pa and velocity of $ 5$ m$/$s. The pipe line leading to the second floor bath room $ 8$ m above has an inner diameter of $ 0.75$ cm. Find the flow velocity and pressure at the pipe outlet in the second floor bathroom. 
\item Define the following terms when applied to fluid flow:
 \begin{itemize}
\item Non-viscous fluid 
\item Steady flow 
\item Line of flow 
\item Turbulent flow
\end{itemize}
\item A horizontal pipeline increases uniformly from $ 0.080$ m diameter to $ 0.160$ m diameter in the direction of flow of water. When $ 96$ litres of water is flowing per second, a pressure gauge at the $ 0.080$ m diameter section reads $ 3.5 \times 10^{5}$ Pa. What should be the reading of the gauge at the $ 0.160$ m diameter section neglecting any loss? 
\item What do you understand by the terms:
 \begin{itemize}
\item Progressive wave 
\item Refraction of waves 
\item Diffraction of waves 
\item Standing wave. 
\end{itemize}
\item Two progressive waves traveling in the opposite direction in the medium are represented by $ Y_{1}=5 \sin(\omega t+\pi/3)$ and  $ Y_{2}=5 \sin(\omega t- \pi/3)$ . If the two progressive waves form a standing wave, determine the resultant amplitude and the phase angle formed. 
\item The shortest length of the resonance tube closed at one end which resounds to fork of frequency $ 256$ Hz is $ 31.6$ cm, The corresponding length for a fork of frequency $ 384$ Hz is $ 20.5$ cm. Determine the end correction for the tube and the velocity of sound in air. 
\item What do you understand by the term interference of waves?
\item A viewing screen is separated from a double-slit source by $ 1.2$ m. The distance between the two slits is $ 0.030$ mm. The second order bright fringe $ (m=2)$ is $ 4.5$ cm from the centre line. Determine the wavelength of the light and the distance between adjacent bright fringes. 
\item Define the term coherent sources of light. 
\item Interference patterns are formed when using Young’s double slit experiment. Mention other three methods that can be used to form interference patterns. 
\item Giving reasons, explain whether either transverse or longitudinal waves could exist, if the vibratory motion causing them were not simple harmonic motion. 
\item A beam of monochromatic light of wavelength $ 680$ nm in air passes into glass.  Calculate: 
 \begin{itemize}
\item The speed of light in glass
\item The frequency of light
\item The wavelength of light in glass
\end{itemize}
\item Light of wavelength $ 644$ nm is incident on a grating with a spacing of $ 2.00 \times 10^{-6}$ m. 
 \begin{itemize}
\item What is the angle to the normal of a second order maximum? 
\item What is the largest number of orders that can be visible? 
\item Find the angular separation between the third and fourth order image.
\end{itemize}
\item Mention any two factors which affect the surface tension of the liquid and in each case explain two typical examples. 
\item Why molecules on the surface of a liquid have more potential energy than those within the liquid? Briefly explain. 
\item Derive an expression for excess pressure inside a soap bubble of radius $ R$ and surface tension $ \gamma $ when the pressures inside and outside the bubble are $ P_{2}$ and $ P_{1}$ respectively. 
\item A soap bubble has a diameter of $ 5$ mm. Calculate the pressure inside it if the atmospheric pressure is $ 10^{5}$ Pa and the surface tension of a soap solution is $ 2.8 \times 10^{-2}$ N$/$m.
\item Water rises up in a glass capillary tube up to a height of $ 9.0$ cm while mercury falls down by $ 3.4$ cm in the same capillary. Assume angles of contact for water-glass and . mercury-glass as $ 0^{\circ}$ and $ 135^{\circ}$ respectively. Determine the ratio of surface tensions of mercury and water. 
\item Briefly explain the following observations as applied to strengths of materials:
 \begin{itemize}
\item Bridges are declared unsafe after long use. 
\item Iron is more elastic than rubber. 
\end{itemize}
\item A composite wire of diameter $ 1$ cm consists of copper and steel wires of lengths $ 2.2$ m and $ 2$ m respectively. Total extension of the wire when stretched by a force is $ 1.2$ mm. Calculate the force, given that Young’s modulus for copper is $ 1.1 \times 10^{11}$ Pa and for steel is $ 2 \times 10^{11}$ Pa. 
\item What do you understand by the following terms?
 \begin{itemize}
\item A perfectly plastic material 
\item The ultimate tensile strength 
\item An elastic limit 
\item Poisson’s ratio. 
\end{itemize}
\item Two rods of different materials but of equal cross-sections and lengths $ 1.0$ m each are joined to make a rod of length $ 2.0$ m. The metal of one rod has coefficient of linear thermal expansion of $ 10^{-5}^{\circ}$C$ ^{-1}$ and Young’s Modulus $ 3 \times 10^{10}$ N$/$m$ ^{2}$ . The other metal has the values $ 2 \times 10^{-5}^{\circ}$C$ ^{-1}$ and $ 10^{10}$ N$/$m$ ^{2}$ respectively. How much pressure must be applied to the ends of the composite rod to prevent its expansion when the temperature is raised by $ 100^{\circ}$C? 
\item Briefly explain the effect of the dielectric material on the capacitance of a capacitor when the capacitor is:
 \begin{itemize}
\item Isolated. 
\item Connected to the battery.
\end{itemize}
\item How are the electrolytic capacitors made? 
\item Two point charges of equal mass $ m$ and charge $ Q$ are suspended at a common point by two threads of negligible mass and length $ L$ . If the two point charges are at equilibrium,  show that;
 \begin{itemize}
\item The distance of separation $ x=({Q^{2}L}/{2\pi\epsilon _{0}mg})^{1/3}$
\item The angle of inclination $ \beta = ^{3}\sqrt{(Q^{2})/(16\pi\epsilon _{0}mgL^{2})}$ 
\end{itemize}
\item Two point charges, $ q_{A}=+3$ $\mu$C and $ q_{b}=-3$ $\mu$C, are located $ 0.2$ m apart in vacuum. Find; 
 \begin{itemize}
\item the electric field at the midpoint of the line joining two charges. 
\item the force experienced by the negative test charge of magnitude $ 1.5 \times 10^{-9}$ C placed at this point.
\end{itemize}
\item What is meant by the term Ballistic galvanometer? 
\item State two conditions to be fulfilled for a galvanometer to be used as a ballistic galvanometer. 
\item Consider a small flat coil which has $ N$ turns of area A and whose plane is perpendicular to a magnetic field of flux density $ B$ . If the search coil is connected to the ballistic galvanometer and the total resistance of the circuit is $ R$ , use the laws of electromagnetic induction to show that the charge delivered to the galvanometer does not depend on how long it takes to remove the search coil from the field. 
\item A circular coil of $ 300$ turns has a radius of $ 10$ cm and carries a current of $ 7.5$ A. Calculate the magnetic field at:
 \begin{itemize}
\item the centre of the coil. 
\item a point which is at a distance of $ 5$ cm from the centre of the coil. 
\end{itemize}
\item Mention the three magnetic materials and briefly explain each one. 
 \begin{itemize}
\item Give the differences between the magnetic materials mentioned above in terms of their magnetic susceptibility. 
\end{itemize}
\item Define the following terms:
 \begin{itemize}
\item Ampere 
\item Hysteresis 
\end{itemize}
\item What do you understand by the term photon. 
\item List down any three properties of a photon. 
\item State any four laws of photoelectric emission. 
\item Briefly explain what led de-Broglie to think that the material particles may also show wave nature and why the wave nature of matter not noticeable in our daily observations? 
\item Prove that de-Broglie wavelength $ \lambda $ , of electrons of kinetic energy $ E$ is given by $ \lambda = h/ \sqrt{2}$ meV  where $ m$ is the mass of the electron, $ e$ is the charge of the electron, $ h$ is the Planck’s constant and v is the accelerating potential difference. 
\item Light of wavelength $ 488$ nm is produced by an argon laser which is used in the photoelectric effect. When light from this spectral line is incident on the emitter, the stopping (cut-off) potential of photoelectrons is $ 0.38$ V. Find the work function of the material from which the emitter is made. 
\item Use the concept of radioactive decay and nuclear reactions to define the following terms:
 \begin{itemize}
\item $ \alpha $ decay
\item $ \beta$ decay
\item $ \gamma $ decay
\item Fission
\item Fusion.
\item For each of the terms above, give one suitable reaction equation. 
\end{itemize}
\item A freshly prepared sample of a radioactive isotope $ Y$ contains $ 10^{12}$ atoms. The half-life of the isotope is $ 15$ hours. Calculate;
 \begin{itemize}
\item the initial activity. 
\item the number of radioactive atoms of $ Y$ remaining after $ 2$ hours, 
\end{itemize}
\item Mention any four important features in the design of a nuclear reactor.
\item Differentiate binding energy from mass defect.
\item Calculate the binding energy per nucleon, in MeV and the packing fraction of an alpha particle.
\item Given: Mass of proton $ =1.0080$ u, Mass of neutron $ =1.0087$ u and Mass of alpha particle $ =4.0026$ u.
 \begin{itemize}
\item State any three limitations of Bohr’s model of the hydrogen atom.
\end{itemize}
\item In a hydrogen atom model, an electron of mass $ m$ and charge $ e$ revolves around the nucleus in a circular orbit of radius $ r$ . Develop an expression for the radius $ 3$ m of the orbit in terms of $ m$ , $ e$ , $ x$ , the quantum number $ n$ , Planck constant $ h$ and the permitting of free space $ \epsilon _{0}$ , and hence, use their values to find the Bohr’s radius. 
\item Distinguish between ionization energy and excitation energy.
\item Why hydrogen spectrum contains a larger number of spectral lines although its  atom has only one electron? 
\item A freshly prepared sample of a radioactive isotope $ Y$ contains $ 10^{12}$ atoms. The half-life of the isotope is $ 15$ hours. Calculate;
 \begin{itemize}
\item the initial activity. 
\item the number of radioactive atoms of $ Y$ remaining after $ 2$ hours, 
\end{itemize}
\item Mention any four important features in the design of a nuclear reactor.
\item Differentiate binding energy from mass defect.
\item Calculate the binding energy per nucleon, in MeV and the packing fraction of an alpha particle.
 \begin{itemize}
\item Given: Mass of proton $ =1.0080$ u, Mass of neutron $ =1.0087$ u and Mass of alpha particle $ =4.0026$ u.
\end{itemize}
\item State any three limitations of Bohr’s model of the hydrogen atom.
\item In a hydrogen atom model, an electron of mass $ m$ and charge $ e$ revolves around the nucleus in a circular orbit of radius $ r$ . Develop an expression for the radius $ 3$ m of the orbit in terms of $ m$ , $ e$ , $ x$ , the quantum number $ n$ , Planck constant $ h$ and the permitting of free space $ \epsilon _{0}$ , and hence, use their values to find the Bohr’s radius. 
\item Distinguish between ionization energy and excitation energy.
\item Why hydrogen spectrum contains a larger number of spectral lines although its  atom has only one electron? 
\end{itemize}

\end{document}
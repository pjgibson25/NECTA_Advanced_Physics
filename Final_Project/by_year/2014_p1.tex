
\documentclass{article}
\usepackage[a4paper, total={6in, 8in}]{geometry}
% \usepackage[utf8]{inputenc}
\usepackage{abstract}
\title{THE UNITED REPUBLIC OF TANZANIA

NATIONAL EXAMINATIONS COUNCIL

ADVANCED CERTIFICATE OF SECONDARY EDUCATION EXAMINATION

\textbf{2014 PHYSICS 1}}
\author{Transcribed by:  PJ Gibson}

\begin{document}

\maketitle

\begin{itemize}
\item Distinguish random error from systematic error.
 \begin{itemize}
\item Give a practical example of random error and systematic error and briefly explain how they can be reduced or eliminated.
\end{itemize}
\item Define the terms error and mistake.
\item An experiment was done to find the acceleration due to gravity by using the formula: $ T=2\pi\sqrt{l/g}$ , where all symbols carry their usual meaning.  If the clock losses $ 3$ seconds in $ 5$ minutes, determine the error in measuring ‘$ g$ ’ given that, $ T=2.22$ sec, $ l=121.6$ cm, $ \Delta T_{1}=0.1$ sec, and  $ \Delta l=\pm 0.05$ .
\item What is the importance of dimensional analysis inspite of its drawbacks?
\item The following measurements were taken by a student fort he length of a piece of rod: $ 21.02$ , $ 20.99$ , $ 20.92$ , $ 21.11$ and $ 20.69$ . Basing on error analysis find the true value at the length of a piece of rod and its associated error.
\item Outline the motions that add up to make projectile motion. 
\item In the first second of its flight, a rocket ejects $ 1/60$ of  its mass with a relative velocity of $ 2400$ m$/$s.
 \begin{itemize}
\item Find its acceleration.
\item What is the final velocity if the ratio of initial to final mass of the rocket is $ 4$ at a time of $ 60$ seconds? 
\end{itemize}
\item A ball is thrown upwards with an initial velocity of $ 33$ m$/$s from a point $ 65^{\circ}$ on the side of a hill which slopes upward uniformly at an angle of $ 28^{\circ}$ .
 \begin{itemize}
\item At what distance up the slope does the ball strike?
\item Calculate the time of flight of the ball. 
\end{itemize}
\item State the principle of conservation of linear momentum. 
 \begin{itemize}
\item Give two examples of the principle of conservation of linear momentum. 
\end{itemize}
\item A cannon of mass $ 1300$ kg fires a $ 72$ kg ball in a horizontal direction with a nuzzle speed of $ 55$ m$/$s, If the cannon is mounted so that it can recoil freely calculate the:
 \begin{itemize}
\item  recoil velocity of the cannon relative to the earth. 
\item horizontal velocity of the ball relative to the earth. 
\end{itemize}
\item Define the term ‘radial acceleration’. 
\item An insect is released from rest at the top of the smooth bowling ball such that it slides over the ball. Prove that it will loose its footing with the ball at an angle of about $ 48^{\circ}$ with the vertical.
\item State where the magnitude of acceleration is greatest in simple harmonic motion.
\item Sketch a graph of acceleration against displacement for a simple harmonic motion.
\item A vertical spring fixed at one end has a mass of $ 0.2$ kg and is attached at the other end.
 \begin{itemize}
\item Determine the:
\item Extension of the spring.
\item Energy stored in the spring.
\end{itemize}
\item The displacement of a particle from the equilibrium position moving with simple harmonic motion is given by $ x=0.05 \sin(6t)$ , where $ t$ is the time in seconds measured at an instant when $ x=0$ .  Calculate the:
 \begin{itemize}
\item Amplitude of oscillations.
\item Period of oscillations. 
\item  Maximum acceleration of the particle. 
\end{itemize}
\item Define the universal gravitational constant.
\item How is the gravitational potential related to gravitational field strength?
\item Write down an expression for the acceleration due to gravity (g) of a body of mass (m) which is at a  distance (r) from the centre of the earth. 
 \begin{itemize}
\item If the Earth were made of lead of relative density of $ 11.3$ kg$/$m$ ^{3}$ , what would he the value of acceleration due to gravity on the surface of the earth?
\end{itemize}
\item Why the value of acceleration due to gravity (g) changes due to the change in latitude? Give two reasons.
\item A rocket is fired from the earth towards the sun. At what point on its path is the gravitational force on the rocket zero?
\item Define torque and give its S.I. unit.
\item A disc of moment of inertia $ 2.5\times10^{-4}$ kg$/$m$ ^{2}$ is rotating freely about an axis through its centre at $ 20$ rev/min. If some wax of mass $ 0.04$ kg is dropped gently on to the disc $ 0.05$ m from its axis, what will be the new revolution per minute of the disc? 
\item Explain briefly why a:
 \begin{itemize}
\item high diver can turn more somersaults before striking the water?
\item dancer on skates can spin faster by folding her arms?
\end{itemize}
\item A heavy flywheel of moment of inertia $ 0.4$ kg$/$m$ ^{2}$ is mounted on a horizontal axle of radius $ 0.01$ m. If a force of $ 60$ N is applied tangentially to the axle:
 \begin{itemize}
\item  Calculate the angular velocity of the flywheel after $ 5$ seconds from rest.
\item List down two assumptions taken to arrive at your answer in above.
\end{itemize}
\item Give two ways in which the internal energy of the system can be changed.
\item List down two simple applications of the First law of thermodynamics in our daily life.
\item One mole of a gas expands from volume, $ V_{1}$ , to a volume $ V_{2}$ . If the gas obeys the Van-der-Waal’s equation, $ (p+ a/v^{2})(v – b)=$ RT, derive the formula for work done in this process.
\item A heat engine works at two temperatures of $ 27^{\circ}$C and $ 227^{\circ}$C. Calculate the:
 \begin{itemize}
\item Efficiency of the engine. 
\item Temperature which will increase the efficiency by $ 10\%$ if the room temperature is kept at $ 27^{\circ}$C. 
\end{itemize}
\item Define thermal convection.
\item Prove that at a very small temperature difference, $ \Delta T=T_{b}$ – $ T_{s}$ ,  Newton's law of cooling obeys the Stefan’s law, whereby $ T_{b}$ , is the temperature of the body and $ T_{s}$ is the temperature of the surrounding. 
\item What is meant by temperature of inversion?
\item A thermometer was wrongly calibrated as mt reads the melting point of ice as $ -10^{\circ}$C and reading a temperature of $ 60^{\circ}$C in place of $ 50^{\circ}$C What would be the temperature of boiling point of water on this scale? 
\item What is meant by the following terms:
 \begin{itemize}
\item Alternating current (a.c.)
\item Effective value of A.C. 
\end{itemize}
\item A $ 60$ V, $ 10$ W lamp is to be run on $ 100$ V, $ 60$ Hz A.C mains.
 \begin{itemize}
\item Calculate the inductance of a choke coil required.
\item If a resistor is used in above instead of choke, what will be value of its resistance.
\end{itemize}
\item An LCR circuit with $ R=70\Omega$ in series with a parallel combination of $ L=1.5$ H and
 \begin{itemize}
\item $ C=30$ $\mu$F is driven by a $ 230$ V supply with angular frequency of $ 300$ rad$/$s.
\item $ (1)$ Find the power in put to the circuit. 
\item  At the frequency $ \omega_{o}=1/(\sqrt{LC})$ , how does the circuit respond?
\end{itemize}
\item Define the following terms:
 \begin{itemize}
\item Current density
\item Conductivity 
\end{itemize}
\item Under what condition is $ \Omega $ ’s law true?
\item Why does the voltage across the terminals of a cell or battery fall when it is delivering a current? 
\item Define temperature coefficient of resistance.
 \begin{itemize}
\item A heating coil of Nichrome wire with cross sectional area of $ 0.1 $ mm$ ^{2}$ operates on a $ 12$ V supply, and has a power of $ 36$ W when immersed in water at $ 373$ K. Calculate the length of the wire.
\end{itemize}
\item What is meant by the following electronic circuits:
 \begin{itemize}
\item Logic gates 
\item Integrated circuits
\end{itemize}
\item What is light emitting diode (LED).
\item Give three advantages of LED's lamp in radio and other electronic system over filament lamps.
\item What is the basic difference between good conductors and semiconductors.
\item Mention two types of transistors.
 \begin{itemize}
\item Which among the transistors mentioned above responds quickly to electrical signal? Give reason for your answer.
\end{itemize}
\item Give the meaning of the following terms:
 \begin{itemize}
\item Bandwidth
\item  Amplitude modulated carrier wave
\end{itemize}
\item What is the purpose of amplifiers in a phone link? 
\item Sketch the frequency spectrum for $ 1500$ m radio waves modulated by $ 4$ kHz audio signal.
\item List down two advantages of digital signals over analogue signals.
\item A carrier of frequency $ 800$ kHz is amplitude modulated by frequencies ranging from $ 1$ kHz to $ 10$ kHz.  What frequency range does each sideband cover?
\item Describe the sources and effects of the following pollutants on the environment:
 \begin{itemize}
\item Air pollution. 
\item Radiation pollution.
\end{itemize}
\item Briefly explain the influence of the following climatic conditions for plant growth and development:
 \begin{itemize}
\item Rain fall and water
\item Wind
\end{itemize}
\end{itemize}

\end{document}
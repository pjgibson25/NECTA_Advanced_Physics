
\documentclass{article}
\usepackage[a4paper, total={6in, 8in}]{geometry}
% \usepackage[utf8]{inputenc}
\usepackage{abstract}
\title{THE UNITED REPUBLIC OF TANZANIA

NATIONAL EXAMINATIONS COUNCIL

ADVANCED CERTIFICATE OF SECONDARY EDUCATION EXAMINATION

\textbf{2015 PHYSICS 1}}
\author{Transcribed by:  PJ Gibson}

\begin{document}

\maketitle

\begin{itemize}
\item What is meant by random errors?
 \begin{itemize}
\item Briefly explain two causes of random errors in measurements. 
\end{itemize}
\item The period $ T$ of oscillation of a body is said to be $ 1.5\pm 0.002$ s while its amplitude A is $ 0.3\pm 0.005$ m and the radius of gyration $ k$ is $ 0.28+0.004$ m. If the acceleration due
 \begin{itemize}
\item to gravity $ g$ was found to be related to $ T$ , $ A$ and $ k$ by the equation $ (gA)/(4\pi^{2})=( A^{2}+k^{2})/T^{2}$ , find the:
\item Numerical value of $ g$ in four decimal places
\item Percentage error in $ g$ .
\end{itemize}
\item State the law of dimensional analysis.
\item The largest mass, $ m$ of a stone that can be moved by the flowing river depends on the velocity of flow v, the density $ \rho $ of water, and the acceleration due to gravity $ g$ . Show that the mass, $ m$ varies to the sixth power of the velocity of flow.
\item Define the term trajectory.
\item Briefly explain why the horizontal component of the initial] velocity of a projectile always remains constant.
\item List down two limitations of projectile motion. 
\item A body projected from the ground at the angle of $ 60^{\circ}$ is required to pass just above the two vertical walls each of height $ 7$ m. If the velocity of projection is $ 100$ m$/$s, calculate the distance between the two walls. 
\item A fireman standing at a horizontal distance of $ 34$ m from the edge of the burning story building aimed to raise streams of water at an angle of $ 60^{\circ}$ into the first floor through an open window which is at $ 20$ m high from the ground level. If water strikes on this floor $ 2$ m away from the outer edge, 
 \begin{itemize}
\item  Sketch a diagram of the trajectory.
\item What speed will the water leave the nozzle of the fire hose?
\end{itemize}
\item Mention three effects of looping the loop.
 \begin{itemize}
\item Why there must be a force acting on a particle moving with uniform speed in a circular path? Write down an expression for its magnitude. 
\end{itemize}
\item A driver negotiating a sharp bend usually tend to reduce the speed of the car.
 \begin{itemize}
\item  What provides the centripetal force on the car?
\item Why is it necessary to reduce its speed?
\end{itemize}
\item A ball of mass $ 0.5$ kg is attached to the end of a cord whose length is $ 1.5$ m then whirled in horizontal circle. If the cord can withstand a maximum tension of $ 50$ N calculate the:
 \begin{itemize}
\item Maximum speed the ball can have before the cord breaks. 
\item Tension in the cord if the ball speed is $ 5$ m$/$s
\end{itemize}
\item Briefly explain why the motion of a simple pendulum is not strictly simple harmonic? 
 \begin{itemize}
\item Why is the velocity and acceleration of a body executing simple harmonic motion (S.H.M.) out of phase? 
\end{itemize}
\item A body of mass $ 0.30$ kg executes simple harmonic motion with a period of $ 2.5$ s and amplitude of $ 4.0\times10^{-2}$ m. Determine the:
 \begin{itemize}
\item Maximum velocity of the body. 
\item Maximum acceleration of the body. 
\item Energy associated with the motion.
\end{itemize}
\item A particle of mass $ 0.25$ kg vibrates with a period of $ 2.0$ s. If its greatest displacement is $ 0.4$ m what is its maximum kinetic energy?
\item  Define moment of inertia of a body.
 \begin{itemize}
\item Briefly explain why there is no unique value for the moment of inertia of a given body?
\end{itemize}
\item State the principle of conservation of angular momentum. 
 \begin{itemize}
\item A horizontal disc rotating freely about a vertical axis makes $ 45$ revolutions per minute. A small piece of putty of mass $ 2.0\times10^{-2}$ kg falls vertically onto the disc and sticks to it at a distance of $ 5.0\times10^{-2}$ m from the axis. If the number of revolutions per minute is thereby reduced to $ 36$ , calculate the moment of inertia of the disc. 
\end{itemize}
\item Define the term tangential velocity.
\item Explain why the astronaut appears to be weightless when traveling in the space vehicle.
\item State Newton's law of gravitation. 
 \begin{itemize}
\item Use Newton’s law of gravitation to derive Kepler’s third law.
\end{itemize}
\item Briefly explain why Newton’s equation of universal gravitation does not hold for bodies falling near the surface of the earth? 
\item Show that the total energy of a satellite in a circular orbit equals half its potential energy.
\item What would be the length of a day if the rate of rotation of the Earth were such that the acceleration due to gravity $ g=0$ at the equator?
\item Calculate the height above the Earth’s surface for a satellite in a parking orbit.
\item What is meant by a thermometric property?
\item Mention three qualities that make a particular property suitable for use in a practical thermometer.
\item Define coefficient of thermal conductivity.
\item Write down two characteristics of a perfectly lagged bar.
\item A thin copper wall of a hot water tank having a total surface area of $ 5.0$ m$ ^{2}$ contains $ 0.8$ cm$ ^{3}$ of water at $ 350$ K and is lagged with a $ 50$ mm thick layer of a material of thermal conductivity $ 4.0\times10^{-2}$ W$/$mK. If the thickness of copper wall is neglected and the temperature of the outside surface is $ 290$ K,
 \begin{itemize}
\item Calculate the electrical power supplied to an immersion heater.
\item If the heater were switched off, how long would it take for the temperature of hot water to fall by $ 1$ K?
\end{itemize}
\item The element of an electric fire with an output of $ 1000$ W is a cylinder of $ 250$ mm long and $ 15$ mm in diameter. If it behaves as a black body, estimate its temperature.
\item What is meant by the following terms:
 \begin{itemize}
\item  Internal resistance of a cell. 
\item  Drift velocity. 
\end{itemize}
\item What is a potentiometer. 
 \begin{itemize}
\item Mention two advantages and two disadvantages of potentiometer.
\end{itemize}
\item Distinguish between ohmic and non-ohmic conductor. Give one example in each
\item Sketch the diagram showing the variation of current with potential difference across the following:
 \begin{itemize}
\item  Filament electric bulb. 
\item Gas-filled diode. 
\end{itemize}
\item A wire of diameter $ 0.1$ mm and resistivity $ 1.69\times10^{-8}\Omega$ m with temperature coefficient
 \begin{itemize}
\item of resistance of $ 4.3\times10^{-3}$ K$ ^{-1}$ was required to make a resistance,
\item  What length of the wire is required to make a coil with a resistance of $ 0.5\Omega $ ?
\item If on passing a Current of $ 2$ A the temperature of the coil above rises  by $ 10^{\circ}$C, what error would arise in taking the potential drop as $ 1.0$ V 
\end{itemize}
\item Mention four important properties of a semiconductor.
\item Applying the concept of doping, explain how a free electron and a positive charge can be created in a semiconductor crystal. 
\item Why a $ p-n$ junction diode when connected in a circuit and then reversed gives a very small leakage current across the junction? 
 \begin{itemize}
\item How is the size of the current stated in above dependent on the temperature of the diode?
\end{itemize}
\item List three properties of operational amplifiers.
\item What is meant by the term negative feedback? Give four advantages of using it in an op-amp or any type of voltage amplifier.
\item Define closed loop gain. 
\item Derive an expression of the closed loop gain for an inverting op-amp voltage amplifier with an input resistor $ R$ , and a feedback resistor.
\item Give one advantage of frequency modulation (FM) as compared to amplitude modulation ( AMT).
\item Briefly explain the importance of bandwidth of an amplitude modulation (AM) signal.
\item State the function of a modulator in radios.
\item Sketch a block diagram to show the general plan of any communication system.
\item The amplitude modulated (AM) broadcast band ranges from $ 450$ to $ 1200$ kHz. If each station modulates with audio frequencies up to $ 5.5$ kHz, determine the
 \begin{itemize}
\item  Bandwidth needed for each station.
\item  Total bandwidth available. 
\end{itemize}
\item What is the origin of earthquake?
\item List down three sources of earth's magnetism. 
\item A large explosion at the earth's surface creates compressional (P) and shear (S) waves moving with a speed of $ 6.0$ km$/$s and $ 3.5$ km$/$s respectively. If both waves arrive at seismological station with $ 30$ s interval, calculate the distance measured between seismological station and the site of explosion. 
\item Explain three techniques applicable for improving soil environment for the best plant growth.
\end{itemize}

\end{document}
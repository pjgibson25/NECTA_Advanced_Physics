
\documentclass{article}
\usepackage[a4paper, total={6in, 8in}]{geometry}
% \usepackage[utf8]{inputenc}
\usepackage{abstract}
\title{THE UNITED REPUBLIC OF TANZANIA

NATIONAL EXAMINATIONS COUNCIL

ADVANCED CERTIFICATE OF SECONDARY EDUCATION EXAMINATION

\textbf{2019 PHYSICS 1}}
\author{Transcribed by:  PJ Gibson}

\begin{document}

\maketitle

\begin{itemize}
\item Identify two basic rules of dimensional analysis.
\item The frequency $ n$ of vibration of a stretched string is a function of its tension $ F$ , the length, $ l$ and mass per unit length $ m$ . Use the method of dimensions to derive the formula relating the stated physical quantities.
\item What causes systematic errors in an experiment? Give four points. 
\item Estimate the numerical value of drag force $ D= 1/2 C \rho  A V^{2}$ with its associated error given that the measurements of the quantities $ C$ , $ A$ , $ \rho $ and $ v$ were recorded as $ (10\pm 0.00)$ unit less $ (5\pm 0.2) $ cm$ ^{2}$ , $ (15\pm 0.15)$ g$/$cm$ ^{3}$ and $ (3\pm 0.5)$ cm$/$sec$ ^{2}$ respectively. 
\item A rocket of mass $ 20$ kg has $ 180$ kg of fuel. If the exhaust velocity of the fuel is $ 1.6$ km/sec, calculate;
 \begin{itemize}
\item The minimum rate of fuel consumption that enable the rocket to rise from the ground. 
\item The ultimate vertical speed gained by the rocket when the rate of fuel consumption ts $ 2$ kg/sec. 
\end{itemize}
\item Determine the least number of pieces required to stop the bullet if a rifle bullet loses $ 1/20$ of its velocity when passing through them.
\item A man of $ 100$ kg jumps into a swimming pool from a height of $ 5$ m. If it takes $ 0.4$ seconds for the water in a pool to reduce its velocity to zero, what average force did  the water exert on the man? 
\item Justify the statement that projectile motion is two dimensional motion.
\item A rocket was launched with a velocity of $ 50$ m$/$s from the surface of the moon at an angle of $ 40^{\circ}$ to the horizontal, Calculate the horizontal distance covered  after half time of flight.
\item Show that the angle of projection $ \theta ^{\circ}$ for a projectile launched from the origin is given by $ \theta ^{\circ}= tan^{-1}(4h_{m}/R)$ , where $ R$ stand for horizontal range and $ h_{m}$ is the maximum vertical height.
\item Determine the angle of projection for which the horizontal range of a projectile is $ 4\sqrt{3}$ times its maximum height. 
\item Provide two typical examples of simple harmonic motion (S.H.M). 
\item Why the velocity and acceleration of a body executing simple harmonic motion are out of phase? 
\item The period of a particle executing simple harmonic motion (S.H.M) is $ 3$ seconds. If its amplitude is $ 25$ cm, calculate the time taken by the particle to move a distance of $ 12.5$ cm on either side from the mean position.
\item A person weighing $ 50$ kg stands on a platform which oscillates with a frequency of $ 2$ Hz and of amplitude $ 0.05$ m. Find his/her minimum weight as recorded by a machine of the platform. 
\item In which aspect does circular motion differ from linear motion? 
\item Why there must be a force acting on a particle moving with uniform speed in a circular path? 
\item A stone tied to the end of string $ 80$ cm long, is whirled in a horizontal circle with a constant speed making $ 25$ revolutions in $ 14$ seconds. Determine the magnitude of its acceleration. 
\item Why the weight of a body becomes zero at the centre of the earth? 
\item How far above the earth surface does the value of acceleration due to gravity becomes $ 36\%$ of its value on the surface? 
\item Compute the period of revolution of a satellite revolving in a circular orbit at a height of $ 3400$ km above the Earth’s surface. 
\item Prove that the angular momentum fora satellite of mass $ M_{s}$ revolving round the
 \begin{itemize}
\item earth of mass $ M_{e}$ in an orbit of radius $ r$ is equal to $ (G M_{e}$  $ M_{s}^{2}r)^{1/2}$ .
\end{itemize}
\item Why water is preferred as a cooling agent in many automobiles?
\item A thermometer has wrong calibration as it reads the melting point of ice as $ -10^{\circ}$C . If it reads $ 40^{\circ}$C in a place where the temperature reads $ 30^{\circ}$C ,  determine the boiling point of water on this scale.
\item Analyze  three practical applications of thermal expansion of solids in daily life situations.
\item A closed metal vessel containing water at $ 75^{\circ}$C , has a surface area of $ 0.5$ m$ ^{2}$ and uniform thickness of $ 4.0$ mm.  If its outside temperature is $ 15^{\circ}$C , calculate the head loss per minute by conduction.
\item Sketch the graph to illustrates how the energy radiated by a black body is distributed among various wavelengths. 
 \begin{itemize}
\item What information would be drawn from the graph above? Give three points.
\end{itemize}
\item Why stainless steel cooking pans are made with extra copper at the bottom?
\item At what temperature will the filament of a $ 10$ W lamp operate if it is supposed to be a perfectly black body of area  $ 1 $ cm$ ^{2}$ ? 
\item Elaborate three significance of dielectric material in a capacitor. 
\item Give the reason behind a loss of electrical energy when two capacitors are joined either in series or parallel. 
\item A researcher has $ 2$ g of gold and wishes to form it into a wire having a resistance of $ 80\Omega $ at $ 0^{\circ}$C . How long should the wire be? 
\item What is the potential difference between two points if $ 5$ Joules of work are required to move $ 10$ Coulombs from one point to another? 
\item Why does a room light turn on at once when the switch is closed? Give comment.
\item A current of $ 3.0$ mA flows in a Television resistor $ R$ when a potential difference of $ 6.0$ V is connected across its terminals. Determine the value of conductance.
\item Why transistors can not be used as rectifiers? 
\item In NPN transistor circuit the collector current is $ 5$ mA. If $ 95\%$ of the emitted electrons reach the collector region, calculate the base current. 
\item What causes damage to transistors? 
\item Distinguish between inverting OP-AMP and non-inverting OP-AMP. 
 \begin{itemize}
\item Give one application of each type of OP-AMP described above.
\end{itemize}
\item Identify three basic elements of a communication system. 
\item Why sky waves are not used for transmission of TV signals? 
\item What 's meant by epicentre and wind belt as used in Geophysics? 
\item Give two positive effects of wind on plant growth.
\item Identify three types of seismic waves.
 \begin{itemize}
\item Outline two characteristics of each type of wave described above.
\end{itemize}
\end{itemize}

\end{document}

\documentclass{article}
\usepackage[a4paper, total={6in, 8in}]{geometry}
% \usepackage[utf8]{inputenc}
\usepackage{abstract}
\title{THE UNITED REPUBLIC OF TANZANIA

NATIONAL EXAMINATIONS COUNCIL

ADVANCED CERTIFICATE OF SECONDARY EDUCATION EXAMINATION

\textbf{2000 PHYSICS 2}}
\author{Transcribed by:  PJ Gibson}

\begin{document}

\maketitle

\begin{itemize}
\item Write down the Bernoulli's equations for fluid flow in a pipe and indicate the term which will disappear when the flow of fluid is stopped.
\item Water flows into a tank of large cross-section area at a rate of $ 10^{-4}$ m$ ^{3}/$s but flows out from a  hole of area 1cm$ ^{2}$ which has been punched through the base. How high does the water rise in the tank?
\item At two points on a horizontal tube of varying circular cross-section carrying water, the radii are 1cm and $ 0.4$ cm and the pressure difference between these points is $ 4.9$ cm of water. How much liquid flows through the tube per second?
\item Define the “bulk modulus” of a gas
\item Find the ratio of the adiabatic bulk modulus of a gas to that of its isothermal bulk modulus in terms of the specific heat capacities of the gas.
\item Explain Young’s Modulus of rigidity
\item Find the work done in stretching a steel wire of $ 1.0$ mm$ ^{2}$ cross-sectional area and $ 2.0$ m in length through $ 0.1$ mm.
\item Find the work done required to break up a drop of water of radius $ 0.5$ cm into drops of water each having radius of $ 1.0$ mm, assuming isothermal condition.
\item What factors lead the real gas to obey the ideal gas equation $ PV = RT$ ?
\item Define the root-mean-square (r.m.s.) speed of the gas molecules. Hence find the r.m.s. speed of oxygen gas molecules at $ 10^{5}$ Pa pressure when the density is $ 1.43$ kg$/$m$ ^{3}$ .
\item Derive an expression for the work done per mole in an isothermal expansion of Vander Waal’s gas from volume $ V_{1}$ to volume $ V_{2}$ .
\item A number of $ 16$ moles of an ideal gas which is kept at constant temperature of $ 320$ K is compressed isothermally from its initial volume of $ 18$ litres to the final volume of $ 4$ litres.
 \begin{itemize}
\item Calculate the total work done in the whole process.
\item Comment on the sign of numerical answer you've obtained.
\end{itemize}
\item A cylinder fitted with a frictionless piston contains $ 1.0$ g of oxygen at a pressure of $ 760$ mmHg and at a temperature of $ 27^{\circ}$C. the following operations are performed in stages: $ (1)$ The oxygen is heated at a constant pressure to $ 127^{\circ}$C and then $ (2)$ it is compressed isothermally to its original volume and finally $ (3)$ it is cooled at a constant volume to its original temperature.
 \begin{itemize}
\item Illustrate these changes in a sketch $ P-V$ diagram.
\item What is the input of heat to the cylinder in stage $ (1)$ above?
\item How much work does the oxygen do in pushing back the piston during stage $ (1)$ ?
\item How much work is done on the oxygen in stage $ (2)$ ?
\item How much heat must be extracted from the oxygen in stage $ (3)$ ? 
\item (For oxygen: density $ =1.43$ kg$/$m$ ^{3}$ (at stp), $ C_v =670$ J kg$ ^{-1}$ K$ ^{-1}$ and molecular mass $ =32$ )
\end{itemize}
\item What is the difference between an “isothermal” process and an “adiabatic” process?
\item How much work is required to compress $ 5$ mol of air at $ 20^{\circ}$C and $ l$ atmosphere to $ 1/10$ th of the original volume by
 \begin{itemize}
\item an isothermal process
\item an adiabatic process?
\item What are the final pressures for the cases and above?
\end{itemize}
\item Explain the fact that the temperature of the ocean at great depths is very nearly constant the year round, at a temperature of about $ 4^{\circ}$C .
\item In a diesel engine, the cylinder compresses air from approximately standard temperature and pressure to about one-sixteenth the original volume and a pressure of about $ 50$ atmospheres. What is the temperature of the compressed air?
\item Give one major similarity and one major difference between heat conduction and wave propagation.
\item Deep bore holes into the earth show that the temperature increases about $ 1^{\circ}$C for each $ 30$ m depth. How much heat flows out from the core of the earth each second for each square metre of surface area.
\item Write two uses of Doppler effect.
\item An observer standing by a railway track notices that the pitch of an engine whistle changes in the ratio of $ 5$:$ 4$ on passing him. What is the speed of the engine?
\item Explain briefiy the necessary conditions for the effects of interference in optics to be observed
\item Interference patterns are formed when using Young’s double slit arrangement. Mention other three methods that can be used to form interference patterns.
\item Explain, giving reasons, whether either transverse or longitudinal waves could exist, if the vibratory motion causing them were not simple harmonic motion.
\item A beam of monochromatic light of wavelength $ 600$ nm in air passes into glass. Calculate:
 \begin{itemize}
\item the speed of light in glass.
\item the frequency of light.
\item the wavelength of light in glass.
\end{itemize}
\item What is meant by “diffraction grating”?
\item A monochromatic light of wavelength $ 5.2 \times 10^{-7}$ m falls normally on a grating which has $ 4 \times 10^{3}$ lines per cm.
 \begin{itemize}
\item What is the largest order of spectrum that can be visible?
\item Find the angular separation between the third and fourth order image.
\end{itemize}
\item What do you understand by the term “drift velocity” as applied to any current carriers in a wire?
\item Determine the drift velocity of electrons in a silver wire of a cross—sectional area $ 4.5 \times 10^{-6}$ m$ ^{2}$ when a current of $ 15$ A flows through it. Given: The density of silver $ =1.05 \times 10^{4}$ kg$/$m$ ^{3}$ . The atomic weight of silver $ =108$ .
\item An unknown wire of $ 1$ mm diameter is found to carry and passes a total charge of $ 90$ C in $ 1$ hour and $ 15$ min. If the wire has $ 5.8 \times 10^{28}$ free electrons per $ m^{3}$ , find
 \begin{itemize}
\item  the current in the wire.
\item the drift velocity of the electrons in m s$ ^{-1}$
\end{itemize}
\item The current of $ 12$ A is made to pass through an aluminium wire of radius $ 1.5$ mm which is joined in series with a copper wire of radius $ 0.8$ mm. Determine.
 \begin{itemize}
\item the current density in an aluminium wire.
\item the drift velocity of the electron tn the copper wire, given that the number of free electrons per unit volume in a copper wire is $ 10^{29}$ .
\end{itemize}
\item Give the statement of Coulomb’s law.
\item A proton of mass $ 1.673 \times 10^{-27}$ kg falls through a distance of $ 1.5$ cm in a uniform electric field of magnitude $ 2.0 \times 10^{4}$ N C$ ^{-1}$ . Determine the time of fall [Neglect $ g$ and air resistance.]
\item A $ 100$ V battery terminals are connected to two large and parallel plates which are $ 2$ cm apart. The field in the region between the plates is nearly uniform. 
 \begin{itemize}
\item If electric field intensity $ E$ is $ 10^{6}$ N C$ ^{-1}$ and points vertically upwards, determine the force of an electron in this field and compare it with the weight of an electron. An electron is released from rest from the upper plate inside the field above.
\item At what velocity will it hit the lower plate?
\item Determine its kinetic energy and the time it takes for the whole journey.
\end{itemize}
\item Mention any three uses of a transistor
\item A certain transistor has a current gain $  \beta =55$ . If it is used in a circuit with common-base configuration, how much change occurs in the collector current if an emitter current is changed by $ 100$ micro A? (Assume the collector potential to be constant and neglect the small collector — current due to the minority current carriers).
\item What is an operational amplifier 
\item List three desirable features of an operational amplifier.
\item In almost all cases, where an operation amplifier is used as a linear voltage amplifier, negative feedback is employed. State the advantage of negative feedback.
\item Using an example of your own choice explain the mechanism behind the production of a laser beam.
\item Describe two applications of a laser
\item A proton is moving in a uniform magnetic field $ B$ . Draw the diagram representing $ B$ and the path of the proton if its initial direction makes an oblique angle to the direction of the field $ B$ . 
\item In the Bohr model of the hydrogen atom, an electron circles the nucleus in an orbit of radius $ r$
 \begin{itemize}
\item Explain what keeps the electron in the orbit and why it does not spiral towards the nucleus.
\item What are the assumptions put forward by Bohr about the orbits of the electron in the hydrogen atom?
\end{itemize}
\item A sample of soil from Olduvai Gorge cave was examined. It was found to contain, among other things, pieces of charcoal. Further investigation on the charcoal revealed that $ 1$ kg of C$ 14$ nuclei decayed each second. It is assumed that this charcoal has resulted from decomposition of the stone-age people who died there (i.e. at the cave) long time ago. Calculate the number of years that have elapsed since these people died.
\item What is the de Broglie wave equation?
\item An electron is accelerated through a potential of $ 400$ V. Determine the de Broglie wavelength of this electron.
\item Determine the de Broglie wavelength for the beam of electron whose total energy is
 \begin{itemize}
\item $ 250$ eV.
\end{itemize}
\item What is a photoelectric cell?
\item The emission of electrons from the surface of a cathode of a certain phototube when irradiated with a light of wavelength $ 3500 \times 10^{-10}$ m is found to stop when the plate potential is $ 1.2$ V with respect to the cathode. Determine the work function of the cathode.
\end{itemize}

\end{document}
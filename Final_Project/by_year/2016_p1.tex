
\documentclass{article}
\usepackage[a4paper, total={6in, 8in}]{geometry}
% \usepackage[utf8]{inputenc}
\usepackage{abstract}
\title{THE UNITED REPUBLIC OF TANZANIA

NATIONAL EXAMINATIONS COUNCIL

ADVANCED CERTIFICATE OF SECONDARY EDUCATION EXAMINATION

\textbf{2016 PHYSICS 1}}
\author{Transcribed by:  PJ Gibson}

\begin{document}

\maketitle

\begin{itemize}
\item Define the term dimension of a physical quantity.
\item The number of particles $ n$ crossing a unit area perpendicular to $ x-$ axis in a unit time is given as $ n=-D(n_{2}-n_{1})/(x_{2}-x_{1})$ where $ n_{1}$ and $ n_{2}$ are the number of particles per unit volume for the values of $ x_{1}$ and $ x_{2}$ respectively.  What are the dimensions of diffusion constant $ D$ ?
\item Give two basic rules of dimensional analysis. 
\item The frequency, $ f$ of a vibrating string depends upon the force applied, $ F$ , the length, $ l$ , of the string and the mass per unit length,$ \mu $ . Using dimension show how $ f$ is related to $ F$ , $ l$ and $ \mu $ .
\item What is meant by least count of a measurement?
\item The period of oscillation of a simple pendulum is given by $ T=2\pi\sqrt{l/g}$ where by $ 100$ vibrations were taken to measure $ 200$ seconds. If the least count for the time and length of a pendulum of $ 1$ m are $ 0.1$ sec and $ 1$ mm respectively, calculate the maximum percentage error in the measurement of $ g$ .
\item Mention two characteristics of projectile motion.
\item If the range of the projectile is $ 120$ m and its time of flight is $ 4$ sec , determine the angle of projection and its initial velocity of projection assuming that the acceleration due to gravity $ g=10$ m$/$s. 
\item State the principles on which the rocket propulsion is based. 
\item A jet engine on a test bed takes in $ 40$ kg of air per second at a velocity of $ 100$ m$/$s  and burns $ 0.80$ kg of fuel per second. After compression and heating the exhaust gases are ejected at $ 600$ m$/$s relative to the air craft. Calculate the thrust of the engine.
\item An object of mass $ 2$ kg is attached to the hook of a spring balance which is suspended vertically to the roof of a lift.  What is the reading on the spring balance when the lift is:
 \begin{itemize}
\item going up with the rate of $ 0.2$ m$/$s$ ^{2}$
\item going down with an acceleration of $ 0.1$ m$/$s$ ^{2}$
\item ascending with uniform velocity of $ 0.15$ m$/$s
\end{itemize}
\item Define the term inertia.
\item Why is Newton’s first law of motion called the law of inertia?
\item  A jet of of water from a fire hose is capable of reaching a height of $ 20$ m.  If the cross sectional area of the hose outlet is $ 4.0	\times 10^{-4}$ m$ ^{2}$ , calculate the:
 \begin{itemize}
\item Minimum speed of water from the hose.
\item Mass of water leaving the hose each second.
\item Force on the hose due to the water jet.
\end{itemize}
\item A boy ties a string around a stone of mass $ 0.15$ kg and then whirls it in a horizontal circle at constant speed. If the period of rotation of the stone is $ 0.4$ sec and the length between the stone and boy’s hand is $ 0.50$ m ;
 \begin{itemize}
\item Calculate the tension in the string. 
\item State one assumption taken to reach the answer above.
\end{itemize}
\item What do you understand by the following terms: 
 \begin{itemize}
\item Damped oscillations. 
\item Undamped oscillations.
\end{itemize}
\item Sketch the waveform diagrams to represent the terms: damped oscillations & undamped oscillations
\item Show that the total energy of a body executing S.H.M. is independent of time.
\item A mass of $ 05$ kg connected to a light spring of force constant $ 20$ N$/$m oscillates on a  horizontal frictionless surface. If the amplitude of the motion $ 1$ s $ 3.0$ cm , calculate the;
 \begin{itemize}
\item Maximum speed of the mass.
\item  Kinetic energy of the system when the displacement is $ 2.0$ cm.
\end{itemize}
\item What is meant by moment of inertia of a body?
\item List two factors on which the moment of inertia of a body depends. 
\item A thin sheet of aluminum of mass $ 0.032$ kg has the length of $ 0.25$ m and width of $ 0.1$ m. Find its moment of inertia on the plane about an axis parallel to the:
 \begin{itemize}
\item Length and passing through its centre of mass, $ m$ .
\item Width and passing through the centre of mass, $ m$ , in its own plane.
\end{itemize}
\item Define the term angular momentum.
\item A thin circular ring of mass, $ M$ , and radius, $ r$ , is rotating about its axis with constant angular velocity, $ w_{1}$ .  If two objects each of mass, $ m$ , are attached gently at the ring, what will be the angular velocity of the rotating wheel?
\item Mention one application of parking orbit.
\item Briefly explain how parking orbit of a satellite is achieved.
\item The earth satellite revolves in a circular orbit at a height of $ 300$ km above the earth’s surface.  Find the; 
 \begin{itemize}
\item Velocity of the satellite
\item Period of the satellite.
\end{itemize}
\item Why are space rockets usually launched from west to east?
\item A spaceship is launched into a circular orbit close to the earth’s surface.  What additional velocity has to be imparted on the spaceship if order to overcome the gravitational pull?
\item Briefly explain why: 
 \begin{itemize}
\item A body with large reflectivity is a poor emitter. 
\item The earth without its atmosphere would be too cold to live.
\end{itemize}
\item Identify two factors on which the coefficient of thermal conductivity of a material depend.
\item A brass boiler of base area $ 1.50\times 10^{-1}$ and thickness $ 1.0$ cm boils water at a rate of $ 6.0$ kg$/$min when placed on a gas Stove. Estimate the temperature of the part of the flame in contact with the boiler.
\item Briefly describe the working principle of a thermocouple. 
\item In a certain thermocouple thermometer the e.m.f. is given by $ E= a \theta + 1/2 b\theta^{2}$ where $ \theta $ is the temperature of hot junction. If a$ =10 $ mV$ ^{\circ}C^{-2}$ , $ b=-1/20 $ mV$ ^{\circ}C^{-2}$ and the cold junction is at $ 0^{\circ}$C, calculate the neutral temperature. 
\item What is meant by thermal radiation?
\item Briefly explain why forced convection is necessary for excess temperate less than $ 20$ K? 
\item Why is the energy of thermal radiation less than that of visible light? 
\item A body with a surface area of $ 5.0 $ cm$ ^{2}$ and a temperature of $ 727^{\circ}$C radiates $ 300$ joules of energy in one minute. Calculate its emissivity.
\item State Newton’s law of cooling. 
\item A body cools from $ 70^{\circ}$C to $ 40^{\circ}$C in $ 5$ minutes. If the temperature of the surroundings is $ 10^{\circ}$C , Calculate the time it takes to cool from $ 50^{\circ}$C to $ 20^{\circ}$C.  
\item Define the term junction as applied in electrical network.
\item What ts the physical significance of Kirchhoff’s first law.
\item Why is Kirchhoff’s second law sometimes referred to as the voltage law?
\item List down five points to be considered when applying Kirchhoff’s second law in formulating analytical problems or equations.
\item What is meant by the following terms:
 \begin{itemize}
\item  Phase of alternating e.m.f.
\item  Root mean square (r.m.s.) value of alternating e.m.f.
\end{itemize}
\item An a.c. circuit consists of a pure resistance of $ 10\Omega $ is connected across an a.c. supply of $ 230$ V , $ 50$ Hz.  Calculate the;
 \begin{itemize}
\item Current flowing in the circuit.
\item Power dissipated
\end{itemize}
\item A $ 25$ $\mu$F capacitor, a $ 0.10$ H inductor and a $ 25\Omega $ resistor are connected in series with an a.c. source whose e.m.f. is given by $ E=310 \sin(314t)$ .  Determine the;
 \begin{itemize}
\item Frequency of the e.m.f.
\item Net reactance of the circuit.
\end{itemize}
\item What is the importance of doping as applied to semiconductors?
\item Distinguish between $ n-$ type and $ p-$ type semiconductors.  Give three points.
\item Why are transistors mostly used in common emitter arrangement?
\item When does a transistor amplifier work as an oscillator?
\item Explain the use of an op-amp as a summing amplifier.
\item Name three electronic circuits in which multivibrators can be constructed.
 \begin{itemize}
\item List down three types of multivibrators.
\item Briefly explain the applications of multivibrators listed above.
\end{itemize}
\item Mention two characteristics of op-amps.
\item Briefly explain why op-amps are sometimes called differential amplifiers?
\item Discuss the mode of action of each of the following sensors:
 \begin{itemize}
\item Thermistor (TH).
\item Light Dependent Resistor (LDR).
\end{itemize}
\item Give symbols, expressions and truth tables for each of the following logic gates: 
 \begin{itemize}
\item NAND gate .
\item Exclusive NOR gate.
\end{itemize}
\item Why is NAND gate considered as basic building block for a variety of logic circuits?
\item What is meant by aerial environment?  Give two examples.
\item Describe three ways at which the aerial environment is threatened.
\item Briefly explain three major concepts on solar wind.
\item How do soil environmental components influence plant growth? Give four points.
\end{itemize}

\end{document}
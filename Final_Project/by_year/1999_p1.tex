
\documentclass{article}
\usepackage[a4paper, total={6in, 8in}]{geometry}
% \usepackage[utf8]{inputenc}
\usepackage{abstract}
\title{THE UNITED REPUBLIC OF TANZANIA

NATIONAL EXAMINATIONS COUNCIL

ADVANCED CERTIFICATE OF SECONDARY EDUCATION EXAMINATION

\textbf{1999 PHYSICS 1}}
\author{Transcribed by:  PJ Gibson}

\begin{document}

\maketitle

\begin{itemize}
\item Mention two applications and two limitations of dimensional analysis.
\item The frequency $ f$ of a note produced by a taut wire stretched between two supports depends on the distance ​ $ l$ ​ between the supports, the mass per unit length of the wire,$ \mu $ , and the tension $ T$ . Use dimensional analysis to find how $ f$ is related to ​ $ l$ ​ , $\mu$, and $ T$ .
\item Define momentum
\item Define impulse of a force
\item A jet of water emerges from a hose pipe of a cross-sectional area $ 5.0\times 10^{-3}$ m​$ ^{2}$ with a velocity of $ 3.0$ m$/$s and strikes a wall at right angle. Assuming the water to be brought to rest by the wall and does not rebound, calculate the force on the wall.
\item What do you understand by the term escape velocity?
\item Calculate the escape velocity from the moon’s surface given that a man on the moon has $ 1/6$ his weight on earth. The mean radius of the moon is $ 1.75 \times 10^6$ m.
\item Give two similarities between simple harmonic motion and circular motion.
\item On the same set of axes, sketch how energy exchange (kinetic to potential) takes place in an oscillator placed in a damping medium.
\item State the parallel axis theorem.
\item Show that the Kinetic energy (K.E.) of rotation of a rigid body about an axis with a constant angular velocity $ w$ is given by $ KE =1/2Iw^{2}$ where i is the moment of inertia of the rigid body about the given axis.
\item Distinguish between static and dynamic friction.
\item With the help of a well labelled diagram briefly explain how you will determine the coefficient of viscosity of a liquid by a constant pressure head apparatus in the laboratory.
\item Explain in terms of surface energy, what is meant by the surface tension, ​ $ \gamma $ ​ of a liquid. 
\item What energy is required to form a soap bubble of radius $ 1.00$ mm if the surface tension of the soap solution is $ 2.5 \times 10$ ​$ E-4$ ​ N$/$m$ ^{2}$ ​ ?
\item Write down the equation of continuity of a fluid defining all your symbols.
\item The velocity at a certain point in a flow pipe is $ 1.0$ ms$ ^{-1}$ and the gauge pressure there is $ 3 \times 10^5 $ N$/$m$ ^{2}$ ​ . The cross-sectional area at a point $ 10$ m above the first is half that at the first point. If the flowing fluid is pure water, calculate the gauge pressure at the second point.
\item What do you understand by the term: Thermodynamic temperature scale
\item What do you understand by the term:   Triple point of water
\item The resistance of a platinum wire at a temperature T​$ ^{\circ}$C measured on a gas scale is given by $ R(T)=R​_{0​}(1+ a T+bT​^{2}​)$ .
 \begin{itemize}
\item What temperature will the platinum thermometer indicate when the temperature on the gas scale is $ 200​^{\circ}$C ? (take a $ =3.8 \times 10^{-3}$ ​ and $ b=-5.6 \times 10^{-7}$ ​)
\end{itemize}
\item What is the coefficient of thermal conductivity of a material?
\item The temperature difference between the inside and outside of a room is $ 25​^{\circ}$C . The room has a window of an area $ 2$ m$ ​^{2}$ and the thickness of the window material is $ 2$ mm. Calculate the heat flow through the window if the coefficient of thermal conductivity of the window material is $ 0.5$ SI units.
\item Write down the equation of state of an ideal gas defining all the symbols used.
\item If the root-mean-square velocity of a hydrogen molecule at $ 0​ ^{\circ}$C is $ 1840$ m$/$s, find the root-mean-square velocity of the molecule at $ 100​ ^{\circ}$ ​ $ C$ .
\item What is the difference between refraction and diffraction as applied to waves?
\item A parallel beam containing two wavelengths $ 600$ nm and $ 602$ nm is incident on a diffraction grating with $ 400$ lines per mm. Calculate the angular separation of the first order spectrum of the two wavelengths. ($ 1$ nm $ =10^{-9}$ m)
\item What is a “Doppler Effect”?
\item A whistle sound of frequency $ 1200$ Hz was directed to an approaching train moving at $ 48$ km$/$h​ . The whistle-man then listened to the beats between the emitted sound and that reflected from the train. What is the beat frequency detected by the whistle-man?
\item Explain why an uncharged metal is attracted by a charged one?
\item Charges $ Q​_{1}=1.2 \times 10$ ​$ E-12$ C and $ Q​_{2}=-4 \times 10$ ​ $ -12$ C are placed $ 5.0$ m apart in air. A third charge $ Q​_{3}=1 \times 10^{-14}$ C is introduced midway between them. Find the resultant force on the third charge.
\item State Kirchhoff’s laws of circuit analysis
\item Write down an expression for the forces on an electron when moving perpendicular to: an electric field
 \begin{itemize}
\item Write down an expression for the forces on an electron when moving perpendicular to: a magnetic field.
\end{itemize}
\item An electron is moving in a uniform electric field of intensity $ 1.2 \times 10^5$ Vm​ $ -1$ .  Find the acceleration of the electron.
\item What is a resonant frequency of an oscillator?
\item Draw the symbol of $ n-p-n$ transistor.
\item Distinguish between insulators, semi-conductors and metals as far as conduction is concerned.
\item What is the “work function” of a metal?
\item The work function of a metal is $ 2.0$ eV. Calculate the stopping potential when the metal is illuminated by light of frequency of $ 6.0 \times 10^{14}$ Hz.
\item What is nuclear fusion 
\item What is nuclear fission?
\end{itemize}

\end{document}
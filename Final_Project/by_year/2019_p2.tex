
\documentclass{article}
\usepackage[a4paper, total={6in, 8in}]{geometry}
% \usepackage[utf8]{inputenc}
\usepackage{abstract}
\title{THE UNITED REPUBLIC OF TANZANIA

NATIONAL EXAMINATIONS COUNCIL

ADVANCED CERTIFICATE OF SECONDARY EDUCATION EXAMINATION

\textbf{2019 PHYSICS 2}}
\author{Transcribed by:  PJ Gibson}

\begin{document}

\maketitle

\begin{itemize}
\item Give the meaning of the terms velocity gradient, tangential stress and coefficient of viscosity as used in fluid dynamics.
\item Write Stokes’ equation defining clearly the meaning of all symbols used.
 \begin{itemize}
\item State two assumptions used to develop the equation above
\end{itemize}
\item Calculate the terminal velocity of the rain drops falling in air assuming that the flow is laminar, the rain drops are spheres of diameter $ 1$ mm and the coefficient of viscosity, $ \eta =1.8 \times 10^{-5}$ Ns$/$m$ ^{2}$ . 
\item Water flows past a horizontal plate of area $ 1.2$ m$ ^{2}$ . If its velocity gradient and coefficient of viscosity adjacent to the plate are $ 10$ s$ ^{-1}$ and $ 1.3 \times 10^{-5}$ Ns$/$m$ ^{2}$ respectively, calculate the force acting on the plate.  
\item A horizontal pipe of cross - sectional area $ 10 $ cm$ ^{2}$ has one section of cross sectional area $ 5 $ cm$ ^{2}$ . If water flows through the pipe, and the pressure difference between the two sections is $ 300$ Pa, how many cubic meters of water will flow out of the pipe in $ 1$ minute?
\item Provide one evidence which proves that sound is a wave.
\item Why thunder of lightning is heard some moments after seeing the flash?
\item What is Doppler effect? 
\item The cyclist moving at $ 10$ m$/$s and the railway train at $ 20$ m$/$s are approaching each other. If the engine driver sounds a warming siren at a frequency of $ 480$ Hz:
 \begin{itemize}
\item calculate the frequency of the note heard by the cyclist before and after the train has passed away. 
\end{itemize}
\item Two sheets of a Polaroid are lined up so that their polarization directions are initially parallel. When one sheet is rotated:
 \begin{itemize}
\item How does the transmitted light intensity vary with the angle between the polarization directions of the polaroid? 
\item What angle must the polaroid be rotated to reduce the light Intensity by $ 50\%$ ?
\end{itemize}
\item Give the meaning of the terms wave function, longitudinal wave and transverse waves.
\item The equation of a Progressive wave traveling in the $ +x$ direction is given by $ y= a \sin(\omega t-kx)$ .  Show that the maximum velocity, $ V_{max}=2\pi a /T$ . 
\item What is meant by diffraction grating?
\item A diffraction grating has $ 500$ lines per millimetre when used with monochromatic light of wavelength $ 6 \times 10^{-7}$ m at normal incidence. Determine the angle at which the bright diffraction images will be observed. 
 \begin{itemize}
\item Why other orders of image above can not be observed? 
\end{itemize}
\item State Huygens’s principle of wave construction.
\item A lens was placed with a convex surface of radius of curvature $ 50.0$ cm in contact with the plane surface such that Newton’s rings were observed when the lens was illuminated with monochromatic light. If the radius of the $ 15$ th ring was $ 2.13$ mm determine the wavelength. 
\item Define Young’s Modulus of a material. 
\item Why work is said to be done in stretching a wire? 
\item A steel wire AB of the length $ 60$ cm and cross-sectional area $ 1.5 \times 10^{-6}$ m$ ^{2}$ is attached at $ B$ to copper wire BC of length $ 39$ cm and cross sectional area $ 3.0 \times 10^{-6}$ m$ ^{2}$ . If the combination of the two wires is suspended vertically from a fixed point at A, and supports a weight of $ 250$ N at $ C$ ; find the extension (in millimeter) of the:
 \begin{itemize}
\item steel wire. 
\item copper wire. 
\end{itemize}
\item Based on the kinetic theory of gases determine:
 \begin{itemize}
\item The average translational kinetic energy of air at a temperature of $ 290$ K.
\item The root mean square seed (r.m.s) of air at the same temperature (above).
\end{itemize}
\item Define the terms electric potential and electric field-strength $ E$ at a point in the electrostatic field.
 \begin{itemize}
\item How the two quantities above related? 
\end{itemize}
\item Outside the sphere, a charged sphere behaves like its charges were concentrated at the centre. If the electric field strength inside the sphere is zero and one sphere of radius $ 5.0$ cm carries a positive charge of $ 6.7$ nC, calculate; 
 \begin{itemize}
\item the potential at the surface of the sphere. 
\item the capacitance of the sphere. 
\end{itemize}
\item What is meant by dielectric constant? 
\item State Coulomb’s law of force between two electrically charged bodies. 
\item Can there be a potential difference between two adjacent conductors carrying the same positive charge? Give a reason. 
\item A parallel plate capacitor with air as a dielectric has plates of area $ 4.0 \times 10^{-2}$ m$ ^{2}$ which are $ 2.0$ mm apart. The capacitor is charged to $ 100$ V battery and connected in parallel with a similar unchanged capacitor with plates of half the area and twice the distance apart. If the edge effect is neglected, calculate the final charge on each plate. 
\item Derive an expression for the total capacitance of two capacitors $ C_{1}$ and $ C_{2}$ connected in series. 
\item Two capacitor of $ 15$ $\mu$F and $ 20$ $\mu$F are connected in series with a $ 600$ V supply.  Calculate the charge and Potential difference across each capacitor. 
\item Based on Balmer series of hydrogen spectra determine the wavelength of the series limit of Paschen series. 
\item Why electrons do not fall into the nucleus due to electrostatic force of attraction?
\item Why hydrogen atom is stable in the ground state? 
\item According to Bohr’s theory, the angular momentum of an electron is an integral multiple of $ h/2\pi$ .  Express this statement. by using a mathematical equation in which angular momentum is represented by the letter Land orbit by the letter $ n$ , 
\item Determine the angular momentum of the electron in the orbit of energy level $ -3.4$ eV given that $ E_{n}=-13.6/n^{2}$ eV, where $ E$ is the energy of an electron and $ n$ is the principal quantum number of hydrogen atom. 
\item What is meant by the following terms as used in nuclear Physics?
 \begin{itemize}
\item Mass defect 
\item Binding energy. 
\end{itemize}
\item Elaborate two aspects on which fission reactions differs from fusion reactions.
\item Why is high temperature required to cause nuclear fusion? 
\item Identify four factors that affect the force experienced by a current-carrying conductor in a magnetic field. 
\item Write the mathematical expression which define magnetic flux density and use it to deduce its S.I. units. 
 \begin{itemize}
\item Apply an expression obtained above to develop the formula for the force on a conductor carrying current i if the conductor and the magnetic fields are not at night angles.
\end{itemize}
\item Distinguish the terms magnetically soft and magnetically hard materials.
\item State the condition which makes the magnetic force on a moving charge in a magnetic field to be maximum. 
\item Determine the magnitude of force experienced by a stationary charge in a uniform magnetic field. 
\item At which position of the rotating coil in the magnetic field, the induced e.m.f. is zero? Give a reason. 
\item Use mathematical expression to justify the statement that there will be no change in the kinetic energy of a charged particle which enters a uniform magnetic field when its initial velocity is directed parallel to the field.

\end{itemize}

\end{document}
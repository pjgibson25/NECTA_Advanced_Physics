
\documentclass{article}
\usepackage[a4paper, total={6in, 8in}]{geometry}
% \usepackage[utf8]{inputenc}
\usepackage{abstract}
\title{THE UNITED REPUBLIC OF TANZANIA

NATIONAL EXAMINATIONS COUNCIL

ADVANCED CERTIFICATE OF SECONDARY EDUCATION EXAMINATION

\textbf{2007 PHYSICS 1}}
\author{Transcribed by:  PJ Gibson}

\begin{document}

\maketitle

\begin{itemize}
\item What is systematic error?
\item The smallest divisions for the voltmeter and ammeter are $ 0.1$ V and $ 0.01$ A respectively.  If $ V=IR$ , find the relative error in the resistance $ R$ , when $ V=2$ V and $ I=0.1$ A.
\item Mention two$ (2)$ uses of dimensional analysis.
\item The frequency $ f$ of a note given by an organ pipe depends on the length, $ l$ , the air pressure $ P$ and the air density $ D$ .  Use the method of dimensions to find a formula for the frequency.
 \begin{itemize}
\item What will be the new frequency of a pipe whos original frequency was $ 256$ Hz if the air density falls by $ 2\%$ and the pressure increases by $ 1\%$ ?
\end{itemize}
\item What is meant by the term "projectile" as applied to projectile motion?
\item Give two $ (2)$ practical applications of projectile motion at your locality.
\item A ball is thrown towards a vertical wall from a point $ 2$ m above the ground and $ 3$ m from the wall.  The initial velocity of the ball is $ 20$ m$/$s at an angle of $ 30$ deg above the horizontal.  If the collision of the ball with the wall is perfectly elastic, how far behind the thrower does the ball hit the ground?
\item The ceiling of a long hall is $ 25$ m high.  Determine the maximum horizontal distance that a ball thrown with a speed of $ 40$ m$/$s can go without hitting the ceiling of the wall.
\item Explain why when catching a fast moving ball, the hands are drawn back will the ball is being brought to rest.
\item Rockets are propelled by the ejection of the products of the combustion of fuel.  Consider a rocket of total mass $ M$ travelling at a speed v in a region of space where the gravitational forces are negligible.  
\item Supposing the combustion products are ejected at a constant speed v, relative to the rocket, show that a fuel "burn" which reduces the total mass $ M$ of the rocket to $ m$ results in an increase in the speed of the rocket to v such that $ v-V=V_{f} \ln (M/m)$ .
\item Supposing that $ 2.1\times10^{6}$ kg of fuel are consumed during a "burn" lasting $ 1.5\times10^{2}$ seconds and given that there is a constant force on the rocket of $ 3.4\times 10^{7}$ N during this burn, calculate v, and increase in speed resulting from the burn if $ M=2.8\times10^{6}$ kg.  
\item What is the initial vertical acceleration that can be imparted to this rocket when it is launched from the Earth if the initital mass is $ 2.8\times 10^{6}$ kg?
\item What is meant by centripetal force?
\item Derive the expression $ a =(v^{2}/r)$ where a, v, and $ r$ stands for the centripetal acceleration, linear velocity and radius of a circular path respectively.  
\item A ball of mass $ 0.5$ kg attached to a light inextensible string rotates in a vertical circle of radius $ 0.75$ m such that it has a speed of $ 5$ m$/$s when the string is horizontal.  Calculate:
 \begin{itemize}
\item  The speed of the ball and the tension in the string at the lowest point of its circular path.
\end{itemize}
\item Evaluate the work done by the Earth's gravitational force and by the tension in the string as the ball moves from its highest to its lowest point.
\item What is meant by a thermometric property of a substance?
\item What qualities make a particular property suitable for use in practical thermometers?
\item Explain why at least two $ (2)$ fixed points are required to define a temperature scale.
\item Mention the type of thermometer which is most suitable for calibration of thermometers.
\item When a metal cylinder of mass $ 2.0x10^{-2}$ kg and specific heat capacity $ 500$ J$/$kgK is heated at constant power, the initial rate of rise of temperature is $ 3.0$ K$/$min.  After a time the heater is switched off and the initial rate of fall of temperature is $ 0.3$ K$/$min.  What is the rate at which the cylinder gains heat energy immediately before the heater is switched off?
\item What is blackbody radiation of a given body?
\item Explain why heat may just mean infrared.
\item State Prvost's theory of heat exchange.
\item Explain why in cold climates, windows of modern buildings are double glazed, ie: There are two pieces of glass with a small air space between them.
\item What is Wien's displacement law?
\item The sun's surface temperature is about $ 6000$ K.  The sun's radiation is maximum at wavelength of $ 0.5\times10^{-6}m$ .  A certain light bulb filament emits radiation with maximum wavelength of $ 2\times10^{-6}m$ .  If both the surface of the sun and of the filament have the same emissive characteristics, what is the temperature of the filament?
\item State Newton’s law of cooling and give one limitation of the law.
\item A body initially at $ 70^{\circ}$C cools to a temperature of $ 55^{\circ}$C in $ 5$ minutes. What will be its temperature after $ 10$ minutes given that the surrounding temperature is $ 31^{\circ}$C ? (Assume Newton’s law of cooling holds true)
\item Give two $ (2)$ differences between progressive and standing waves.
\item Two progressive waves travelling along the same line in a medium are represented by $ Y_{1}=10 \sin(\omega t +\pi/2)$ and $ Y_{2}=10 \sin(\omega t +\pi/6)$
 \begin{itemize}
\item If the two progressive waves form a standing wave, determine the resultant amplitude and phase angle of the wave formed.
\end{itemize}
\item State the modes of vibrations in closed and open pipes.  
\item A metre-long tube at one end, with a movable piston at the other end, shows resonance with a fixed frequency source (a tuning fork) of frequency $ 340$ Hz when the tube length is $ 25.5$ cm or $ 79.3$ cm.  Estimate the speed of sound in air at the temperature of the experiment (ignore edge effects).
\item The shortest length of the resonance tube closed at one end which resounds to a fork of frequency $ 256$ Hz is $ 32.0$ cm.  The corresponding length for a fork of frequency $ 384$ Hz is $ 20.8$ cm.  Determine the end correction for the tube and the velocity of sound in air.
\item Define the internal resistance (r) of a cell and the terminal potential difference.
\item The e.m.f. of a cell is a special terminal potential difference.  Comment.
\item State Kirchhoff's laws of electrical network.
\item List three $ (3)$ classes of magnetic materials on the basis of magnetic susceptibility and give one example for each class.
\item How are the magnetic susceptibility and relative permeability of a magnetic material related to each other?
\item Define the magnetic field intensity.
\item A long solenoid has $ 10$ turns per cm and carries a current of $ 2.0$ A.  Calculate the magnetic field intensity at its centre.
\item An a.c. generator consists of a coil of $ 50$ turns and an area of $ 2.5$ m$ ^{2}$ , rotates at an angular speed of $ 60$ rad$/$s in a uniform magnetic field of $ 0.30$ T between two fixed pole pieces.  The resistance of the circuit including that of the coil is $ 500\Omega $ .  
 \begin{itemize}
\item  What is the maximum current that can be drawn from the generator?
\item  What is the magnetic flux through the coil if the current is maximum?
\end{itemize}
\item How does the arrangement of the energy level in a semiconductor differ from that of an insulator?
\item Using the notation of energy bands, explain the following optical properties of solids.
 \begin{itemize}
\item  All metals are opaque to light of all wavelengths.
\item  Semi-conductors are transparent to infrared light although opaque to visible light.
\item  Most insulators are transparent to visible light.
\end{itemize}
\item What is the potential at the centre of the square of side $ 1.0$ m, due to charges:
 \begin{itemize}
\item $ q_{1}=+1.0\times10^{-8}$ C , $ q_{2}=-2.0\times10^{-8}$ C , $ q_{3}=+3.0\times10^{-8}$ C , $ q_{1}= +2.0\times10^{-8}$ C
\item situated at the corners of the square?
\end{itemize}
\item What do you understand by an electrostatic generator?
\item The belt of a Van de Graaf generator carries a charge of $ 100$ $\mu$C per metre.  If the diameter of the lower pulley is $ 10$ cm and its angular velocity is $ 5$ rad$/$s, what p.d. will the upper conductor attain in $ 5$ minutes if its capacitance to ground is $ 5x10^{-12}$ F and if there is no leakage of charge?
\item Two similar balls of mass $ m$ are hung from silk thread of length "a" and carry a similar charge $ q$ .  Assume $ \theta $ is small enough that $ X = (\frac{q^2 a}{2 \pi \epsilon_0 m g})^{1/3}$
 \begin{itemize}
\item where $ X$ is the distance of separation.
\end{itemize}
\item A charge $ Q$ is distributed over the concentric hollow spheres of radii $ r$ and $ R$ $ (R>r)$ such that the surface densities are the same.  Calculate the potential at the common centre of the two spheres.
\item Make well labelled diagram of the cathode ray oscilloscope and explain briefly how a sinusoidal voltage signal is displayed on its screen.
\item Mention three $ (3)$ practical applications of the cathode ray oscilloscope.
\item An electron having $ 450$ eV of energy enters at right angles to a uniform magnetic field of strength $ 1.50x10^{-3}$ T.  Show that the path traced by the electron in a uniform magnetic field is circular and estimate its radius.
\item A charged oil drop of mass $ 6.0x10^{-15}$ kg falls vertically in air with a steady velocity between two long parallel vertical plates $ 5.0$ mm apart.  When a potential difference of $ 3000$ V is applied between the plates the drop falls with a steady velocity at an angle of $ 58^{\circ}$ to the vertical.
 \begin{itemize}
\item Determine the charge $ Q$ , on the oil drop.
\end{itemize}
\item What are the difference between $ P$ and s waves?
\item Explain how the two terms of waves ($ P$ and $ S$ ) can be used in studying the internal structure of the earth. 
\item Write short notes on the following terms in relation to changes in the Earth's magnetic field:  long-term (secular) changes, short-period (regular) changes and short-term (irregular) changes.
\item What is geomagnetic micropulsation.
\item Give a summary of location, constitution and practical uses of the stratosphere, ionosphere, and mesosphere.
\end{itemize}

\end{document}

\documentclass{article}
\usepackage[a4paper, total={6in, 8in}]{geometry}
% \usepackage[utf8]{inputenc}
\usepackage{abstract}
\title{THE UNITED REPUBLIC OF TANZANIA

NATIONAL EXAMINATIONS COUNCIL

ADVANCED CERTIFICATE OF SECONDARY EDUCATION EXAMINATION

\textbf{2000 PHYSICS 1}}
\author{Transcribed by:  PJ Gibson}

\begin{document}

\maketitle

\begin{itemize}
\item What is an error? Mention two causes of systematic and two causes of random errors.
\item The pressure $ P$ is calculated from the relation $ P=F/( \pi R^{2})$ where $ F$ is the force and $ R$ the radius. If the percentage possible errors are $ +2\%$ for $ F$ and $ +1\%$ for $ R$ . Calculate the possible percentage error for $ P$ .
\item The speed v of a wave is found to depend on the tension $ T$ in the string and the mass per unit length $ u$ (linear mass density). Using dimensional analysis derive the relationship between v, $ T$ and $ u$ .
\item The longitudinal wave speed in gases is given by $ v=\sqrt{\gamma p/ \rho }$ ; where $ \gamma =C_{p}/C_{v}$ , $ P$ is the pressure and $ \rho $ the density of gas. If $ v_{1,}$ and $ v_{2,}$ are the speeds of sound in air at temperature $ T_{1}$ and $ T_{2}$ respectively, show that $ v_{1/v_2}=\sqrt{T_{1}/T_{2}}$
 \begin{itemize}
\item NOTE: $ C_{p}$ and $ C_{v}$ are the specific heats of the gas at constant pressure and constant volume respectively.
\end{itemize}
\item Show that the period of a body of mass $ m$ revolving in a horizontal circle with constant velocity v at the end of a string of length $ l$ is independent of the mass of the object.
\item A ball of mass $ 100$ g is attached to the end of a string and is swung in a circle of radius $ 100$ cm at a constant velocity of $ 200$ cm$/$s. While in motion the string is shortened to $ 50$ cm. Calculate:
 \begin{itemize}
\item The new velocity of the motion.
\item The new period of the motion.
\end{itemize}
\item A car travels over a humpback bridge of radius of curvature $ 45$ m. Calculate the maximum speed of the car if the wheels are to remain in contact with the bridge.
\item Mention two motions that add up to make projectile motion.
\item In long jumps does it matter how high you jump? State the factors which determine the span of the jump. 
\item Derive an expression that relates the span of the jump and the factors you have mentioned.
\item A bullet is fired from a gun on the top of a cliff $ 140$ m high with a velocity of $ 150$ m$/$s at an elevation of $ 30^{\circ}$ to the horizontal. Find the horizontal distance from the foot of a cliff to the point where the bullet lands on the ground.
\item Define simple harmonic motion.
\item Two simple pendulums of length $ 0.4$ m and $ 0.6$ m respectively are set oscillating in step. 
 \begin{itemize}
\item After what further time will the two pendulums be in step again? 
\item Find the number of oscillations made by each pendulum during the time found above.
\end{itemize}
\item Cite two examples of SHM which are of importance to everyday life experience.
\item What does one require in order to establish a scale of temperature?
\item A copper-constantan thermocouple with its cold junction at $ 0^{\circ}$C had an emf of $ 4.28$ mV when its other hot junction was at $ 100^{\circ}$C. The emf became $ 9.29$ mV when the temperature of the hot junction was $ 200^{\circ}$C. If the emf $ E$ is related to the temperature difference $ 8$ between hot and cold junctions by the equation $ E= A(\theta )+B(\theta ^{2})$ , calculate:
 \begin{itemize}
\item The values of $ A$ and $ B$ .
\item The range of temperature for which $ E$ may be assumed proportional to $ 8$ without incurring an error of more than $ 1\%$ .
\end{itemize}
\item The resistance $ R$ , of a platinum varies with temperature $ t$ according to the equation $ R_{t}=R_{o}(1+8000bt -b t^{2})$ where $ b$ is a constant. Calculate the temperature on platinum scale corresponding to $ 400^{\circ}$C on the gas scale. 
\item Define the thermal conductivity of a material
\item Write down a formula for the rate of cooling under natural convection and define all the symbols used. 
\item Heat is supplied at a rate of $ 80$ W to one end of a well lagged copper bar of uniform cross section area $ 10$ cm? having a total length of $ 20$ cm. The heat is removed by water cooling at the other end of the bar. Temperature recorded by two thermometers $ T_{1}$ and $ T_{2}$ at distances $ 5$ cm and $ 15$ cm from the hot end are $ 48^{\circ}$C and $ 28^{\circ}$C respectively.
 \begin{itemize}
\item Calculate the thermal conductivity of copper.
\item Estimate the rate of flow (in g$/$min) of cooling water sufficient for the water temperature to rise $ 5$ K. 
\item What is the temperature at the cold end of the bar? 
\end{itemize}
\item What vibrates in the following types of wave motion?
 \begin{itemize}
\item Light waves
\item Sound waves
\item X-rays
\item Water waves
\end{itemize}
\item A plane progressive wave on a water surface is given by the equation $ y=2 \sin 2x(100t -x/30)$ ; where $ x$ is the distance covered in a time $ t$ . $ x$ , $ y$ and $ t$ are in cm and seconds respectively.  Find:
 \begin{itemize}
\item the wavelength, and frequency of the wave motion.
\item the phase difference between two points on the water surface that are $ 60$ cm apart.
\end{itemize}
\item Show how wavelength and frequency of a wave are related.
\item Two open organ pipes of length $ 50$ cm and $ 51$ cm respectively give beat frequency of $ 6.0$ Hz when sounding their fundamental notes together, neglecting end corrections. What value does this give for the velocity of sound in air?
\item What is electric potential at a point in an electrostatic field? 
\item Derive an expression for an electric potential at a point a distance a from a positive point charge $ Q$ .
\item Positive charge is distributed over a solid spherical volume of radius $ R$ and the charge per unit volume is $ \sigma $
 \begin{itemize}
\item Show that the electric field inside the volume at a distance $ r<R$ from the centre is given by $ E=(\sigma r/3e_{0})$
\item What is the electric field at a point $ r>R$ (i.e. outside the spherical volume).
\end{itemize}
\item What is meant by the terms electrical resistivity and ohmic conductor.
\item A $ 4$ m long resistance wire has a cross-sectional area of $ 0.8$ mm? and has a resistance of $ 2.80\Omega $ .  Determine:
 \begin{itemize}
\item The resistivity of the wire.
\item The length of a similar wire which when joined in parallel will give a total resistance of $ 2.0\Omega $ .
\end{itemize}
\item State Kirchhoff’s laws of electric circuits.
\item Two cells of emf $ 1.5$ V and $ 2.0$ V and internal resistances of $ 1\Omega $ and $ 2.0\Omega $ respectively are connected in parallel and across them an external resistance of $ 5.0$ Q. Calculate the currents in each of the three branches of the network. 
\item An electron with charge $ e$ and mass $ m_{e}$ is initially projected with a speed v at right angles to a uniform magnetic field of flux density $ B$ .
 \begin{itemize}
\item Explain why the path of the electron $ 1$ s circular.
\item Show also that the time to describe one complete circle is independent of the speed of the electron.
\end{itemize}
\item Calculate the radius of the path traversed by an electron of energy $ 450$ eV moving at right angles to a uniform magnetic field of flux density $ 1.5\times 10^{-3}$ T.
\item Distinguish between metals and semiconductors in terms of energy bands. 
\item Briefly discuss the formation of the potential difference barrier (depletion layer) of a $ p-n$ junction diode.
\item What is a rectifier?
\item Using $ p-n$ junction diodes, draw the arrangement of a full-wave rectifier and briefly explain how it works.
 \begin{itemize}
\item Define the electron – volt.
\end{itemize}
\item Electrons in a certain television tube are accelerated through a potential difference of $ 2.0$ kV
 \begin{itemize}
\item Calculate the velocity acquired by the electrons.
\item If these electrons lose all their energy on impact and given that $ 10^{12}$ electrons pass per second in the TV tube, calculate the power dissipated.
\end{itemize}
\item Explain why Audio amplification is necessary for a practical radio set.
\item A coil and a capacitor in parallel are used to make a tuning circuit for a radio receiver. Sketch the resonance curve for the circuit. State two ways of changing the circuit to increase the resonant frequency.
\item Mention any three uses of a CRO.
\item A proton is placed in a uniform electric field $ E$ . What must be the magnitude and direction of the field if the electrostatic force acting on the proton $ 1$ s just to balance its weight?
\item With reference to an earthquake on a certain point of the earth explain the terms ‘Focus’ and ‘Epicentre’.
\item What is the importance of the following layers of the atmosphere?
 \begin{itemize}
\item The lowest layer
\item The ionosphere
\end{itemize}
\item Describe two ways by which seismic waves may be produced.
 \begin{itemize}
\item Describe briefly the meaning and application of “seismic prospecting”. 
\end{itemize}
\end{itemize}

\end{document}
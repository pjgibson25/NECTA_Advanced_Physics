
\documentclass{article}
\usepackage[a4paper, total={6in, 8in}]{geometry}
% \usepackage[utf8]{inputenc}
\usepackage{abstract}
\newcommand{\mysection}[2]{\setcounter{section}{#1}\addtocounter{section}{-1}\section{#2}}

\title{\textbf{9 - Current Electricity}}
\author{PJ Gibson - Peace Corps Tanzania}
\date{May 2020}

\begin{document}

\maketitle


\mysection{9}{Current Electricity}

\subsection{Electric Conduction in Metals}
\begin{itemize}
\item (1999)  State Kirchhoff’s laws of circuit analysis
\item (2000)  State Kirchhoff’s laws of electric circuits.
\item (2000)  What do you understand by the term “drift velocity” as applied to any current carriers in a wire?
\item (2000)  Determine the drift velocity of electrons in a silver wire of a cross—sectional area $ 4.5 \times 10^{-6}$ m$ ^{2}$ when a current of $ 15$ A flows through it. Given: The density of silver $ =1.05 \times 10^{4}$ kg$/$m$ ^{3}$ . The atomic weight of silver $ =108$ .
\item (2000)  An unknown wire of $ 1$ mm diameter is found to carry and passes a total charge of $ 90$ C in $ 1$ hour and $ 15$ min. If the wire has $ 5.8 \times 10^{28}$ free electrons per $ m^{3}$ , find
 \begin{itemize}
\item  the current in the wire.
\item the drift velocity of the electrons in m s$ ^{-1}$
\end{itemize}
\item (2000)  The current of $ 12$ A is made to pass through an aluminium wire of radius $ 1.5$ mm which is joined in series with a copper wire of radius $ 0.8$ mm. Determine.
 \begin{itemize}
\item the current density in an aluminium wire.
\item the drift velocity of the electron tn the copper wire, given that the number of free electrons per unit volume in a copper wire is $ 10^{29}$ .
\end{itemize}
\item (2007)  Define the internal resistance (r) of a cell and the terminal potential difference.
\item (2007)  The e.m.f. of a cell is a special terminal potential difference.  Comment.
\item (2007)  State Kirchhoff's laws of electrical network.
\item (2007)  Discuss two $ (2)$ harmful effects of electrolysis. 
\item (2009)  Explain the mechanism of electric conduction in:
 \begin{itemize}
\item Gases
\item Electrolytes
\end{itemize}
\item (2010)  Define the temperature coefficient of resistance
\item (2013)  What is meant by “power rating" as regards to a resistor?
 \begin{itemize}
\item Mention two distinct velocities of an electron in a wire.
\end{itemize}
\item (2013)  Explain why it is better to use a small current for a long time to plate a metal with a given thickness of silver than using a larger current for a short time? 
\item (2013)  Give four difference between the passage of electricity through metals and  ionized solution.
\item (2014)  Define the following terms:
 \begin{itemize}
\item Current density
\item Conductivity 
\end{itemize}
\item (2014)  Under what condition is $ \Omega $ ’s law true?
\item (2014)  Why does the voltage across the terminals of a cell or battery fall when it is delivering a current? 
\item (2014)  Define temperature coefficient of resistance.
 \begin{itemize}
\item A heating coil of Nichrome wire with cross sectional area of $ 0.1 $ mm$ ^{2}$ operates on a $ 12$ V supply, and has a power of $ 36$ W when immersed in water at $ 373$ K. Calculate the length of the wire.
\end{itemize}
\item (2015)  What is meant by the following terms:
 \begin{itemize}
\item  Internal resistance of a cell. 
\item  Drift velocity. 
\end{itemize}
\item (2015)  What is a potentiometer. 
 \begin{itemize}
\item Mention two advantages and two disadvantages of potentiometer.
\end{itemize}
\item (2015)  Distinguish between ohmic and non-ohmic conductor. Give one example in each
\item (2016)  What ts the physical significance of Kirchhoff’s first law.
\item (2016)  Why is Kirchhoff’s second law sometimes referred to as the voltage law?
\item (2016)  List down five points to be considered when applying Kirchhoff’s second law in formulating analytical problems or equations.
\item (2017)  What is the advantage of using a greater length of potentiometer wire?
\item (2017)  Why is Wheatstone bridge not suitable for measuring very high resistance?
\item (2017)  List two factors on which the resistivity of a material depends. 
\item (2017)  A wire of resistivity, $ \rho $ , is stretched to double its length. What will be its new resistivity? Give reason for your answer. 
\item (2017)  Why a high voltage supply should have high internal resistance?
\item (2017)  Justify the statement that ‘it is not possible to verify Ohm's law by using a filament lamp’.
\item (2017)  A potential difference of $ 4$ V is connected to $ 4$ uniform resistance wire of length $ 3.0$ m and cross-sectional area $ 9\times 10^{-9}$ , when a current of $ 0.2$ A is flowing in the wire. Find the:
 \begin{itemize}
\item Resistivity of the wire.
\item Conductivity of the wire. 
\end{itemize}
\item (2018)  Outline three important points which are usually referred as sign convection in  solving Kirchhoff’s second law problems. 
\item (2018)  How is ohmic conductor differ from non-ohmic conductor? Give one example in each case. 
\item (2018)  State a condition that could be employed to make an insulator conduct some electricity. 
\item (2018)  What is meant by the term Ballistic galvanometer? 
\item (2018)  State two conditions to be fulfilled for a galvanometer to be used as a ballistic galvanometer. 
\item (2019)  A researcher has $ 2$ g of gold and wishes to form it into a wire having a resistance of $ 80\Omega $ at $ 0^{\circ}$C . How long should the wire be? 
\end{itemize}

\subsection{Electric Conduction in Gases}
\begin{itemize}
\item (1998)  What is thermionic emission?
\item (2013)  Explain the following observation:
 \begin{itemize}
\item Light in the bulb comes on once the switch is kept on despite the drift velocity of electrons being very low.
\item The potentiometer is said to be a better device for measuring the potential difference (p.d) than a moving coil voltmeter.
\end{itemize}
\item (2013)  A milliameter connected in series with a hydrogen discharge tube indicates a current of $ 1.0 \times 10^{-3}$ A. If the number of electrons passing the cross section of the tube at a particular point is $ 4.0 \times 10^{15}$ per second, find the number of protons that pass the same cross section per second. 
\item (2015)  Sketch the diagram showing the variation of current with potential difference across the following:
 \begin{itemize}
\item  Filament electric bulb. 
\item Gas-filled diode. 
\end{itemize}
\item (2018)  Distinguish between ionization energy and excitation energy.
\end{itemize}

\subsection{Alternating Current (ac)}
\begin{itemize}
\item (1999)  What is a resonant frequency of an oscillator?
\item (1999)  An inductance of $ 4$ mH is connected in series with a resistance of $ 20\Omega $ together with a battery:
 \begin{itemize}
\item Determine how the current will vary with time in this circuit.
\item Sketch the current of above against time
\item Calculate the inductive time constant
\end{itemize}
\item (2000)  What is meant by the terms electrical resistivity and ohmic conductor.
\item (2000)  A $ 4$ m long resistance wire has a cross-sectional area of $ 0.8$ mm? and has a resistance of $ 2.80\Omega $ .  Determine:
 \begin{itemize}
\item The resistivity of the wire.
\item The length of a similar wire which when joined in parallel will give a total resistance of $ 2.0\Omega $ .
\end{itemize}
\item (2000)  Two cells of emf $ 1.5$ V and $ 2.0$ V and internal resistances of $ 1\Omega $ and $ 2.0\Omega $ respectively are connected in parallel and across them an external resistance of $ 5.0$ Q. Calculate the currents in each of the three branches of the network. 
\item (2000)  What is a rectifier?
\item (2007)  An a.c. generator consists of a coil of $ 50$ turns and an area of $ 2.5$ m$ ^{2}$ , rotates at an angular speed of $ 60$ rad$/$s in a uniform magnetic field of $ 0.30$ T between two fixed pole pieces.  The resistance of the circuit including that of the coil is $ 500\Omega $ .  
 \begin{itemize}
\item  What is the maximum current that can be drawn from the generator?
\item  What is the magnetic flux through the coil if the current is maximum?
\end{itemize}
\item (2013)  A $ 20$ k$ \Omega$ resistor is to be connected across a potential difference of $ 300$ V Calculate the required power rating.
\item (2013)  Derive an expression for impedance of a series $ R-C$ circuit. 
\item (2013)  Write down two advantages of digital circuits over the analogue circuits.
\item (2014)  What is meant by the following terms:
 \begin{itemize}
\item Alternating current (a.c.)
\item Effective value of A.C. 
\end{itemize}
\item (2014)  A $ 60$ V, $ 10$ W lamp is to be run on $ 100$ V, $ 60$ Hz A.C mains.
 \begin{itemize}
\item Calculate the inductance of a choke coil required.
\item If a resistor is used in above instead of choke, what will be value of its resistance.
\end{itemize}
\item (2014)  An LCR circuit with $ R=70\Omega$ in series with a parallel combination of $ L=1.5$ H and
 \begin{itemize}
\item $ C=30$ $\mu$F is driven by a $ 230$ V supply with angular frequency of $ 300$ rad$/$s.
\item $ (1)$ Find the power in put to the circuit. 
\item  At the frequency $ \omega_{o}=1/(\sqrt{LC})$ , how does the circuit respond?
\end{itemize}
\item (2015)  Explain the statement that, a sinusoidal current, of peak value $ 5$ A passed through an a.c. ammeter reads $ 5/\sqrt{2}$ A.  
\item (2015)  Show that the average power transferred to an a.c. circuit is, in general, given by $ EIR/Z$ , where $ R$ is the resistance in the circuit defined to be the real part of complex impedance and $ Z$ is its impedance.
\item (2015)  A coil which has an inductance of $ 0.2$ H and negligible resistance is in series in a resistor, whose resistance is $ 60\Omega $ . The pair is connected across a $ 50$ V supply alternating at $ 100/\pi$ Hz.  Calculate the toal impedance of the circuit and its power factor.
\item (2016)  An a.c. circuit consists of a pure resistance of $ 10\Omega $ is connected across an a.c. supply of $ 230$ V , $ 50$ Hz.  Calculate the;
 \begin{itemize}
\item Current flowing in the circuit.
\item Power dissipated
\end{itemize}
\item (2016)  An X-ray tube, operated at a d.c. potential difference of $ 60$ kV , produces heat at the target at the rate of $ 840$ W .  Assuming $ 0.65\%$ of the energy of the incident electrons is converted into X-radiation, calculate:
 \begin{itemize}
\item The number of electrons per second striking the target.
\item The velocity of the incident electrons.
\item The energy of incident electrons
\end{itemize}
\item (2018)  Calculate the current flowing in the circuit when three similar cells each of emf $ 1.5$ V and internal resistance $ 0.3\Omega $ are connected in parallel across a $ 2\Omega $ resistor. 
\item (2018)  Why choke coil is preferred over resistance to control alternating current?
\item (2018)  Explain what could be done to light a $ 30$ V bulb from a $ 220$ volt A.C. supply?
\item (2019)  A current of $ 3.0$ mA flows in a Television resistor $ R$ when a potential difference of $ 6.0$ V is connected across its terminals. Determine the value of conductance.
\end{itemize}

\end{document}
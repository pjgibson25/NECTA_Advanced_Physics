
\documentclass{article}
\usepackage[a4paper, total={6in, 8in}]{geometry}
% \usepackage[utf8]{inputenc}
\usepackage{abstract}
\newcommand{\mysection}[2]{\setcounter{section}{#1}\addtocounter{section}{-1}\section{#2}}

\title{\textbf{10 - Electronics}}
\author{PJ Gibson - Peace Corps Tanzania}
\date{May 2020}

\begin{document}

\maketitle


\mysection{10}{Electronics}

\subsection{The Band Theory of Solids}
\begin{itemize}
\item (2013)  What is band theory?
\item (2013)  How does the band theory explain electrical properties of solids?
\item (2013)  In an intrinsic semiconductor, the energy gap $ E_{g}=1.2$ eV, and its hole mobility is very much smaller than electron mobility which is Independent of temperature. Assuming that the temperature dependence of intrinsic carrier concentration, $ n_{i}$ is expressed as:
 \begin{itemize}
\item $ N_{i}=n_{o}$ exp$ (-E_{g}/(K_{B}T))$ , where $ n_{o}$ and $ K_{B}$ are constants, $ T$ is temperature and $ E_{g}$ is an energy equal to $ E_{q}/2$ .  
\item What is the ratio between conductivity at $ 600$ K and that at $ 300$ K?
\item Comment on the result obtained above.
\end{itemize}
\item (2017)  How does the forbidden energy gap of an intrinsic semiconductor vary with increase in temperature? 
\end{itemize}

\subsection{Semiconductors}
\begin{itemize}
\item (1998)  Describe the function of each of;
 \begin{itemize}
\item the electron gun
\item the deflection system and
\item the display system of the Cathode ray Oscilloscope.
\end{itemize}
\item (1999)  Distinguish between insulators, semi-conductors and metals as far as conduction is concerned.
\item (2000)  Distinguish between metals and semiconductors in terms of energy bands. 
\item (2007)  How does the arrangement of the energy level in a semiconductor differ from that of an insulator?
\item (2013)  Mention one application of LED. What type of a semiconductor is it?
\item (2014)  What is light emitting diode (LED).
\item (2014)  Give three advantages of LED's lamp in radio and other electronic system over filament lamps.
\item (2014)  What is the basic difference between good conductors and semiconductors.
\item (2015)  Mention four important properties of a semiconductor.
\item (2015)  Applying the concept of doping, explain how a free electron and a positive charge can be created in a semiconductor crystal. 
\item (2016)  What is the importance of doping as applied to semiconductors?
\item (2016)  Distinguish between $ n-$ type and $ p-$ type semiconductors.  Give three points.
\item (2016)  Discuss the mode of action of each of the following sensors:
 \begin{itemize}
\item Thermistor (TH).
\item Light Dependent Resistor (LDR).
\end{itemize}
\item (2017)  Define the term semiconductor.
 \begin{itemize}
\item Give three examples of semiconductor materials. 
\end{itemize}
\item (2017)  Outline two factors on which electrical conductivity of a pure semiconductor depends. 
\item (2017)  Explain the meaning of the following terms:
 \begin{itemize}
\item $ P-$ type semiconductor.
\item $ N-$ type semiconductor. 
\end{itemize}
\item (2018)  List two chief properties of semiconductors. 
\item (2018)  Why is it easier to establish the current in a semiconductor than in an insulator?
\item (2018)  Distinguish between conductors and semiconductors on the basis of their energy band structures. 
\end{itemize}

\subsection{Transistors}
\begin{itemize}
\item (1999)  Draw the symbol of $ n-p-n$ transistor.
\item (1999)  With the help of illustrative diagrams explain the action of a choke in a circuit.
\item (1999)  Explain the term “thermal run away” as regards a transistor amplifier.
\item (2000)  Briefly discuss the formation of the potential difference barrier (depletion layer) of a $ p-n$ junction diode.
\item (2000)  Using $ p-n$ junction diodes, draw the arrangement of a full-wave rectifier and briefly explain how it works.
 \begin{itemize}
\item Define the electron – volt.
\end{itemize}
\item (2000)  Mention any three uses of a transistor
\item (2000)  A certain transistor has a current gain $  \beta =55$ . If it is used in a circuit with common-base configuration, how much change occurs in the collector current if an emitter current is changed by $ 100$ micro A? (Assume the collector potential to be constant and neglect the small collector — current due to the minority current carriers).
\item (2010)  Briefly explain why a $ P-N$ junction is referred as a junction diode.
\item (2013)  What is meant by transistor action?
\item (2013)  Briefly explain why the collector of a transistor is made wider than the emitter and base?
\item (2013)  Derive the closed – loop gain A of an inverting amplifier.  If the input resistor is equal to the feedback resistor, what would be the value of the gain A?
\item (2014)  Mention two types of transistors.
 \begin{itemize}
\item Which among the transistors mentioned above responds quickly to electrical signal? Give reason for your answer.
\end{itemize}
\item (2015)  A wire of diameter $ 0.1$ mm and resistivity $ 1.69\times10^{-8}\Omega$ m with temperature coefficient
 \begin{itemize}
\item of resistance of $ 4.3\times10^{-3}$ K$ ^{-1}$ was required to make a resistance,
\item  What length of the wire is required to make a coil with a resistance of $ 0.5\Omega $ ?
\item If on passing a Current of $ 2$ A the temperature of the coil above rises  by $ 10^{\circ}$C, what error would arise in taking the potential drop as $ 1.0$ V 
\end{itemize}
\item (2015)  Why a $ p-n$ junction diode when connected in a circuit and then reversed gives a very small leakage current across the junction? 
 \begin{itemize}
\item How is the size of the current stated in above dependent on the temperature of the diode?
\end{itemize}
\item (2015)  Define closed loop gain. 
\item (2016)  Define the term junction as applied in electrical network.
\item (2016)  Why are transistors mostly used in common emitter arrangement?
\item (2016)  When does a transistor amplifier work as an oscillator?
\item (2017)  List three types of transistor configurations.
\item (2017)  Why is collector of a transistor made wider than emitter and base? 
\item (2018)  What do you understand by the term node as applied to electric circuit?
\item (2018)  Mention four types of energy losses suffered by a transformer.  
\item (2018)  What is meant by depletion layer as used in pn -junction device? 
\item (2018)  Describe the effect of applying a reverse bias to the junction diode. 
\item (2018)  Sketch the graph of transfer characteristic of a transistor. 
 \begin{itemize}
\item State the significance of the slope from the graph above.
\end{itemize}
\item (2018)  What is the basic condition for a transistor to operate properly as an amplifier? 
\item (2018)  Briefly explain how a junction transistor can be connected to act as a current operated device. 
\item (2018)  Why the magnitude of output frequency of a full wave rectifier is twice the input frequency? 
\item (2018)  Draw a simple basic transistor switching circuit diagram. 
\item (2019)  Why transistors can not be used as rectifiers? 
\item (2019)  In NPN transistor circuit the collector current is $ 5$ mA. If $ 95\%$ of the emitted electrons reach the collector region, calculate the base current. 
\item (2019)  What causes damage to transistors? 
\end{itemize}

\subsection{Logic Gates}
\begin{itemize}
\item (2009)  Define the following:
 \begin{itemize}
\item Logic gate.
\item Integrated circuit.
\item Modulation.
\end{itemize}
\item (2014)  What is meant by the following electronic circuits:
 \begin{itemize}
\item Logic gates 
\item Integrated circuits
\end{itemize}
\item (2016)  Give symbols, expressions and truth tables for each of the following logic gates: 
 \begin{itemize}
\item NAND gate .
\item Exclusive NOR gate.
\end{itemize}
\item (2016)  Why is NAND gate considered as basic building block for a variety of logic circuits?
\item (2018)  What is meant by a logic gate? 
\item (2018)  List three basic logic gates that make up all digital circuits. 
\end{itemize}

\subsection{Operational Amplifiers}
\begin{itemize}
\item (1998)  Sketch the traces seen on the screen of a cathode ray oscilloscope when two sinusoidal potential differences of the same frequency — and amplitude are applied simultaneously to $ X$ and $ Y$ plates of  a cathode ray oscilloscope, when the phase difference between them is:
 \begin{itemize}
\item $ 0^{\circ}$ $ 45^{\circ}$ $ 90^{\circ}$ .
\end{itemize}
\item (1999)  Briefly describe the major factors that you would consider when designing a voltage amplifier.
\item (1999)  With the help of clear diagrams, explain how you would overcome thermal run away in a voltage amplifier.
\item (2000)  Mention any three uses of a CRO.
\item (2000)  What is an operational amplifier 
\item (2000)  List three desirable features of an operational amplifier.
\item (2000)  In almost all cases, where an operation amplifier is used as a linear voltage amplifier, negative feedback is employed. State the advantage of negative feedback.
\item (2007)  Make well labelled diagram of the cathode ray oscilloscope and explain briefly how a sinusoidal voltage signal is displayed on its screen.
\item (2007)  Mention three $ (3)$ practical applications of the cathode ray oscilloscope.
\item (2007)  Explain the terms output saturation and negative feedback as applied to op-amplifiers. 
\item (2007)  For an ideal operational amplifier, what are the values of the:
 \begin{itemize}
\item current into both inputs of the op-amp? 
\item voltage between the inputs if the output is not saturated? 
\end{itemize}
\item (2007)  What is a non-inverting amplifier? 
\item (2009)  Explain the following terms:
 \begin{itemize}
\item Forward bias.
\item Reverse bias.
\item Inverting and non-inverting amplifier. 
\end{itemize}
\item (2009)  An operational amplifier is to have a voltage gain of $ 100$ .  Calculate the required values for the external resistances $ R_{1}$ and $ R_{2}$ when the following gains are required:
 \begin{itemize}
\item non-inverting.
\item Inverting.
\end{itemize}
\item (2013)  Briefly explain why Cathode Ray Oscilloscope (C.R.O.) is said to be an excellent instrument for measuring the emf 
\item (2013)  Draw a well labeled circuit diagram of an inverting amplifier.
\item (2014)  What is the purpose of amplifiers in a phone link? 
\item (2015)  List three properties of operational amplifiers.
\item (2015)  What is meant by the term negative feedback? Give four advantages of using it in an op-amp or any type of voltage amplifier.
\item (2015)  Derive an expression of the closed loop gain for an inverting op-amp voltage amplifier with an input resistor $ R$ , and a feedback resistor.
\item (2016)  Explain the use of an op-amp as a summing amplifier.
\item (2016)  Name three electronic circuits in which multivibrators can be constructed.
 \begin{itemize}
\item List down three types of multivibrators.
\item Briefly explain the applications of multivibrators listed above.
\end{itemize}
\item (2016)  Mention two characteristics of op-amps.
\item (2016)  Briefly explain why op-amps are sometimes called differential amplifiers?
\item (2016)  Describe the structure and the mode of action of a simplified version of the Van de Graaff generator.
\item (2017)  Briefly explain the function of the following:
 \begin{itemize}
\item Oscilloscope
\item Op-amps
\end{itemize}
\item (2017)  A change of $ 100$ A in the base current produces a change of $ 3$ mA in the collector current. Calculate:
 \begin{itemize}
\item The current amplification factor, $ \beta$
\item The current gain, $ \alpha $
\end{itemize}
\item (2019)  Distinguish between inverting OP-AMP and non-inverting OP-AMP. 
 \begin{itemize}
\item Give one application of each type of OP-AMP described above.
\end{itemize}
\end{itemize}

\subsection{Telecommunication}
\begin{itemize}
\item (1998)  Give the reason for better reception of radio waves for high Frequency signals at night than during the day time.
\item (1998)  Explain briefly three different types of radio waves travelling from a transmitting station to a receiving antenna.
\item (2000)  Explain why Audio amplification is necessary for a practical radio set.
\item (2013)  An electron gun fires electrons at the screen of a TV tube. The electrons start from rest and are accelerated through a potential difference of $ 30$ kV. What is the speed of impact of electrons on the screen of the picture tube?
\item (2013)  Briefly explain why long distance radio broadcasts make use of short wave bands.
\item (2014)  Give the meaning of the following terms:
 \begin{itemize}
\item Bandwidth
\item  Amplitude modulated carrier wave
\end{itemize}
\item (2014)  Sketch the frequency spectrum for $ 1500$ m radio waves modulated by $ 4$ kHz audio signal.
\item (2014)  List down two advantages of digital signals over analogue signals.
\item (2014)  A carrier of frequency $ 800$ kHz is amplitude modulated by frequencies ranging from $ 1$ kHz to $ 10$ kHz.  What frequency range does each sideband cover?
\item (2015)  Give one advantage of frequency modulation (FM) as compared to amplitude modulation ( AMT).
\item (2015)  Briefly explain the importance of bandwidth of an amplitude modulation (AM) signal.
\item (2015)  State the function of a modulator in radios.
\item (2015)  Sketch a block diagram to show the general plan of any communication system.
\item (2015)  The amplitude modulated (AM) broadcast band ranges from $ 450$ to $ 1200$ kHz. If each station modulates with audio frequencies up to $ 5.5$ kHz, determine the
 \begin{itemize}
\item  Bandwidth needed for each station.
\item  Total bandwidth available. 
\end{itemize}
\item (2017)  List three basic elements of communication system. 
\item (2018)  Identify two difficulties which would arise when two straight wires are used to transmit electricity direct from the source to the city station. 
\item (2019)  Identify three basic elements of a communication system. 
\item (2019)  Why sky waves are not used for transmission of TV signals? 
\end{itemize}

\end{document}
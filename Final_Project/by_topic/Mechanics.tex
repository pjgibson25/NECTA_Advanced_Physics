
\documentclass{article}
\usepackage[a4paper, total={6in, 8in}]{geometry}
% \usepackage[utf8]{inputenc}
\usepackage{abstract}
\title{2 - Mechanics}
\author{PJ Gibson - Peace Corps Tanzania}
\date{May 2020}

\begin{document}

\maketitle


\section{Mechanics}

\subsection{Newton’s Laws of Motion}
\begin{itemize}
\item (2019)  A rocket of mass $ 20$ kg has $ 180$ kg of fuel. If the exhaust velocity of the fuel is $ 1.6$ km/sec, calculate;
 \begin{itemize}
\item The minimum rate of fuel consumption that enable the rocket to rise from the ground. 
\item The ultimate vertical speed gained by the rocket when the rate of fuel consumption ts $ 2$ kg/sec. 
\end{itemize}
\item (2019)  Determine the least number of pieces required to stop the bullet if a rifle bullet loses $ 1/20$ of its velocity when passing through them.
\item (2019)  A man of $ 100$ kg jumps into a swimming pool from a height of $ 5$ m. If it takes $ 0.4$ seconds for the water in a pool to reduce its velocity to zero, what average force did  the water exert on the man? 
\end{itemize}

\subsection{Projectile Motion}
\begin{itemize}
\item (2019)  Justify the statement that projectile motion is two dimensional motion.
\item (2019)  A rocket was launched with a velocity of $ 50$ m$/$s from the surface of the moon at an angle of $ 40^{\circ}$ to the horizontal, Calculate the horizontal distance covered  after half time of flight.
\item (2019)  Show that the angle of projection $ \theta ^{\circ}$ for a projectile launched from the origin is given by $ \theta ^{\circ}= tan^{-1}(4h_{m}/R)$ , where $ R$ stand for horizontal range and $ h_{m}$ is the maximum vertical height.
\item (2019)  Determine the angle of projection for which the horizontal range of a projectile is $ 4\sqrt{3}$ times its maximum height. 
\end{itemize}

\subsection{Simple Harmonic Motion}
\begin{itemize}
\item (2019)  Provide two typical examples of simple harmonic motion (S.H.M). 
\item (2019)  Why the velocity and acceleration of a body executing simple harmonic motion are out of phase? 
\item (2019)  The period of a particle executing simple harmonic motion (S.H.M) is $ 3$ seconds. If its amplitude is $ 25$ cm, calculate the time taken by the particle to move a distance of $ 12.5$ cm on either side from the mean position.
\item (2019)  A person weighing $ 50$ kg stands on a platform which oscillates with a frequency of $ 2$ Hz and of amplitude $ 0.05$ m. Find his/her minimum weight as recorded by a machine of the platform. 
\end{itemize}

\subsection{Uniform Circular Motion}
\begin{itemize}
\item (2019)  In which aspect does circular motion differ from linear motion? 
\item (2019)  Why there must be a force acting on a particle moving with uniform speed in a circular path? 
\item (2019)  A stone tied to the end of string $ 80$ cm long, is whirled in a horizontal circle with a constant speed making $ 25$ revolutions in $ 14$ seconds. Determine the magnitude of its acceleration. 
\end{itemize}

\subsection{Gravitation}
\begin{itemize}
\item (2019)  Why the weight of a body becomes zero at the centre of the earth? 
\item (2019)  How far above the earth surface does the value of acceleration due to gravity becomes $ 36\%$ of its value on the surface? 
\item (2019)  Compute the period of revolution of a satellite revolving in a circular orbit at a height of $ 3400$ km above the Earth’s surface. 
\item (2019)  Prove that the angular momentum fora satellite of mass $ M_{s}$ revolving round the
 \begin{itemize}
\item earth of mass $ M_{e}$ in an orbit of radius $ r$ is equal to $ (G M_{e}$  $ M_{s}^{2}r)^{1/2}$ .
\end{itemize}
\end{itemize}

\end{document}
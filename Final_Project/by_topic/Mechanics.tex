
\documentclass{article}
\usepackage[a4paper, total={6in, 8in}]{geometry}
% \usepackage[utf8]{inputenc}
\usepackage{abstract}
\newcommand{\mysection}[2]{\setcounter{section}{#1}\addtocounter{section}{-1}\section{#2}}

\title{\textbf{2 - Mechanics}}
\author{PJ Gibson - Peace Corps Tanzania}
\date{May 2020}

\begin{document}

\maketitle


\mysection{2}{Mechanics}

\subsection{Newton’s Laws of Motion}
\begin{itemize}
\item (1998)  State Newton's laws of motion.
\item (1998)  A ball of mass $ 0.4$ kg is dropped vertically from a height of $ 2.5$ m on to a horizontal table and bounces to a height of $ 1.5$ m.
 \begin{itemize}
\item Find the kinetic energy of the ball just before striking the table.
\item Find the kinetic energy just after impact.
\item Suggest reasons for the difference between these two values of kinetic energy.
\item What height would you expect the ball to reach after its next bounce from the table?
\end{itemize}
\item (1998)  A jet of water flowing with a velocity of $ 20$ ms$ ^{-1}$ from a pipe of cross-sectional area, $ 5.0 \times 10^{-3}$ m$ ^{2}$ , strikes a wall at right angles and loses all its velocity.
 \begin{itemize}
\item What is the mass of water striking the wall per second?
\item What is the change in momentum per second of the water hitting the wall?
\item What is the force exerted on the wall?
\end{itemize}
\item (1999)  Define momentum
\item (1999)  Define impulse of a force
\item (1999)  A jet of water emerges from a hose pipe of a cross-sectional area $ 5.0\times 10^{-3}$ m​$ ^{2}$ with a velocity of $ 3.0$ m$/$s and strikes a wall at right angle. Assuming the water to be brought to rest by the wall and does not rebound, calculate the force on the wall.
\item (1999)  Distinguish between static and dynamic friction.
\item (2007)  A ball is thrown towards a vertical wall from a point $ 2$ m above the ground and $ 3$ m from the wall.  The initial velocity of the ball is $ 20$ m$/$s at an angle of $ 30$ deg above the horizontal.  If the collision of the ball with the wall is perfectly elastic, how far behind the thrower does the ball hit the ground?
\item (2007)  Explain why when catching a fast moving ball, the hands are drawn back will the ball is being brought to rest.
\item (2007)  Rockets are propelled by the ejection of the products of the combustion of fuel.  Consider a rocket of total mass $ M$ travelling at a speed v in a region of space where the gravitational forces are negligible.  
\item (2007)  Supposing the combustion products are ejected at a constant speed v, relative to the rocket, show that a fuel "burn" which reduces the total mass $ M$ of the rocket to $ m$ results in an increase in the speed of the rocket to v such that $ v-V=V_{f} \ln (M/m)$ .
\item (2007)  Supposing that $ 2.1\times10^{6}$ kg of fuel are consumed during a "burn" lasting $ 1.5\times10^{2}$ seconds and given that there is a constant force on the rocket of $ 3.4\times 10^{7}$ N during this burn, calculate v, and increase in speed resulting from the burn if $ M=2.8\times10^{6}$ kg.  
\item (2007)  What is the initial vertical acceleration that can be imparted to this rocket when it is launched from the Earth if the initital mass is $ 2.8\times 10^{6}$ kg?
\item (2007)  State and define Newton’s 2nd law of motion with respect to angular motion. 
\item (2013)  A man stands in a lift which is being accelerated upwards at $ 3.2$ m$/$s$ ^{2}$ . If the man has a mass of $ 65$ kg, what is the net force exerted on the man by the floor of the lift?
\item (2013)  A rubber cord of a $ Y-$ shaped object has a cross sectional area of $ 4 \times 10^{-6}$ m$ ^{2}$ ? And relaxation length of $ 100$ mm. If the arms of the catapult are $ 70$ mm apart, calculate the: 
 \begin{itemize}
\item tension in the rubber. 
\item force required to stretch it when the rubber cord is pulled back until its length doubles. 
\end{itemize}
\item (2014)  State the principle of conservation of linear momentum. 
 \begin{itemize}
\item Give two examples of the principle of conservation of linear momentum. 
\end{itemize}
\item (2014)  An insect is released from rest at the top of the smooth bowling ball such that it slides over the ball. Prove that it will loose its footing with the ball at an angle of about $ 48^{\circ}$ with the vertical.
\item (2014)  A vertical spring fixed at one end has a mass of $ 0.2$ kg and is attached at the other end.
 \begin{itemize}
\item Determine the:
\item Extension of the spring.
\item Energy stored in the spring.
\end{itemize}
\item (2014)  Define torque and give its S.I. unit.
\item (2014)  Give two ways in which the internal energy of the system can be changed.
\item (2016)  State the principles on which the rocket propulsion is based. 
\item (2016)  A jet engine on a test bed takes in $ 40$ kg of air per second at a velocity of $ 100$ m$/$s  and burns $ 0.80$ kg of fuel per second. After compression and heating the exhaust gases are ejected at $ 600$ m$/$s relative to the air craft. Calculate the thrust of the engine.
\item (2016)  An object of mass $ 2$ kg is attached to the hook of a spring balance which is suspended vertically to the roof of a lift.  What is the reading on the spring balance when the lift is:
 \begin{itemize}
\item going up with the rate of $ 0.2$ m$/$s$ ^{2}$
\item going down with an acceleration of $ 0.1$ m$/$s$ ^{2}$
\item ascending with uniform velocity of $ 0.15$ m$/$s
\end{itemize}
\item (2016)  Define the term inertia.
\item (2017)  A $ 75$ kg hunter fires a bullet of mass $ 10$ g with a velocity of $ 400$ m$/$s from a gun of mass $ 5$ kg. Calculate the:
 \begin{itemize}
\item Recoil velocity of the gun. 
\item Velocity acquired by the hunter during firing.
\end{itemize}
\item (2017)  A traffic light is suspended with two steel wires of equal lengths and radii of $ 0.5$ cm. If the wires make an angle of $ 15^{\circ}$ with the horizontal, what is the fractional increase in their length due to the weight of the light? 
\item (2018)  Under what condition a passenger in a lift feels weightless? 
\item (2018)  Calculate the tension in the supporting cable of an elevator of mass $ 500$ kg which was originally moving downwards at $ 4$ m$/$s and brought to rest with constant acceleration at a distance of $ 20$ m. 
\item (2018)  The rotating blades of a hovering helicopter swept out an area of radius $ 2$ m imparting a downward velocity of $ 8$ m$/$s of the air displaced. Find the mass of a helicopter. 
\item (2019)  A rocket of mass $ 20$ kg has $ 180$ kg of fuel. If the exhaust velocity of the fuel is $ 1.6$ km/sec, calculate;
 \begin{itemize}
\item The minimum rate of fuel consumption that enable the rocket to rise from the ground. 
\item The ultimate vertical speed gained by the rocket when the rate of fuel consumption ts $ 2$ kg/sec. 
\end{itemize}
\item (2019)  Determine the least number of pieces required to stop the bullet if a rifle bullet loses $ 1/20$ of its velocity when passing through them.
\item (2019)  A man of $ 100$ kg jumps into a swimming pool from a height of $ 5$ m. If it takes $ 0.4$ seconds for the water in a pool to reduce its velocity to zero, what average force did  the water exert on the man? 
\end{itemize}

\subsection{Projectile Motion}
\begin{itemize}
\item (2000)  Mention two motions that add up to make projectile motion.
\item (2000)  In long jumps does it matter how high you jump? State the factors which determine the span of the jump. 
\item (2000)  Derive an expression that relates the span of the jump and the factors you have mentioned.
\item (2000)  A bullet is fired from a gun on the top of a cliff $ 140$ m high with a velocity of $ 150$ m$/$s at an elevation of $ 30^{\circ}$ to the horizontal. Find the horizontal distance from the foot of a cliff to the point where the bullet lands on the ground.
\item (2007)  What is meant by the term "projectile" as applied to projectile motion?
\item (2007)  Give two $ (2)$ practical applications of projectile motion at your locality.
\item (2007)  The ceiling of a long hall is $ 25$ m high.  Determine the maximum horizontal distance that a ball thrown with a speed of $ 40$ m$/$s can go without hitting the ceiling of the wall.
\item (2010)  Mention two examples of projectile motion. 
\item (2010)  Define the trajectory. 
\item (2010)  Mention two uses of projectile motion.
\item (2010)  Find the velocity and angle of projection of a particle which passes in a horizontal direction Just over the top of a wall which is $ 12$ m high and $ 32$ m away. 
\item (2013)  List down two main assumptions in deriving the equation of projectile motion.
\item (2013)  Why the horizontal motion of a projectile constant? 
\item (2013)  A ball is thrown horizontally with a speed of $ 14.0$ m$/$s from a point $ 6.4$ m above the ground, calculate:
 \begin{itemize}
\item The horizontal distance traveled in that time.
\item Its velocity when it reaches the ground.
\end{itemize}
\item (2014)  Outline the motions that add up to make projectile motion. 
\item (2014)  In the first second of its flight, a rocket ejects $ 1/60$ of  its mass with a relative velocity of $ 2400$ m$/$s.
 \begin{itemize}
\item Find its acceleration.
\item What is the final velocity if the ratio of initial to final mass of the rocket is $ 4$ at a time of $ 60$ seconds? 
\end{itemize}
\item (2014)  A ball is thrown upwards with an initial velocity of $ 33$ m$/$s from a point $ 65^{\circ}$ on the side of a hill which slopes upward uniformly at an angle of $ 28^{\circ}$ .
 \begin{itemize}
\item At what distance up the slope does the ball strike?
\item Calculate the time of flight of the ball. 
\end{itemize}
\item (2014)  A cannon of mass $ 1300$ kg fires a $ 72$ kg ball in a horizontal direction with a nuzzle speed of $ 55$ m$/$s, If the cannon is mounted so that it can recoil freely calculate the:
 \begin{itemize}
\item  recoil velocity of the cannon relative to the earth. 
\item horizontal velocity of the ball relative to the earth. 
\end{itemize}
\item (2015)  Define the term trajectory.
\item (2015)  Briefly explain why the horizontal component of the initial] velocity of a projectile always remains constant.
\item (2015)  List down two limitations of projectile motion. 
\item (2015)  A body projected from the ground at the angle of $ 60^{\circ}$ is required to pass just above the two vertical walls each of height $ 7$ m. If the velocity of projection is $ 100$ m$/$s, calculate the distance between the two walls. 
\item (2015)  A fireman standing at a horizontal distance of $ 34$ m from the edge of the burning story building aimed to raise streams of water at an angle of $ 60^{\circ}$ into the first floor through an open window which is at $ 20$ m high from the ground level. If water strikes on this floor $ 2$ m away from the outer edge, 
 \begin{itemize}
\item  Sketch a diagram of the trajectory.
\item What speed will the water leave the nozzle of the fire hose?
\end{itemize}
\item (2016)  Mention two characteristics of projectile motion.
\item (2016)  If the range of the projectile is $ 120$ m and its time of flight is $ 4$ sec , determine the angle of projection and its initial velocity of projection assuming that the acceleration due to gravity $ g=10$ m$/$s. 
\item (2017)  A jumbo jet traveling horizontally at $ 50$ m$/$s at a height of $ 500$ m from sea level drops a luggage of food to a disaster area.
 \begin{itemize}
\item At what horizontal distance from the target should the luggage be dropped?
\item Find the velocity of the luggage as it hit the ground. 
\end{itemize}
\item (2018)  How does projectile motion differ from uniform circular motion? 
\item (2018)  A rifle shoots a bullet with a muzzle velocity of $ 1000$ m$/$s at a small target $ 200$ m away. How high above the target must the rifle be aimed so that the bullet will hit the target? 
 \begin{itemize}
\item Where does the object strike the ground when thrown horizontally with a velocity of $ 15$ m$/$s from the top of a $ 40$ m high building? 
\item Find the speed of travel when a man jumps a maximum horizontal distance of $ 1$ m spending a minimum time on the ground.
\end{itemize}
\item (2019)  Justify the statement that projectile motion is two dimensional motion.
\item (2019)  A rocket was launched with a velocity of $ 50$ m$/$s from the surface of the moon at an angle of $ 40^{\circ}$ to the horizontal, Calculate the horizontal distance covered  after half time of flight.
\item (2019)  Show that the angle of projection $ \theta ^{\circ}$ for a projectile launched from the origin is given by $ \theta ^{\circ}= tan^{-1}(4h_{m}/R)$ , where $ R$ stand for horizontal range and $ h_{m}$ is the maximum vertical height.
\item (2019)  Determine the angle of projection for which the horizontal range of a projectile is $ 4\sqrt{3}$ times its maximum height. 
\end{itemize}

\subsection{Uniform Circular Motion}
\begin{itemize}
\item (2000)  Show that the period of a body of mass $ m$ revolving in a horizontal circle with constant velocity v at the end of a string of length $ l$ is independent of the mass of the object.
\item (2000)  A ball of mass $ 100$ g is attached to the end of a string and is swung in a circle of radius $ 100$ cm at a constant velocity of $ 200$ cm$/$s. While in motion the string is shortened to $ 50$ cm. Calculate:
 \begin{itemize}
\item The new velocity of the motion.
\item The new period of the motion.
\end{itemize}
\item (2000)  A car travels over a humpback bridge of radius of curvature $ 45$ m. Calculate the maximum speed of the car if the wheels are to remain in contact with the bridge.
\item (2007)  What is meant by centripetal force?
\item (2007)  Derive the expression $ a =(v^{2}/r)$ where a, v, and $ r$ stands for the centripetal acceleration, linear velocity and radius of a circular path respectively.  
\item (2007)  A ball of mass $ 0.5$ kg attached to a light inextensible string rotates in a vertical circle of radius $ 0.75$ m such that it has a speed of $ 5$ m$/$s when the string is horizontal.  Calculate:
 \begin{itemize}
\item  The speed of the ball and the tension in the string at the lowest point of its circular path.
\end{itemize}
\item (2010)  What is the origin of centripetal force for:
 \begin{itemize}
\item A satellite orbiting around the Earth. 
\item An electron in the hydrogen atom?
\end{itemize}
\item (2010)  A small mass of $ 0.15$ kg is suspended from a fixed point by a thread of a fixed length. The mass is given a push so that it moves along a circular path of radius $ 1.82$ m in a horizontal plane at a Steady speed, taking $ 18.0$ s to make $ 10$ complete revolutions. Calculate:
 \begin{itemize}
\item The speed of the small mass.
\item The centripetal acceleration. 
\item The tension in the thread. 
\end{itemize}
\item (2013)  Why is it technically advised to bank a road at corners?
\item (2013)  A wheel rotates at a constant rate of $ 10$ revolutions per second. Calculate the centripetal acceleration at a distance of $ 0.80$ m from the centre of the wheel.
\item (2014)  Define the term ‘radial acceleration’. 
\item (2015)  Mention three effects of looping the loop.
 \begin{itemize}
\item Why there must be a force acting on a particle moving with uniform speed in a circular path? Write down an expression for its magnitude. 
\end{itemize}
\item (2015)  A driver negotiating a sharp bend usually tend to reduce the speed of the car.
 \begin{itemize}
\item  What provides the centripetal force on the car?
\item Why is it necessary to reduce its speed?
\end{itemize}
\item (2015)  A ball of mass $ 0.5$ kg is attached to the end of a cord whose length is $ 1.5$ m then whirled in horizontal circle. If the cord can withstand a maximum tension of $ 50$ N calculate the:
 \begin{itemize}
\item Maximum speed the ball can have before the cord breaks. 
\item Tension in the cord if the ball speed is $ 5$ m$/$s
\end{itemize}
\item (2015)  Define the term tangential velocity.
\item (2016)  A boy ties a string around a stone of mass $ 0.15$ kg and then whirls it in a horizontal circle at constant speed. If the period of rotation of the stone is $ 0.4$ sec and the length between the stone and boy’s hand is $ 0.50$ m ;
 \begin{itemize}
\item Calculate the tension in the string. 
\item State one assumption taken to reach the answer above.
\end{itemize}
\item (2017)  A car is moving with a speed of $ 30$ m$/$s on a circular track of radius $ 500$ m. If its speed is increasing at the rate of $ 2$ m$/$s, find its resultant linear acceleration.
\item (2017)  An object of mass $ 1$ kg is attached to the lower end of a string $ 1$ m long whose upper end is fixed and made to rotate in a horizontal circle of radius $ 0.6$ m. If the circular speed of the mass is constant, find the:
 \begin{itemize}
\item Tension in the string. 
\item Period of motion. 
\end{itemize}
\item (2019)  In which aspect does circular motion differ from linear motion? 
\item (2019)  Why there must be a force acting on a particle moving with uniform speed in a circular path? 
\item (2019)  A stone tied to the end of string $ 80$ cm long, is whirled in a horizontal circle with a constant speed making $ 25$ revolutions in $ 14$ seconds. Determine the magnitude of its acceleration. 
\end{itemize}

\subsection{Simple Harmonic Motion}
\begin{itemize}
\item (1998)  Define simple harmonic motion.
\item (1998)  Prove that, the velocity v of a particle moving in simple harmonic motion is given by: $ v=w(A^{2}-y^{2})^{0.5}$ , where A is the amplitude of oscillation, $ w$ the angular frequency and $ y$ the displacement from the mean position.
\item (1998)  A simple pendulum has a period of $ 2.8$ seconds. When its length is shortened by $ 1.0$ metre, the period becomes $ 2.0$ seconds. From this information, determine the acceleration $ g$ , of gravity and the original length of the pendulum.
\item (1998)  A particle rests on a horizontal platform which is moving vertically in simple harmonic motion with an amplitude of $ 50$ mm. Above a certain frequency the particle ceases to remain in contact with the platform throughout the motion. With a help of a diagram and illustrative equations, find;
 \begin{itemize}
\item the lowest frequency at which this situation occurs.
\item the position at which contact ceases.
\end{itemize}
\item (1999)  Give two similarities between simple harmonic motion and circular motion.
\item (1999)  On the same set of axes, sketch how energy exchange (kinetic to potential) takes place in an oscillator placed in a damping medium.
\item (2000)  Define simple harmonic motion.
\item (2000)  Two simple pendulums of length $ 0.4$ m and $ 0.6$ m respectively are set oscillating in step. 
 \begin{itemize}
\item After what further time will the two pendulums be in step again? 
\item Find the number of oscillations made by each pendulum during the time found above.
\end{itemize}
\item (2000)  Cite two examples of SHM which are of importance to everyday life experience.
\item (2000)  Explain, giving reasons, whether either transverse or longitudinal waves could exist, if the vibratory motion causing them were not simple harmonic motion.
\item (2014)  State where the magnitude of acceleration is greatest in simple harmonic motion.
\item (2014)  Sketch a graph of acceleration against displacement for a simple harmonic motion.
\item (2014)  The displacement of a particle from the equilibrium position moving with simple harmonic motion is given by $ x=0.05 \sin(6t)$ , where $ t$ is the time in seconds measured at an instant when $ x=0$ .  Calculate the:
 \begin{itemize}
\item Amplitude of oscillations.
\item Period of oscillations. 
\item  Maximum acceleration of the particle. 
\end{itemize}
\item (2015)  Briefly explain why the motion of a simple pendulum is not strictly simple harmonic? 
 \begin{itemize}
\item Why is the velocity and acceleration of a body executing simple harmonic motion (S.H.M.) out of phase? 
\end{itemize}
\item (2015)  A body of mass $ 0.30$ kg executes simple harmonic motion with a period of $ 2.5$ s and amplitude of $ 4.0\times10^{-2}$ m. Determine the:
 \begin{itemize}
\item Maximum velocity of the body. 
\item Maximum acceleration of the body. 
\item Energy associated with the motion.
\end{itemize}
\item (2015)  A particle of mass $ 0.25$ kg vibrates with a period of $ 2.0$ s. If its greatest displacement is $ 0.4$ m what is its maximum kinetic energy?
\item (2016)  Show that the total energy of a body executing S.H.M. is independent of time.
\item (2016)  A mass of $ 05$ kg connected to a light spring of force constant $ 20$ N$/$m oscillates on a  horizontal frictionless surface. If the amplitude of the motion $ 1$ s $ 3.0$ cm , calculate the;
 \begin{itemize}
\item Maximum speed of the mass.
\item  Kinetic energy of the system when the displacement is $ 2.0$ cm.
\end{itemize}
\item (2017)  The equation of simple harmonic motion is given as $ x=6 \sin(10\pi t)+8 \sin(10\pi t)$ , where $ x$ is in centimeters and $ t$ in seconds. Determine the:
 \begin{itemize}
\item Amplitude 
\item Initial phase of motion. 
\end{itemize}
\item (2017)  Show that the total energy of a body executing simple harmonic motion is independent of time. 
\item (2017)  Find the periodic time of a cubical body of side $ 0.2$ m and mass $ 0.004$ kg floating in water then pressed and released such that it oscillates vertically. 
\item (2018)  What is meant by the following terms as used in simple harmonic motion (S.H.M)?
 \begin{itemize}
\item Periodic motion. 
\item Oscillatory motion. 
\end{itemize}
\item (2018)  List four important properties of a particle executing simple harmonic motion (S.H.M). 
\item (2018)  Sketch a labeled graph that represents the total energy of a particle executing simple harmonic motion (S.H.M). 
\item (2018)  The periodic time of a body executing S.H.M is $ 4$ seconds. How much time interval from time, $ t=0$ will its displacement be half its amplitude? 
\item (2018)  Giving reasons, explain whether either transverse or longitudinal waves could exist, if the vibratory motion causing them were not simple harmonic motion. 
\item (2019)  Provide two typical examples of simple harmonic motion (S.H.M). 
\item (2019)  Why the velocity and acceleration of a body executing simple harmonic motion are out of phase? 
\item (2019)  The period of a particle executing simple harmonic motion (S.H.M) is $ 3$ seconds. If its amplitude is $ 25$ cm, calculate the time taken by the particle to move a distance of $ 12.5$ cm on either side from the mean position.
\item (2019)  A person weighing $ 50$ kg stands on a platform which oscillates with a frequency of $ 2$ Hz and of amplitude $ 0.05$ m. Find his/her minimum weight as recorded by a machine of the platform. 
\end{itemize}

\subsection{Gravitation}
\begin{itemize}
\item (1999)  What do you understand by the term escape velocity?
\item (1999)  Calculate the escape velocity from the moon’s surface given that a man on the moon has $ 1/6$ his weight on earth. The mean radius of the moon is $ 1.75 \times 10^6$ m.
\item (1999)  Explain the meaning of the following terms:
 \begin{itemize}
\item Gravitational Potential of the Earth.
\item Gravitational Field Strength of the Earth.
\item How are the above quantities in and related?
\end{itemize}
\item (1999)  Show that the total energy of a satellite in a circular orbit equals half its potential energy.
\item (1999)  Calculate the height above the Earth's surface for a satellite in a parking orbit.
\item (1999)  What would be the length of a day if the rate of rotation of the Earth were such that the acceleration of gravity $ g=0$ at the equator?
\item (2007)  Evaluate the work done by the Earth's gravitational force and by the tension in the string as the ball moves from its highest to its lowest point.
\item (2007)  Two small spheres each of mass $ 10g$ are attached to a light rod $ 50$ cm long. The system Is set into oscillation and the period of torsional oscillation is found to be $ 770$ seconds. To produce maximum torsion to the system two large spheres each of mass $ 10$ kg are placed near each suspended sphere, if the angular deflection of the suspended rod Is $ 3.96 \times 10^{-3}$ rad. and the distance between the centres of the large spheres and small spheres is $ 10$ cm, determine the value of the universal constant of gravitation, $ G$ , from the given information. 
\item (2007)  On the basis of Newton’s universal law of gravitation, derive Kepler’s third law of planetary motion. 
\item (2007)  A planet has half the density of earth but twice its radius. What will be the speed of a satellite moving fast past the surface of the planet which has on no atmosphere?
 \begin{itemize}
\item ( Radius of earth $ R_{E}=6.4 \times 10^{3}$ km and gravitational potential energy $ g_{E}=9.81$ N$/$kg )
\end{itemize}
\item (2009)  State Kepler's laws of planetary motion.
\item (2009)  Explain the variation of acceleration due to gravity, $ g$ . inside and outside the earth.
\item (2009)  Derive the formula for mass and density of the earth.
\item (2009)  What do you understand by the term satellite?
\item (2009)  A satellite of mass $ 100$ kg moves in a circular orbit of radius $ 7000$ km around the earth, assumed to be a sphere of radius $ 6400$ km.  Calculate the total energy needed to place the satellite in orbit from the earth assuming $ g=10$ N$/$kg at the earth’s surface.
\item (2013)  With the aid of a labeled diagram, sketch the possible orbits for a satellite launched from the earth.
 \begin{itemize}
\item From the diagram above, write down an expression for the velocity of a satellite corresponding to each orbit.
\end{itemize}
\item (2014)  Define the universal gravitational constant.
\item (2014)  How is the gravitational potential related to gravitational field strength?
\item (2014)  Write down an expression for the acceleration due to gravity (g) of a body of mass (m) which is at a  distance (r) from the centre of the earth. 
 \begin{itemize}
\item If the Earth were made of lead of relative density of $ 11.3$ kg$/$m$ ^{3}$ , what would he the value of acceleration due to gravity on the surface of the earth?
\end{itemize}
\item (2014)  Why the value of acceleration due to gravity (g) changes due to the change in latitude? Give two reasons.
\item (2014)  A rocket is fired from the earth towards the sun. At what point on its path is the gravitational force on the rocket zero?
\item (2015)  Explain why the astronaut appears to be weightless when traveling in the space vehicle.
\item (2015)  State Newton's law of gravitation. 
 \begin{itemize}
\item Use Newton’s law of gravitation to derive Kepler’s third law.
\end{itemize}
\item (2015)  Briefly explain why Newton’s equation of universal gravitation does not hold for bodies falling near the surface of the earth? 
\item (2015)  Show that the total energy of a satellite in a circular orbit equals half its potential energy.
\item (2015)  Calculate the height above the Earth’s surface for a satellite in a parking orbit.
\item (2015)  A $ 10$ kg satellite circles the Earth once every $ 2$ hours in an orbit having a radius of $ 8000$ km.  Assuming Bohr’s angular momentum postulate applies to the satellite just as it does to an electron in the hydrogen atom,  find the quantum number of the orbit of the satellite.
\item (2016)  Mention one application of parking orbit.
\item (2016)  Briefly explain how parking orbit of a satellite is achieved.
\item (2016)  The earth satellite revolves in a circular orbit at a height of $ 300$ km above the earth’s surface.  Find the; 
 \begin{itemize}
\item Velocity of the satellite
\item Period of the satellite.
\end{itemize}
\item (2016)  A spaceship is launched into a circular orbit close to the earth’s surface.  What additional velocity has to be imparted on the spaceship if order to overcome the gravitational pull?
\item (2017)  Why does the kinetic energy of an earth satellite change in the elliptical orbit?
\item (2017)  A space craft is launched from the earth to the moon, If the mass of the earth is $ 81$ times that of the moon and the distance from the centre of the earth to that of the moon is about $ 4.0 \times 10^{5}$ km;
 \begin{itemize}
\item Draw a sketch showing how the gravitational force on the spacecraft varies during its journey. 
\item Calculate the distance from the centre of the earth where the resultant gravitational force becomes zero. 
\end{itemize}
\item (2018)  A satellite of mass $ 600$ kg is in a circular orbit at a height $ 2 \times 10^{6}$ km above the earth’s surface. Determine the:
 \begin{itemize}
\item Orbital speed. 
\item Gravitational potential energy. 
\end{itemize}
\item (2018)  What would happen if gravity suddenly disappears?  
\item (2018)  Two base of a mountain are at sea level where the gravitational field strength is $ 9.81$ N$/$kg . If the value of gravitational field at the top of the mountain is $ 9.7$ N$/$kg, calculate the height of the mountain above the sea level. 
\item (2019)  Why the weight of a body becomes zero at the centre of the earth? 
\item (2019)  How far above the earth surface does the value of acceleration due to gravity becomes $ 36\%$ of its value on the surface? 
\item (2019)  Compute the period of revolution of a satellite revolving in a circular orbit at a height of $ 3400$ km above the Earth’s surface. 
\item (2019)  Prove that the angular momentum fora satellite of mass $ M_{s}$ revolving round the
 \begin{itemize}
\item earth of mass $ M_{e}$ in an orbit of radius $ r$ is equal to $ (G M_{e}$  $ M_{s}^{2}r)^{1/2}$ .
\end{itemize}
\end{itemize}

\subsection{Rotation of Rigid Bodies}
\begin{itemize}
\item (1999)  State the parallel axis theorem.
\item (1999)  Show that the Kinetic energy (K.E.) of rotation of a rigid body about an axis with a constant angular velocity $ w$ is given by $ KE =1/2Iw^{2}$ where i is the moment of inertia of the rigid body about the given axis.
\item (1999)  What do you understand by the term "moments of inertia" of a rigid body?
\item (1999)  State the perpendicular axes theorem of moments of inertia for a body in the form of a lamina
\item (1999)  Calculate the moments of inertia of a thin circular disc of radius $ 50$ cm and mass $ 2$ kg about an axis along a diameter of the disc.
\item (1999)  A wheel mounted on an axle that is not frictionless is initially at rest. A constant external torque of $ 50$ Nm is applied to the wheel for $ 20$ s. At the end of the $ 20$ s, the wheel has an angular velocity of
 \begin{itemize}
\item $ 600$ rev/min. The external torque is the removed, and the wheel comes to rest after $ 120$ s more.
\item Determine the moments of inertia of the wheel.
\item Calculate the frictional torque which is assumed to be constant. 
\end{itemize}
\item (2007)  The $ T$ is then suspended from the free end of rod $ Y$ and the pendulum swings in the plane of $ T$ about the axis Of rotation.
 \begin{itemize}
\item Calculate the moment of inertia i of the $ T$ about the axis of rotation. 
\item Obtain the expression for the k.e. and p.e. in terms of the angle $ \theta $ of inclination to the vertical oscillation of the pendulum. 
\item Show that the period of oscillation is $ 2\pi\sqrt{17L/18g}$ . 
\item ( Moment of inertia of a thin rod about its centre $ I_{C}=mL^{2}/12$ . )
\end{itemize}
\item (2009)  Define angular momentum and give its dimensions.
\item (2009)  A grinding wheel in a form of solid cylinder of $ 0.2$ m diameter and $ 3$ kg mass is rotated at $ 3600$ rev/minute.
 \begin{itemize}
\item What is its kinetic energy?
\item Find how far it would have to fall to acquire the same kinetic energy as in the question above.
\end{itemize}
\item (2014)  A disc of moment of inertia $ 2.5\times10^{-4}$ kg$/$m$ ^{2}$ is rotating freely about an axis through its centre at $ 20$ rev/min. If some wax of mass $ 0.04$ kg is dropped gently on to the disc $ 0.05$ m from its axis, what will be the new revolution per minute of the disc? 
\item (2014)  Explain briefly why a:
 \begin{itemize}
\item high diver can turn more somersaults before striking the water?
\item dancer on skates can spin faster by folding her arms?
\end{itemize}
\item (2014)  A heavy flywheel of moment of inertia $ 0.4$ kg$/$m$ ^{2}$ is mounted on a horizontal axle of radius $ 0.01$ m. If a force of $ 60$ N is applied tangentially to the axle:
 \begin{itemize}
\item  Calculate the angular velocity of the flywheel after $ 5$ seconds from rest.
\item List down two assumptions taken to arrive at your answer in above.
\end{itemize}
\item (2015)   Define moment of inertia of a body.
 \begin{itemize}
\item Briefly explain why there is no unique value for the moment of inertia of a given body?
\end{itemize}
\item (2015)  State the principle of conservation of angular momentum. 
 \begin{itemize}
\item A horizontal disc rotating freely about a vertical axis makes $ 45$ revolutions per minute. A small piece of putty of mass $ 2.0\times10^{-2}$ kg falls vertically onto the disc and sticks to it at a distance of $ 5.0\times10^{-2}$ m from the axis. If the number of revolutions per minute is thereby reduced to $ 36$ , calculate the moment of inertia of the disc. 
\end{itemize}
\item (2015)  What would be the length of a day if the rate of rotation of the Earth were such that the acceleration due to gravity $ g=0$ at the equator?
\item (2016)  Why is Newton’s first law of motion called the law of inertia?
\item (2016)  What is meant by moment of inertia of a body?
\item (2016)  List two factors on which the moment of inertia of a body depends. 
\item (2016)  A thin sheet of aluminum of mass $ 0.032$ kg has the length of $ 0.25$ m and width of $ 0.1$ m. Find its moment of inertia on the plane about an axis parallel to the:
 \begin{itemize}
\item Length and passing through its centre of mass, $ m$ .
\item Width and passing through the centre of mass, $ m$ , in its own plane.
\end{itemize}
\item (2016)  Define the term angular momentum.
\item (2016)  A thin circular ring of mass, $ M$ , and radius, $ r$ , is rotating about its axis with constant angular velocity, $ w_{1}$ .  If two objects each of mass, $ m$ , are attached gently at the ring, what will be the angular velocity of the rotating wheel?
\item (2016)  Why are space rockets usually launched from west to east?
\item (2017)  Justify the statement that ‘If no external torque acts on a body, its angular velocity will not conserved.
\item (2018)  Why is flywheel designed such that most of its mass is concentrated at the rim? Briefly explain. 
\item (2018)  Estimate the couple that will bring the wheel to rest in $ 10$ seconds when a grinding wheel of radius $ 40$ cm and mass $ 3$ kg is rotating at $ 3600$ revolutions per minute. 
\item (2018)  Why an ice skater rotates at relatively low speed when stretches her arms and a leg outward? 
\item (2018)  Calculate the moment of inertia of a sphere about an axis which is a tangent to its surface given that the mass and radius of the sphere are $ 10$ kg and $ 0.2$ m respectively. 
\end{itemize}

\end{document}
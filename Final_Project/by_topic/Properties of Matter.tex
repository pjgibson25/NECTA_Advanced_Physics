
\documentclass{article}
\usepackage[a4paper, total={6in, 8in}]{geometry}
% \usepackage[utf8]{inputenc}
\usepackage{abstract}
\title{4 - Properties of Matter}
\author{PJ Gibson - Peace Corps Tanzania}
\date{May 2020}

\begin{document}

\maketitle


\section{Properties of Matter}

\subsection{Elasticity}
\begin{itemize}
\item (2019)  Define Young’s Modulus of a material. 
\item (2019)  Why work is said to be done in stretching a wire? 
\item (2019)  A steel wire AB of the length $ 60$ cm and cross-sectional area $ 1.5 \times 10^{-6}$ m$ ^{2}$ is attached at $ B$ to copper wire BC of length $ 39$ cm and cross sectional area $ 3.0 \times 10^{-6}$ m$ ^{2}$ . If the combination of the two wires is suspended vertically from a fixed point at A, and supports a weight of $ 250$ N at $ C$ ; find the extension (in millimeter) of the:
 \begin{itemize}
\item steel wire. 
\item copper wire. 
\end{itemize}
\end{itemize}

\subsection{Kinetic Theory of Gases}
\begin{itemize}
\item (2019)  Based on the kinetic theory of gases determine:
 \begin{itemize}
\item The average translational kinetic energy of air at a temperature of $ 290$ K.
\item The root mean square seed (r.m.s) of air at the same temperature (above).
\end{itemize}
\end{itemize}

\end{document}
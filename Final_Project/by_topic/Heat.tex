
\documentclass{article}
\usepackage[a4paper, total={6in, 8in}]{geometry}
% \usepackage[utf8]{inputenc}
\usepackage{abstract}
\newcommand{\mysection}[2]{\setcounter{section}{#1}\addtocounter{section}{-1}\section{#2}}

\title{\textbf{5 - Heat}}
\author{PJ Gibson - Peace Corps Tanzania}
\date{May 2020}

\begin{document}

\maketitle


\mysection{5}{Heat}

\subsection{Thermometers}
\begin{itemize}
\item (1999)  What do you understand by the term:   Triple point of water
\item (1999)  The resistance of a platinum wire at a temperature T​$ ^{\circ}$C measured on a gas scale is given by $ R(T)=R​_{0​}(1+ a T+bT​^{2}​)$ .
 \begin{itemize}
\item What temperature will the platinum thermometer indicate when the temperature on the gas scale is $ 200​^{\circ}$C ? (take a $ =3.8 \times 10^{-3}$ ​ and $ b=-5.6 \times 10^{-7}$ ​)
\end{itemize}
\item (2000)  What does one require in order to establish a scale of temperature?
\item (2000)  A copper-constantan thermocouple with its cold junction at $ 0^{\circ}$C had an emf of $ 4.28$ mV when its other hot junction was at $ 100^{\circ}$C. The emf became $ 9.29$ mV when the temperature of the hot junction was $ 200^{\circ}$C. If the emf $ E$ is related to the temperature difference $ 8$ between hot and cold junctions by the equation $ E= A(\theta )+B(\theta ^{2})$ , calculate:
 \begin{itemize}
\item The values of $ A$ and $ B$ .
\item The range of temperature for which $ E$ may be assumed proportional to $ 8$ without incurring an error of more than $ 1\%$ .
\end{itemize}
\item (2000)  The resistance $ R$ , of a platinum varies with temperature $ t$ according to the equation $ R_{t}=R_{o}(1+8000bt -b t^{2})$ where $ b$ is a constant. Calculate the temperature on platinum scale corresponding to $ 400^{\circ}$C on the gas scale. 
\item (2000)  Heat is supplied at a rate of $ 80$ W to one end of a well lagged copper bar of uniform cross section area $ 10$ cm? having a total length of $ 20$ cm. The heat is removed by water cooling at the other end of the bar. Temperature recorded by two thermometers $ T_{1}$ and $ T_{2}$ at distances $ 5$ cm and $ 15$ cm from the hot end are $ 48^{\circ}$C and $ 28^{\circ}$C respectively.
 \begin{itemize}
\item Calculate the thermal conductivity of copper.
\item Estimate the rate of flow (in g$/$min) of cooling water sufficient for the water temperature to rise $ 5$ K. 
\item What is the temperature at the cold end of the bar? 
\end{itemize}
\item (2007)  What is meant by a thermometric property of a substance?
\item (2007)  What qualities make a particular property suitable for use in practical thermometers?
\item (2007)  Explain why at least two $ (2)$ fixed points are required to define a temperature scale.
\item (2007)  Mention the type of thermometer which is most suitable for calibration of thermometers.
\item (2010)  In a special type thermometer a fixed mass of a gas has a volume of $ 100$ cm? at a pressure of $ 81.6$ cmHg at the ice point and volume of $ 124$ cm$ ^{3}$ and pressure of $ 90$ cmHg at steam point. Determine the temperature if its volume is $ 120$ cm$ ^{3}$ and pressure of $ 85$ cmHg.
 \begin{itemize}
\item What value does the scale of this thermometer give for absolute
\item zero? 
\end{itemize}
\item (2013)  Name the temperature of a thermocouple at which the thermo,
 \begin{itemize}
\item e.m.f. changes its sign.
\item electric power becomes zero.
\end{itemize}
\item (2013)  A Nichrome-coustantan thermocouple gives about $ 70$ $\mu$V for each $ 1^{\circ}$C difference in temperature between the junctions. If $ 100$ such thermocouples are made into a thermopile, what voltage is produced when the junctions are at $ 20^{\circ}$C and $ 240^{\circ}$C? 
\item (2014)  What is meant by temperature of inversion?
\item (2014)  A thermometer was wrongly calibrated as mt reads the melting point of ice as $ -10^{\circ}$C and reading a temperature of $ 60^{\circ}$C in place of $ 50^{\circ}$C What would be the temperature of boiling point of water on this scale? 
\item (2015)  What is meant by a thermometric property?
\item (2015)  Mention three qualities that make a particular property suitable for use in a practical thermometer.
\item (2016)  Briefly describe the working principle of a thermocouple. 
\item (2016)  In a certain thermocouple thermometer the e.m.f. is given by $ E= a \theta + 1/2 b\theta^{2}$ where $ \theta $ is the temperature of hot junction. If a$ =10 $ mV$ ^{\circ}C^{-2}$ , $ b=-1/20 $ mV$ ^{\circ}C^{-2}$ and the cold junction is at $ 0^{\circ}$C, calculate the neutral temperature. 
\item (2017)  The value of the property $ X$ of a certain substance Is given by $ X_{\theta}=X_{0}+0.5\theta +2\times 10^{-4}\theta ^{2}$  , Where $ \theta $ is the temperature in degree Celsius. What would be the Celsius temperature defined by the property $ X$ which corresponds to a temperature of $ 50^{\circ}$C on this gas thermometer scale? 
\item (2018)  Which type of thermometer is most suitable for calibration of other thermometers? 
\item (2018)  Why at least two fixed points are required to define a temperature scale?
\item (2018)  List two qualities which makes a particular property suitable for use in practical thermometers. 
\item (2018)  Describe how mercury in glass thermometer could be made sensitive.
\item (2018)  What is meant by triple point of water? 
\item (2018)  Evaluate the temperature in Kelvin if the pressure recorded by a constant volume gas thermometer is $ 6.8 \times 10^{4}$ Nm$ ^{-2}$ given that the pressure at triple point $ 273.16$ K is $ 4.6 \times 10^{4}$ Nm$ ^{-2}$ .
\item (2019)  A thermometer has wrong calibration as it reads the melting point of ice as $ -10^{\circ}$C . If it reads $ 40^{\circ}$C in a place where the temperature reads $ 30^{\circ}$C ,  determine the boiling point of water on this scale.
\end{itemize}

\subsection{Thermal Conduction}
\begin{itemize}
\item (1999)  What is the coefficient of thermal conductivity of a material?
\item (1999)  The temperature difference between the inside and outside of a room is $ 25​^{\circ}$C . The room has a window of an area $ 2$ m$ ​^{2}$ and the thickness of the window material is $ 2$ mm. Calculate the heat flow through the window if the coefficient of thermal conductivity of the window material is $ 0.5$ SI units.
\item (2000)  Define the thermal conductivity of a material
\item (2000)  Give one major similarity and one major difference between heat conduction and wave propagation.
\item (2000)  Deep bore holes into the earth show that the temperature increases about $ 1^{\circ}$C for each $ 30$ m depth. How much heat flows out from the core of the earth each second for each square metre of surface area.
\item (2007)  Explain why in cold climates, windows of modern buildings are double glazed, ie: There are two pieces of glass with a small air space between them.
\item (2010)  A cylindrical element of $ 1$ kW electric fire $ 1$ s $ 30$ cm long and $ 1.0$ cm in  diameter. If the temperature of the surroundings is $ 20^{\circ}$C , estimate the working temperature of the element.
\item (2013)  Compare the law governing the conduction of heat and electricity pointing out the corresponding quantities in each case.
\item (2013)  A Lagged copper rod is uniformly heated by a passage of an electric current. Show by considering a small section dx that the temperature $ \theta $ varies with distance $ x$ along a rod in a way that, $ k\frac{d^{2}T}{dx^{2}}=-H$ , where $ k$ is a thermal conductivity and $ H$ is the rate of heat generation per unit volume.
\item (2015)  Define coefficient of thermal conductivity.
\item (2015)  Write down two characteristics of a perfectly lagged bar.
\item (2015)  A thin copper wall of a hot water tank having a total surface area of $ 5.0$ m$ ^{2}$ contains $ 0.8$ cm$ ^{3}$ of water at $ 350$ K and is lagged with a $ 50$ mm thick layer of a material of thermal conductivity $ 4.0\times10^{-2}$ W$/$mK. If the thickness of copper wall is neglected and the temperature of the outside surface is $ 290$ K,
 \begin{itemize}
\item Calculate the electrical power supplied to an immersion heater.
\item If the heater were switched off, how long would it take for the temperature of hot water to fall by $ 1$ K?
\end{itemize}
\item (2016)  Identify two factors on which the coefficient of thermal conductivity of a material depend.
\item (2016)  A brass boiler of base area $ 1.50\times 10^{-1}$ and thickness $ 1.0$ cm boils water at a rate of $ 6.0$ kg$/$min when placed on a gas Stove. Estimate the temperature of the part of the flame in contact with the boiler.
\item (2019)  A closed metal vessel containing water at $ 75^{\circ}$C , has a surface area of $ 0.5$ m$ ^{2}$ and uniform thickness of $ 4.0$ mm.  If its outside temperature is $ 15^{\circ}$C , calculate the head loss per minute by conduction.
\end{itemize}

\subsection{Thermal Convection}
\begin{itemize}
\item (2000)  Write down a formula for the rate of cooling under natural convection and define all the symbols used. 
\item (2007)  State Newton’s law of cooling and give one limitation of the law.
\item (2007)  A body initially at $ 70^{\circ}$C cools to a temperature of $ 55^{\circ}$C in $ 5$ minutes. What will be its temperature after $ 10$ minutes given that the surrounding temperature is $ 31^{\circ}$C ? (Assume Newton’s law of cooling holds true)
\item (2010)  Define thermal convection.
\item (2010)  State Newton’s law of cooling.
\item (2010)  A glass disc of radius $ 5$ cm and uniform thickness of $ 2$ mm had one of its sides maintained at $ 100^{\circ}$C while copper block in good thermal contact with this side was found to be $ 70^{\circ}$C . The copper block weighs $ 0.75$ kg. The cooling of copper was studied over a range of temperature and the rate of cooling at $ 70^{\circ}$C was found to be $ 16.5$ K$/$min. Determine the thermal conductivity of glass.
\item (2013)  A person sitting on a bench on a calm hot summer day is aware of a cool breeze blowing from the sea. Briefly explain why there is a natural convection?
\item (2013)  A cup of tea kept in a room with temperature of $ 22^{\circ}$C cools from $ 66^{\circ}$C to $ 63^{\circ}$C in $ 1$ minute. How long will the same cup of tea take to cool from the temperature of $ 43^{\circ}$C to $ 40^{\circ}$C under the same condition?
\item (2014)  Define thermal convection.
\item (2014)  Prove that at a very small temperature difference, $ \Delta T=T_{b}$ – $ T_{s}$ ,  Newton's law of cooling obeys the Stefan’s law, whereby $ T_{b}$ , is the temperature of the body and $ T_{s}$ is the temperature of the surrounding. 
\item (2016)  Briefly explain why forced convection is necessary for excess temperate less than $ 20$ K? 
\item (2016)  State Newton’s law of cooling. 
\item (2016)  A body cools from $ 70^{\circ}$C to $ 40^{\circ}$C in $ 5$ minutes. If the temperature of the surroundings is $ 10^{\circ}$C , Calculate the time it takes to cool from $ 50^{\circ}$C to $ 20^{\circ}$C.  
\end{itemize}

\subsection{Thermal Radiation}
\begin{itemize}
\item (2007)  What is blackbody radiation of a given body?
\item (2007)  Explain why heat may just mean infrared.
\item (2007)  State Prvost's theory of heat exchange.
\item (2007)  What is Wien's displacement law?
\item (2007)  The sun's surface temperature is about $ 6000$ K.  The sun's radiation is maximum at wavelength of $ 0.5\times10^{-6}m$ .  A certain light bulb filament emits radiation with maximum wavelength of $ 2\times10^{-6}m$ .  If both the surface of the sun and of the filament have the same emissive characteristics, what is the temperature of the filament?
\item (2010)  State Stefan’s law of thermal radiation.
\item (2010)  A solid copper sphere cools at the rate of $ 2.8^{\circ}$C$/$min when its temperature is $ 127^{\circ}$C. At what rate will a solid copper sphere of twice the radius cool when its temperature is $ 227^{\circ}$C? In both cases the surroundings are kept at $ 27^{\circ}$C and conditions are such that  Stefan’s law may be applied.
\item (2010)  Explain the observation that a piece of wire when steadily heated up appears reddish in color before turning bluish. 
\item (2013)  A black body of temperature $ \theta $ is placed in a blackened enclosure maintained at a temperature of $ 100^{\circ}$C. When its temperature rises to $ 30^{\circ}$C the net rate of loss of energy from the body was found to be $ 10$ Watts. Find the power generated by the body at $ 50^{\circ}$C if the energy exchange takes place solely by the process of forced convection.
\item (2013)  Write down three laws governing the black body radiation.
\item (2015)  The element of an electric fire with an output of $ 1000$ W is a cylinder of $ 250$ mm long and $ 15$ mm in diameter. If it behaves as a black body, estimate its temperature.
\item (2016)  Briefly explain why: 
 \begin{itemize}
\item A body with large reflectivity is a poor emitter. 
\item The earth without its atmosphere would be too cold to live.
\end{itemize}
\item (2016)  What is meant by thermal radiation?
\item (2016)  Why is the energy of thermal radiation less than that of visible light? 
\item (2016)  A body with a surface area of $ 5.0 $ cm$ ^{2}$ and a temperature of $ 727^{\circ}$C radiates $ 300$ joules of energy in one minute. Calculate its emissivity.
\item (2017)  State the following according to heat exchange:
 \begin{itemize}
\item  Prevost’s theory. 
\item Wien's displacement law.
\end{itemize}
\item (2018)  Why during emission of radiations from black body its temperature does not  reach zero Kelvin? 
\item (2018)  A black ball of radius $ 1$ m is maintained at a temperature of $ 30^{\circ}$C . How much  heat is radiated by the ball in $ 4$ seconds? 
\item (2019)  Sketch the graph to illustrates how the energy radiated by a black body is distributed among various wavelengths. 
 \begin{itemize}
\item What information would be drawn from the graph above? Give three points.
\end{itemize}
\item (2019)  At what temperature will the filament of a $ 10$ W lamp operate if it is supposed to be a perfectly black body of area  $ 1 $ cm$ ^{2}$ ? 
\end{itemize}

\subsection{First Law of Thermodynamics}
\begin{itemize}
\item (1999)  What do you understand by the term: Thermodynamic temperature scale
\item (2000)  The longitudinal wave speed in gases is given by $ v=\sqrt{\gamma p/ \rho }$ ; where $ \gamma =C_{p}/C_{v}$ , $ P$ is the pressure and $ \rho $ the density of gas. If $ v_{1,}$ and $ v_{2,}$ are the speeds of sound in air at temperature $ T_{1}$ and $ T_{2}$ respectively, show that $ v_{1/v_2}=\sqrt{T_{1}/T_{2}}$
 \begin{itemize}
\item NOTE: $ C_{p}$ and $ C_{v}$ are the specific heats of the gas at constant pressure and constant volume respectively.
\end{itemize}
\item (2000)  A number of $ 16$ moles of an ideal gas which is kept at constant temperature of $ 320$ K is compressed isothermally from its initial volume of $ 18$ litres to the final volume of $ 4$ litres.
 \begin{itemize}
\item Calculate the total work done in the whole process.
\item Comment on the sign of numerical answer you've obtained.
\end{itemize}
\item (2000)  A cylinder fitted with a frictionless piston contains $ 1.0$ g of oxygen at a pressure of $ 760$ mmHg and at a temperature of $ 27^{\circ}$C. the following operations are performed in stages: $ (1)$ The oxygen is heated at a constant pressure to $ 127^{\circ}$C and then $ (2)$ it is compressed isothermally to its original volume and finally $ (3)$ it is cooled at a constant volume to its original temperature.
 \begin{itemize}
\item Illustrate these changes in a sketch $ P-V$ diagram.
\item What is the input of heat to the cylinder in stage $ (1)$ above?
\item How much work does the oxygen do in pushing back the piston during stage $ (1)$ ?
\item How much work is done on the oxygen in stage $ (2)$ ?
\item How much heat must be extracted from the oxygen in stage $ (3)$ ? 
\item (For oxygen: density $ =1.43$ kg$/$m$ ^{3}$ (at stp), $ C_v =670$ J kg$ ^{-1}$ K$ ^{-1}$ and molecular mass $ =32$ )
\end{itemize}
\item (2000)  What is the difference between an “isothermal” process and an “adiabatic” process?
\item (2000)  How much work is required to compress $ 5$ mol of air at $ 20^{\circ}$C and $ l$ atmosphere to $ 1/10$ th of the original volume by
 \begin{itemize}
\item an isothermal process
\item an adiabatic process?
\item What are the final pressures for the cases and above?
\end{itemize}
\item (2000)  Explain the fact that the temperature of the ocean at great depths is very nearly constant the year round, at a temperature of about $ 4^{\circ}$C .
\item (2000)  In a diesel engine, the cylinder compresses air from approximately standard temperature and pressure to about one-sixteenth the original volume and a pressure of about $ 50$ atmospheres. What is the temperature of the compressed air?
\item (2007)  When a metal cylinder of mass $ 2.0x10^{-2}$ kg and specific heat capacity $ 500$ J$/$kgK is heated at constant power, the initial rate of rise of temperature is $ 3.0$ K$/$min.  After a time the heater is switched off and the initial rate of fall of temperature is $ 0.3$ K$/$min.  What is the rate at which the cylinder gains heat energy immediately before the heater is switched off?
\item (2007)  State the expression for the $ 1$ st law of thermodynamics.
\item (2007)  What do you understand by the terms:
 \begin{itemize}
\item critical temperature? 
\item adiabatic change?
\end{itemize}
\item (2007)  Find the number of molecules and their mean kinetic energy for a cylinder of volume $ 5 \times 10^{-4}m^{3}$ containing oxygen at a pressure of $ 2 \times 10^{5}$ Pa and a temperature of $ 300K$ . 
 \begin{itemize}
\item When the gas is compressed adiabatically to a volume of $ 2 \times 10^{-4}m^{3}$ , the temperature rises to $ 434K$ . Determine $ \gamma $ , the ratio of the principal heat capacities.
\item [ Molar gas constant $ R=8.31$ J$/$mol$/$K ,$ N  =6 \times 10^{32}$ mol$^{-1}$ ] 
\end{itemize}
\item (2009)  What is the difference between isothermal and adiabatic processes?
 \begin{itemize}
\item Write down the equation of state obeyed by each process in the question above.
\end{itemize}
\item (2009)  Using the same graph and under the same conditions sketch the isotherms and the adiabatics.
\item (2009)  Derive the expression for the work done by the gas when it expands from volume $ V_{1}$ to volume $ V_{2}$ during an:
 \begin{itemize}
\item Isothermal process
\item Adiabatic process
\end{itemize}
\item (2009)  When water is boiled under a pressure of $ 2$ atmospheres the boiling point is $ 120^{\circ}$C. At this pressure $ 1$ kg of water has a volume of $ 10^{-3}$ m$ ^{3}$ and $ 2$ kg of steam have a volume of $ 1.648$ m$ ^{3}$ . Compute the work done when $ 1$ kg of steam is formed at this temperature increase in the internal energy. 
\item (2010)  Briefly describe an experiment to measure temperature coefficient of a wire.
\item (2010)  A heating coil is made of a nichrome wire which will operate on a $ 12$ V supply and will have a power of $ 36$ W when immersed in water at $ 373$ K. The wire available has a cross-sectional area of $ 0.10$ mm$ ^{2}$ . What length of the wire will be required? 
\item (2013)  Briefly give comments on the following observations:
 \begin{itemize}
\item Polyatomic and diatomic gases have larger molar heat capacities than monatomic gases. 
\item  Cubical container is used for the derivation of pressure of an ideal gas.
\end{itemize}
\item (2013)  What is meant by a gas constant. 
\item (2013)  When a gas expand adiabatically it does work on its surroundings although there is no heat input to the gas. Explain where this energy is coming from.
\item (2013)  An ideal gas at $ 17^{\circ}$C and $ 750$ mmHg is compressed isothermally Until its volume is reached to ¾ of its initial value If it then allowed to expand adiabatically to a volume of $ 20\%$ greater than its original value. calculate the final temperature and pressure of the gas. 
\item (2013)  How does the first law of thermodynamics change under isothermal and adiabatic processes? 
\item (2013)  Show that the specific heat capacities of an ideal gas are related by the relation $ C_{p}=C_{v}+nR$ .
 \begin{itemize}
\item Explain the meaning of all the symbols used in the equation above.
\end{itemize}
\item (2013)  One mole of an ideal monatomic gas is heated at constant volume from the temperature of $ 300$ K to $ 600$ K. Calculate the:
 \begin{itemize}
\item amount of heat added 
\item work done by the gas 
\item change in its infernal energy
\end{itemize}
\item (2013)  The piston of a bicycle pump at room temperature of $ 290$ K is slowly moved in until the volume of air enclosed is one — fifth of the total volume of the pump. The outlet is then sealed and the piston suddenly drawn out to full extension. If no air passes the piston, find the temperature of the air in the pump immediately after withdrawing the piston, assuming that air ts an ideal gas with cryoscopic constant, $ \gamma =1.4$ .
\item (2014)  List down two simple applications of the First law of thermodynamics in our daily life.
\item (2014)  A heat engine works at two temperatures of $ 27^{\circ}$C and $ 227^{\circ}$C. Calculate the:
 \begin{itemize}
\item Efficiency of the engine. 
\item Temperature which will increase the efficiency by $ 10\%$ if the room temperature is kept at $ 27^{\circ}$C. 
\end{itemize}
\item (2017)  Give a common example of adiabatic process. 
\item (2017)  What happens to the internal energy of a gas during adiabatic expansion?
\item (2017)  A mass of an ideal gas of volume $ 400$ cm$ ^{3}$ at $ 288$ K expands adiabatically. If its temperature falls to $ 273$ K;
 \begin{itemize}
\item Find the new volume of the gas. 
\item Calculate the final volume of the gas if it is then compressed isothermally until the pressure returns to its original value.
\end{itemize}
\item (2017)  Briefly explain why:
 \begin{itemize}
\item Steam pipes are wrapped with insulating materials?
\item Stainless steel cooking pans fitted with extra copper at the bottom are more preferred?
\end{itemize}
\item (2017)  The capacitance $ C$ of a capacitor ts full charged by a $ 200$ V battery. It is then discharged through a small coil of resistance wire embedded in a thermally insulated block of specific heat capacity $ 2.5 \times 10^{2}$ J$/$kgK and of mass of $ 0.1$ kg.  If the temperature of the block rises by $ 0.4$ K. what is the value of $ C$ ?
\item (2018)  One gram of water becomes $ 1671 $ cm$ ^{3}$ of steam at a pressure of $ 1$ atmosphere. If the latent heat of vaporization at this pressure is $ 2256$ J$/$g, determine the:
 \begin{itemize}
\item external work done. 
\item increase in internal energy 
\end{itemize}
\item (2019)  Why water is preferred as a cooling agent in many automobiles?
\item (2019)  Analyze  three practical applications of thermal expansion of solids in daily life situations.
\item (2019)  Why stainless steel cooking pans are made with extra copper at the bottom?
\end{itemize}

\end{document}

\documentclass{article}
\usepackage[a4paper, total={6in, 8in}]{geometry}
% \usepackage[utf8]{inputenc}
\usepackage{abstract}
\newcommand{\mysection}[2]{\setcounter{section}{#1}\addtocounter{section}{-1}\section{#2}}

\title{\textbf{6 - Vibrations and Waves}}
\author{PJ Gibson - Peace Corps Tanzania}
\date{May 2020}

\begin{document}

\maketitle


\mysection{6}{Vibrations and Waves}

\subsection{Mechanical Vibrations}
\begin{itemize}
\item (2000)  Show how wavelength and frequency of a wave are related.
\item (2007)  State the modes of vibrations in closed and open pipes.  
\item (2013)  What is meant by dispersion of waves? 
\item (2013)  Briefly explain if it is possible for dispersion to take place on a wave whose frequency lies in the audible range.
\item (2015)  Define the following terms:
 \begin{itemize}
\item Damped oscillations
\item Forced oscillations
\item Resonance
\end{itemize}
\item (2016)  What do you understand by the following terms: 
 \begin{itemize}
\item Damped oscillations. 
\item Undamped oscillations.
\end{itemize}
\item (2016)  Sketch the waveform diagrams to represent the terms: damped oscillations & undamped oscillations
\item (2016)  A steel wire hangs vertically from a fixed point, supporting a weight of $ 80$ N as its lower end.  The length of the wire from the fixed point to the weight is $ 1.5$ m.  Calculate the fundamental frequency emitted by the wire when it is plucked if its diameter is $ 0.5$ mm. 
\item (2017)  A $ 40$ cm long wire is in unison with a tuning fork of frequency $ 256$ Hz, when stretched by a load of density $ 9$ gm$ ^{-3}$ hanging vertically. The load is then immersed in water. By how much the length of the wire should be reduced to bring it again in unison with the same tuning fork,
\end{itemize}

\subsection{Wave Motion}
\begin{itemize}
\item (2000)  What vibrates in the following types of wave motion?
 \begin{itemize}
\item Light waves
\item Sound waves
\item X-rays
\item Water waves
\end{itemize}
\item (2000)  A plane progressive wave on a water surface is given by the equation $ y=2 \sin 2x(100t -x/30)$ ; where $ x$ is the distance covered in a time $ t$ . $ x$ , $ y$ and $ t$ are in cm and seconds respectively.  Find:
 \begin{itemize}
\item the wavelength, and frequency of the wave motion.
\item the phase difference between two points on the water surface that are $ 60$ cm apart.
\end{itemize}
\item (2007)  Give two $ (2)$ differences between progressive and standing waves.
\item (2007)  Two progressive waves travelling along the same line in a medium are represented by $ Y_{1}=10 \sin(\omega t +\pi/2)$ and $ Y_{2}=10 \sin(\omega t +\pi/6)$
 \begin{itemize}
\item If the two progressive waves form a standing wave, determine the resultant amplitude and phase angle of the wave formed.
\end{itemize}
\item (2010)  Distinguish between stationary waves and progressive waves.
\item (2010)  A wave is represented by the equation $ y=10 \sin(0.42\pi(60$ t-x)), where the distance parameters are measured in metres and the time in seconds.
 \begin{itemize}
\item State whether the wave is stationary or progressive.
\item Determine the wavelength and frequency of the wave.
\item What will be the phase difference between two points which are $ 40$ cm apart? 
\item Calculate the period and amplitude of the wave. 
\end{itemize}
\item (2013)  Define the term standing wave.
\item (2013)  State the position in a stationary wave where a man can hear a louder sound.
\item (2016)  State the principle of:
 \begin{itemize}
\item Superposition of waves
\item Huygens construction of wave fronts.
\end{itemize}
\item (2017)  The equation $ y= a  \sin(\omega t – kx)$ represents a plane wave traveling in a medium along the $ x$ - direction, $ y$ being the displacement at the point $ x$ at time $ t$ . Deduce whether the wave is traveling in the positive $ x$ – direction or in the negative $ x$ – direction.
 \begin{itemize}
\item If $ z=1.1 \times 10^{-7}$ m , $ \omega = 6.5 \times 10^{3}$ s$ ^{-1}$ , $ k=19$ m$ ^{-1}$ ; determine the speed of the wave.
\end{itemize}
\item (2018)  What do you understand by the terms:
 \begin{itemize}
\item Progressive wave 
\item Refraction of waves 
\item Diffraction of waves 
\item Standing wave. 
\end{itemize}
\item (2018)  Two progressive waves traveling in the opposite direction in the medium are represented by $ Y_{1}=5 \sin(\omega t+\pi/3)$ and  $ Y_{2}=5 \sin(\omega t- \pi/3)$ . If the two progressive waves form a standing wave, determine the resultant amplitude and the phase angle formed. 
\item (2019)  Give the meaning of the terms wave function, longitudinal wave and transverse waves.
\item (2019)  The equation of a Progressive wave traveling in the $ +x$ direction is given by $ y= a \sin(\omega t-kx)$ .  Show that the maximum velocity, $ V_{max}=2\pi a /T$ . 
\end{itemize}

\subsection{Sound}
\begin{itemize}
\item (2000)  Two open organ pipes of length $ 50$ cm and $ 51$ cm respectively give beat frequency of $ 6.0$ Hz when sounding their fundamental notes together, neglecting end corrections. What value does this give for the velocity of sound in air?
\item (2007)  A metre-long tube at one end, with a movable piston at the other end, shows resonance with a fixed frequency source (a tuning fork) of frequency $ 340$ Hz when the tube length is $ 25.5$ cm or $ 79.3$ cm.  Estimate the speed of sound in air at the temperature of the experiment (ignore edge effects).
\item (2007)  The shortest length of the resonance tube closed at one end which resounds to a fork of frequency $ 256$ Hz is $ 32.0$ cm.  The corresponding length for a fork of frequency $ 384$ Hz is $ 20.8$ cm.  Determine the end correction for the tube and the velocity of sound in air.
\item (2013)  A small speaker emitting $ 4$ note of frequency $ 250$ Hz is placed over the open upper end of a vertical tube which is full of water. When the water is gradually run out of the tube the air column resonates. If the initial and final position of the water surface below the top are $ 0.31$ m and $ 0.998$ m respectively, calculate the speed of sound in air and the end-correction of the tube. 
\item (2015)  A source of sound emits waves of frequency, $ f$ , and is moving with a speed of $ u_{s}$ towards the listener and away from the listener.  Derive an expression for apparent frequency $ f_{A}$ of sound in each case if the velocity of sound wave in air is v.  
\item (2016)  Define the following terms:
 \begin{itemize}
\item Intensity of sound
\item Beats
\item Ultrasonic
\item Overtones
\end{itemize}
\item (2016)  Give any two applications of ultrasonic as applied to sound waves.
\item (2017)  Briefly explain why diffraction is common in sound but not in light.
\item (2018)  The shortest length of the resonance tube closed at one end which resounds to fork of frequency $ 256$ Hz is $ 31.6$ cm, The corresponding length for a fork of frequency $ 384$ Hz is $ 20.5$ cm. Determine the end correction for the tube and the velocity of sound in air. 
\item (2019)  Provide one evidence which proves that sound is a wave.
\item (2019)  Why thunder of lightning is heard some moments after seeing the flash?
\end{itemize}

\subsection{Electromagnetic Waves (em-waves)}
\begin{itemize}
\item (2018)  What do you understand by the term photon. 
\item (2018)  List down any three properties of a photon. 
\end{itemize}

\subsection{Physical Optics (interferance/diffraction/polarization)}
\begin{itemize}
\item (1998)  What is a diffraction grating?
\item (1998)  A diffraction grating has $ 5000$ lines per centimetre. At what angles will bright diffraction images be observed, if it is used with monochromatic light of wavelength $ 6.0 \times 10^{-7}$ m at normal incidence?
\item (1998)  A lamp emits two wavelengths, $ 4.2 \times 10^{-7}$ m and $ 6.0 \times 10^{-7}$ m. Find the angular separation of these two waves in the third order diffraction pattern produced by a diffraction grating having $ 4000$ lines per centimetre, when light is at normal incidence on the grating?
\item (1999)  What is the difference between refraction and diffraction as applied to waves?
\item (1999)  A parallel beam containing two wavelengths $ 600$ nm and $ 602$ nm is incident on a diffraction grating with $ 400$ lines per mm. Calculate the angular separation of the first order spectrum of the two wavelengths. ($ 1$ nm $ =10^{-9}$ m)
\item (2000)  Explain briefiy the necessary conditions for the effects of interference in optics to be observed
\item (2000)  Interference patterns are formed when using Young’s double slit arrangement. Mention other three methods that can be used to form interference patterns.
\item (2000)  A beam of monochromatic light of wavelength $ 600$ nm in air passes into glass. Calculate:
 \begin{itemize}
\item the speed of light in glass.
\item the frequency of light.
\item the wavelength of light in glass.
\end{itemize}
\item (2000)  What is meant by “diffraction grating”?
\item (2000)  A monochromatic light of wavelength $ 5.2 \times 10^{-7}$ m falls normally on a grating which has $ 4 \times 10^{3}$ lines per cm.
 \begin{itemize}
\item What is the largest order of spectrum that can be visible?
\item Find the angular separation between the third and fourth order image.
\end{itemize}
\item (2007)  Using the notation of energy bands, explain the following optical properties of solids.
 \begin{itemize}
\item  All metals are opaque to light of all wavelengths.
\item  Semi-conductors are transparent to infrared light although opaque to visible light.
\item  Most insulators are transparent to visible light.
\end{itemize}
\item (2007)  Describe briefly the formation of Newton rings. How would you measure the wavelength of yellow light by use of Newton’s rings? 
\item (2007)  What would happen to the central spot when air rests between the lens and the plate of the apparatus for Newton’s rings? 
\item (2007)  State Rayleigh’s criterion for the resolution of two objects. 
\item (2007)  The diameter of the pupil of the human eye is $ 2$ mm in bright light.
 \begin{itemize}
\item What is its resolving power with light of wavelength lamda $ =5 \times 10^{-7}m$ ? 
\item Would it be possible to resolve two large birds $ 30$ cm apart sitting on a wire$ 1.5 \times 10^{3}m$ away at daytime? 
\item What would the situation be at night when the pupil dilates to $ 4$ mm? 
\end{itemize}
\item (2007)  What is meant by the back e.m.f. (polarization potential) in a water voltameter? 
\item (2009)  What is interference?  Explain the term path difference with reference to the interference of two wave-trains.
\item (2009)  Why is it not possible to see interference when the light beams from head lamps of a car overlap?
\item (2009)  Discuss whether it is possible to observe an interference pattern when white light is shone on a Young’s double slit experiment.
\item (2009)  A grating has $ 500$ lines per millimetre and is illuminated normally with monochromatic light of wavelength $ 5.89 \times 10^{-7}$ m.
 \begin{itemize}
\item How many diffraction maxima may be observed?
\item Calculate the angular separation.
\end{itemize}
\item (2013)  What is an electron microscope? 
\item (2013)  Outline three disadvantages of electron microscope.
\item (2013)  Draw a schematic diagram of an electron microscope showing its main parts.
 \begin{itemize}
\item Give the order of resolution of electron microscope in the question above.
\end{itemize}
\item (2013)  What is meant by crossed polaroids? 
\item (2013)  Briefly describe the appearance of fringes produced by monochromatic fight.
\item (2013)  Give two difference between diffracting grating spectra and prism spectra.
\item (2013)  A diffraction grating used at normal incidence gives a yellow line. $ \lambda =5750$ A in a certain spectral order: superimposed on a blue line, $ \lambda =4600$ A of the next higher order, If the angle of diffraction is $ 30^{\circ}$ , what is the spacing between the grating lines? 
\item (2013)  State Huygens principle of wave construction. 
\item (2013)  A thin wedge of air of small angle ts enclosed by two thin glass plates. When the plates are illuminated by a parallel beam of monochromatic light of wavelength $ 589$ nm, the distance apart of the fringes is $ 0.8$ mm. Calculate the angle of the wedge. 
\item (2015)  What is meant by the statement that light is plane polarized.
\item (2015)  State Brewster’s law.
\item (2015)  Sunlight is reflected from a calm lake.  The reflected sunlight is totally polarized.  What is the angle between the sun and the horizon.
\item (2015)  State four conditions for sustained interference of light.
\item (2015)  In a Young’s double slit experiment the interval between the slits is $ 0.2$ mm.  For the light of wavelength $ 6.0\times 10^{-7}$ m, Find the distance of the second dark fringe from the central fringe.
\item (2015)  Distinguish between diffraction and diffraction grating.
\item (2015)  A parallel beam of the monochromatic light is incident normally on a diffraction grating.  The angle between the two first-order spectra on either side of the normal is $ 30^{\circ}$ .  Assume that the wavelength of the light is $ 5893\times 10^{14}$ m. Find the number of ruling per mm on the grating and the greatest number of bright images obtained. 
\item (2016)  The incident parallel light is a monochromatic beam of wavelength $ 450$ nm.  The two slits $ A$ and $ B$ have their centres, a distance of $ 0.3$ mm apart.  The screen is situated a distance of $ 2.0$ m from the slits.
 \begin{itemize}
\item Calculate the spacing between fringes observed on the screen.
\item How would you expect the pattern to change when the slits $ A$ and $ B$ are each made wider?
\end{itemize}
\item (2016)  Describe the formation of interference patterns by using Newton’s rings experiment.
 \begin{itemize}
\item Calculate the radius of curvature of a Plano-convex lens used to produce Newton’s rings with a flat glass plate if the diameter of the tenth dark ring is $ 4.48$ mm, viewed by normally reflected light of wavelength $ 5.0 \times 10^{-7}$ m.  What is the diameter of the twentieth bright ring?
\end{itemize}
\item (2017)  Explain the advantage of using optical fibre systems instead of coaxial cable systems in telecommunication processes.
\item (2017)  In a Young's double - slit experiment a total of $ 23$ bright fringes occupying $ 4$ total distance of $ 3.9$ mm were visible in traveling microscope, which was focused on a plane being at a distance of $ 31$ cm from the double slit. If the wavelength of light being used was $ 5.5 \times 10^{-7}$ m; determine the separation of the double slit.
\item (2017)  When a grating with $ 300$ lines per millimeters is illuminated normally with parallel beam of monochromatic light a second order principal maximum is observed at $ 18.9^{\circ}$ to the straight through direction. Find the wavelength of the light.
\item (2017)  A white light fall on a slit of width ‘a’: for what value of 'a' will be the first minimum of light falling at the angle of $ 30^{\circ}$ when the wavelength of light is $ 6500$ nm? 
\item (2018)  What do you understand by the term interference of waves?
\item (2018)  A viewing screen is separated from a double-slit source by $ 1.2$ m. The distance between the two slits is $ 0.030$ mm. The second order bright fringe $ (m=2)$ is $ 4.5$ cm from the centre line. Determine the wavelength of the light and the distance between adjacent bright fringes. 
\item (2018)  Define the term coherent sources of light. 
\item (2018)  Interference patterns are formed when using Young’s double slit experiment. Mention other three methods that can be used to form interference patterns. 
\item (2018)  A beam of monochromatic light of wavelength $ 680$ nm in air passes into glass.  Calculate: 
 \begin{itemize}
\item The speed of light in glass
\item The frequency of light
\item The wavelength of light in glass
\end{itemize}
\item (2018)  Light of wavelength $ 644$ nm is incident on a grating with a spacing of $ 2.00 \times 10^{-6}$ m. 
 \begin{itemize}
\item What is the angle to the normal of a second order maximum? 
\item What is the largest number of orders that can be visible? 
\item Find the angular separation between the third and fourth order image.
\end{itemize}
\item (2018)  State any four laws of photoelectric emission. 
\item (2019)  Two sheets of a Polaroid are lined up so that their polarization directions are initially parallel. When one sheet is rotated:
 \begin{itemize}
\item How does the transmitted light intensity vary with the angle between the polarization directions of the polaroid? 
\item What angle must the polaroid be rotated to reduce the light Intensity by $ 50\%$ ?
\end{itemize}
\item (2019)  What is meant by diffraction grating?
\item (2019)  A diffraction grating has $ 500$ lines per millimetre when used with monochromatic light of wavelength $ 6 \times 10^{-7}$ m at normal incidence. Determine the angle at which the bright diffraction images will be observed. 
 \begin{itemize}
\item Why other orders of image above can not be observed? 
\end{itemize}
\item (2019)  State Huygens’s principle of wave construction.
\item (2019)  A lens was placed with a convex surface of radius of curvature $ 50.0$ cm in contact with the plane surface such that Newton’s rings were observed when the lens was illuminated with monochromatic light. If the radius of the $ 15$ th ring was $ 2.13$ mm determine the wavelength. 
\end{itemize}

\subsection{Doppler Effect}
\begin{itemize}
\item (1999)  What is a “Doppler Effect”?
\item (1999)  A whistle sound of frequency $ 1200$ Hz was directed to an approaching train moving at $ 48$ km$/$h​ . The whistle-man then listened to the beats between the emitted sound and that reflected from the train. What is the beat frequency detected by the whistle-man?
\item (2000)  Write two uses of Doppler effect.
\item (2000)  An observer standing by a railway track notices that the pitch of an engine whistle changes in the ratio of $ 5$:$ 4$ on passing him. What is the speed of the engine?
\item (2007)  What is meant by Doppler effect? 
\item (2007)  Mention two $ (2)$ common applications of the Doppler shift. 
\item (2007)  Ultra sound of frequency $ 5 \times 10^{6}$ Hz is incident at an angle of $ 30^{\circ}$ to the blood vessel of a patient and a doppler shift of $ 4.5$ KHz is observed. If the blood vessel has a diameter $ 10^{-3}m$ and the velocity of ultrasound is $ 1.5 \times 10^{3}$  $ m/s$ . Calculate the:
 \begin{itemize}
\item blood flow velocity. 
\item volume rate of blood flow. 
\end{itemize}
\item (2015)  What is meant by Doppler effect?
 \begin{itemize}
\item Write down three uses of Doppler effect.
\end{itemize}
\item (2015)  A whistle emitting a sound of frequency $ 440$ Hz is tied to a string of $ 1.5$ m length and rotated with an angular velocity of $ 20$ rad$/$s in the horizontal plane.  Calculate the range of frequencies heard by an observer stationed at a large distance.
\item (2015)  A police on duty detects a drop of a $ 10\%$ in the pitch of the horn of a motor car as it crosses him. Calculate the speed of the car.
\item (2016)  Ultrasound of frequency $ 4.0$ MHz is incident at an angle of $ 30^{\circ}$ to a blood vessel of diameter $ 1.6$ mm.  If a Doppler shift of $ 3.2$ kHz is observed, calculate the blood flow velocity and the volume rate of blood flow.  Assume that the speed of ultrasound is $ 1.5$ km$/$s.
\item (2016)  The absorption spectrum of a faint galaxy is measured and the wavelength of one of the lines identified as the calcium $ H$ line is found to be $ 478$ nm.  The same line has a wavelength of $ 397$ nm when measured in a laboratory. 
 \begin{itemize}
\item Is the galaxy moving towards or away from the observer on the Earth?
\item Determine the speed of the galaxy relative to observer on the Earth.
\end{itemize}
\item (2017)  A cyclist and a railway train are approaching each other with a speed of $ 10$ m$/$s and $ 20$ m$/$s respectively. If the engine driver sounds a warning siren at a frequency of  $ 480$ Hz, calculate the frequency of the noise heard by the cyclist:
 \begin{itemize}
\item Before the train has passed.
\item After the tram has passed. 
\end{itemize}
\item (2019)  What is Doppler effect? 
\item (2019)  The cyclist moving at $ 10$ m$/$s and the railway train at $ 20$ m$/$s are approaching each other. If the engine driver sounds a warming siren at a frequency of $ 480$ Hz:
 \begin{itemize}
\item calculate the frequency of the note heard by the cyclist before and after the train has passed away. 
\end{itemize}
\end{itemize}

\end{document}

\documentclass{article}
\usepackage[a4paper, total={6in, 8in}]{geometry}
% \usepackage[utf8]{inputenc}
\usepackage{abstract}
\newcommand{\mysection}[2]{\setcounter{section}{#1}\addtocounter{section}{-1}\section{#2}}

\title{\textbf{12 - Environmental Physics}}
\author{PJ Gibson - Peace Corps Tanzania}
\date{May 2020}

\begin{document}

\maketitle


\mysection{12}{Environmental Physics}

\subsection{Agricultural Physics}
\begin{itemize}
\item (2013)  Describe the soil temperature with regard to agricultural physics which causes lower crop growth at a particular area.
\item (2014)  Briefly explain the influence of the following climatic conditions for plant growth and development:
 \begin{itemize}
\item Rain fall and water
\item Wind
\end{itemize}
\item (2015)  Explain three techniques applicable for improving soil environment for the best plant growth.
\item (2016)  How do soil environmental components influence plant growth? Give four points.
\item (2017)  Discuss two advantages of windbreaks to plant environment. 
\item (2019)  Give two positive effects of wind on plant growth.
\end{itemize}

\subsection{Energy from the environment}
\begin{itemize}
\item (2016)  Briefly explain three major concepts on solar wind.
\item (2017)  State three sources of heat energy within the interior of the earth. 
\item (2018)  What is meant by solar constant? 
\item (2018)  List two factors on which the solar constant depends. 
\item (2018)  Give two advantages of photovoltaic system. 
\item (2018)  Briefly explain how photovoltaic cells work. 
\item (2018)  Estimate the maximum power available from $ 10$ m$ ^{2}$ of solar panels.  Calculate the volume of water per second which must pass through if the inlet and outlet temperature of the panels are at $ 10^{\circ}$C and $ 60^{\circ}$C respectively. (Assume the wave carries away energy at the same rate as the maximum power available)
\end{itemize}

\subsection{Earthquakes}
\begin{itemize}
\item (1998)  Explain the following terms: Earthquake, Earthquake focus, Epicentre and Body waves.
\item (1998)  List down three $ (3)$ sources of earthquakes.
\item (2000)  With reference to an earthquake on a certain point of the earth explain the terms ‘Focus’ and ‘Epicentre’.
\item (2000)  Describe two ways by which seismic waves may be produced.
 \begin{itemize}
\item Describe briefly the meaning and application of “seismic prospecting”. 
\end{itemize}
\item (2007)  What are the difference between $ P$ and s waves?
\item (2007)  Explain how the two terms of waves ($ P$ and $ S$ ) can be used in studying the internal structure of the earth. 
\item (2007)  What is geomagnetic micropulsation.
\item (2010)  Explain the following terms Earthquake, Earthquake focus and Epicenter.
\item (2010)  Describe clearly how $ P$ and s waves are used to ascertain that the outer core of the Earth is in liquid form. 
\item (2013)  The main interior of the earth (core) is believed to be in molten form. What seismic evidence supports this belief?
\item (2015)  What is the origin of earthquake?
\item (2015)  A large explosion at the earth's surface creates compressional (P) and shear (S) waves moving with a speed of $ 6.0$ km$/$s and $ 3.5$ km$/$s respectively. If both waves arrive at seismological station with $ 30$ s interval, calculate the distance measured between seismological station and the site of explosion. 
\item (2019)  What 's meant by epicentre and wind belt as used in Geophysics? 
\item (2019)  Identify three types of seismic waves.
 \begin{itemize}
\item Outline two characteristics of each type of wave described above.
\end{itemize}
\end{itemize}

\subsection{Environmental Pollution}
\begin{itemize}
\item (1998)  Define ionosphere.
\item (1998)  Mention the ionospheric layers that exist during the day time.
\item (2000)  What is the importance of the following layers of the atmosphere?
 \begin{itemize}
\item The lowest layer
\item The ionosphere
\end{itemize}
\item (2007)  Give a summary of location, constitution and practical uses of the stratosphere, ionosphere, and mesosphere.
\item (2010)  Define the ionosphere and give one basic use of it.
\item (2010)  Why is the ionosphere obstacle to radio astronomy?
\item (2013)  Explain why the small ozone layer on the top of the stratosphere is crucial for human survival
\item (2013)  Electrical properties of the atmosphere are significantly exhibited in the ionosphere.
 \begin{itemize}
\item  What is the layer composed of and what do you think is the origin of such constituents.
\item  Mention two uses of the ionosphere.
\end{itemize}
\item (2013)  Briefly explain on the following types of environmental pollution:
 \begin{itemize}
\item  Thermal pollution.
\item  Water pollution.
\end{itemize}
\item (2014)  Describe the sources and effects of the following pollutants on the environment:
 \begin{itemize}
\item Air pollution. 
\item Radiation pollution.
\end{itemize}
\item (2016)  What is meant by aerial environment?  Give two examples.
\item (2016)  Describe three ways at which the aerial environment is threatened.
\item (2017)  Give two factors which determine whether a planet has an atmosphere or not.
\item (2017)  Briefly explain the major causes of the following types of environmental pollution:
 \begin{itemize}
\item Water pollution. 
\item  Air pollution.
\end{itemize}

\end{itemize}

\end{document}
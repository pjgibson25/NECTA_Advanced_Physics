
\documentclass{article}
\usepackage[a4paper, total={6in, 8in}]{geometry}
% \usepackage[utf8]{inputenc}
\usepackage{abstract}
\newcommand{\mysection}[2]{\setcounter{section}{#1}\addtocounter{section}{-1}\section{#2}}

\title{\textbf{8 - Electromagnetism}}
\author{PJ Gibson - Peace Corps Tanzania}
\date{May 2020}

\begin{document}

\maketitle


\mysection{8}{Electromagnetism}

\subsection{Magnetic Fields}
\begin{itemize}
\item (2000)  A proton is moving in a uniform magnetic field $ B$ . Draw the diagram representing $ B$ and the path of the proton if its initial direction makes an oblique angle to the direction of the field $ B$ . 
\item (2007)  Define the magnetic field intensity.
\item (2007)  A long solenoid has $ 10$ turns per cm and carries a current of $ 2.0$ A.  Calculate the magnetic field intensity at its centre.
\item (2007)  An electron having $ 450$ eV of energy enters at right angles to a uniform magnetic field of strength $ 1.50x10^{-3}$ T.  Show that the path traced by the electron in a uniform magnetic field is circular and estimate its radius.
\item (2007)  A charged oil drop of mass $ 6.0x10^{-15}$ kg falls vertically in air with a steady velocity between two long parallel vertical plates $ 5.0$ mm apart.  When a potential difference of $ 3000$ V is applied between the plates the drop falls with a steady velocity at an angle of $ 58^{\circ}$ to the vertical.
 \begin{itemize}
\item Determine the charge $ Q$ , on the oil drop.
\end{itemize}
\item (2007)  A coil having $ 475$ turns and cross sectional area $ 20 cm^{2}$ , rotates at $ 600r.p.m$ . in a uniform magnetic field of $ 0.01$ T. Find:
 \begin{itemize}
\item the peak e.m.f and the r.m.s. e.m.f induced in the coil. 
\item show these values on a graph of $ E$ vs time. 
\end{itemize}
\item (2009)  Outline four applications of eddy currents.
\item (2010)  Distinguish between magnetic flux density and magnetic induction.
\item (2010)  Describe using a sketch graph how magnetic flux density varies with the axis (both inside and at the ends) of a long solenoid carrying current. 
\item (2010)  A solenoid $ 80$ m long has a cross-sectional area of $ 16$ cm$ ^{2}$ and a total of $ 3500$ turns closely wound. If the coil is filled with air and carries a current of $ 3$ A, Calculate:
 \begin{itemize}
\item Magnetic field density $ B$ at the middle of the coil.
\item Magnetic flux inside the coil. 
\item Magnetic force $ H$ at the centre of the coil. 
\item Magnetic induction at the end of the coil.
\item $ (v$ ) Magnetic field intensity at the middle of the coil. 
\end{itemize}
\item (2013)  Mention the factors which determine the magnitude and direction of the force experienced by a current-carrying conductor in a magnetic field.
\item (2013)  What is the maximum torque on a $ 400-$ turns circular coil of radius $ 0.75$ cm that carrying a current of $ 1.6$ mA and resides in a uniform magnetic field of $ 0.25$ T?
\item (2013)  Brielfly explain how you can demonstrate that there are two types of charges in nature.
\item (2013)  A $ 10$ eV proton is circulating in a plane at right angles to a uniform magnetic field of magnetic flux density of $ 1.0 \times 10^{-4}$ Wb$/$m$ ^{2}$ Calculate the cyclotron frequency of a proton.
\item (2013)  A toroid of inner radius $ 25$ cm and an outer radius of $ 28$ cm has $ 4500$ turns of wound around it which passes a Current of $ 12$ A. What will be the induction of the magnetic flux;
 \begin{itemize}
\item Outside the toroid. 
\item inside the core of the toroid, 
\item in an empty space surrounding the toroid. 
\end{itemize}
\item (2016)  What is meant by the following terms:
 \begin{itemize}
\item  Phase of alternating e.m.f.
\item  Root mean square (r.m.s.) value of alternating e.m.f.
\end{itemize}
\item (2016)  State the following laws or theorems as applied in magnetism.
 \begin{itemize}
\item Biot-Savart law
\item Ampere’s theorem
\end{itemize}
\item (2016)  Derive an expression for the magnetic flux density $ B$ at the centre of the circular coil of radius $ r$ and $ N$ turns placed in air carrying a current i.
\item (2016)  The diameter of a $ 40$ turn circular coil is $ 16$ cm and it has a current of $ 5$ A.  Calculate:
 \begin{itemize}
\item The magnetic induction at the centre of the coil
\item The magnetic moment of the coil.
\item The torque action on the coil if it is suspended in a uniform magnetic field of $ 0.76$ T such that its plane is parallel to the field.
\end{itemize}
\item (2017)  Draw the diagram of the solenoid with certain number of tums placed in the magnetic field and indicate any suitable directions of the flow of current in it.
\item (2017)  Write down the formula for the magnetic field induced at the centre of solenoid. 
\item (2017)  It is desired to design a solenoid that produces a magnetic field of $ 0.1$ T at the centre. If the radius of solenoid is $ 5$ cm, its length is $ 50$ cm and carries a current of $ 10$ A; Calculate:
 \begin{itemize}
\item The number of turns per unit length of the solenoid. 
\item The total length of a wire required. 
\end{itemize}
\item (2017)  State the Biot-Savart law. 
\item (2017)  In a hydrogen atom, an electron keeps moving around its nucleus with a constant speed of $ 2.18 \times 10^{6}$ m$/$s. Assuming that the orbit is a circular of radius $ 5.3 \times 10^{-11}$ m. determine the magnetic flux density produced at the site of the proton in the nucleus. 
\item (2018)  A circular coil of $ 300$ turns has a radius of $ 10$ cm and carries a current of $ 7.5$ A. Calculate the magnetic field at:
 \begin{itemize}
\item the centre of the coil. 
\item a point which is at a distance of $ 5$ cm from the centre of the coil. 
\end{itemize}
\item (2019)  Identify four factors that affect the force experienced by a current-carrying conductor in a magnetic field. 
\item (2019)  Write the mathematical expression which define magnetic flux density and use it to deduce its S.I. units. 
 \begin{itemize}
\item Apply an expression obtained above to develop the formula for the force on a conductor carrying current i if the conductor and the magnetic fields are not at night angles.
\end{itemize}
\item (2019)  State the condition which makes the magnetic force on a moving charge in a magnetic field to be maximum. 
\item (2019)  Use mathematical expression to justify the statement that there will be no change in the kinetic energy of a charged particle which enters a uniform magnetic field when its initial velocity is directed parallel to the field.
\end{itemize}

\subsection{Magnetic Properties of Materials}
\begin{itemize}
\item (1999)  With the help of clear diagrams, explain briefly how you would convert a sensitive galvanometer into:
 \begin{itemize}
\item an ammeter
\item a voltmeter
\end{itemize}
\item (2007)  List three $ (3)$ classes of magnetic materials on the basis of magnetic susceptibility and give one example for each class.
\item (2007)  How are the magnetic susceptibility and relative permeability of a magnetic material related to each other?
\item (2007)  State the main differences between.
 \begin{itemize}
\item diamagnetism and paramagnetism. 
\item ferromagnetism and auntiferromagnetism. 
\item ferromagnetism and ferrielectricity. 
\end{itemize}
\item (2007)  Draw hysteresis loops diagrams for soft iron and hard steel and use them to discuss:
 \begin{itemize}
\item permanent magnets.
\item electromagnets.
\item transformer cores. 
\end{itemize}
\item (2016)  Draw hysteresis loops diagram for soft iron and hard steel and use them to discuss permanent magnets.
\item (2016)  Define permeability constant.
\item (2018)  Mention the three magnetic materials and briefly explain each one. 
 \begin{itemize}
\item Give the differences between the magnetic materials mentioned above in terms of their magnetic susceptibility. 
\end{itemize}
\item (2018)  Define the following terms:
 \begin{itemize}
\item Ampere 
\item Hysteresis 
\end{itemize}
\item (2019)  Distinguish the terms magnetically soft and magnetically hard materials.
\end{itemize}

\subsection{Magnetic Forces}
\begin{itemize}
\item (1998)  An electron is projected horizontally with a velocity of $ 2.0 \times 10^{6}$ ms$ ^{-1}$ into a large evacuated enclosure. A magnetic field which has a flux density of $ 15 \times 10^{-4}$ tesla is directed vertically downwards throughout the enclosure. Find
 \begin{itemize}
\item the radius of curvature of the electron's path.
\item how many complete loops must the electron describe before it falls by $ 1.0$ cm under the influence of gravity?
\item What would be the effect of changing the direction of the magnetic field to upwards?
\end{itemize}
\item (2000)  An electron with charge $ e$ and mass $ m_{e}$ is initially projected with a speed v at right angles to a uniform magnetic field of flux density $ B$ .
 \begin{itemize}
\item Explain why the path of the electron $ 1$ s circular.
\item Show also that the time to describe one complete circle is independent of the speed of the electron.
\end{itemize}
\item (2000)  Calculate the radius of the path traversed by an electron of energy $ 450$ eV moving at right angles to a uniform magnetic field of flux density $ 1.5\times 10^{-3}$ T.
\item (2009)  Develop an equation for the torque acting on a current carrying coil of dimensions lxb placed in a magnetic field.  How is this effect applied in a moving coil galvanometer?
\item (2009)  A galvanometer coil has $ 50$ turns, each with an area of $ 1.0 $ cm$ ^{2}$ .  If the coil is in a radian field of $ 10^{-2}$ T and suspended by a suspension of torsion constant $ 2 \times 10^{-9}$ Nm per degree, what current is needed to give a deflection of $ 30^{\circ}$ ?
\item (2009)  Give a general form expressing the force exerted on the wire carrying current i if its length $ l$ is inclined at angle angle $ \theta $ to the magnetic field $ B$ .  
\item (2009)  A wire carrying a current of $ 2$ A has a length of $ 100$ mm in a uniform magnetic field of $ 0.8$ Wb$/$m$ ^{2}$ .  Find the force acting on the wire when the field is at $ 60^{\circ}$ to the wire.
\item (2009)  A wire carrying a current of $ 25$ A and $ 8$ m long is placed in a magnetic field of flux density $ 0.42$ T . What is the force on the wire if it is placed:
 \begin{itemize}
\item At right angles to the field?
\item At $ 45^{\circ}$ to the field?
\item Along the field?
\end{itemize}
\item (2013)  Derive the formula for the torque acting of the rectangular current-carrying coil in a magnetic field
\item (2013)  Give comment on the statement that, an electron suffers no force when it moves parallel to the magnetic field, $ B$ .
\item (2015)  A horizontal straight wire $ 0.05$ m long weighing $ 2.4$ g$/$m is placed perpendicular to a uniform horizontal magnetic field of flux density $ 0.8$ T.  If the resistance of the wire is $ 7.6\Omega /$m, calculate the potential difference that has to be applied between the ends of the wire to make it just self-supporting.
\item (2015)  Two very long wires made of copper and of equal lengths are placed parallel to each other in such a way that they are $ 10$ cm apart.  If the total power dissipated in the two wires is $ 75$ W, find the force between them if the resistivity of the copper wire is $ 1.69	imes 10^{-8}\Omega m$ and of diameter $ 2$ mm.
\item (2017)  State the law of force acting on a conductor of length $ l$ carrying an electric current in a magnetic field. 
\item (2019)  Determine the magnitude of force experienced by a stationary charge in a uniform magnetic field. 
\end{itemize}

\subsection{Electromagnetic Induction}
\begin{itemize}
\item (1998)  Define the term self inductance for a coil.
\item (1998)  Give the S.I units of self inductance.
\item (1998)  Derive an expression for the coefficient of self induction of a uniformly wound solenoid; of length $ 1$ , cross-sectional area A having $ N$ turns in air.
\item (1998)  Two coils $ A$ and $ B$ have $ 200$ and $ 800$ turns respectively. A current of $ 2$ amperes in A produces a magnetic flux of $ 1.8 \times 10^{-4}$ Wb in each turn of $ B$ . Compute:
 \begin{itemize}
\item the mutual inductance.
\item the magnetic flux through A when there is a current of $ 4.0$ amperes in $ B$ and
\item the emf induced in $ B$ when the current in A changes from $ 3$ amperes to $ 1$ ampere in $ 0.2$ seconds.
\end{itemize}
\item (1999)  State the laws of electromagnetic induction and describe briefly experiments (one in each case) which can be used to demonstrate them.
\item (2007)  State Faraday’s two $ (2)$ laws of electrolysis and calculate the value of Faradays constant given that the e.c.e. of copper is $ 3.30 \times 10^{-7}$ kg$/C$ and the copper is a divalent element. 
\item (2007)  A piece of metal weighing $ 200g$ is to be electroplated with $ 5\%$ of its weight in gold. If the strength of the available current is $ 2$ A, how long would it take to deposit the required amount of gold?
\item (2007)  State Faraday’s law of electromagnetic induction. 
\item (2007)  A coil of cross section area A rotates with an angular velocity $ \omega $ in a uniform. magnetic field, $ B$ . Derive the equation for induced e.m.f. of the system.
\item (2009)  State the laws of electromagnetic induction.
\item (2009)  A coil of $ 100$ turns is rotated at $ 1500$ revolutions per minute in a magnetic field of uniform density $ 0.05$ T.  If the axis of rotation is at right angles to the direction of the flux and the area per turn is $ 4000 $ mm$ ^{2}$ .  Calculate the:
 \begin{itemize}
\item Frequency
\item Period
\item Maximum induced e.m.f.
\item Maximum value of the induced e.m.f. when the coil has rotated through $ 30^{\circ}$ from the position of zero e.m.f.
\end{itemize}
\item (2013)  State the laws of electromagnetic induction.
\item (2013)  State Lenz’s Jaw of electromagnetic induction.
\item (2015)  Distinguish between self-inductance and mutual inductance.
\item (2018)  Consider a small flat coil which has $ N$ turns of area A and whose plane is perpendicular to a magnetic field of flux density $ B$ . If the search coil is connected to the ballistic galvanometer and the total resistance of the circuit is $ R$ , use the laws of electromagnetic induction to show that the charge delivered to the galvanometer does not depend on how long it takes to remove the search coil from the field. 
\item (2019)  At which position of the rotating coil in the magnetic field, the induced e.m.f. is zero? Give a reason. 
\end{itemize}

\subsection{Magnetic Field of the Earth}
\begin{itemize}
\item (1999)  A flat coil of $ 100$ turns and mean radius $ 5.0$ cm is tying on a horizontal surface and is turned over in $ 0.20$ sec. against the vertical component of the Earth's magnetic field. Calculate the average e.m.f. induced.
\item (2007)  Write short notes on the following terms in relation to changes in the Earth's magnetic field:  long-term (secular) changes, short-period (regular) changes and short-term (irregular) changes.
\item (2013)  An aircraft is flying horizontally at $ 200$ m$/$s through the region where the vertical component of the earth magnetic field is $ 4.0 \times 10^{-5}$ T. If the air craft has a wing span of $ 40$ m, what will be the potential difference (p.d.) produced between the wing tips? 
\item (2015)  List down three sources of earth's magnetism. 
\item (2016)  State any three magnetic components of the earth’s magnetic field.
\item (2016)  The horizontal and vertical components of the Earth’s magnetic field at a certain location are $ 2.7 \times 10^{-5}$ T and $ 2.0 \times 10^{-5}$ T respectively.  Determine the Earth’s magnetic field at the location and its angle of inclination i.
\end{itemize}

\end{document}

\documentclass{article}
\usepackage[a4paper, total={6in, 8in}]{geometry}
% \usepackage[utf8]{inputenc}
\usepackage{abstract}
\title{8 - Electromagnetism}
\author{PJ Gibson - Peace Corps Tanzania}
\date{May 2020}

\begin{document}

\maketitle


\section{Electromagnetism}

\subsection{Magnetic Fields}
\begin{itemize}
\item (2019)  Identify four factors that affect the force experienced by a current-carrying conductor in a magnetic field. 
\item (2019)  Write the mathematical expression which define magnetic flux density and use it to deduce its S.I. units. 
 \begin{itemize}
\item Apply an expression obtained above to develop the formula for the force on a conductor carrying current i if the conductor and the magnetic fields are not at night angles.
\end{itemize}
\end{itemize}

\subsection{Magnetic Properties of Materials}
\begin{itemize}
\item (2019)  Distinguish the terms magnetically soft and magnetically hard materials.
\end{itemize}

\subsection{Magnetic Fields}
\begin{itemize}
\item (2019)  State the condition which makes the magnetic force on a moving charge in a magnetic field to be maximum. 
\end{itemize}

\subsection{Magnetic Forces}
\begin{itemize}
\item (2019)  Determine the magnitude of force experienced by a stationary charge in a uniform magnetic field. 
\end{itemize}

\subsection{Electromagnetic Induction}
\begin{itemize}
\item (2019)  At which position of the rotating coil in the magnetic field, the induced e.m.f. is zero? Give a reason. 
\end{itemize}

\subsection{Magnetic Fields}
\begin{itemize}
\item (2019)  Use mathematical expression to justify the statement that there will be no change in the kinetic energy of a charged particle which enters a uniform magnetic field when its initial velocity is directed parallel to the field.

\end{itemize}

\end{document}
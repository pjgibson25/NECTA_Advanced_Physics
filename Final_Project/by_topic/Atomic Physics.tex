
\documentclass{article}
\usepackage[a4paper, total={6in, 8in}]{geometry}
% \usepackage[utf8]{inputenc}
\usepackage{abstract}
\newcommand{\mysection}[2]{\setcounter{section}{#1}\addtocounter{section}{-1}\section{#2}}

\title{\textbf{11 - Atomic Physics}}
\author{PJ Gibson - Peace Corps Tanzania}
\date{May 2020}

\begin{document}

\maketitle


\mysection{11}{Atomic Physics}

\subsection{Structure of the Atom}
\begin{itemize}
\item (1999)  State Bohr’s postulates of the atomic model.
\item (1999)  Show that for an electron in a hydrogen atom, the possible radii of an electron orbit are given by:
 \begin{itemize}
\item $ r_{n}=a_{0}n^{2}$ , $ n=1$ , $ 2$ , $ 3$ , ...
\end{itemize}
\item (2000)  In the Bohr model of the hydrogen atom, an electron circles the nucleus in an orbit of radius $ r$
 \begin{itemize}
\item Explain what keeps the electron in the orbit and why it does not spiral towards the nucleus.
\item What are the assumptions put forward by Bohr about the orbits of the electron in the hydrogen atom?
\end{itemize}
\item (2007)  Develop an expression for electrical energy spent in the decomposition of water. 
\item (2007)  In a hydrogen atom model an electron of mass $ m$ and charge $ e$ rotates about a heavy nucleus of charge $ e$ in a circular orbit of radius $ r$ . Develop an expression for the angular momentum of the electron in terms of $ m$ , $ e$ , $ r$ , $ \pi$ and $ \epsilon  _{0}-$ the permitting of free space.
\item (2007)  The four lowest energy levels in a mercury atom are $ -10.4$ eV, $ -5.5$ eV, $ -3.7$ eV and $ -1.6$ eV.
 \begin{itemize}
\item Determine the ionization energy of mercury in joules. 
\item Calculate the wavelength of the radiation emitted when an electron jumps from $ -1.6$ eV to $ -5.5$ eV energy levels. 
\item What will happen if a mercury atom in its excited state is bombarded with electrons having an energy of $ 11$ eV. 
\end{itemize}
\item (2013)  Given that Rydberg’s constant is approximately $ 1.1 \times 10^{7}$ m$ ^{-1}$ Calculate the corresponding range of frequency for emitted radiation in the:
 \begin{itemize}
\item Lyman series. 
\item Balmer series. 
\end{itemize}
\item (2015)  Why are the energy levels labelled with negative energies?
\item (2016)  The first member of the Balmer series of hydrogen spectrum has wavelength of $ 6563 \times 10^{-10}$ m. Calculate the wavelength of its second member.
\item (2017)  Use the Rydberg constant, $ R_{H}=1.0974 \times 10^{7}$ m$ ^{-1}$ to calculate the shortest wavelength of the Balmer series. 
\item (2017)  Use the Bohr's theory for hydrogen atom to determine the:
 \begin{itemize}
\item Radius of the first orbit of the hydrogen atom in A units. 
\item Velocity of the electron in the first orbit. 
\end{itemize}
\item (2017)  What is ionization potential of an atom?
\item (2017)  Show that the ionization potential of hydrogen is $ 13.6$ eV. 
\item (2017)  How can you account for the chemical behavior of atoms on the basis of the atomic electrons and shells? 
\item (2017)  How can you account for the chemical behavior of atoms on the basis of the atomic electrons and shells? 
\item (2018)  Given: Mass of proton $ =1.0080$ u, Mass of neutron $ =1.0087$ u and Mass of alpha particle $ =4.0026$ u.
 \begin{itemize}
\item State any three limitations of Bohr’s model of the hydrogen atom.
\end{itemize}
\item (2018)  Why hydrogen spectrum contains a larger number of spectral lines although its  atom has only one electron? 
\item (2018)  State any three limitations of Bohr’s model of the hydrogen atom.
\item (2018)  Distinguish between ionization energy and excitation energy.
\item (2018)  Why hydrogen spectrum contains a larger number of spectral lines although its  atom has only one electron? 
\item (2019)  Based on Balmer series of hydrogen spectra determine the wavelength of the series limit of Paschen series. 
\item (2019)  Why hydrogen atom is stable in the ground state? 
\item (2019)  According to Bohr’s theory, the angular momentum of an electron is an integral multiple of $ h/2\pi$ .  Express this statement. by using a mathematical equation in which angular momentum is represented by the letter Land orbit by the letter $ n$ , 
\end{itemize}

\subsection{Quantum Physics}
\begin{itemize}
\item (1999)  What is the “work function” of a metal?
\item (1999)  The work function of a metal is $ 2.0$ eV. Calculate the stopping potential when the metal is illuminated by light of frequency of $ 6.0 \times 10^{14}$ Hz.
\item (2000)  What is the de Broglie wave equation?
\item (2000)  An electron is accelerated through a potential of $ 400$ V. Determine the de Broglie wavelength of this electron.
\item (2000)  Determine the de Broglie wavelength for the beam of electron whose total energy is
 \begin{itemize}
\item $ 250$ eV.
\end{itemize}
\item (2000)  What is a photoelectric cell?
\item (2000)  The emission of electrons from the surface of a cathode of a certain phototube when irradiated with a light of wavelength $ 3500 \times 10^{-10}$ m is found to stop when the plate potential is $ 1.2$ V with respect to the cathode. Determine the work function of the cathode.
\item (2007)  A certain diatomic gas is contained in a vessel whose inner surface is a small absorber which retains any atoms or molecules of gas which strike it.  Show that if doubling the absolute temperature causes one half of the molecules to dissociate into atoms then the rate at which the absorber is gaining mass increases by a factor $ 1+1/\sqrt{2}$ .
\item (2007)  What is a line spectrum? 
\item (2009)  Write down Bragg’s equation for the study of the atomic structure of the crystals by $ X-$ rays.
\item (2009)  The radiation from an $ X$ — ray tube which operates at $ 50$ kV is diffracted by is diffracted by a cubic KCl crystal of molecular mass $ 74.6$ and density $ 1.99 \times 10^{3}$ kg$/$m$ ^{3}$ .  Calculate:
 \begin{itemize}
\item The shortest wavelength limit of the spectrum from the tube.
\item The glancing angle for first order reflection from the planes of the crystal for that wavelength and angle of deviation of a diffracted beam.
\end{itemize}
\item (2009)  The radiation emitted by an $ X$ — ray tube consists of continuous spectrum with a line spectrum superimposed on it. Explain how the continuous spectrum and the line spectrum are produced.
 \begin{itemize}
\item Draw the graph of the spectra stated. ‘
\end{itemize}
\item (2013)  If the energy necessary to cause the ejection of an electron by photoelectric effect from the $ N$ — shell and $ K-$ shell of an atom is $ 10$ eV and $ 20$ eV respectively, calculate the maximum wavelength of radiation for each level.
\item (2015)  Show that the de Broglie hypothesis of matter wave are in agreement with Bohr’s theory.
\item (2015)  Ultraviolet light of wavelength $ 3600 \times 10^{-10}$ m is made to fall on a smooth surface of potassium. Determine:
 \begin{itemize}
\item The maximum energy of emitted photoelectrons
\item The stopping potential.
\item The velocity of the most energetic photoelectrons given that work function for potassium is $ 2$ eV.
\end{itemize}
\item (2016)  Briefly explain the production of X-rays.
\item (2016)  List down any three uses of X-rays.
\item (2016)  How are the intensity and penetrating power of an X-ray beam controlled?
\item (2018)  Briefly explain what led de-Broglie to think that the material particles may also show wave nature and why the wave nature of matter not noticeable in our daily observations? 
\item (2018)  Prove that de-Broglie wavelength $ \lambda $ , of electrons of kinetic energy $ E$ is given by $ \lambda = h/ \sqrt{2}$ meV  where $ m$ is the mass of the electron, $ e$ is the charge of the electron, $ h$ is the Planck’s constant and v is the accelerating potential difference. 
\item (2018)  Light of wavelength $ 488$ nm is produced by an argon laser which is used in the photoelectric effect. When light from this spectral line is incident on the emitter, the stopping (cut-off) potential of photoelectrons is $ 0.38$ V. Find the work function of the material from which the emitter is made. 
\item (2018)  In a hydrogen atom model, an electron of mass $ m$ and charge $ e$ revolves around the nucleus in a circular orbit of radius $ r$ . Develop an expression for the radius $ 3$ m of the orbit in terms of $ m$ , $ e$ , $ x$ , the quantum number $ n$ , Planck constant $ h$ and the permitting of free space $ \epsilon _{0}$ , and hence, use their values to find the Bohr’s radius. 
\item (2018)  In a hydrogen atom model, an electron of mass $ m$ and charge $ e$ revolves around the nucleus in a circular orbit of radius $ r$ . Develop an expression for the radius $ 3$ m of the orbit in terms of $ m$ , $ e$ , $ x$ , the quantum number $ n$ , Planck constant $ h$ and the permitting of free space $ \epsilon _{0}$ , and hence, use their values to find the Bohr’s radius. 
\item (2019)  Why electrons do not fall into the nucleus due to electrostatic force of attraction?
\item (2019)  Determine the angular momentum of the electron in the orbit of energy level $ -3.4$ eV given that $ E_{n}=-13.6/n^{2}$ eV, where $ E$ is the energy of an electron and $ n$ is the principal quantum number of hydrogen atom. 
\end{itemize}

\subsection{LASER}
\begin{itemize}
\item (2000)  Using an example of your own choice explain the mechanism behind the production of a laser beam.
\item (2000)  Describe two applications of a laser
\item (2007)  Explain breifly the action of a helium-neon laser.
\item (2009)  Define the terms laser and maser. 
\item (2009)  Give three applications of laser. 
\item (2009)  A laser beam has a power of $ 20 \times 10^{9}$ watts and a diameter of $ 2$ mm.  Calculate the peak values of electric field and magnetic fields.
\item (2015)  Give any four uses of LASER lgith.
\end{itemize}

\subsection{Nuclear Physics}
\begin{itemize}
\item (1998)  Explain the terms: atomic mass unit, mass defect, packing fraction and binding energy.
\item (1998)  Discuss carbon dating.
\item (1998)  Find the age at death of an organism, if the ratio of amount of C$ 14$ at death to that of the present time is $ 10^{8}$ and that the half life of Cl$ 4$ is $ 5600$ years.
\item (1999)  What is nuclear fusion 
\item (1999)  What is nuclear fission?
\item (1999)  Define the term “binding energy” of a nuclide.
\item (1999)  Distinguish between:
 \begin{itemize}
\item $ \beta -$ decay and $ \beta +$ decay.
\item nuclear fission and nuclear fusion
\item activity and half-life of a radioactive material.
\item Taking the half-life of Radium $ -226$ to be $ 1600$ years, what fraction of a given sample remains after $ 4800$ years?
\end{itemize}
\item (2000)  A sample of soil from Olduvai Gorge cave was examined. It was found to contain, among other things, pieces of charcoal. Further investigation on the charcoal revealed that $ 1$ kg of C$ 14$ nuclei decayed each second. It is assumed that this charcoal has resulted from decomposition of the stone-age people who died there (i.e. at the cave) long time ago. Calculate the number of years that have elapsed since these people died.
\item (2007)  It is not possible to separate the different isotopes of an element by chemical means.  Explain.
\item (2007)  Define a mass spectrometer. 
\item (2007)  Ion A of mass $ 24$ and charge $ +e$ and ion $ B$ of mass $ 22$ and charge $ +2e$ both enter the magnetic field of a mass spectrometer with the same speed. If the radius of A is $ 2.5 \times 10^{-1}m$ , calculate the radius of the circular path of $ B$ . 
\item (2007)  If the ratio of mass of lead – $ 206$  to mass of uranium – $ 238$ in a certain rock was found to be $ 0.45$ and that the rock originally contained no lead – $ 206$ , estimate the age of the rock given that the half life of uranium – $ 238$ is $ 4.5 \times 10^{9}$ years.
\item (2007)  Define the following terms:
 \begin{itemize}
\item Atomic mass unit
\item Binding energy
\item Mass defect.
\end{itemize}
\item (2009)  Explain the following observations:
 \begin{itemize}
\item A radioactive source is placed in front of a detector which can detect all forms of radioactive emissions. It is found that the activity registered as noticeably reduced when a thin sheet of paper is placed between the source and detector.
\item When a brass plate with a narrow vertical shit is placed in front of the radioactive source (above) and a horizontal: magnetic field normal to the line joining the source and the detector is applied, its found that the activity is further reduced.
\item The magnetic field (above) is removed and a sheet of aluminum is placed in front of the source. The activity recorded is similarly reduced.
\end{itemize}
\item (2009)  A $ 2.71$ g sample of Kcl from the chemistry stock is found to be radioactive and decays at a constant rate of $ 4490$ disintegrations per second.  The decays are traced to the element potassium and in particular to the isotope $ ^{40}$ K which constitutes $ 1.17\%$ of normal potassium.  Calculate the half life of the nuclide.
\item (2013)  Distinguish between white spectrum and line spectrum. 
\item (2013)  What is the significance of the binding energy per nucleon? 
\item (2013)  Briefly explain why the $ \beta$  — particles emitted from a radioactive source differ from the electrons obtained by thermionic emission? 
\item (2013)  The mass of a particular radioisotope in « sample is initially $ 6.4 \times 10^{-3}$ kg, After $ 42$ days the isotope was separated from the sample and found to have a mass of $ 1.0 \times 10^{-4}$ kg. Calculate the half- life of the isotope.
\item (2015)  Define activity and half-life.
\item (2015)  The half-life of radioactive substance is $ 1$ hour.  How long will it take for $ 60\%$ of the substance to decay?
\item (2015)  What is a nuclear reactor?
 \begin{itemize}
\item Briefly explain any three main components in a nuclear reactor.
\end{itemize}
\item (2015)  Sketch the binding energy curve.
 \begin{itemize}
\item State any two conclusions that can be drawn from the curve above.
\end{itemize}
\item (2015)  If the mass of deuterium nucleus is $ 2.015$ a.m.u, that of one isotope of helium is $ 3.017$ a.m.u. and that of neutron is $ 1.009$ a.m.u., calculate the energy released by the fusion of $ 1$ kg of deuterium. 
 \begin{itemize}
\item Suppose $ 50\%$ of this energy was used to produce $ 1$ MW of electricity, for how many days would be able to function.
\end{itemize}
\item (2016)  The number of particles $ n$ crossing a unit area perpendicular to $ x-$ axis in a unit time is given as $ n=-D(n_{2}-n_{1})/(x_{2}-x_{1})$ where $ n_{1}$ and $ n_{2}$ are the number of particles per unit volume for the values of $ x_{1}$ and $ x_{2}$ respectively.  What are the dimensions of diffusion constant $ D$ ?
\item (2016)  Differentiate natural radioactivity from artificial radioactivity.
\item (2016)  Name three applications of radioisotopes in medicine.
\item (2016)  State two conditions for stability of nuclides referring to light nuclides and heavy nuclides.
\item (2016)  Derive an expression for the half-life using the radioactive decay law.
\item (2016)  What is carbon $ -14$ ?  Explain its production and how it is used in the dating process.
\item (2016)  Living wood has an activity of $ 16.0$ counts per minute per gram of carbon.  A certain sample of dead wood is found to have an activity of $ 18.4$ counts per minute for $ 4.0$ grams.  Calculate the age of the sample of dead wood.  Assume the half-life of carbon $ -14$ is $ 5568$ years.
\item (2017)  What is meant by the following?
 \begin{itemize}
\item Atomic Mass Unit (a.m.u.)
\item Binding energy. 
\item Mass defect
\end{itemize}
\item (2017)  Write down the equation for the disintegration.
\item (2018)  Use the concept of radioactive decay and nuclear reactions to define the following terms:
 \begin{itemize}
\item $ \alpha $ decay
\item $ \beta$ decay
\item $ \gamma $ decay
\item Fission
\item Fusion.
\item For each of the terms above, give one suitable reaction equation. 
\end{itemize}
\item (2018)  A freshly prepared sample of a radioactive isotope $ Y$ contains $ 10^{12}$ atoms. The half-life of the isotope is $ 15$ hours. Calculate;
 \begin{itemize}
\item the initial activity. 
\item the number of radioactive atoms of $ Y$ remaining after $ 2$ hours, 
\end{itemize}
\item (2018)  Mention any four important features in the design of a nuclear reactor.
\item (2018)  Differentiate binding energy from mass defect.
\item (2018)  Calculate the binding energy per nucleon, in MeV and the packing fraction of an alpha particle.
\item (2018)  A freshly prepared sample of a radioactive isotope $ Y$ contains $ 10^{12}$ atoms. The half-life of the isotope is $ 15$ hours. Calculate;
 \begin{itemize}
\item the initial activity. 
\item the number of radioactive atoms of $ Y$ remaining after $ 2$ hours, 
\end{itemize}
\item (2018)  Mention any four important features in the design of a nuclear reactor.
\item (2018)  Differentiate binding energy from mass defect.
\item (2018)  Calculate the binding energy per nucleon, in MeV and the packing fraction of an alpha particle.
 \begin{itemize}
\item Given: Mass of proton $ =1.0080$ u, Mass of neutron $ =1.0087$ u and Mass of alpha particle $ =4.0026$ u.
\end{itemize}
\item (2019)  What is meant by the following terms as used in nuclear Physics?
 \begin{itemize}
\item Mass defect 
\item Binding energy. 
\end{itemize}
\item (2019)  Elaborate two aspects on which fission reactions differs from fusion reactions.
\item (2019)  Why is high temperature required to cause nuclear fusion? 
\end{itemize}

\end{document}
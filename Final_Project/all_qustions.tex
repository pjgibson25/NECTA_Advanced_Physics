
\documentclass{article}
\usepackage[a4paper, total={6in, 8in}]{geometry}
% \usepackage[utf8]{inputenc}
\usepackage{abstract}
\renewcommand{\abstractname}{Description}
\title{Categorized Advanced Physics ACSEE Questions}
\author{PJ Gibson - Peace Corps Tanzania}
\date{May 2020}

\begin{document}

\maketitle

\begin{abstract}

Below are 1046 physics questions from previous Tanzanian ACSEE physics exams.  This resource is intended to help students learn physics material categorically and to help teachers better prepare students for national exam-level questions. I compiled this document using test images and PDFs made freely available at Maktaba, a great resource for learning.  I apologize ahead of time for small erros and miscategorized questions.  Additionally, questions that required graphs were left out of this catalogue.  

\end{abstract}

\section{Measurement}

\subsection{Physical Quantities}
\begin{itemize}
\item (1999)  Mention two applications and two limitations of dimensional analysis.
\item (1999)  The frequency $ f$ of a note produced by a taut wire stretched between two supports depends on the distance ​ $ l$ ​ between the supports, the mass per unit length of the wire,$ \mu $ , and the tension $ T$ . Use dimensional analysis to find how $ f$ is related to ​ $ l$ ​ , $\mu$, and $ T$ .
\item (1999)  The period $ T$ of vibrations of a tuning fork may be expected to depend on the density $ D$ , Young's Modulus $ Y$ of the material of which it is made and the length a of its prongs. Using dimensional analysis deduce an expression for $ T$ in terms of $ D$ , $ Y$ and a.
\item (2000)  The speed v of a wave is found to depend on the tension $ T$ in the string and the mass per unit length $ u$ (linear mass density). Using dimensional analysis derive the relationship between v, $ T$ and $ u$ .
\item (2007)  Mention two$ (2)$ uses of dimensional analysis.
\item (2007)  The frequency $ f$ of a note given by an organ pipe depends on the length, $ l$ , the air pressure $ P$ and the air density $ D$ .  Use the method of dimensions to find a formula for the frequency.\begin{itemize}
\item What will be the new frequency of a pipe whos original frequency was $ 256$ Hz if the air density falls by $ 2\%$ and the pressure increases by $ 1\%$ ?
\end{itemize}
\item (2010)  Mention two uses of dimensional analysis.
\item (2010)  The critical velocity of a liquid flowing in a certain pipe is $ 3$ m$/$s, assuming that the critical velocity v depends on the density $ \rho $ of the liquid. its viscosity mu, and the diameter $ d$ . of the pipe. \begin{itemize}
\item Use the method of dimensional analysis to derive the equation of the critical velocity of the liquid in a pipe of half the diameter.
\item Calculate the value of critical velocity.
\end{itemize}
\item (2014)  What is the importance of dimensional analysis inspite of its drawbacks?
\item (2015)  State the law of dimensional analysis.
\item (2015)  The largest mass, $ m$ of a stone that can be moved by the flowing river depends on the velocity of flow v, the density $ \rho $ of water, and the acceleration due to gravity $ g$ . Show that the mass, $ m$ varies to the sixth power of the velocity of flow.
\item (2016)  Define the term dimension of a physical quantity.
\item (2016)  Give two basic rules of dimensional analysis. 
\item (2016)  The frequency, $ f$ of a vibrating string depends upon the force applied, $ F$ , the length, $ l$ , of the string and the mass per unit length,$ \mu $ . Using dimension show how $ f$ is related to $ F$ , $ l$ and $ \mu $ .
\item (2016)  What is meant by least count of a measurement?
\item (2017)  Define the term dimensions of a physical quantity. 
\item (2017)  Identify two uses of dimensional equations.
\item (2017)  What is the basic requirement for a physical relation to be correct? 
\item (2017)  List two quantities whose dimension is [ $ ML^{2}T^{-1}$ ]
\item (2017)  The frequency ‘$ f$ ’ of vibration of a stretched string depends on the tension ‘$ F$ ’, the length ‘$ l$ ’ and the mass per unit length $ $ $\mu$of the string. Derive the formula relating the physical quantities by the method of dimensions. 
\item (2017)  Use dimensional analysis to prove the correctness of the relation, $ \rho = \frac{3g}{4 \pi RG}$ where by $ \rho =$ density of the earth, $ g=$ acceleration due to gravity, $ R=$ radius of the earth and $ G=$ gravitational constant.
\item (2018)  State the law of dimensional analysis. 
\item (2018)  If the speed v of the transverse wave along a wire of tension, $ T$ and mass, $ m$ is given by, $ V=\sqrt{T/m}$ .  Apply dimensional analysis to check whether the given expression is correct or not.  
\item (2019)  Identify two basic rules of dimensional analysis.
\item (2019)  The frequency $ n$ of vibration of a stretched string is a function of its tension $ F$ , the length, $ l$ and mass per unit length $ m$ . Use the method of dimensions to derive the formula relating the stated physical quantities.
\end{itemize}

\subsection{Errors}
\begin{itemize}
\item (2000)  What is an error? Mention two causes of systematic and two causes of random errors.
\item (2000)  The pressure $ P$ is calculated from the relation $ P=F/( \pi R^{2})$ where $ F$ is the force and $ R$ the radius. If the percentage possible errors are $ +2\%$ for $ F$ and $ +1\%$ for $ R$ . Calculate the possible percentage error for $ P$ .
\item (2007)  What is systematic error?
\item (2007)  The smallest divisions for the voltmeter and ammeter are $ 0.1$ V and $ 0.01$ A respectively.  If $ V=IR$ , find the relative error in the resistance $ R$ , when $ V=2$ V and $ I=0.1$ A.
\item (2010)  Define an error.
\item (2010)  In an experiment to determine the acceleration due to gravity $ g$ , a small ball bearing is timed while falling freely from rest through a measured vertical height. The following data were obtained: vertical height $ h=(600\pm 1)$ mm, time taken $ t=(350\pm 1)$ ms. Calculate the numerical value of $ g$ from the experimental data, clearly specify the errors. 
\item (2013)  What is the difference between degree of accuracy and precision.
\item (2013)  In an experiment to determine Young's modulus of a wooden material the following measurements were recorded:\begin{itemize}
\item length $ l=80.0\pm 0.05$ cm 
\item breadth $ b=28.65\pm 0.03$ mm
\item thickness $ t=6.40\pm 0.03$ mm and
\item slope $ G=0.035\pm 0.001$ cm/gm
\item Given that the Young’s modulus $ Y$ is given by:
\item $ Y=(4/Gb)(l/t)^{3}$
\item Calculate the maximum percentage error in the value of $ Y$ .
\end{itemize}
\item (2014)  Distinguish random error from systematic error.\begin{itemize}
\item Give a practical example of random error and systematic error and briefly explain how they can be reduced or eliminated.
\end{itemize}
\item (2014)  Define the terms error and mistake.
\item (2014)  An experiment was done to find the acceleration due to gravity by using the formula: $ T=2\pi\sqrt{l/g}$ , where all symbols carry their usual meaning.  If the clock losses $ 3$ seconds in $ 5$ minutes, determine the error in measuring ‘$ g$ ’ given that, $ T=2.22$ sec, $ l=121.6$ cm, $ \Delta T_{1}=0.1$ sec, and  $ \Delta l=\pm 0.05$ .
\item (2014)  The following measurements were taken by a student fort he length of a piece of rod: $ 21.02$ , $ 20.99$ , $ 20.92$ , $ 21.11$ and $ 20.69$ . Basing on error analysis find the true value at the length of a piece of rod and its associated error.
\item (2015)  What is meant by random errors?\begin{itemize}
\item Briefly explain two causes of random errors in measurements. 
\end{itemize}
\item (2015)  The period $ T$ of oscillation of a body is said to be $ 1.5\pm 0.002$ s while its amplitude A is $ 0.3\pm 0.005$ m and the radius of gyration $ k$ is $ 0.28+0.004$ m. If the acceleration due\begin{itemize}
\item to gravity $ g$ was found to be related to $ T$ , $ A$ and $ k$ by the equation $ (gA)/(4\pi^{2})=( A^{2}+k^{2})/T^{2}$ , find the:
\item Numerical value of $ g$ in four decimal places
\item Percentage error in $ g$ .
\end{itemize}
\item (2016)  The period of oscillation of a simple pendulum is given by $ T=2\pi\sqrt{l/g}$ where by $ 100$ vibrations were taken to measure $ 200$ seconds. If the least count for the time and length of a pendulum of $ 1$ m are $ 0.1$ sec and $ 1$ mm respectively, calculate the maximum percentage error in the measurement of $ g$ .
\item (2017)  Give the meaning of the following terms as used in error analysis:\begin{itemize}
\item Absolute error. 
\item Relative error. 
\end{itemize}
\item (2017)  The force ‘$ F$ ’ acting on an object of mass ‘$ m$ ’, travelling at velocity ‘v’ in a circle of radius ‘$ r$ ’ is given by: $ F= \frac{mv^{2}}{r}$ If the measurements are recorded as: $ m=(3.5 \pm 0.1)$ kg, $ V=(20\pm 1)$ m$/$s, $ r=(12.5\pm 0.5)m$ ; find the maximum possible\begin{itemize}
\item Fractional error. 
\item Percentage error in the measurement of force.
\item Show how you will record the reading of force, ‘$ F$ ’ in the question above. 
\end{itemize}
\item (2018)  How can random and Systematic errors be minimized during an experiment?
\item (2018)  Estimate the precision to which the Young’s modulus, $ \gamma $ of the wire can be determined from the formula $ \gamma =(4Fl)/(\pi d^{2} e)$ , given that the applied tension, $ F=500$ N, the length of the loaded wire,  $ l=3$ m, the diameter of the wire, $ d=1$ mm, the extension of the wire, $ e=5$ mm and the errors associated with these quantities are $ 0.5$ N, $ 2$ mm, $ 0.01$ mm and $ 0.1$ mm respectively. 
\item (2019)  What causes systematic errors in an experiment? Give four points. 
\item (2019)  Estimate the numerical value of drag force $ D= 1/2 C \rho  A V^{2}$ with its associated error given that the measurements of the quantities $ C$ , $ A$ , $ \rho $ and $ v$ were recorded as $ (10\pm 0.00)$ unit less $ (5\pm 0.2) $ cm$ ^{2}$ , $ (15\pm 0.15)$ g$/$cm$ ^{3}$ and $ (3\pm 0.5)$ cm$/$sec$ ^{2}$ respectively. 
\end{itemize}


\section{Mechanics}

\subsection{Newton’s Laws of Motion}
\begin{itemize}
\item (1998)  State Newton's laws of motion.
\item (1998)  A ball of mass $ 0.4$ kg is dropped vertically from a height of $ 2.5$ m on to a horizontal table and bounces to a height of $ 1.5$ m.\begin{itemize}
\item Find the kinetic energy of the ball just before striking the table.
\item Find the kinetic energy just after impact.
\item Suggest reasons for the difference between these two values of kinetic energy.
\item What height would you expect the ball to reach after its next bounce from the table?
\end{itemize}
\item (1998)  A jet of water flowing with a velocity of $ 20$ ms$ ^{-1}$ from a pipe of cross-sectional area, $ 5.0 \times 10^{-3}$ m$ ^{2}$ , strikes a wall at right angles and loses all its velocity.\begin{itemize}
\item What is the mass of water striking the wall per second?
\item What is the change in momentum per second of the water hitting the wall?
\item What is the force exerted on the wall?
\end{itemize}
\item (1999)  Define momentum
\item (1999)  Define impulse of a force
\item (1999)  A jet of water emerges from a hose pipe of a cross-sectional area $ 5.0\times 10^{-3}$ m​$ ^{2}$ with a velocity of $ 3.0$ m$/$s and strikes a wall at right angle. Assuming the water to be brought to rest by the wall and does not rebound, calculate the force on the wall.
\item (1999)  Distinguish between static and dynamic friction.
\item (2007)  A ball is thrown towards a vertical wall from a point $ 2$ m above the ground and $ 3$ m from the wall.  The initial velocity of the ball is $ 20$ m$/$s at an angle of $ 30$ deg above the horizontal.  If the collision of the ball with the wall is perfectly elastic, how far behind the thrower does the ball hit the ground?
\item (2007)  Explain why when catching a fast moving ball, the hands are drawn back will the ball is being brought to rest.
\item (2007)  Rockets are propelled by the ejection of the products of the combustion of fuel.  Consider a rocket of total mass $ M$ travelling at a speed v in a region of space where the gravitational forces are negligible.  
\item (2007)  Supposing the combustion products are ejected at a constant speed v, relative to the rocket, show that a fuel "burn" which reduces the total mass $ M$ of the rocket to $ m$ results in an increase in the speed of the rocket to v such that $ v-V=V_{f} \ln (M/m)$ .
\item (2007)  Supposing that $ 2.1\times10^{6}$ kg of fuel are consumed during a "burn" lasting $ 1.5\times10^{2}$ seconds and given that there is a constant force on the rocket of $ 3.4\times 10^{7}$ N during this burn, calculate v, and increase in speed resulting from the burn if $ M=2.8\times10^{6}$ kg.  
\item (2007)  What is the initial vertical acceleration that can be imparted to this rocket when it is launched from the Earth if the initital mass is $ 2.8\times 10^{6}$ kg?
\item (2007)  State and define Newton’s 2nd law of motion with respect to angular motion. 
\item (2013)  A man stands in a lift which is being accelerated upwards at $ 3.2$ m$/$s$ ^{2}$ . If the man has a mass of $ 65$ kg, what is the net force exerted on the man by the floor of the lift?
\item (2013)  A rubber cord of a $ Y-$ shaped object has a cross sectional area of $ 4 \times 10^{-6}$ m$ ^{2}$ ? And relaxation length of $ 100$ mm. If the arms of the catapult are $ 70$ mm apart, calculate the: \begin{itemize}
\item tension in the rubber. 
\item force required to stretch it when the rubber cord is pulled back until its length doubles. 
\end{itemize}
\item (2014)  State the principle of conservation of linear momentum. \begin{itemize}
\item Give two examples of the principle of conservation of linear momentum. 
\end{itemize}
\item (2014)  An insect is released from rest at the top of the smooth bowling ball such that it slides over the ball. Prove that it will loose its footing with the ball at an angle of about $ 48^{\circ}$ with the vertical.
\item (2014)  A vertical spring fixed at one end has a mass of $ 0.2$ kg and is attached at the other end.\begin{itemize}
\item Determine the:
\item Extension of the spring.
\item Energy stored in the spring.
\end{itemize}
\item (2014)  Define torque and give its S.I. unit.
\item (2014)  Give two ways in which the internal energy of the system can be changed.
\item (2016)  State the principles on which the rocket propulsion is based. 
\item (2016)  A jet engine on a test bed takes in $ 40$ kg of air per second at a velocity of $ 100$ m$/$s  and burns $ 0.80$ kg of fuel per second. After compression and heating the exhaust gases are ejected at $ 600$ m$/$s relative to the air craft. Calculate the thrust of the engine.
\item (2016)  An object of mass $ 2$ kg is attached to the hook of a spring balance which is suspended vertically to the roof of a lift.  What is the reading on the spring balance when the lift is:\begin{itemize}
\item going up with the rate of $ 0.2$ m$/$s$ ^{2}$
\item going down with an acceleration of $ 0.1$ m$/$s$ ^{2}$
\item ascending with uniform velocity of $ 0.15$ m$/$s
\end{itemize}
\item (2016)  Define the term inertia.
\item (2017)  A $ 75$ kg hunter fires a bullet of mass $ 10$ g with a velocity of $ 400$ m$/$s from a gun of mass $ 5$ kg. Calculate the:\begin{itemize}
\item Recoil velocity of the gun. 
\item Velocity acquired by the hunter during firing.
\end{itemize}
\item (2017)  A traffic light is suspended with two steel wires of equal lengths and radii of $ 0.5$ cm. If the wires make an angle of $ 15^{\circ}$ with the horizontal, what is the fractional increase in their length due to the weight of the light? 
\item (2018)  Under what condition a passenger in a lift feels weightless? 
\item (2018)  Calculate the tension in the supporting cable of an elevator of mass $ 500$ kg which was originally moving downwards at $ 4$ m$/$s and brought to rest with constant acceleration at a distance of $ 20$ m. 
\item (2018)  The rotating blades of a hovering helicopter swept out an area of radius $ 2$ m imparting a downward velocity of $ 8$ m$/$s of the air displaced. Find the mass of a helicopter. 
\item (2019)  A rocket of mass $ 20$ kg has $ 180$ kg of fuel. If the exhaust velocity of the fuel is $ 1.6$ km/sec, calculate;\begin{itemize}
\item The minimum rate of fuel consumption that enable the rocket to rise from the ground. 
\item The ultimate vertical speed gained by the rocket when the rate of fuel consumption ts $ 2$ kg/sec. 
\end{itemize}
\item (2019)  Determine the least number of pieces required to stop the bullet if a rifle bullet loses $ 1/20$ of its velocity when passing through them.
\item (2019)  A man of $ 100$ kg jumps into a swimming pool from a height of $ 5$ m. If it takes $ 0.4$ seconds for the water in a pool to reduce its velocity to zero, what average force did  the water exert on the man? 
\end{itemize}

\subsection{Projectile Motion}
\begin{itemize}
\item (2000)  Mention two motions that add up to make projectile motion.
\item (2000)  In long jumps does it matter how high you jump? State the factors which determine the span of the jump. 
\item (2000)  Derive an expression that relates the span of the jump and the factors you have mentioned.
\item (2000)  A bullet is fired from a gun on the top of a cliff $ 140$ m high with a velocity of $ 150$ m$/$s at an elevation of $ 30^{\circ}$ to the horizontal. Find the horizontal distance from the foot of a cliff to the point where the bullet lands on the ground.
\item (2007)  What is meant by the term "projectile" as applied to projectile motion?
\item (2007)  Give two $ (2)$ practical applications of projectile motion at your locality.
\item (2007)  The ceiling of a long hall is $ 25$ m high.  Determine the maximum horizontal distance that a ball thrown with a speed of $ 40$ m$/$s can go without hitting the ceiling of the wall.
\item (2010)  Mention two examples of projectile motion. 
\item (2010)  Define the trajectory. 
\item (2010)  Mention two uses of projectile motion.
\item (2010)  Find the velocity and angle of projection of a particle which passes in a horizontal direction Just over the top of a wall which is $ 12$ m high and $ 32$ m away. 
\item (2013)  List down two main assumptions in deriving the equation of projectile motion.
\item (2013)  Why the horizontal motion of a projectile constant? 
\item (2013)  A ball is thrown horizontally with a speed of $ 14.0$ m$/$s from a point $ 6.4$ m above the ground, calculate:\begin{itemize}
\item The horizontal distance traveled in that time.
\item Its velocity when it reaches the ground.
\end{itemize}
\item (2014)  Outline the motions that add up to make projectile motion. 
\item (2014)  In the first second of its flight, a rocket ejects $ 1/60$ of  its mass with a relative velocity of $ 2400$ m$/$s.\begin{itemize}
\item Find its acceleration.
\item What is the final velocity if the ratio of initial to final mass of the rocket is $ 4$ at a time of $ 60$ seconds? 
\end{itemize}
\item (2014)  A ball is thrown upwards with an initial velocity of $ 33$ m$/$s from a point $ 65^{\circ}$ on the side of a hill which slopes upward uniformly at an angle of $ 28^{\circ}$ .\begin{itemize}
\item At what distance up the slope does the ball strike?
\item Calculate the time of flight of the ball. 
\end{itemize}
\item (2014)  A cannon of mass $ 1300$ kg fires a $ 72$ kg ball in a horizontal direction with a nuzzle speed of $ 55$ m$/$s, If the cannon is mounted so that it can recoil freely calculate the:\begin{itemize}
\item  recoil velocity of the cannon relative to the earth. 
\item horizontal velocity of the ball relative to the earth. 
\end{itemize}
\item (2015)  Define the term trajectory.
\item (2015)  Briefly explain why the horizontal component of the initial] velocity of a projectile always remains constant.
\item (2015)  List down two limitations of projectile motion. 
\item (2015)  A body projected from the ground at the angle of $ 60^{\circ}$ is required to pass just above the two vertical walls each of height $ 7$ m. If the velocity of projection is $ 100$ m$/$s, calculate the distance between the two walls. 
\item (2015)  A fireman standing at a horizontal distance of $ 34$ m from the edge of the burning story building aimed to raise streams of water at an angle of $ 60^{\circ}$ into the first floor through an open window which is at $ 20$ m high from the ground level. If water strikes on this floor $ 2$ m away from the outer edge, \begin{itemize}
\item  Sketch a diagram of the trajectory.
\item What speed will the water leave the nozzle of the fire hose?
\end{itemize}
\item (2016)  Mention two characteristics of projectile motion.
\item (2016)  If the range of the projectile is $ 120$ m and its time of flight is $ 4$ sec , determine the angle of projection and its initial velocity of projection assuming that the acceleration due to gravity $ g=10$ m$/$s. 
\item (2017)  A jumbo jet traveling horizontally at $ 50$ m$/$s at a height of $ 500$ m from sea level drops a luggage of food to a disaster area.\begin{itemize}
\item At what horizontal distance from the target should the luggage be dropped?
\item Find the velocity of the luggage as it hit the ground. 
\end{itemize}
\item (2018)  How does projectile motion differ from uniform circular motion? 
\item (2018)  A rifle shoots a bullet with a muzzle velocity of $ 1000$ m$/$s at a small target $ 200$ m away. How high above the target must the rifle be aimed so that the bullet will hit the target? \begin{itemize}
\item Where does the object strike the ground when thrown horizontally with a velocity of $ 15$ m$/$s from the top of a $ 40$ m high building? 
\item Find the speed of travel when a man jumps a maximum horizontal distance of $ 1$ m spending a minimum time on the ground.
\end{itemize}
\item (2019)  Justify the statement that projectile motion is two dimensional motion.
\item (2019)  A rocket was launched with a velocity of $ 50$ m$/$s from the surface of the moon at an angle of $ 40^{\circ}$ to the horizontal, Calculate the horizontal distance covered  after half time of flight.
\item (2019)  Show that the angle of projection $ \theta ^{\circ}$ for a projectile launched from the origin is given by $ \theta ^{\circ}= tan^{-1}(4h_{m}/R)$ , where $ R$ stand for horizontal range and $ h_{m}$ is the maximum vertical height.
\item (2019)  Determine the angle of projection for which the horizontal range of a projectile is $ 4\sqrt{3}$ times its maximum height. 
\end{itemize}

\subsection{Uniform Circular Motion}
\begin{itemize}
\item (2000)  Show that the period of a body of mass $ m$ revolving in a horizontal circle with constant velocity v at the end of a string of length $ l$ is independent of the mass of the object.
\item (2000)  A ball of mass $ 100$ g is attached to the end of a string and is swung in a circle of radius $ 100$ cm at a constant velocity of $ 200$ cm$/$s. While in motion the string is shortened to $ 50$ cm. Calculate:\begin{itemize}
\item The new velocity of the motion.
\item The new period of the motion.
\end{itemize}
\item (2000)  A car travels over a humpback bridge of radius of curvature $ 45$ m. Calculate the maximum speed of the car if the wheels are to remain in contact with the bridge.
\item (2007)  What is meant by centripetal force?
\item (2007)  Derive the expression $ a =(v^{2}/r)$ where a, v, and $ r$ stands for the centripetal acceleration, linear velocity and radius of a circular path respectively.  
\item (2007)  A ball of mass $ 0.5$ kg attached to a light inextensible string rotates in a vertical circle of radius $ 0.75$ m such that it has a speed of $ 5$ m$/$s when the string is horizontal.  Calculate:\begin{itemize}
\item  The speed of the ball and the tension in the string at the lowest point of its circular path.
\end{itemize}
\item (2010)  What is the origin of centripetal force for:\begin{itemize}
\item A satellite orbiting around the Earth. 
\item An electron in the hydrogen atom?
\end{itemize}
\item (2010)  A small mass of $ 0.15$ kg is suspended from a fixed point by a thread of a fixed length. The mass is given a push so that it moves along a circular path of radius $ 1.82$ m in a horizontal plane at a Steady speed, taking $ 18.0$ s to make $ 10$ complete revolutions. Calculate:\begin{itemize}
\item The speed of the small mass.
\item The centripetal acceleration. 
\item The tension in the thread. 
\end{itemize}
\item (2013)  Why is it technically advised to bank a road at corners?
\item (2013)  A wheel rotates at a constant rate of $ 10$ revolutions per second. Calculate the centripetal acceleration at a distance of $ 0.80$ m from the centre of the wheel.
\item (2014)  Define the term ‘radial acceleration’. 
\item (2015)  Mention three effects of looping the loop.\begin{itemize}
\item Why there must be a force acting on a particle moving with uniform speed in a circular path? Write down an expression for its magnitude. 
\end{itemize}
\item (2015)  A driver negotiating a sharp bend usually tend to reduce the speed of the car.\begin{itemize}
\item  What provides the centripetal force on the car?
\item Why is it necessary to reduce its speed?
\end{itemize}
\item (2015)  A ball of mass $ 0.5$ kg is attached to the end of a cord whose length is $ 1.5$ m then whirled in horizontal circle. If the cord can withstand a maximum tension of $ 50$ N calculate the:\begin{itemize}
\item Maximum speed the ball can have before the cord breaks. 
\item Tension in the cord if the ball speed is $ 5$ m$/$s
\end{itemize}
\item (2015)  Define the term tangential velocity.
\item (2016)  A boy ties a string around a stone of mass $ 0.15$ kg and then whirls it in a horizontal circle at constant speed. If the period of rotation of the stone is $ 0.4$ sec and the length between the stone and boy’s hand is $ 0.50$ m ;\begin{itemize}
\item Calculate the tension in the string. 
\item State one assumption taken to reach the answer above.
\end{itemize}
\item (2017)  A car is moving with a speed of $ 30$ m$/$s on a circular track of radius $ 500$ m. If its speed is increasing at the rate of $ 2$ m$/$s, find its resultant linear acceleration.
\item (2017)  An object of mass $ 1$ kg is attached to the lower end of a string $ 1$ m long whose upper end is fixed and made to rotate in a horizontal circle of radius $ 0.6$ m. If the circular speed of the mass is constant, find the:\begin{itemize}
\item Tension in the string. 
\item Period of motion. 
\end{itemize}
\item (2019)  In which aspect does circular motion differ from linear motion? 
\item (2019)  Why there must be a force acting on a particle moving with uniform speed in a circular path? 
\item (2019)  A stone tied to the end of string $ 80$ cm long, is whirled in a horizontal circle with a constant speed making $ 25$ revolutions in $ 14$ seconds. Determine the magnitude of its acceleration. 
\end{itemize}

\subsection{Simple Harmonic Motion}
\begin{itemize}
\item (1998)  Define simple harmonic motion.
\item (1998)  Prove that, the velocity v of a particle moving in simple harmonic motion is given by: $ v=w(A^{2}-y^{2})^{0.5}$ , where A is the amplitude of oscillation, $ w$ the angular frequency and $ y$ the displacement from the mean position.
\item (1998)  A simple pendulum has a period of $ 2.8$ seconds. When its length is shortened by $ 1.0$ metre, the period becomes $ 2.0$ seconds. From this information, determine the acceleration $ g$ , of gravity and the original length of the pendulum.
\item (1998)  A particle rests on a horizontal platform which is moving vertically in simple harmonic motion with an amplitude of $ 50$ mm. Above a certain frequency the particle ceases to remain in contact with the platform throughout the motion. With a help of a diagram and illustrative equations, find;\begin{itemize}
\item the lowest frequency at which this situation occurs.
\item the position at which contact ceases.
\end{itemize}
\item (1999)  Give two similarities between simple harmonic motion and circular motion.
\item (1999)  On the same set of axes, sketch how energy exchange (kinetic to potential) takes place in an oscillator placed in a damping medium.
\item (2000)  Define simple harmonic motion.
\item (2000)  Two simple pendulums of length $ 0.4$ m and $ 0.6$ m respectively are set oscillating in step. \begin{itemize}
\item After what further time will the two pendulums be in step again? 
\item Find the number of oscillations made by each pendulum during the time found above.
\end{itemize}
\item (2000)  Cite two examples of SHM which are of importance to everyday life experience.
\item (2000)  Explain, giving reasons, whether either transverse or longitudinal waves could exist, if the vibratory motion causing them were not simple harmonic motion.
\item (2014)  State where the magnitude of acceleration is greatest in simple harmonic motion.
\item (2014)  Sketch a graph of acceleration against displacement for a simple harmonic motion.
\item (2014)  The displacement of a particle from the equilibrium position moving with simple harmonic motion is given by $ x=0.05 \sin(6t)$ , where $ t$ is the time in seconds measured at an instant when $ x=0$ .  Calculate the:\begin{itemize}
\item Amplitude of oscillations.
\item Period of oscillations. 
\item  Maximum acceleration of the particle. 
\end{itemize}
\item (2015)  Briefly explain why the motion of a simple pendulum is not strictly simple harmonic? \begin{itemize}
\item Why is the velocity and acceleration of a body executing simple harmonic motion (S.H.M.) out of phase? 
\end{itemize}
\item (2015)  A body of mass $ 0.30$ kg executes simple harmonic motion with a period of $ 2.5$ s and amplitude of $ 4.0\times10^{-2}$ m. Determine the:\begin{itemize}
\item Maximum velocity of the body. 
\item Maximum acceleration of the body. 
\item Energy associated with the motion.
\end{itemize}
\item (2015)  A particle of mass $ 0.25$ kg vibrates with a period of $ 2.0$ s. If its greatest displacement is $ 0.4$ m what is its maximum kinetic energy?
\item (2016)  Show that the total energy of a body executing S.H.M. is independent of time.
\item (2016)  A mass of $ 05$ kg connected to a light spring of force constant $ 20$ N$/$m oscillates on a  horizontal frictionless surface. If the amplitude of the motion $ 1$ s $ 3.0$ cm , calculate the;\begin{itemize}
\item Maximum speed of the mass.
\item  Kinetic energy of the system when the displacement is $ 2.0$ cm.
\end{itemize}
\item (2017)  The equation of simple harmonic motion is given as $ x=6 \sin(10\pi t)+8 \sin(10\pi t)$ , where $ x$ is in centimeters and $ t$ in seconds. Determine the:\begin{itemize}
\item Amplitude 
\item Initial phase of motion. 
\end{itemize}
\item (2017)  Show that the total energy of a body executing simple harmonic motion is independent of time. 
\item (2017)  Find the periodic time of a cubical body of side $ 0.2$ m and mass $ 0.004$ kg floating in water then pressed and released such that it oscillates vertically. 
\item (2018)  What is meant by the following terms as used in simple harmonic motion (S.H.M)?\begin{itemize}
\item Periodic motion. 
\item Oscillatory motion. 
\end{itemize}
\item (2018)  List four important properties of a particle executing simple harmonic motion (S.H.M). 
\item (2018)  Sketch a labeled graph that represents the total energy of a particle executing simple harmonic motion (S.H.M). 
\item (2018)  The periodic time of a body executing S.H.M is $ 4$ seconds. How much time interval from time, $ t=0$ will its displacement be half its amplitude? 
\item (2018)  Giving reasons, explain whether either transverse or longitudinal waves could exist, if the vibratory motion causing them were not simple harmonic motion. 
\item (2019)  Provide two typical examples of simple harmonic motion (S.H.M). 
\item (2019)  Why the velocity and acceleration of a body executing simple harmonic motion are out of phase? 
\item (2019)  The period of a particle executing simple harmonic motion (S.H.M) is $ 3$ seconds. If its amplitude is $ 25$ cm, calculate the time taken by the particle to move a distance of $ 12.5$ cm on either side from the mean position.
\item (2019)  A person weighing $ 50$ kg stands on a platform which oscillates with a frequency of $ 2$ Hz and of amplitude $ 0.05$ m. Find his/her minimum weight as recorded by a machine of the platform. 
\end{itemize}

\subsection{Gravitation}
\begin{itemize}
\item (1999)  What do you understand by the term escape velocity?
\item (1999)  Calculate the escape velocity from the moon’s surface given that a man on the moon has $ 1/6$ his weight on earth. The mean radius of the moon is $ 1.75 \times 10^6$ m.
\item (1999)  Explain the meaning of the following terms:\begin{itemize}
\item Gravitational Potential of the Earth.
\item Gravitational Field Strength of the Earth.
\item How are the above quantities in and related?
\end{itemize}
\item (1999)  Show that the total energy of a satellite in a circular orbit equals half its potential energy.
\item (1999)  Calculate the height above the Earth's surface for a satellite in a parking orbit.
\item (1999)  What would be the length of a day if the rate of rotation of the Earth were such that the acceleration of gravity $ g=0$ at the equator?
\item (2007)  Evaluate the work done by the Earth's gravitational force and by the tension in the string as the ball moves from its highest to its lowest point.
\item (2007)  Two small spheres each of mass $ 10g$ are attached to a light rod $ 50$ cm long. The system Is set into oscillation and the period of torsional oscillation is found to be $ 770$ seconds. To produce maximum torsion to the system two large spheres each of mass $ 10$ kg are placed near each suspended sphere, if the angular deflection of the suspended rod Is $ 3.96 \times 10^{-3}$ rad. and the distance between the centres of the large spheres and small spheres is $ 10$ cm, determine the value of the universal constant of gravitation, $ G$ , from the given information. 
\item (2007)  On the basis of Newton’s universal law of gravitation, derive Kepler’s third law of planetary motion. 
\item (2007)  A planet has half the density of earth but twice its radius. What will be the speed of a satellite moving fast past the surface of the planet which has on no atmosphere?\begin{itemize}
\item ( Radius of earth $ R_{E}=6.4 \times 10^{3}$ km and gravitational potential energy $ g_{E}=9.81$ N$/$kg )
\end{itemize}
\item (2009)  State Kepler's laws of planetary motion.
\item (2009)  Explain the variation of acceleration due to gravity, $ g$ . inside and outside the earth.
\item (2009)  Derive the formula for mass and density of the earth.
\item (2009)  What do you understand by the term satellite?
\item (2009)  A satellite of mass $ 100$ kg moves in a circular orbit of radius $ 7000$ km around the earth, assumed to be a sphere of radius $ 6400$ km.  Calculate the total energy needed to place the satellite in orbit from the earth assuming $ g=10$ N$/$kg at the earth’s surface.
\item (2013)  With the aid of a labeled diagram, sketch the possible orbits for a satellite launched from the earth.\begin{itemize}
\item From the diagram above, write down an expression for the velocity of a satellite corresponding to each orbit.
\end{itemize}
\item (2014)  Define the universal gravitational constant.
\item (2014)  How is the gravitational potential related to gravitational field strength?
\item (2014)  Write down an expression for the acceleration due to gravity (g) of a body of mass (m) which is at a  distance (r) from the centre of the earth. \begin{itemize}
\item If the Earth were made of lead of relative density of $ 11.3$ kg$/$m$ ^{3}$ , what would he the value of acceleration due to gravity on the surface of the earth?
\end{itemize}
\item (2014)  Why the value of acceleration due to gravity (g) changes due to the change in latitude? Give two reasons.
\item (2014)  A rocket is fired from the earth towards the sun. At what point on its path is the gravitational force on the rocket zero?
\item (2015)  Explain why the astronaut appears to be weightless when traveling in the space vehicle.
\item (2015)  State Newton's law of gravitation. \begin{itemize}
\item Use Newton’s law of gravitation to derive Kepler’s third law.
\end{itemize}
\item (2015)  Briefly explain why Newton’s equation of universal gravitation does not hold for bodies falling near the surface of the earth? 
\item (2015)  Show that the total energy of a satellite in a circular orbit equals half its potential energy.
\item (2015)  Calculate the height above the Earth’s surface for a satellite in a parking orbit.
\item (2015)  A $ 10$ kg satellite circles the Earth once every $ 2$ hours in an orbit having a radius of $ 8000$ km.  Assuming Bohr’s angular momentum postulate applies to the satellite just as it does to an electron in the hydrogen atom,  find the quantum number of the orbit of the satellite.
\item (2016)  Mention one application of parking orbit.
\item (2016)  Briefly explain how parking orbit of a satellite is achieved.
\item (2016)  The earth satellite revolves in a circular orbit at a height of $ 300$ km above the earth’s surface.  Find the; \begin{itemize}
\item Velocity of the satellite
\item Period of the satellite.
\end{itemize}
\item (2016)  A spaceship is launched into a circular orbit close to the earth’s surface.  What additional velocity has to be imparted on the spaceship if order to overcome the gravitational pull?
\item (2017)  Why does the kinetic energy of an earth satellite change in the elliptical orbit?
\item (2017)  A space craft is launched from the earth to the moon, If the mass of the earth is $ 81$ times that of the moon and the distance from the centre of the earth to that of the moon is about $ 4.0 \times 10^{5}$ km;\begin{itemize}
\item Draw a sketch showing how the gravitational force on the spacecraft varies during its journey. 
\item Calculate the distance from the centre of the earth where the resultant gravitational force becomes zero. 
\end{itemize}
\item (2018)  A satellite of mass $ 600$ kg is in a circular orbit at a height $ 2 \times 10^{6}$ km above the earth’s surface. Determine the:\begin{itemize}
\item Orbital speed. 
\item Gravitational potential energy. 
\end{itemize}
\item (2018)  What would happen if gravity suddenly disappears?  
\item (2018)  Two base of a mountain are at sea level where the gravitational field strength is $ 9.81$ N$/$kg . If the value of gravitational field at the top of the mountain is $ 9.7$ N$/$kg, calculate the height of the mountain above the sea level. 
\item (2019)  Why the weight of a body becomes zero at the centre of the earth? 
\item (2019)  How far above the earth surface does the value of acceleration due to gravity becomes $ 36\%$ of its value on the surface? 
\item (2019)  Compute the period of revolution of a satellite revolving in a circular orbit at a height of $ 3400$ km above the Earth’s surface. 
\item (2019)  Prove that the angular momentum fora satellite of mass $ M_{s}$ revolving round the\begin{itemize}
\item earth of mass $ M_{e}$ in an orbit of radius $ r$ is equal to $ (G M_{e}$  $ M_{s}^{2}r)^{1/2}$ .
\end{itemize}
\end{itemize}

\subsection{Rotation of Rigid Bodies}
\begin{itemize}
\item (1999)  State the parallel axis theorem.
\item (1999)  Show that the Kinetic energy (K.E.) of rotation of a rigid body about an axis with a constant angular velocity $ w$ is given by $ KE =1/2Iw^{2}$ where i is the moment of inertia of the rigid body about the given axis.
\item (1999)  What do you understand by the term "moments of inertia" of a rigid body?
\item (1999)  State the perpendicular axes theorem of moments of inertia for a body in the form of a lamina
\item (1999)  Calculate the moments of inertia of a thin circular disc of radius $ 50$ cm and mass $ 2$ kg about an axis along a diameter of the disc.
\item (1999)  A wheel mounted on an axle that is not frictionless is initially at rest. A constant external torque of $ 50$ Nm is applied to the wheel for $ 20$ s. At the end of the $ 20$ s, the wheel has an angular velocity of\begin{itemize}
\item $ 600$ rev/min. The external torque is the removed, and the wheel comes to rest after $ 120$ s more.
\item Determine the moments of inertia of the wheel.
\item Calculate the frictional torque which is assumed to be constant. 
\end{itemize}
\item (2007)  The $ T$ is then suspended from the free end of rod $ Y$ and the pendulum swings in the plane of $ T$ about the axis Of rotation.\begin{itemize}
\item Calculate the moment of inertia i of the $ T$ about the axis of rotation. 
\item Obtain the expression for the k.e. and p.e. in terms of the angle $ \theta $ of inclination to the vertical oscillation of the pendulum. 
\item Show that the period of oscillation is $ 2\pi\sqrt{17L/18g}$ . 
\item ( Moment of inertia of a thin rod about its centre $ I_{C}=mL^{2}/12$ . )
\end{itemize}
\item (2009)  Define angular momentum and give its dimensions.
\item (2009)  A grinding wheel in a form of solid cylinder of $ 0.2$ m diameter and $ 3$ kg mass is rotated at $ 3600$ rev/minute.\begin{itemize}
\item What is its kinetic energy?
\item Find how far it would have to fall to acquire the same kinetic energy as in the question above.
\end{itemize}
\item (2014)  A disc of moment of inertia $ 2.5\times10^{-4}$ kg$/$m$ ^{2}$ is rotating freely about an axis through its centre at $ 20$ rev/min. If some wax of mass $ 0.04$ kg is dropped gently on to the disc $ 0.05$ m from its axis, what will be the new revolution per minute of the disc? 
\item (2014)  Explain briefly why a:\begin{itemize}
\item high diver can turn more somersaults before striking the water?
\item dancer on skates can spin faster by folding her arms?
\end{itemize}
\item (2014)  A heavy flywheel of moment of inertia $ 0.4$ kg$/$m$ ^{2}$ is mounted on a horizontal axle of radius $ 0.01$ m. If a force of $ 60$ N is applied tangentially to the axle:\begin{itemize}
\item  Calculate the angular velocity of the flywheel after $ 5$ seconds from rest.
\item List down two assumptions taken to arrive at your answer in above.
\end{itemize}
\item (2015)   Define moment of inertia of a body.\begin{itemize}
\item Briefly explain why there is no unique value for the moment of inertia of a given body?
\end{itemize}
\item (2015)  State the principle of conservation of angular momentum. \begin{itemize}
\item A horizontal disc rotating freely about a vertical axis makes $ 45$ revolutions per minute. A small piece of putty of mass $ 2.0\times10^{-2}$ kg falls vertically onto the disc and sticks to it at a distance of $ 5.0\times10^{-2}$ m from the axis. If the number of revolutions per minute is thereby reduced to $ 36$ , calculate the moment of inertia of the disc. 
\end{itemize}
\item (2015)  What would be the length of a day if the rate of rotation of the Earth were such that the acceleration due to gravity $ g=0$ at the equator?
\item (2016)  Why is Newton’s first law of motion called the law of inertia?
\item (2016)  What is meant by moment of inertia of a body?
\item (2016)  List two factors on which the moment of inertia of a body depends. 
\item (2016)  A thin sheet of aluminum of mass $ 0.032$ kg has the length of $ 0.25$ m and width of $ 0.1$ m. Find its moment of inertia on the plane about an axis parallel to the:\begin{itemize}
\item Length and passing through its centre of mass, $ m$ .
\item Width and passing through the centre of mass, $ m$ , in its own plane.
\end{itemize}
\item (2016)  Define the term angular momentum.
\item (2016)  A thin circular ring of mass, $ M$ , and radius, $ r$ , is rotating about its axis with constant angular velocity, $ w_{1}$ .  If two objects each of mass, $ m$ , are attached gently at the ring, what will be the angular velocity of the rotating wheel?
\item (2016)  Why are space rockets usually launched from west to east?
\item (2017)  Justify the statement that ‘If no external torque acts on a body, its angular velocity will not conserved.
\item (2018)  Why is flywheel designed such that most of its mass is concentrated at the rim? Briefly explain. 
\item (2018)  Estimate the couple that will bring the wheel to rest in $ 10$ seconds when a grinding wheel of radius $ 40$ cm and mass $ 3$ kg is rotating at $ 3600$ revolutions per minute. 
\item (2018)  Why an ice skater rotates at relatively low speed when stretches her arms and a leg outward? 
\item (2018)  Calculate the moment of inertia of a sphere about an axis which is a tangent to its surface given that the mass and radius of the sphere are $ 10$ kg and $ 0.2$ m respectively. 
\end{itemize}


\section{Fluid Dynamics}

\subsection{Streamline Flow and Continuity}
\begin{itemize}
\item (1998)  What is terminal velocity?
\item (1998)  Briefly explain an experiment designed to measure terminal velocity.
\item (1998)  A small sphere of radius $ r$ and density $ \sigma $ is released from the bottom of a column of liquid of density $ \rho $ which is slightly higher than $ \sigma $ . Deduce expressions for;\begin{itemize}
\item the initial acceleration of the sphere.
\item the terminal velocity of the sphere.
\end{itemize}
\item (1998)  Explain why a length of horse pipe which is lying in a curve on a smooth horizontal surface, straightens out when a fast flowing stream of water passes through it.
\item (1999)  Write down the equation of continuity of a fluid defining all your symbols.
\item (2000)  At two points on a horizontal tube of varying circular cross-section carrying water, the radii are 1cm and $ 0.4$ cm and the pressure difference between these points is $ 4.9$ cm of water. How much liquid flows through the tube per second?
\item (2007)  Write the Continuity and Bernoullis’ equations as applied to fluid dynamics. 
\item (2007)  Develop an equation to determine the velocity of a fluid in a venture meter pipe.\begin{itemize}
\item What amount of fluid passes through a section at any given time? 
\end{itemize}
\item (2013)  What is meant by Newtonian fluid? 
\item (2015)  Name the principle on which the continuity equation is based.
\item (2015)  Air is moving fast horizontally past an air-plane.  The speed over the top surface is $ 60$ m$/$s and under the bottom surface is $ 45$ m$/$s.  Calculate the difference in pressure.
\item (2016)   A jet of of water from a fire hose is capable of reaching a height of $ 20$ m.  If the cross sectional area of the hose outlet is $ 4.0	\times 10^{-4}$ m$ ^{2}$ , calculate the:\begin{itemize}
\item Minimum speed of water from the hose.
\item Mass of water leaving the hose each second.
\item Force on the hose due to the water jet.
\end{itemize}
\item (2017)  What is the terminal velocity?
\item (2018)  Compute the mass of water striking the wall per second when a jet of water with a velocity of $ 5$ m$/$s and cross-sectional area of $ 3 \times 10^{-2}$ m$ ^{2}$ strikes the wall at right angle losing its velocity to zero. 
\item (2018)  Define the following terms when applied to fluid flow:\begin{itemize}
\item Non-viscous fluid 
\item Steady flow 
\item Line of flow 
\item Turbulent flow
\end{itemize}
\end{itemize}

\subsection{Bernoulli's Principle}
\begin{itemize}
\item (1999)  The velocity at a certain point in a flow pipe is $ 1.0$ ms$ ^{-1}$ and the gauge pressure there is $ 3 \times 10^5 $ N$/$m$ ^{2}$ ​ . The cross-sectional area at a point $ 10$ m above the first is half that at the first point. If the flowing fluid is pure water, calculate the gauge pressure at the second point.
\item (2000)  Write down the Bernoulli's equations for fluid flow in a pipe and indicate the term which will disappear when the flow of fluid is stopped.
\item (2000)  Water flows into a tank of large cross-section area at a rate of $ 10^{-4}$ m$ ^{3}/$s but flows out from a  hole of area 1cm$ ^{2}$ which has been punched through the base. How high does the water rise in the tank?
\item (2007)  Under what conditions is the Bernoullis’ equation applicable?
\item (2007)  Discuss two $ (2)$ applications of the Bernoullis equation. 
\item (2013)  A submarine model is situated in a part of a tube with diameter $ 5.1$ cm where water moves at $ 2.4$ m$/$s.  Determine the:\begin{itemize}
\item velocity of flow in the water supply pipe of diameter $ 25.4$ cm. 
\item pressure difference between the narrow and the wide tube. 
\end{itemize}
\item (2015)  Write down the Bernoulli’s equation for fluid flow in a pipe and indicate the term which will disappear when the fluid is stopped.
\item (2015)  Basing on the applications of Bernoulli’s principle, briefly explain why two ships which are moving parallel and close to each other experience an attractive force.
\item (2015)  Water is flowing through a horizontal pipe having different cross-sections at two points $ A$ and $ B$ .  The diameters of the ippe at $ A$ and $ B$ are $ 0.6$ m and $ 0.2$ m respectively.   The pressure difference between points $ A$ and $ B$ is $ 1$ m column of water.  Calculate the volume of water flowing per second.
\item (2016)  Distinguish between static pressure, dynamic pressure and total pressure when applied to streamline or laminar fluid flow and write down expression at a point in the fluid in terms of the fluid velocity v, the fluid density $ \rho $ , pressure $ P$ and the height $ h$ , of the point with respect to a datum.  
\item (2016)  The static pressure in a horizontal pipeline is $ 4.3 \times 10^{4}$ Pa, the total pressure is $ 4.7 \times 10^{4}$ Pa and the area of cross-section is $ 20 $ cm$ ^{2}$ . The fluid may be considered to be incompressible and non-viscous and has a density of $ 1000$ kg$/$m$ ^{3}$ .  Calculate the flow velocity and the volume flow rate in the pipeline.
\item (2016)  Briefly explain the carburetor of a car as applied to Bernoulli’s theorem.
\item (2016)  Three capillaries of the same length but with internal radii $ 3R$ , $ 4R$ , and $ 5R$ are connected in series and a liquid flows through them under streamline conditions.  If the pressure across the third capillary is $ 8.1$ mm of liquid, find the pressure across the first capillary.
\item (2017)  State Bernoulli's theorem for the horizontal flow. 
\item (2017)  On which principle does the Bernoulli's theorem based. 
\item (2017)  A pipe is running full of water. At a certain point $ A$ , it tapers from $ 30$ cm diameter to $ 10$ cm diameter at $ B$ , the pressure difference between point $ A$ and $ B$ is $ 100$ cm of water column. Find the rate of flow of water through the pipe. 
\item (2017)  Two capillaries of the same length and radii in the ratio of $ 1$:$ 2$ are connected in series and the liquid flow through the system under stream line conditions. If the pressure across the two extreme ends of the combination is  $ 1$ m of water, what is the pressure difference across the first capillary?
\item (2018)  Given the Bernoulli’s equation: $ p+\rho gh+\rho v^{2}=$ constant where all the symbols carry their usual meaning.\begin{itemize}
\item What quantity does each expression on the left hand side of the equation represent? 
\item Mention any three conditions which make the equation to be valid. 
\end{itemize}
\item (2018)  Water is supplied to a house at ground level through a pipe of inner diameter $ 1.5$ cm at an absolute pressure of $ 6.5 \times 10^{5}$ Pa and velocity of $ 5$ m$/$s. The pipe line leading to the second floor bath room $ 8$ m above has an inner diameter of $ 0.75$ cm. Find the flow velocity and pressure at the pipe outlet in the second floor bathroom. 
\item (2018)  A horizontal pipeline increases uniformly from $ 0.080$ m diameter to $ 0.160$ m diameter in the direction of flow of water. When $ 96$ litres of water is flowing per second, a pressure gauge at the $ 0.080$ m diameter section reads $ 3.5 \times 10^{5}$ Pa. What should be the reading of the gauge at the $ 0.160$ m diameter section neglecting any loss? 
\item (2019)  A horizontal pipe of cross - sectional area $ 10 $ cm$ ^{2}$ has one section of cross sectional area $ 5 $ cm$ ^{2}$ . If water flows through the pipe, and the pressure difference between the two sections is $ 300$ Pa, how many cubic meters of water will flow out of the pipe in $ 1$ minute?
\end{itemize}

\subsection{Viscosity and Turbulent Flow}
\begin{itemize}
\item (1998)  Two equal drops of water are falling through air with a steady velocity of $ 0.15$ ms$ ^{-1}$ , If the drops coalesce, find their new terminal velocity.
\item (1999)  With the help of a well labelled diagram briefly explain how you will determine the coefficient of viscosity of a liquid by a constant pressure head apparatus in the laboratory.
\item (2010)  In the form of Millikan’s experiment, an oil drop was observe fall with a constant velocity of $ 2.5	\times 10^{-4}m/s$ in the absence of an electric field. When a p.d of $ 1000$ V was applied between the plates $ 10$ mm apart, the drop remained stationary between them. i the density of oil is $ 9 \times 10^{2}$ kg$/$m$ ^{3}$ , density of air is $ 1.2$ kg$/$m$ ^{3}$ and viscosity of air is $ 1.8\times 10^{-5}$ Ns$/$m$ ^{2}$ , Calculate the radius of the oil drop and the number of electric charges it carries.
\item (2013)  Write down the Poiscuille’s equation for a viscous fluid flowing through a tube defining all the symbols.\begin{itemize}
\item What assumptions are used to develop the equation above. 
\end{itemize}
\item (2015)  A sphere is dropped under gravity through a fluid of viscosity, $ \eta $ .  Taking average acceleration as half of the initial acceleration, show that the time taken to attain terminal velocity is independent of fluid density.
\item (2015)  The flow rate of water from a tap of diameter $ 1.25$ cm is $ 3$ litres per minute.  The coefficient of viscosity of water is $ 10^{-3}$ Ns/m$ ^{2}$ .  Determine the Reynolds’ number and then state the type of flow of water.
\item (2016)  State Newton’s law of viscosity and hence deduce the dimensions of the coefficient of viscosity.
\item (2016)  In an experiment to determine the coefficient of viscosity of motor oil, the following measurements are made:\begin{itemize}
\item Mass of glass sphere $ =1.2 \times 10^{-4}$ kg.
\item Diameter of sphere $ =4.0 \times 10^{-3}$ m.
\item Terminal velocity of sphere $ =5.4 \times 10^{-5}$ m$/$s.
\item Density of oil $ =860$ kg$/$m$ ^{3}$
\item Calculate the coefficient of viscosity of the oil.
\end{itemize}
\item (2016)  Give reasons for the following observations as applied in fluid dynamics.\begin{itemize}
\item A flag flutter when strong winds are blowing on a certain day.
\item A parachute is used while jumping from an airplane.
\item Hotter liquids flow faster than cold ones.
\end{itemize}
\item (2017)  Derive an expression for the terminal velocity of a spherical body falling  from rest through a viscous fluid. 
\item (2019)  Give the meaning of the terms velocity gradient, tangential stress and coefficient of viscosity as used in fluid dynamics.
\item (2019)  Write Stokes’ equation defining clearly the meaning of all symbols used.\begin{itemize}
\item State two assumptions used to develop the equation above
\end{itemize}
\item (2019)  Calculate the terminal velocity of the rain drops falling in air assuming that the flow is laminar, the rain drops are spheres of diameter $ 1$ mm and the coefficient of viscosity, $ \eta =1.8 \times 10^{-5}$ Ns$/$m$ ^{2}$ . 
\item (2019)  Water flows past a horizontal plate of area $ 1.2$ m$ ^{2}$ . If its velocity gradient and coefficient of viscosity adjacent to the plate are $ 10$ s$ ^{-1}$ and $ 1.3 \times 10^{-5}$ Ns$/$m$ ^{2}$ respectively, calculate the force acting on the plate.  
\end{itemize}


\section{Properties of Matter}

\subsection{Surface Tension}
\begin{itemize}
\item (1999)  Explain in terms of surface energy, what is meant by the surface tension, ​ $ \gamma $ ​ of a liquid. 
\item (1999)  What energy is required to form a soap bubble of radius $ 1.00$ mm if the surface tension of the soap solution is $ 2.5 \times 10$ ​$ E-4$ ​ N$/$m$ ^{2}$ ​ ?
\item (2000)  Find the work done required to break up a drop of water of radius $ 0.5$ cm into drops of water each having radius of $ 1.0$ mm, assuming isothermal condition.
\item (2010)  State surface tension In terms of energy. 
\item (2010)  The Surface tension of water at $ 20^{\circ}$C is $ 7.28 \times 10^{-2}N/m^{2}$ . The vapor pressure of water at this temperature is $ 2.33 \times 10^{3}$ Pa Determine the radius of smallest spherical water droplet which it can form without evaporating
\item (2010)  A circular ring of thin wire $ 3$ cm in radius is suspended with its plane horizontal by a thread passing through the $ 10$ cm mark of a metre rule pivoted at its centre and is balanced by $ 8$ g weight suspended at the $ 80$ cm mark. When the ring is just brought in contact with the surface of a liquid, the $ 8$ g weight has to be moved to the $ 90$ cm mark to just detach the ring from the liquid. Find the surface tension of the liquid (assume zero angle of contact.)
\item (2013)  Using the method of dimensions, indicate which of the following equations are dimensionally correct and which are not, given that, $ f=$ frequency, $ \gamma =$ surface tension, $ \rho =$ density, $ r=$ radius and $ k=$ dimensionless constant.\begin{itemize}
\item  $ \rho^{2}=k\sqrt{r^{3}f/\gamma }$
\item  $ f=(kr^{3}\sqrt{\gamma })/(\rho^{1/2})$
\item  $ f=(k\gamma^{1/2})/(\sqrt{\rho}r^{3/2})$
\end{itemize}
\item (2013)  Distinguish surface tension from surface energy.
\item (2013)  Explain the phenomenon of surface tension in terms of the molecular theory.
\item (2013)  A clean open ended glass U-tube has vertical limbs one of which has a uniform internal diameter of $ 4.0$ mm and the other of $ 20.0$ mm. Mercury is poured into the tube; and observed that the height of mercury column in the two limbs ts different.\begin{itemize}
\item Explain this observation
\item Calculate the difference in levels
\end{itemize}
\item (2016)  Define the following terms:\begin{itemize}
\item Free surface energy
\item Capillary action
\item Angle of contact
\end{itemize}
\item (2016)  Briefly explain the following observations:\begin{itemize}
\item Soap solution is a better cleansing agent than ordinary water.
\item When a piece of chalk is put into water, it emits bubbles in all directions.
\end{itemize}
\item (2016)  Two spherical soap bubbles are combined.  If v is the change in volume of the contained air, $ A$ is the change in total surface area, show that $ 3P_{A}V+4A T=0$ . Where $ T$ is the surface tension and $ P_{A}$ is the atmospheric pressure.
\item (2016)  There is a soap bubble of radius $ 3.6 \times 10^{-4}$ m in air cylinder which is originally at a pressure of $ 10^{5}$ N$/$m$ ^{2}$ . The air in the cylinder is now compressed isothermally until the radius of the bubble is halved. Calculate the pressure of air in the cylinder.
\item (2017)  \item (2017)  A spherical drop of mercury of radius $ 5$ mm falls on the ground and breaks into $ 1000$ droplets. Calculate the work done in breaking the drop. 
\item (2018)  Mention any two factors which affect the surface tension of the liquid and in each case explain two typical examples. 
\item (2018)  Why molecules on the surface of a liquid have more potential energy than those within the liquid? Briefly explain. 
\item (2018)  Derive an expression for excess pressure inside a soap bubble of radius $ R$ and surface tension $ \gamma $ when the pressures inside and outside the bubble are $ P_{2}$ and $ P_{1}$ respectively. 
\item (2018)  A soap bubble has a diameter of $ 5$ mm. Calculate the pressure inside it if the atmospheric pressure is $ 10^{5}$ Pa and the surface tension of a soap solution is $ 2.8 \times 10^{-2}$ N$/$m.
\item (2018)  Water rises up in a glass capillary tube up to a height of $ 9.0$ cm while mercury falls down by $ 3.4$ cm in the same capillary. Assume angles of contact for water-glass and . mercury-glass as $ 0^{\circ}$ and $ 135^{\circ}$ respectively. Determine the ratio of surface tensions of mercury and water. 
\end{itemize}

\subsection{Elasticity}
\begin{itemize}
\item (1999)  Define "Young's Modulus" of a material and give its SI units.
\item (1999)  With the aid of a sketch graph, explain what happens when a steel wire is stretched gradually by an increasing load until it breaks. 
\item (1999)  A force $ F$ is applied to a long steel wire of length $ L$ and cross-sectional area A.\begin{itemize}
\item Show that if the wire is considered to be a spring, the force constant $ k$ is given by: $ k= AY/L$ , where $ Y$ is Young's Modulus of the wire.
\item Show that the energy stored in the wire is $ U=1/2F \Delta L$ where $ \Delta{L}$ is the extension of the wire
\end{itemize}
\item (2000)  Define the “bulk modulus” of a gas
\item (2000)  Find the ratio of the adiabatic bulk modulus of a gas to that of its isothermal bulk modulus in terms of the specific heat capacities of the gas.
\item (2000)  Explain Young’s Modulus of rigidity
\item (2000)  Find the work done in stretching a steel wire of $ 1.0$ mm$ ^{2}$ cross-sectional area and $ 2.0$ m in length through $ 0.1$ mm.
\item (2007)  With the aid of a diagram describe a simple laboratory experiment to measure Young’s modulus of a wooden bar acting as a loaded cantilever from its period of vibration given that the depression s is given by $ S=(WL^{3})/(3IE)$ . 
\item (2007)  Differentiate between tensile and shear stress. 
\item (2007)  A lift is designed to hold a maximum of $ 12$ people. The lift cage has a mass of $ 500$ kg and the distance from the top floor of the building to the ground floor is $ 50$ m.\begin{itemize}
\item What minimum cross-sectional area should the cable have in order to support the lift and the people in it?
\item Why should the cable have to be thicker than the minimum cross-sectional area above in practice? 
\item How much will the lift cable above stretch if $ 10$ people get into the lift at the ground floor, assuming that the lift cable has a cross section of $ 1.36$ cm? 
\item Note: Mass of an average person $ =70$ kg . $ E_{steel}=2 \times 10^{11}$ N$/$m$^{2}$ , Tensile strength of steel $ =4 \times 10^{11}$ N$/$m$^{2}$ .
\end{itemize}
\item (2009)  Define the following terms:\begin{itemize}
\item Tensile stress
\item Tensile strain
\item Young’s modulus
\end{itemize}
\item (2009)  Derive the expression for the work done in stretching a wire of length $ L$ by a load $ W$ through an extension $ X$ .
\item (2009)  A vertical wire made of steel of length $ 2.0$ m and $ 1.0$ mm diameter has a load of $ 5.0$ kg applied to its lower end.  What is the energy stored in the wire?
\item (2009)  A copper wire $ 2.0$ m long and $ 1.22 \times 10^{-3}$ m diameter is fixed horizontally to two rigid supports $ 2.0$ m apart.  Find the mass in kg of the load, which when suspended at the mid point of the wire, produces a sag of $ 2.0 \times 10^{-2}$ m at the point.
\item (2013)  The bulk modulus of elasticity for lead is $ 8 \times 10^{9}$ N$/$m$ ^{2}$ . Find the density of lead if the pressure applied is $ 2 \times 10^{8}$ N$/$m$ ^{2}$ . 
\item (2013)  Define the terms: proportional limit, elastic limit, yield point and elasticity.
\item (2013)  \item (2015)  Define the following materials as classified on the basis of elastic properties:\begin{itemize}
\item  Ductile materials 
\item Brittle materials
\item Elastomers
\end{itemize}
\item (2015)  Briefly explain why the stretching of a coil spring is determined by its shear modulus.
\item (2015)  A copper wire of negligible mass, $ 1$ m long and cross-sectional area $ 10^{-5}$ m$ ^{2}$ is kept on a smooth horizontal table with one end fixed.  A ball of $ 1$ kg is attached to the other end.  The wire and the ball are rotating with an angular velocity of $ 35$ rad$/$s.  If the elongation of the wire is $ 10^{-3}$ m, find Young’s modulus of wire.  If on increasing the angular velocity to $ 100$ rad$/$s, the wire breaks down, find the breaking stress.
\item (2015)  Differentiate bulk modulus from shear modulus.
\item (2015)  Two wires, one of steel and one of phosphor bronze each $ 1.5$ m long and $ 2$ mm diameter are joined end to end as a composite wire of length $ 3$ cm.  What tension in the composite wire will produce total extension of $ 0.064$ cm?
\item (2016)  What is strain energy?
\item (2017)  A steel rod of length $ 0.60$ m and cross-sectional area $ 2.5 \times 10^{-5}$ m$ ^{2}$ at a temperature of $ 100^{\circ}$C is clamped so that when it cools was unable to contract. Find the tension in the rod when it has cooled to $ 20^{\circ}$C. 
\item (2017)  A spring $ 60$ cm long is stretched by $ 2$ cm for the application of load of $ 200$ g. What will be the length when a load of $ 500$ g is applied? 
\item (2017)  Calculate the percentage increase in length of a wire of diameter $ 2.2$ mm stretched by a load of $ 100$ kg. ( Young's modulus of wire is $ 12.5 \times 10^{10}$ N$/$m$ ^{2})$
\item (2017)  Define tensile stress and tensile strain. 
\item (2017)  Calculate the work done in a stretching copper wire of $ 100$ cm long and $ 0.03 $ cm$ ^{2}$ cross — sectional area when a load of $ 120$ N is applied. 
\item (2017)  Mention any two factors on which modulus of elasticity of a material depends.
\item (2018)  Briefly explain the following observations as applied to strengths of materials:\begin{itemize}
\item Bridges are declared unsafe after long use. 
\item Iron is more elastic than rubber. 
\end{itemize}
\item (2018)  A composite wire of diameter $ 1$ cm consists of copper and steel wires of lengths $ 2.2$ m and $ 2$ m respectively. Total extension of the wire when stretched by a force is $ 1.2$ mm. Calculate the force, given that Young’s modulus for copper is $ 1.1 \times 10^{11}$ Pa and for steel is $ 2 \times 10^{11}$ Pa. 
\item (2018)  What do you understand by the following terms?\begin{itemize}
\item A perfectly plastic material 
\item The ultimate tensile strength 
\item An elastic limit 
\item Poisson’s ratio. 
\end{itemize}
\item (2018)  Two rods of different materials but of equal cross-sections and lengths $ 1.0$ m each are joined to make a rod of length $ 2.0$ m. The metal of one rod has coefficient of linear thermal expansion of $ 10^{-5}^{\circ}$C$ ^{-1}$ and Young’s Modulus $ 3 \times 10^{10}$ N$/$m$ ^{2}$ . The other metal has the values $ 2 \times 10^{-5}^{\circ}$C$ ^{-1}$ and $ 10^{10}$ N$/$m$ ^{2}$ respectively. How much pressure must be applied to the ends of the composite rod to prevent its expansion when the temperature is raised by $ 100^{\circ}$C? 
\item (2019)  Define Young’s Modulus of a material. 
\item (2019)  Why work is said to be done in stretching a wire? 
\item (2019)  A steel wire AB of the length $ 60$ cm and cross-sectional area $ 1.5 \times 10^{-6}$ m$ ^{2}$ is attached at $ B$ to copper wire BC of length $ 39$ cm and cross sectional area $ 3.0 \times 10^{-6}$ m$ ^{2}$ . If the combination of the two wires is suspended vertically from a fixed point at A, and supports a weight of $ 250$ N at $ C$ ; find the extension (in millimeter) of the:\begin{itemize}
\item steel wire. 
\item copper wire. 
\end{itemize}
\end{itemize}

\subsection{Kinetic Theory of Gases}
\begin{itemize}
\item (1999)  Write down the equation of state of an ideal gas defining all the symbols used.
\item (1999)  If the root-mean-square velocity of a hydrogen molecule at $ 0​ ^{\circ}$C is $ 1840$ m$/$s, find the root-mean-square velocity of the molecule at $ 100​ ^{\circ}$ ​ $ C$ .
\item (1999)  State the main assumptions of the “kinetic theory" of gases.
\item (1999)  Derive an expression for the pressure exerted by an ideal gas on the walls of its container.
\item (1999)  How does the average translational kinetic energy of a molecule of an ideal gas change if\begin{itemize}
\item the pressure is doubled while the volume is kept constant?
\item the volume is doubled while the pressure is kept constant?
\end{itemize}
\item (1999)  Calculate the value of the root mean-square speed of molecules of helium at $ 0^{\circ}$C .
\item (2000)  What factors lead the real gas to obey the ideal gas equation $ PV = RT$ ?
\item (2000)  Define the root-mean-square (r.m.s.) speed of the gas molecules. Hence find the r.m.s. speed of oxygen gas molecules at $ 10^{5}$ Pa pressure when the density is $ 1.43$ kg$/$m$ ^{3}$ .
\item (2000)  Derive an expression for the work done per mole in an isothermal expansion of Vander Waal’s gas from volume $ V_{1}$ to volume $ V_{2}$ .
\item (2007)  Define an ideal gas.
\item (2007)  State the four $ (4)$ assumptions necessary for an ideal gas that are used to develop the expression $ p=$ ½ $ \rho C^{2}$ .
\item (2007)  How is pressure explained in terms of the kinetic theory? 
\item (2007)  Without a detailed mathematical analysis argue the steps to follow in deriving the relation $ p=$ ½ $ \rho C^{2}$ .
\item (2007)  Define the temperature of an ideal gas as a consequence of the kinetic theory.
\item (2007)  A mole of an ideal gas at $ 300K$ is subjected to a pressure of $ 10^{5}N/m^{2}$ and its volume is $ 2.5 \times 10^{-2}m^{3}$ .  Calculate the:\begin{itemize}
\item molar gas constant $ R$
\item Boltzmann constant $ k$
\item average transnational kinetic energy of a molecule of the gas.
\end{itemize}
\item (2013)  Define comprehensibility of a gas in terms of the elasticity of gases. 
\item (2013)  Helium gas occupies a volume of $ 4 \times 10^{-2}$ m$ ^{3}$ at a pressure of $ 2 \times 10^{5}$ Pa and temperature of $ 300$ K. Calculate the mass of helium and the r.m.s speed of its molecules.
\item (2014)  One mole of a gas expands from volume, $ V_{1}$ , to a volume $ V_{2}$ . If the gas obeys the Van-der-Waal’s equation, $ (p+ a/v^{2})(v – b)=$ RT, derive the formula for work done in this process.
\item (2019)  Based on the kinetic theory of gases determine:\begin{itemize}
\item The average translational kinetic energy of air at a temperature of $ 290$ K.
\item The root mean square seed (r.m.s) of air at the same temperature (above).
\end{itemize}
\end{itemize}


\section{Heat}

\subsection{Thermometers}
\begin{itemize}
\item (1999)  What do you understand by the term:   Triple point of water
\item (1999)  The resistance of a platinum wire at a temperature T​$ ^{\circ}$C measured on a gas scale is given by $ R(T)=R​_{0​}(1+ a T+bT​^{2}​)$ .\begin{itemize}
\item What temperature will the platinum thermometer indicate when the temperature on the gas scale is $ 200​^{\circ}$C ? (take a $ =3.8 \times 10^{-3}$ ​ and $ b=-5.6 \times 10^{-7}$ ​)
\end{itemize}
\item (2000)  What does one require in order to establish a scale of temperature?
\item (2000)  A copper-constantan thermocouple with its cold junction at $ 0^{\circ}$C had an emf of $ 4.28$ mV when its other hot junction was at $ 100^{\circ}$C. The emf became $ 9.29$ mV when the temperature of the hot junction was $ 200^{\circ}$C. If the emf $ E$ is related to the temperature difference $ 8$ between hot and cold junctions by the equation $ E= A(\theta )+B(\theta ^{2})$ , calculate:\begin{itemize}
\item The values of $ A$ and $ B$ .
\item The range of temperature for which $ E$ may be assumed proportional to $ 8$ without incurring an error of more than $ 1\%$ .
\end{itemize}
\item (2000)  The resistance $ R$ , of a platinum varies with temperature $ t$ according to the equation $ R_{t}=R_{o}(1+8000bt -b t^{2})$ where $ b$ is a constant. Calculate the temperature on platinum scale corresponding to $ 400^{\circ}$C on the gas scale. 
\item (2000)  Heat is supplied at a rate of $ 80$ W to one end of a well lagged copper bar of uniform cross section area $ 10$ cm? having a total length of $ 20$ cm. The heat is removed by water cooling at the other end of the bar. Temperature recorded by two thermometers $ T_{1}$ and $ T_{2}$ at distances $ 5$ cm and $ 15$ cm from the hot end are $ 48^{\circ}$C and $ 28^{\circ}$C respectively.\begin{itemize}
\item Calculate the thermal conductivity of copper.
\item Estimate the rate of flow (in g$/$min) of cooling water sufficient for the water temperature to rise $ 5$ K. 
\item What is the temperature at the cold end of the bar? 
\end{itemize}
\item (2007)  What is meant by a thermometric property of a substance?
\item (2007)  What qualities make a particular property suitable for use in practical thermometers?
\item (2007)  Explain why at least two $ (2)$ fixed points are required to define a temperature scale.
\item (2007)  Mention the type of thermometer which is most suitable for calibration of thermometers.
\item (2010)  In a special type thermometer a fixed mass of a gas has a volume of $ 100$ cm? at a pressure of $ 81.6$ cmHg at the ice point and volume of $ 124$ cm$ ^{3}$ and pressure of $ 90$ cmHg at steam point. Determine the temperature if its volume is $ 120$ cm$ ^{3}$ and pressure of $ 85$ cmHg.\begin{itemize}
\item What value does the scale of this thermometer give for absolute
\item zero? 
\end{itemize}
\item (2013)  Name the temperature of a thermocouple at which the thermo,\begin{itemize}
\item e.m.f. changes its sign.
\item electric power becomes zero.
\end{itemize}
\item (2013)  A Nichrome-coustantan thermocouple gives about $ 70$ $\mu$V for each $ 1^{\circ}$C difference in temperature between the junctions. If $ 100$ such thermocouples are made into a thermopile, what voltage is produced when the junctions are at $ 20^{\circ}$C and $ 240^{\circ}$C? 
\item (2014)  What is meant by temperature of inversion?
\item (2014)  A thermometer was wrongly calibrated as mt reads the melting point of ice as $ -10^{\circ}$C and reading a temperature of $ 60^{\circ}$C in place of $ 50^{\circ}$C What would be the temperature of boiling point of water on this scale? 
\item (2015)  What is meant by a thermometric property?
\item (2015)  Mention three qualities that make a particular property suitable for use in a practical thermometer.
\item (2016)  Briefly describe the working principle of a thermocouple. 
\item (2016)  In a certain thermocouple thermometer the e.m.f. is given by $ E= a \theta + 1/2 b\theta^{2}$ where $ \theta $ is the temperature of hot junction. If a$ =10 $ mV$ ^{\circ}C^{-2}$ , $ b=-1/20 $ mV$ ^{\circ}C^{-2}$ and the cold junction is at $ 0^{\circ}$C, calculate the neutral temperature. 
\item (2017)  The value of the property $ X$ of a certain substance Is given by $ X_{\theta}=X_{0}+0.5\theta +2\times 10^{-4}\theta ^{2}$  , Where $ \theta $ is the temperature in degree Celsius. What would be the Celsius temperature defined by the property $ X$ which corresponds to a temperature of $ 50^{\circ}$C on this gas thermometer scale? 
\item (2018)  Which type of thermometer is most suitable for calibration of other thermometers? 
\item (2018)  Why at least two fixed points are required to define a temperature scale?
\item (2018)  List two qualities which makes a particular property suitable for use in practical thermometers. 
\item (2018)  Describe how mercury in glass thermometer could be made sensitive.
\item (2018)  What is meant by triple point of water? 
\item (2018)  Evaluate the temperature in Kelvin if the pressure recorded by a constant volume gas thermometer is $ 6.8 \times 10^{4}$ Nm$ ^{-2}$ given that the pressure at triple point $ 273.16$ K is $ 4.6 \times 10^{4}$ Nm$ ^{-2}$ .
\item (2019)  A thermometer has wrong calibration as it reads the melting point of ice as $ -10^{\circ}$C . If it reads $ 40^{\circ}$C in a place where the temperature reads $ 30^{\circ}$C ,  determine the boiling point of water on this scale.
\end{itemize}

\subsection{Thermal Conduction}
\begin{itemize}
\item (1999)  What is the coefficient of thermal conductivity of a material?
\item (1999)  The temperature difference between the inside and outside of a room is $ 25​^{\circ}$C . The room has a window of an area $ 2$ m$ ​^{2}$ and the thickness of the window material is $ 2$ mm. Calculate the heat flow through the window if the coefficient of thermal conductivity of the window material is $ 0.5$ SI units.
\item (2000)  Define the thermal conductivity of a material
\item (2000)  Give one major similarity and one major difference between heat conduction and wave propagation.
\item (2000)  Deep bore holes into the earth show that the temperature increases about $ 1^{\circ}$C for each $ 30$ m depth. How much heat flows out from the core of the earth each second for each square metre of surface area.
\item (2007)  Explain why in cold climates, windows of modern buildings are double glazed, ie: There are two pieces of glass with a small air space between them.
\item (2010)  A cylindrical element of $ 1$ kW electric fire $ 1$ s $ 30$ cm long and $ 1.0$ cm in  diameter. If the temperature of the surroundings is $ 20^{\circ}$C , estimate the working temperature of the element.
\item (2013)  Compare the law governing the conduction of heat and electricity pointing out the corresponding quantities in each case.
\item (2013)  A Lagged copper rod is uniformly heated by a passage of an electric current. Show by considering a small section dx that the temperature $ \theta $ varies with distance $ x$ along a rod in a way that, $ k\frac{d^{2}T}{dx^{2}}=-H$ , where $ k$ is a thermal conductivity and $ H$ is the rate of heat generation per unit volume.
\item (2015)  Define coefficient of thermal conductivity.
\item (2015)  Write down two characteristics of a perfectly lagged bar.
\item (2015)  A thin copper wall of a hot water tank having a total surface area of $ 5.0$ m$ ^{2}$ contains $ 0.8$ cm$ ^{3}$ of water at $ 350$ K and is lagged with a $ 50$ mm thick layer of a material of thermal conductivity $ 4.0\times10^{-2}$ W$/$mK. If the thickness of copper wall is neglected and the temperature of the outside surface is $ 290$ K,\begin{itemize}
\item Calculate the electrical power supplied to an immersion heater.
\item If the heater were switched off, how long would it take for the temperature of hot water to fall by $ 1$ K?
\end{itemize}
\item (2016)  Identify two factors on which the coefficient of thermal conductivity of a material depend.
\item (2016)  A brass boiler of base area $ 1.50\times 10^{-1}$ and thickness $ 1.0$ cm boils water at a rate of $ 6.0$ kg$/$min when placed on a gas Stove. Estimate the temperature of the part of the flame in contact with the boiler.
\item (2019)  A closed metal vessel containing water at $ 75^{\circ}$C , has a surface area of $ 0.5$ m$ ^{2}$ and uniform thickness of $ 4.0$ mm.  If its outside temperature is $ 15^{\circ}$C , calculate the head loss per minute by conduction.
\end{itemize}

\subsection{Thermal Convection}
\begin{itemize}
\item (2000)  Write down a formula for the rate of cooling under natural convection and define all the symbols used. 
\item (2007)  State Newton’s law of cooling and give one limitation of the law.
\item (2007)  A body initially at $ 70^{\circ}$C cools to a temperature of $ 55^{\circ}$C in $ 5$ minutes. What will be its temperature after $ 10$ minutes given that the surrounding temperature is $ 31^{\circ}$C ? (Assume Newton’s law of cooling holds true)
\item (2010)  Define thermal convection.
\item (2010)  State Newton’s law of cooling.
\item (2010)  A glass disc of radius $ 5$ cm and uniform thickness of $ 2$ mm had one of its sides maintained at $ 100^{\circ}$C while copper block in good thermal contact with this side was found to be $ 70^{\circ}$C . The copper block weighs $ 0.75$ kg. The cooling of copper was studied over a range of temperature and the rate of cooling at $ 70^{\circ}$C was found to be $ 16.5$ K$/$min. Determine the thermal conductivity of glass.
\item (2013)  A person sitting on a bench on a calm hot summer day is aware of a cool breeze blowing from the sea. Briefly explain why there is a natural convection?
\item (2013)  A cup of tea kept in a room with temperature of $ 22^{\circ}$C cools from $ 66^{\circ}$C to $ 63^{\circ}$C in $ 1$ minute. How long will the same cup of tea take to cool from the temperature of $ 43^{\circ}$C to $ 40^{\circ}$C under the same condition?
\item (2014)  Define thermal convection.
\item (2014)  Prove that at a very small temperature difference, $ \Delta T=T_{b}$ – $ T_{s}$ ,  Newton's law of cooling obeys the Stefan’s law, whereby $ T_{b}$ , is the temperature of the body and $ T_{s}$ is the temperature of the surrounding. 
\item (2016)  Briefly explain why forced convection is necessary for excess temperate less than $ 20$ K? 
\item (2016)  State Newton’s law of cooling. 
\item (2016)  A body cools from $ 70^{\circ}$C to $ 40^{\circ}$C in $ 5$ minutes. If the temperature of the surroundings is $ 10^{\circ}$C , Calculate the time it takes to cool from $ 50^{\circ}$C to $ 20^{\circ}$C.  
\end{itemize}

\subsection{Thermal Radiation}
\begin{itemize}
\item (2007)  What is blackbody radiation of a given body?
\item (2007)  Explain why heat may just mean infrared.
\item (2007)  State Prvost's theory of heat exchange.
\item (2007)  What is Wien's displacement law?
\item (2007)  The sun's surface temperature is about $ 6000$ K.  The sun's radiation is maximum at wavelength of $ 0.5\times10^{-6}m$ .  A certain light bulb filament emits radiation with maximum wavelength of $ 2\times10^{-6}m$ .  If both the surface of the sun and of the filament have the same emissive characteristics, what is the temperature of the filament?
\item (2010)  State Stefan’s law of thermal radiation.
\item (2010)  A solid copper sphere cools at the rate of $ 2.8^{\circ}$C$/$min when its temperature is $ 127^{\circ}$C. At what rate will a solid copper sphere of twice the radius cool when its temperature is $ 227^{\circ}$C? In both cases the surroundings are kept at $ 27^{\circ}$C and conditions are such that  Stefan’s law may be applied.
\item (2010)  Explain the observation that a piece of wire when steadily heated up appears reddish in color before turning bluish. 
\item (2013)  A black body of temperature $ \theta $ is placed in a blackened enclosure maintained at a temperature of $ 100^{\circ}$C. When its temperature rises to $ 30^{\circ}$C the net rate of loss of energy from the body was found to be $ 10$ Watts. Find the power generated by the body at $ 50^{\circ}$C if the energy exchange takes place solely by the process of forced convection.
\item (2013)  Write down three laws governing the black body radiation.
\item (2015)  The element of an electric fire with an output of $ 1000$ W is a cylinder of $ 250$ mm long and $ 15$ mm in diameter. If it behaves as a black body, estimate its temperature.
\item (2016)  Briefly explain why: \begin{itemize}
\item A body with large reflectivity is a poor emitter. 
\item The earth without its atmosphere would be too cold to live.
\end{itemize}
\item (2016)  What is meant by thermal radiation?
\item (2016)  Why is the energy of thermal radiation less than that of visible light? 
\item (2016)  A body with a surface area of $ 5.0 $ cm$ ^{2}$ and a temperature of $ 727^{\circ}$C radiates $ 300$ joules of energy in one minute. Calculate its emissivity.
\item (2017)  State the following according to heat exchange:\begin{itemize}
\item  Prevost’s theory. 
\item Wien's displacement law.
\end{itemize}
\item (2018)  Why during emission of radiations from black body its temperature does not  reach zero Kelvin? 
\item (2018)  A black ball of radius $ 1$ m is maintained at a temperature of $ 30^{\circ}$C . How much  heat is radiated by the ball in $ 4$ seconds? 
\item (2019)  Sketch the graph to illustrates how the energy radiated by a black body is distributed among various wavelengths. \begin{itemize}
\item What information would be drawn from the graph above? Give three points.
\end{itemize}
\item (2019)  At what temperature will the filament of a $ 10$ W lamp operate if it is supposed to be a perfectly black body of area  $ 1 $ cm$ ^{2}$ ? 
\end{itemize}

\subsection{First Law of Thermodynamics}
\begin{itemize}
\item (1999)  What do you understand by the term: Thermodynamic temperature scale
\item (2000)  The longitudinal wave speed in gases is given by $ v=\sqrt{\gamma p/ \rho }$ ; where $ \gamma =C_{p}/C_{v}$ , $ P$ is the pressure and $ \rho $ the density of gas. If $ v_{1,}$ and $ v_{2,}$ are the speeds of sound in air at temperature $ T_{1}$ and $ T_{2}$ respectively, show that $ v_{1/v_2}=\sqrt{T_{1}/T_{2}}$\begin{itemize}
\item NOTE: $ C_{p}$ and $ C_{v}$ are the specific heats of the gas at constant pressure and constant volume respectively.
\end{itemize}
\item (2000)  A number of $ 16$ moles of an ideal gas which is kept at constant temperature of $ 320$ K is compressed isothermally from its initial volume of $ 18$ litres to the final volume of $ 4$ litres.\begin{itemize}
\item Calculate the total work done in the whole process.
\item Comment on the sign of numerical answer you've obtained.
\end{itemize}
\item (2000)  A cylinder fitted with a frictionless piston contains $ 1.0$ g of oxygen at a pressure of $ 760$ mmHg and at a temperature of $ 27^{\circ}$C. the following operations are performed in stages: $ (1)$ The oxygen is heated at a constant pressure to $ 127^{\circ}$C and then $ (2)$ it is compressed isothermally to its original volume and finally $ (3)$ it is cooled at a constant volume to its original temperature.\begin{itemize}
\item Illustrate these changes in a sketch $ P-V$ diagram.
\item What is the input of heat to the cylinder in stage $ (1)$ above?
\item How much work does the oxygen do in pushing back the piston during stage $ (1)$ ?
\item How much work is done on the oxygen in stage $ (2)$ ?
\item How much heat must be extracted from the oxygen in stage $ (3)$ ? 
\item (For oxygen: density $ =1.43$ kg$/$m$ ^{3}$ (at stp), $ C_v =670$ J kg$ ^{-1}$ K$ ^{-1}$ and molecular mass $ =32$ )
\end{itemize}
\item (2000)  What is the difference between an “isothermal” process and an “adiabatic” process?
\item (2000)  How much work is required to compress $ 5$ mol of air at $ 20^{\circ}$C and $ l$ atmosphere to $ 1/10$ th of the original volume by\begin{itemize}
\item an isothermal process
\item an adiabatic process?
\item What are the final pressures for the cases and above?
\end{itemize}
\item (2000)  Explain the fact that the temperature of the ocean at great depths is very nearly constant the year round, at a temperature of about $ 4^{\circ}$C .
\item (2000)  In a diesel engine, the cylinder compresses air from approximately standard temperature and pressure to about one-sixteenth the original volume and a pressure of about $ 50$ atmospheres. What is the temperature of the compressed air?
\item (2007)  When a metal cylinder of mass $ 2.0x10^{-2}$ kg and specific heat capacity $ 500$ J$/$kgK is heated at constant power, the initial rate of rise of temperature is $ 3.0$ K$/$min.  After a time the heater is switched off and the initial rate of fall of temperature is $ 0.3$ K$/$min.  What is the rate at which the cylinder gains heat energy immediately before the heater is switched off?
\item (2007)  State the expression for the $ 1$ st law of thermodynamics.
\item (2007)  What do you understand by the terms:\begin{itemize}
\item critical temperature? 
\item adiabatic change?
\end{itemize}
\item (2007)  Find the number of molecules and their mean kinetic energy for a cylinder of volume $ 5 \times 10^{-4}m^{3}$ containing oxygen at a pressure of $ 2 \times 10^{5}$ Pa and a temperature of $ 300K$ . \begin{itemize}
\item When the gas is compressed adiabatically to a volume of $ 2 \times 10^{-4}m^{3}$ , the temperature rises to $ 434K$ . Determine $ \gamma $ , the ratio of the principal heat capacities.
\item [ Molar gas constant $ R=8.31$ J$/$mol$/$K ,$ N  =6 \times 10^{32}$ mol$^{-1}$ ] 
\end{itemize}
\item (2009)  What is the difference between isothermal and adiabatic processes?\begin{itemize}
\item Write down the equation of state obeyed by each process in the question above.
\end{itemize}
\item (2009)  Using the same graph and under the same conditions sketch the isotherms and the adiabatics.
\item (2009)  Derive the expression for the work done by the gas when it expands from volume $ V_{1}$ to volume $ V_{2}$ during an:\begin{itemize}
\item Isothermal process
\item Adiabatic process
\end{itemize}
\item (2009)  When water is boiled under a pressure of $ 2$ atmospheres the boiling point is $ 120^{\circ}$C. At this pressure $ 1$ kg of water has a volume of $ 10^{-3}$ m$ ^{3}$ and $ 2$ kg of steam have a volume of $ 1.648$ m$ ^{3}$ . Compute the work done when $ 1$ kg of steam is formed at this temperature increase in the internal energy. 
\item (2010)  Briefly describe an experiment to measure temperature coefficient of a wire.
\item (2010)  A heating coil is made of a nichrome wire which will operate on a $ 12$ V supply and will have a power of $ 36$ W when immersed in water at $ 373$ K. The wire available has a cross-sectional area of $ 0.10$ mm$ ^{2}$ . What length of the wire will be required? 
\item (2013)  Briefly give comments on the following observations:\begin{itemize}
\item Polyatomic and diatomic gases have larger molar heat capacities than monatomic gases. 
\item  Cubical container is used for the derivation of pressure of an ideal gas.
\end{itemize}
\item (2013)  What is meant by a gas constant. 
\item (2013)  When a gas expand adiabatically it does work on its surroundings although there is no heat input to the gas. Explain where this energy is coming from.
\item (2013)  An ideal gas at $ 17^{\circ}$C and $ 750$ mmHg is compressed isothermally Until its volume is reached to ¾ of its initial value If it then allowed to expand adiabatically to a volume of $ 20\%$ greater than its original value. calculate the final temperature and pressure of the gas. 
\item (2013)  How does the first law of thermodynamics change under isothermal and adiabatic processes? 
\item (2013)  Show that the specific heat capacities of an ideal gas are related by the relation $ C_{p}=C_{v}+nR$ .\begin{itemize}
\item Explain the meaning of all the symbols used in the equation above.
\end{itemize}
\item (2013)  One mole of an ideal monatomic gas is heated at constant volume from the temperature of $ 300$ K to $ 600$ K. Calculate the:\begin{itemize}
\item amount of heat added 
\item work done by the gas 
\item change in its infernal energy
\end{itemize}
\item (2013)  The piston of a bicycle pump at room temperature of $ 290$ K is slowly moved in until the volume of air enclosed is one — fifth of the total volume of the pump. The outlet is then sealed and the piston suddenly drawn out to full extension. If no air passes the piston, find the temperature of the air in the pump immediately after withdrawing the piston, assuming that air ts an ideal gas with cryoscopic constant, $ \gamma =1.4$ .
\item (2014)  List down two simple applications of the First law of thermodynamics in our daily life.
\item (2014)  A heat engine works at two temperatures of $ 27^{\circ}$C and $ 227^{\circ}$C. Calculate the:\begin{itemize}
\item Efficiency of the engine. 
\item Temperature which will increase the efficiency by $ 10\%$ if the room temperature is kept at $ 27^{\circ}$C. 
\end{itemize}
\item (2017)  Give a common example of adiabatic process. 
\item (2017)  What happens to the internal energy of a gas during adiabatic expansion?
\item (2017)  A mass of an ideal gas of volume $ 400$ cm$ ^{3}$ at $ 288$ K expands adiabatically. If its temperature falls to $ 273$ K;\begin{itemize}
\item Find the new volume of the gas. 
\item Calculate the final volume of the gas if it is then compressed isothermally until the pressure returns to its original value.
\end{itemize}
\item (2017)  Briefly explain why:\begin{itemize}
\item Steam pipes are wrapped with insulating materials?
\item Stainless steel cooking pans fitted with extra copper at the bottom are more preferred?
\end{itemize}
\item (2017)  The capacitance $ C$ of a capacitor ts full charged by a $ 200$ V battery. It is then discharged through a small coil of resistance wire embedded in a thermally insulated block of specific heat capacity $ 2.5 \times 10^{2}$ J$/$kgK and of mass of $ 0.1$ kg.  If the temperature of the block rises by $ 0.4$ K. what is the value of $ C$ ?
\item (2018)  One gram of water becomes $ 1671 $ cm$ ^{3}$ of steam at a pressure of $ 1$ atmosphere. If the latent heat of vaporization at this pressure is $ 2256$ J$/$g, determine the:\begin{itemize}
\item external work done. 
\item increase in internal energy 
\end{itemize}
\item (2019)  Why water is preferred as a cooling agent in many automobiles?
\item (2019)  Analyze  three practical applications of thermal expansion of solids in daily life situations.
\item (2019)  Why stainless steel cooking pans are made with extra copper at the bottom?
\end{itemize}


\section{Vibrations and Waves}

\subsection{Mechanical Vibrations}
\begin{itemize}
\item (2000)  Show how wavelength and frequency of a wave are related.
\item (2007)  State the modes of vibrations in closed and open pipes.  
\item (2013)  What is meant by dispersion of waves? 
\item (2013)  Briefly explain if it is possible for dispersion to take place on a wave whose frequency lies in the audible range.
\item (2015)  Define the following terms:\begin{itemize}
\item Damped oscillations
\item Forced oscillations
\item Resonance
\end{itemize}
\item (2016)  What do you understand by the following terms: \begin{itemize}
\item Damped oscillations. 
\item Undamped oscillations.
\end{itemize}
\item (2016)  Sketch the waveform diagrams to represent the terms: damped oscillations & undamped oscillations
\item (2016)  A steel wire hangs vertically from a fixed point, supporting a weight of $ 80$ N as its lower end.  The length of the wire from the fixed point to the weight is $ 1.5$ m.  Calculate the fundamental frequency emitted by the wire when it is plucked if its diameter is $ 0.5$ mm. 
\item (2017)  A $ 40$ cm long wire is in unison with a tuning fork of frequency $ 256$ Hz, when stretched by a load of density $ 9$ gm$ ^{-3}$ hanging vertically. The load is then immersed in water. By how much the length of the wire should be reduced to bring it again in unison with the same tuning fork,
\end{itemize}

\subsection{Wave Motion}
\begin{itemize}
\item (2000)  What vibrates in the following types of wave motion?\begin{itemize}
\item Light waves
\item Sound waves
\item X-rays
\item Water waves
\end{itemize}
\item (2000)  A plane progressive wave on a water surface is given by the equation $ y=2 \sin 2x(100t -x/30)$ ; where $ x$ is the distance covered in a time $ t$ . $ x$ , $ y$ and $ t$ are in cm and seconds respectively.  Find:\begin{itemize}
\item the wavelength, and frequency of the wave motion.
\item the phase difference between two points on the water surface that are $ 60$ cm apart.
\end{itemize}
\item (2007)  Give two $ (2)$ differences between progressive and standing waves.
\item (2007)  Two progressive waves travelling along the same line in a medium are represented by $ Y_{1}=10 \sin(\omega t +\pi/2)$ and $ Y_{2}=10 \sin(\omega t +\pi/6)$\begin{itemize}
\item If the two progressive waves form a standing wave, determine the resultant amplitude and phase angle of the wave formed.
\end{itemize}
\item (2010)  Distinguish between stationary waves and progressive waves.
\item (2010)  A wave is represented by the equation $ y=10 \sin(0.42\pi(60$ t-x)), where the distance parameters are measured in metres and the time in seconds.\begin{itemize}
\item State whether the wave is stationary or progressive.
\item Determine the wavelength and frequency of the wave.
\item What will be the phase difference between two points which are $ 40$ cm apart? 
\item Calculate the period and amplitude of the wave. 
\end{itemize}
\item (2013)  Define the term standing wave.
\item (2013)  State the position in a stationary wave where a man can hear a louder sound.
\item (2016)  State the principle of:\begin{itemize}
\item Superposition of waves
\item Huygens construction of wave fronts.
\end{itemize}
\item (2017)  The equation $ y= a  \sin(\omega t – kx)$ represents a plane wave traveling in a medium along the $ x$ - direction, $ y$ being the displacement at the point $ x$ at time $ t$ . Deduce whether the wave is traveling in the positive $ x$ – direction or in the negative $ x$ – direction.\begin{itemize}
\item If $ z=1.1 \times 10^{-7}$ m , $ \omega = 6.5 \times 10^{3}$ s$ ^{-1}$ , $ k=19$ m$ ^{-1}$ ; determine the speed of the wave.
\end{itemize}
\item (2018)  What do you understand by the terms:\begin{itemize}
\item Progressive wave 
\item Refraction of waves 
\item Diffraction of waves 
\item Standing wave. 
\end{itemize}
\item (2018)  Two progressive waves traveling in the opposite direction in the medium are represented by $ Y_{1}=5 \sin(\omega t+\pi/3)$ and  $ Y_{2}=5 \sin(\omega t- \pi/3)$ . If the two progressive waves form a standing wave, determine the resultant amplitude and the phase angle formed. 
\item (2019)  Give the meaning of the terms wave function, longitudinal wave and transverse waves.
\item (2019)  The equation of a Progressive wave traveling in the $ +x$ direction is given by $ y= a \sin(\omega t-kx)$ .  Show that the maximum velocity, $ V_{max}=2\pi a /T$ . 
\end{itemize}

\subsection{Sound}
\begin{itemize}
\item (2000)  Two open organ pipes of length $ 50$ cm and $ 51$ cm respectively give beat frequency of $ 6.0$ Hz when sounding their fundamental notes together, neglecting end corrections. What value does this give for the velocity of sound in air?
\item (2007)  A metre-long tube at one end, with a movable piston at the other end, shows resonance with a fixed frequency source (a tuning fork) of frequency $ 340$ Hz when the tube length is $ 25.5$ cm or $ 79.3$ cm.  Estimate the speed of sound in air at the temperature of the experiment (ignore edge effects).
\item (2007)  The shortest length of the resonance tube closed at one end which resounds to a fork of frequency $ 256$ Hz is $ 32.0$ cm.  The corresponding length for a fork of frequency $ 384$ Hz is $ 20.8$ cm.  Determine the end correction for the tube and the velocity of sound in air.
\item (2013)  A small speaker emitting $ 4$ note of frequency $ 250$ Hz is placed over the open upper end of a vertical tube which is full of water. When the water is gradually run out of the tube the air column resonates. If the initial and final position of the water surface below the top are $ 0.31$ m and $ 0.998$ m respectively, calculate the speed of sound in air and the end-correction of the tube. 
\item (2015)  A source of sound emits waves of frequency, $ f$ , and is moving with a speed of $ u_{s}$ towards the listener and away from the listener.  Derive an expression for apparent frequency $ f_{A}$ of sound in each case if the velocity of sound wave in air is v.  
\item (2016)  Define the following terms:\begin{itemize}
\item Intensity of sound
\item Ultrasonic
\item Overtones
\end{itemize}
\item (2016)  Give any two applications of ultrasonic as applied to sound waves.
\item (2017)  Briefly explain why diffraction is common in sound but not in light.
\item (2018)  The shortest length of the resonance tube closed at one end which resounds to fork of frequency $ 256$ Hz is $ 31.6$ cm, The corresponding length for a fork of frequency $ 384$ Hz is $ 20.5$ cm. Determine the end correction for the tube and the velocity of sound in air. 
\item (2019)  Provide one evidence which proves that sound is a wave.
\item (2019)  Why thunder of lightning is heard some moments after seeing the flash?
\end{itemize}

\subsection{Electromagnetic Waves (em-waves)}
\begin{itemize}
\item (2018)  What do you understand by the term photon. 
\item (2018)  List down any three properties of a photon. 
\end{itemize}

\subsection{Physical Optics (interferance/diffraction/polarization)}
\begin{itemize}
\item (1998)  What is a diffraction grating?
\item (1998)  A diffraction grating has $ 5000$ lines per centimetre. At what angles will bright diffraction images be observed, if it is used with monochromatic light of wavelength $ 6.0 \times 10^{-7}$ m at normal incidence?
\item (1998)  A lamp emits two wavelengths, $ 4.2 \times 10^{-7}$ m and $ 6.0 \times 10^{-7}$ m. Find the angular separation of these two waves in the third order diffraction pattern produced by a diffraction grating having $ 4000$ lines per centimetre, when light is at normal incidence on the grating?
\item (1999)  What is the difference between refraction and diffraction as applied to waves?
\item (1999)  A parallel beam containing two wavelengths $ 600$ nm and $ 602$ nm is incident on a diffraction grating with $ 400$ lines per mm. Calculate the angular separation of the first order spectrum of the two wavelengths. ($ 1$ nm $ =10^{-9}$ m)
\item (2000)  Explain briefiy the necessary conditions for the effects of interference in optics to be observed
\item (2000)  Interference patterns are formed when using Young’s double slit arrangement. Mention other three methods that can be used to form interference patterns.
\item (2000)  A beam of monochromatic light of wavelength $ 600$ nm in air passes into glass. Calculate:\begin{itemize}
\item the speed of light in glass.
\item the frequency of light.
\item the wavelength of light in glass.
\end{itemize}
\item (2000)  What is meant by “diffraction grating”?
\item (2000)  A monochromatic light of wavelength $ 5.2 \times 10^{-7}$ m falls normally on a grating which has $ 4 \times 10^{3}$ lines per cm.\begin{itemize}
\item What is the largest order of spectrum that can be visible?
\item Find the angular separation between the third and fourth order image.
\end{itemize}
\item (2007)  Using the notation of energy bands, explain the following optical properties of solids.\begin{itemize}
\item  All metals are opaque to light of all wavelengths.
\item  Semi-conductors are transparent to infrared light although opaque to visible light.
\item  Most insulators are transparent to visible light.
\end{itemize}
\item (2007)  Describe briefly the formation of Newton rings. How would you measure the wavelength of yellow light by use of Newton’s rings? 
\item (2007)  What would happen to the central spot when air rests between the lens and the plate of the apparatus for Newton’s rings? 
\item (2007)  State Rayleigh’s criterion for the resolution of two objects. 
\item (2007)  The diameter of the pupil of the human eye is $ 2$ mm in bright light.\begin{itemize}
\item What is its resolving power with light of wavelength lamda $ =5 \times 10^{-7}m$ ? 
\item Would it be possible to resolve two large birds $ 30$ cm apart sitting on a wire$ 1.5 \times 10^{3}m$ away at daytime? 
\item What would the situation be at night when the pupil dilates to $ 4$ mm? 
\end{itemize}
\item (2007)  What is meant by the back e.m.f. (polarization potential) in a water voltameter? 
\item (2009)  What is interference?  Explain the term path difference with reference to the interference of two wave-trains.
\item (2009)  Why is it not possible to see interference when the light beams from head lamps of a car overlap?
\item (2009)  Discuss whether it is possible to observe an interference pattern when white light is shone on a Young’s double slit experiment.
\item (2009)  A grating has $ 500$ lines per millimetre and is illuminated normally with monochromatic light of wavelength $ 5.89 \times 10^{-7}$ m.\begin{itemize}
\item How many diffraction maxima may be observed?
\item Calculate the angular separation.
\end{itemize}
\item (2013)  What is an electron microscope? 
\item (2013)  Outline three disadvantages of electron microscope.
\item (2013)  Draw a schematic diagram of an electron microscope showing its main parts.\begin{itemize}
\item Give the order of resolution of electron microscope in the question above.
\end{itemize}
\item (2013)  What is meant by crossed polaroids? 
\item (2013)  Briefly describe the appearance of fringes produced by monochromatic fight.
\item (2013)  Give two difference between diffracting grating spectra and prism spectra.
\item (2013)  A diffraction grating used at normal incidence gives a yellow line. $ \lambda =5750$ A in a certain spectral order: superimposed on a blue line, $ \lambda =4600$ A of the next higher order, If the angle of diffraction is $ 30^{\circ}$ , what is the spacing between the grating lines? 
\item (2013)  State Huygens principle of wave construction. 
\item (2013)  A thin wedge of air of small angle ts enclosed by two thin glass plates. When the plates are illuminated by a parallel beam of monochromatic light of wavelength $ 589$ nm, the distance apart of the fringes is $ 0.8$ mm. Calculate the angle of the wedge. 
\item (2015)  What is meant by the statement that light is plane polarized.
\item (2015)  State Brewster’s law.
\item (2015)  Sunlight is reflected from a calm lake.  The reflected sunlight is totally polarized.  What is the angle between the sun and the horizon.
\item (2015)  State four conditions for sustained interference of light.
\item (2015)  In a Young’s double slit experiment the interval between the slits is $ 0.2$ mm.  For the light of wavelength $ 6.0\times 10^{-7}$ m, Find the distance of the second dark fringe from the central fringe.
\item (2015)  Distinguish between diffraction and diffraction grating.
\item (2015)  A parallel beam of the monochromatic light is incident normally on a diffraction grating.  The angle between the two first-order spectra on either side of the normal is $ 30^{\circ}$ .  Assume that the wavelength of the light is $ 5893\times 10^{14}$ m. Find the number of ruling per mm on the grating and the greatest number of bright images obtained. 
\item (2016)  The incident parallel light is a monochromatic beam of wavelength $ 450$ nm.  The two slits $ A$ and $ B$ have their centres, a distance of $ 0.3$ mm apart.  The screen is situated a distance of $ 2.0$ m from the slits.\begin{itemize}
\item Calculate the spacing between fringes observed on the screen.
\item How would you expect the pattern to change when the slits $ A$ and $ B$ are each made wider?
\end{itemize}
\item (2016)  \item (2017)  Explain the advantage of using optical fibre systems instead of coaxial cable systems in telecommunication processes.
\item (2017)  In a Young's double - slit experiment a total of $ 23$ bright fringes occupying $ 4$ total distance of $ 3.9$ mm were visible in traveling microscope, which was focused on a plane being at a distance of $ 31$ cm from the double slit. If the wavelength of light being used was $ 5.5 \times 10^{-7}$ m; determine the separation of the double slit.
\item (2017)  When a grating with $ 300$ lines per millimeters is illuminated normally with parallel beam of monochromatic light a second order principal maximum is observed at $ 18.9^{\circ}$ to the straight through direction. Find the wavelength of the light.
\item (2017)  A white light fall on a slit of width ‘a’: for what value of 'a' will be the first minimum of light falling at the angle of $ 30^{\circ}$ when the wavelength of light is $ 6500$ nm? 
\item (2018)  What do you understand by the term interference of waves?
\item (2018)  A viewing screen is separated from a double-slit source by $ 1.2$ m. The distance between the two slits is $ 0.030$ mm. The second order bright fringe $ (m=2)$ is $ 4.5$ cm from the centre line. Determine the wavelength of the light and the distance between adjacent bright fringes. 
\item (2018)  Define the term coherent sources of light. 
\item (2018)  Interference patterns are formed when using Young’s double slit experiment. Mention other three methods that can be used to form interference patterns. 
\item (2018)  A beam of monochromatic light of wavelength $ 680$ nm in air passes into glass.  Calculate: \begin{itemize}
\item The speed of light in glass
\item The frequency of light
\item The wavelength of light in glass
\end{itemize}
\item (2018)  Light of wavelength $ 644$ nm is incident on a grating with a spacing of $ 2.00 \times 10^{-6}$ m. \begin{itemize}
\item What is the angle to the normal of a second order maximum? 
\item What is the largest number of orders that can be visible? 
\item Find the angular separation between the third and fourth order image.
\end{itemize}
\item (2018)  State any four laws of photoelectric emission. 
\item (2019)  Two sheets of a Polaroid are lined up so that their polarization directions are initially parallel. When one sheet is rotated:\begin{itemize}
\item How does the transmitted light intensity vary with the angle between the polarization directions of the polaroid? 
\item What angle must the polaroid be rotated to reduce the light Intensity by $ 50\%$ ?
\end{itemize}
\item (2019)  What is meant by diffraction grating?
\item (2019)  A diffraction grating has $ 500$ lines per millimetre when used with monochromatic light of wavelength $ 6 \times 10^{-7}$ m at normal incidence. Determine the angle at which the bright diffraction images will be observed. \begin{itemize}
\item Why other orders of image above can not be observed? 
\end{itemize}
\item (2019)  State Huygens’s principle of wave construction.
\item (2019)  A lens was placed with a convex surface of radius of curvature $ 50.0$ cm in contact with the plane surface such that Newton’s rings were observed when the lens was illuminated with monochromatic light. If the radius of the $ 15$ th ring was $ 2.13$ mm determine the wavelength. 
\end{itemize}

\subsection{Doppler Effect}
\begin{itemize}
\item (1999)  What is a “Doppler Effect”?
\item (1999)  A whistle sound of frequency $ 1200$ Hz was directed to an approaching train moving at $ 48$ km$/$h​ . The whistle-man then listened to the beats between the emitted sound and that reflected from the train. What is the beat frequency detected by the whistle-man?
\item (2000)  Write two uses of Doppler effect.
\item (2000)  An observer standing by a railway track notices that the pitch of an engine whistle changes in the ratio of $ 5$:$ 4$ on passing him. What is the speed of the engine?
\item (2007)  What is meant by Doppler effect? 
\item (2007)  Mention two $ (2)$ common applications of the Doppler shift. 
\item (2007)  Ultra sound of frequency $ 5 \times 10^{6}$ Hz is incident at an angle of $ 30^{\circ}$ to the blood vessel of a patient and a doppler shift of $ 4.5$ KHz is observed. If the blood vessel has a diameter $ 10^{-3}m$ and the velocity of ultrasound is $ 1.5 \times 10^{3}$  $ m/s$ . Calculate the:\begin{itemize}
\item blood flow velocity. 
\item volume rate of blood flow. 
\end{itemize}
\item (2015)  What is meant by Doppler effect?\begin{itemize}
\item Write down three uses of Doppler effect.
\end{itemize}
\item (2015)  A whistle emitting a sound of frequency $ 440$ Hz is tied to a string of $ 1.5$ m length and rotated with an angular velocity of $ 20$ rad$/$s in the horizontal plane.  Calculate the range of frequencies heard by an observer stationed at a large distance.
\item (2015)  A police on duty detects a drop of a $ 10\%$ in the pitch of the horn of a motor car as it crosses him. Calculate the speed of the car.
\item (2016)  Ultrasound of frequency $ 4.0$ MHz is incident at an angle of $ 30^{\circ}$ to a blood vessel of diameter $ 1.6$ mm.  If a Doppler shift of $ 3.2$ kHz is observed, calculate the blood flow velocity and the volume rate of blood flow.  Assume that the speed of ultrasound is $ 1.5$ km$/$s.
\item (2016)  The absorption spectrum of a faint galaxy is measured and the wavelength of one of the lines identified as the calcium $ H$ line is found to be $ 478$ nm.  The same line has a wavelength of $ 397$ nm when measured in a laboratory. \begin{itemize}
\item Is the galaxy moving towards or away from the observer on the Earth?
\item Determine the speed of the galaxy relative to observer on the Earth.
\end{itemize}
\item (2017)  A cyclist and a railway train are approaching each other with a speed of $ 10$ m$/$s and $ 20$ m$/$s respectively. If the engine driver sounds a warning siren at a frequency of  $ 480$ Hz, calculate the frequency of the noise heard by the cyclist:\begin{itemize}
\item Before the train has passed.
\item After the tram has passed. 
\end{itemize}
\item (2019)  What is Doppler effect? 
\item (2019)  The cyclist moving at $ 10$ m$/$s and the railway train at $ 20$ m$/$s are approaching each other. If the engine driver sounds a warming siren at a frequency of $ 480$ Hz:\begin{itemize}
\item calculate the frequency of the note heard by the cyclist before and after the train has passed away. 
\end{itemize}
\end{itemize}


\section{Electrostatics}

\subsection{The Electric Field}
\begin{itemize}
\item (1998)  The distance between the electron and proton in the hydrogen atom is about $ 5.3 \times 10^{11}$ m. Calculate the electrical and gravitational forces between these particles. How do they compare?
\item (1999)  Explain why an uncharged metal is attracted by a charged one?
\item (1999)  Charges $ Q​_{1}=1.2 \times 10$ ​$ E-12$ C and $ Q​_{2}=-4 \times 10$ ​ $ -12$ C are placed $ 5.0$ m apart in air. A third charge $ Q​_{3}=1 \times 10^{-14}$ C is introduced midway between them. Find the resultant force on the third charge.
\item (2000)  Derive an expression for an electric potential at a point a distance a from a positive point charge $ Q$ .
\item (2000)  Positive charge is distributed over a solid spherical volume of radius $ R$ and the charge per unit volume is $ \sigma $\begin{itemize}
\item Show that the electric field inside the volume at a distance $ r<R$ from the centre is given by $ E=(\sigma r/3e_{0})$
\item What is the electric field at a point $ r>R$ (i.e. outside the spherical volume).
\end{itemize}
\item (2000)  A proton is placed in a uniform electric field $ E$ . What must be the magnitude and direction of the field if the electrostatic force acting on the proton $ 1$ s just to balance its weight?
\item (2000)  Give the statement of Coulomb’s law.
\item (2000)  A $ 100$ V battery terminals are connected to two large and parallel plates which are $ 2$ cm apart. The field in the region between the plates is nearly uniform. \begin{itemize}
\item If electric field intensity $ E$ is $ 10^{6}$ N C$ ^{-1}$ and points vertically upwards, determine the force of an electron in this field and compare it with the weight of an electron. An electron is released from rest from the upper plate inside the field above.
\item At what velocity will it hit the lower plate?
\item Determine its kinetic energy and the time it takes for the whole journey.
\end{itemize}
\item (2007)  Two similar balls of mass $ m$ are hung from silk thread of length "a" and carry a similar charge $ q$ .  Assume $ \theta $ is small enough that $ X = (\frac{q^2 a}{2 \pi \epsilon_0 m g})^{1/3}$\begin{itemize}
\item where $ X$ is the distance of separation.
\end{itemize}
\item (2010)  State Coulomb’s law for charged particles.
\item (2010)  Does the coulomb force that one charge exert on another charge change when a third charge is brought nearby? Explain.
\item (2010)  The electric field intensity inside a capacitor is $ E$ . What is the work done in displacing a charge $ q$ over a closed rectangular surface?
\item (2010)  Explain the following observations:\begin{itemize}
\item A dressing table mirror becomes dusty when wiped with a dry cloth on a warm day.
\item A charged metal ball comes into contact with an uncharged identical ball.  (Illustrate your answer by using diagrams).
\end{itemize}
\item (2010)  Without giving any experimental or theoretical detail explain how the results of Millikan’s experiment led to the idea that charge comes in ‘packets’, the size of the smallest packet being carried by an electron. 
\item (2013)  Describe Coulomb’s law and give the dimensions of each quantity.
\item (2015)  Two bodies $ A$ and $ B$ are $ 0.1$ m apart.  A point charge of $ 3\times 10^{-3}$ $\mu$C is placed at A and a point charge of $ 1\times 10^{-9}\mu C$ is placed at $ B$ .  $ C$ is the point on the straight line between $ A$ and $ B$ , where the electric potential is zero.  Calculate the distance between $ A$ and $ C$ .
\item (2016)  State coulomb’s law of electrostatics.
\item (2016)  Define electric field strength, $ E$ at any point.
\item (2016)  Mention two common properties of electric field lines.
\item (2016)  By using the coulomb’s law of electrostatics, derive an expression for the electric field strength $ E$ , due to a point charge if the material is surrounded by a material of permittivity $ \epsilon $ , and hence show how it relates with charge density $ \rho $ .
\item (2018)  Two point charges of equal mass $ m$ and charge $ Q$ are suspended at a common point by two threads of negligible mass and length $ L$ . If the two point charges are at equilibrium,  show that;\begin{itemize}
\item The distance of separation $ x=({Q^{2}L}/{2\pi\epsilon _{0}mg})^{1/3}$
\item The angle of inclination $ \beta = ^{3}\sqrt{(Q^{2})/(16\pi\epsilon _{0}mgL^{2})}$ 
\end{itemize}
\item (2018)  Two point charges, $ q_{A}=+3$ $\mu$C and $ q_{b}=-3$ $\mu$C, are located $ 0.2$ m apart in vacuum. Find; \begin{itemize}
\item the electric field at the midpoint of the line joining two charges. 
\item the force experienced by the negative test charge of magnitude $ 1.5 \times 10^{-9}$ C placed at this point.
\end{itemize}
\item (2019)  State Coulomb’s law of force between two electrically charged bodies. 
\end{itemize}

\subsection{Electric Potential}
\begin{itemize}
\item (1998)  Describe and explain briefly a method for measuring the specific charge. Mention the errors expected in this method.
\item (1999)  Write down an expression for the forces on an electron when moving perpendicular to: an electric field\begin{itemize}
\item Write down an expression for the forces on an electron when moving perpendicular to: a magnetic field.
\end{itemize}
\item (1999)  An electron is moving in a uniform electric field of intensity $ 1.2 \times 10^5$ Vm​ $ -1$ .  Find the acceleration of the electron.
\item (2000)  What is electric potential at a point in an electrostatic field? 
\item (2000)  A proton of mass $ 1.673 \times 10^{-27}$ kg falls through a distance of $ 1.5$ cm in a uniform electric field of magnitude $ 2.0 \times 10^{4}$ N C$ ^{-1}$ . Determine the time of fall [Neglect $ g$ and air resistance.]
\item (2007)  What is the potential at the centre of the square of side $ 1.0$ m, due to charges:\begin{itemize}
\item $ q_{1}=+1.0\times10^{-8}$ C , $ q_{2}=-2.0\times10^{-8}$ C , $ q_{3}=+3.0\times10^{-8}$ C , $ q_{1}= +2.0\times10^{-8}$ C
\item situated at the corners of the square?
\end{itemize}
\item (2007)  A charge $ Q$ is distributed over the concentric hollow spheres of radii $ r$ and $ R$ $ (R>r)$ such that the surface densities are the same.  Calculate the potential at the common centre of the two spheres.
\item (2010)  Show that the path of an electron moving In an electric field is a parabola.
\item (2013)  Define electric potential.
\item (2013)  A radioactive source in the form of metallic sphere of radius $ 1.0$ cm emits Beta particles at the rate of $ 5.0 \times 10^{10}$ particles per second.  If the source is electrically insulated, how long will it take for its electric potential to be raised by $ 2.0$ V? (assuming that $ 40\%$ of the emitted Beta-particles escape the source).
\item (2013)  A silver and copper voltammeter are connected in parallel across a $ 6$ V battery of negligible internal resistance. In half an hour $ 1.0$ g of copper and $ 2.0$ g of silver are deposited. Calculate the rate at which the energy is supplied by the battery. 
\item (2015)  Differentiate electric potential from electric potential difference.
\item (2015)  Sketch a graph of variation of electrical potential from the centre of a hollow charged conducting sphere of radius, $ r$ , up to infinity.  Explain the shape of the graph.
\item (2015)  A square ABCD has each side of $ 100$ cm.  Four points charges of $ +0.04$ $\mu$C, $ -0.05$ $\mu$C, $ +0.06$ $\mu$C, and $ +0.05$ $\mu$C are placed at $ A$ , $ B$ , $ C$ , and $ D$ respectively.  Calculate the electric potential at the centre of the square.
\item (2016)  A proton of mass $ 16.7 \times 10^{-28}$ kg falls through a distance of $ 2.5$ cm in a uniform electric field of magnitude $ 2.65 \times 10^{4}$ V$/$m.  Determine the time of fall if the air resistance and the acceleration due to gravity, $ g$ , are neglected.
\item (2017)  Define the terms capacitance and electric potential. 
\item (2018)  Why the emf of a cell is sometimes called a special terminal potential difference? 
\item (2019)  What is the potential difference between two points if $ 5$ Joules of work are required to move $ 10$ Coulombs from one point to another? 
\item (2019)  Define the terms electric potential and electric field-strength $ E$ at a point in the electrostatic field.\begin{itemize}
\item How the two quantities above related? 
\end{itemize}
\item (2019)  Can there be a potential difference between two adjacent conductors carrying the same positive charge? Give a reason. 
\end{itemize}

\subsection{Capacitance}
\begin{itemize}
\item (1998)  A girl is holding a metal rod in her hand and rubs its surface with fur. Explain what happens to the rod.
\item (1998)  Can charge be conserved? Give at least two examples to support your answer.
\item (1998)  A capacitor of capacitance $ 3$ micro$ -F$ is charged until a potential difference of $ 200$ V is developed across its plates. Another capacitor of capacitance $ 2$ micro$ -F$ developed a p.d. of $ 100$ V across its plates on being charged.\begin{itemize}
\item What is the energy stored in each capacitor?
\item The capacitors are then connected by wires of negligible resistance, so that the plates carrying like charges are connected together. What is the total energy stored in the combined capacitors?
\item What would the time constant of the circuit be, if the resistance of each wire connecting the plates was $ 10\Omega $ ?
\end{itemize}
\item (1999)  What is "capacitance"?
\item (1999)  List three factors that govern the capacitance of a parallel plate capacitor.
\item (1999)  Show that the energy per unit volume stored in a parallel plate capacitor is given by: $ U=1/2\epsilon E^{2}$ and define all the symbols in this equation.
\item (1999)  Given that the distance of separation between the parallel plates of a capacitor is $ 5$ mm, and the plates have an area of $ 5$ m$ ^{2}$ . A potential difference of $ 10$ kV is applied across the capacitor which is\begin{itemize}
\item parallel in vacuum. Compute:
\item the capacitance
\item the electric intensity in the space between the plates
\item the change in the stored energy if the separation of the plates is increased from $ 5$ mm to $ 5.5$ mm.
\end{itemize}
\item (1999)  When an impedance consisting of an inductance $ L$ and a resistance $ R$ in series is connected across a $ 12$ V, $ 50$ Hz power supply, a current of $ 0.050$ A flows, which differs in phase from that of the applied potential difference by $ 60^{\circ}$ .\begin{itemize}
\item Find the value of $ R$ and $ L$ .
\item Find the capacitance of the capacitor which, when connected in series in the above circuit, has the effect of bringing the current into phase with the applied voltage.
\end{itemize}
\item (1999)  (i) Show that the possible energy levels (in Joules) for the hydrogen atom are given by the formula:\begin{itemize}
\item $ E_{n}=-me^{4}/(8h^{2}\epsilon _{0}^{2}$ * $ 1/n^{2}$
\item where $ m=$ mass of the electron
\item $ e=$ electronic charge
\item $ h=$ Planck's constant
\item $ \epsilon _{0}=$ permittivity constant of vacuum
\item What does the negative sign signify in the formula for $ E$ , in above?
\end{itemize}
\item (2000)  Electrons in a certain television tube are accelerated through a potential difference of $ 2.0$ kV\begin{itemize}
\item Calculate the velocity acquired by the electrons.
\item If these electrons lose all their energy on impact and given that $ 10^{12}$ electrons pass per second in the TV tube, calculate the power dissipated.
\end{itemize}
\item (2000)  A coil and a capacitor in parallel are used to make a tuning circuit for a radio receiver. Sketch the resonance curve for the circuit. State two ways of changing the circuit to increase the resonant frequency.
\item (2007)  What do you understand by an electrostatic generator?
\item (2007)  The belt of a Van de Graaf generator carries a charge of $ 100$ $\mu$C per metre.  If the diameter of the lower pulley is $ 10$ cm and its angular velocity is $ 5$ rad$/$s, what p.d. will the upper conductor attain in $ 5$ minutes if its capacitance to ground is $ 5x10^{-12}$ F and if there is no leakage of charge?
\item (2010)  Describe the action of dielectric in a capacitor.
\item (2010)  A capacitor of $ 12$ $\mu$F is connected in series with a resistor of $ 0.7$ M$ \Omega $ across a $ 250$ V d.c supply. Calculate the current and p.d across the capacitor after $ 4.2$ seconds.
\item (2010)  Show that the unit of CR (time constant) is seconds and prove that for a discharging capacitor it is the time taken for the charge to fall by $ 37\%$ . 
\item (2010)  The variable radio capacitor can be charged from $ 50$ pF to $ 950$ pF by turning the dial from $ 0$ degrees to $ 180$ degrees. With the dial at $ 180$ degrees, the capacitor is connected to a $ 400$ V battery. After charging the capacitor is disconnected from the battery and the dial is turned to $ 0$ degrees. What is the charge on the capacitor? What is the p.d across the capacitor when the dial reads $ 0$ degrees and the work done required to turn the dial to $ 0$ degrees? (Neglect frictional effects).
\item (2013)  Define electric discharge and give one example.
\item (2013)  An alternating current (a.c) of $ 0.2$ A r.m.s and frequency of $ 110/2\pi$ Hz flow in a circuit containing a series arrangement of a resistor $ R$ of resistance $ 20\Omega$ , an inductor $ L$ of $ 0.15$ H and a capacitor $ C$ of capacitance $ 500$ $\mu$F . Calculate the potential difference (p.d) and the impedence of the circuit.
\item (2015)  What do you understand by dielectric constant?
\item (2015)  When are the capacitors said to be connected in parallel?
\item (2015)  The parallel plate capacitor consisting of two metal plates each of area $ 20$ cm$ ^{2}$ placed at $ 1$ cm apart are connected to the terminals of an electrostatic voltmeter.  The system is charged to give a reading of $ 120$ V on the voltmeter scale.  When the space between the plates is filled with a glass of dielectric constant of $ 5$ , the voltmeter reading falls to $ 50$ V.  What is the capacitance of the voltmeter?  You may assume that volutage recorded by a voltmeter is directly proportional to the scale reading.
\item (2015)  A $ 4.0$ $\mu$F capacitor is charged by $ 12$ V supply and is then discharged through $ 1.5M\Omega $ resistor.  \begin{itemize}
\item Obtain the time constant.
\item Calculate the charge on the capacitor at the start of the discharge.
\item What will the value of the charge on the capacitor, the potential difference across the capacitor and the current in the circuit be $ 2$ seconds after the discharge starts?
\end{itemize}
\item (2016)  A $ 25$ $\mu$F capacitor, a $ 0.10$ H inductor and a $ 25\Omega $ resistor are connected in series with an a.c. source whose e.m.f. is given by $ E=310 \sin(314t)$ .  Determine the;\begin{itemize}
\item Frequency of the e.m.f.
\item Net reactance of the circuit.
\end{itemize}
\item (2016)  Two capacitors $ C_{1}$ and $ C_{2}$ each of area $ 36$ cm$ ^{2}$ separated by $ 4$ cm have capacities of $ 6$ $\mu$C and $ 8$ $\mu$C respectively.  The capacitor $ C_{1}$ is charged to a potential difference of $ 110$ V whereas the capacitor $ C_{2}$ is charged to a potential difference of $ 140$ V.  The capacitors are now joined with plates of like charges connected together.\begin{itemize}
\item What will be the loss of energy transferred to heat in the connecting wires?
\item What will be the loss of energy per unit volume transferred to heat in the connecting wires?
\end{itemize}
\item (2016)  Define the following terms:\begin{itemize}
\item Capacitance
\item Charge density
\item Equipotential surface
\end{itemize}
\item (2016)  Identify any three factors on which the capacitance of parallel plate capacitor depends.
\item (2016)  A parallel plate capacitor is made of a paper $ 40$ mm wider and $ 3.0 \times 10^{-2}$ mm thick.  Determine the length of the paper sheet required to construct a capacitance of $ 15$ $\mu$F , if its relative permitting is $ 2.5$ .
\item (2016)  Show that the possible energy levels (in joules) for the hydrogen atom are given by the formula: $ E_{n}=-k^{2}(2\pi^{2}me^{4}/h^{2})(1/n^{2})$ .  Where $ m$ is the mass of electron, $ e$ is the electronic charge, $ h$ is the Planck’s constant, $ k=1/4\pi\epsilon _{0}$ and $ \epsilon _{0}$ is the permittivity constant of vacuum.  \begin{itemize}
\item What does the negative sign signify in the formula above?
\end{itemize}
\item (2017)  A parallel plate capacitor has plates each of area $ 0.24$ m$ ^{2}$ separated by a small distance\begin{itemize}
\item $ 0.50$ mm. If the capacitor is full charged by a battery of electromotive force of $ 24$ V, calculate:
\item the capacitance of the capacitor. 
\item the energy stared tn the capacitor. 
\end{itemize}
\item (2017)  Comment on the assertion that, the safest way of protecting yourself from lightning is to be inside a car. 
\item (2018)  A series LCR circuit with inductance, $ L=0.12H$ , capacitance, $ C=480$ nF and resistance, $ R=23\Omega $ is connected to a $ 230V$ variable frequency supply. Determine the:\begin{itemize}
\item Maximum current flowing in the circuit. 
\item Source frequency for which the current is maximum. 
\end{itemize}
\item (2018)  Briefly explain the effect of the dielectric material on the capacitance of a capacitor when the capacitor is:\begin{itemize}
\item Isolated. 
\item Connected to the battery.
\end{itemize}
\item (2018)  How are the electrolytic capacitors made? 
\item (2019)  Elaborate three significance of dielectric material in a capacitor. 
\item (2019)  Give the reason behind a loss of electrical energy when two capacitors are joined either in series or parallel. 
\item (2019)  Why does a room light turn on at once when the switch is closed? Give comment.
\item (2019)  Outside the sphere, a charged sphere behaves like its charges were concentrated at the centre. If the electric field strength inside the sphere is zero and one sphere of radius $ 5.0$ cm carries a positive charge of $ 6.7$ nC, calculate; \begin{itemize}
\item the potential at the surface of the sphere. 
\item the capacitance of the sphere. 
\end{itemize}
\item (2019)  What is meant by dielectric constant? 
\item (2019)  A parallel plate capacitor with air as a dielectric has plates of area $ 4.0 \times 10^{-2}$ m$ ^{2}$ which are $ 2.0$ mm apart. The capacitor is charged to $ 100$ V battery and connected in parallel with a similar unchanged capacitor with plates of half the area and twice the distance apart. If the edge effect is neglected, calculate the final charge on each plate. 
\item (2019)  Derive an expression for the total capacitance of two capacitors $ C_{1}$ and $ C_{2}$ connected in series. 
\item (2019)  Two capacitor of $ 15$ $\mu$F and $ 20$ $\mu$F are connected in series with a $ 600$ V supply.  Calculate the charge and Potential difference across each capacitor. 
\end{itemize}


\section{Electromagnetism}

\subsection{Magnetic Fields}
\begin{itemize}
\item (2000)  A proton is moving in a uniform magnetic field $ B$ . Draw the diagram representing $ B$ and the path of the proton if its initial direction makes an oblique angle to the direction of the field $ B$ . 
\item (2007)  Define the magnetic field intensity.
\item (2007)  A long solenoid has $ 10$ turns per cm and carries a current of $ 2.0$ A.  Calculate the magnetic field intensity at its centre.
\item (2007)  An electron having $ 450$ eV of energy enters at right angles to a uniform magnetic field of strength $ 1.50x10^{-3}$ T.  Show that the path traced by the electron in a uniform magnetic field is circular and estimate its radius.
\item (2007)  A charged oil drop of mass $ 6.0x10^{-15}$ kg falls vertically in air with a steady velocity between two long parallel vertical plates $ 5.0$ mm apart.  When a potential difference of $ 3000$ V is applied between the plates the drop falls with a steady velocity at an angle of $ 58^{\circ}$ to the vertical.\begin{itemize}
\item Determine the charge $ Q$ , on the oil drop.
\end{itemize}
\item (2007)  A coil having $ 475$ turns and cross sectional area $ 20 cm^{2}$ , rotates at $ 600r.p.m$ . in a uniform magnetic field of $ 0.01$ T. Find:\begin{itemize}
\item the peak e.m.f and the r.m.s. e.m.f induced in the coil. 
\item show these values on a graph of $ E$ vs time. 
\end{itemize}
\item (2009)  Outline four applications of eddy currents.
\item (2010)  Distinguish between magnetic flux density and magnetic induction.
\item (2010)  Describe using a sketch graph how magnetic flux density varies with the axis (both inside and at the ends) of a long solenoid carrying current. 
\item (2010)  A solenoid $ 80$ m long has a cross-sectional area of $ 16$ cm$ ^{2}$ and a total of $ 3500$ turns closely wound. If the coil is filled with air and carries a current of $ 3$ A, Calculate:\begin{itemize}
\item Magnetic field density $ B$ at the middle of the coil.
\item Magnetic flux inside the coil. 
\item Magnetic force $ H$ at the centre of the coil. 
\item Magnetic induction at the end of the coil.
\item $ (v$ ) Magnetic field intensity at the middle of the coil. 
\end{itemize}
\item (2013)  Mention the factors which determine the magnitude and direction of the force experienced by a current-carrying conductor in a magnetic field.
\item (2013)  What is the maximum torque on a $ 400-$ turns circular coil of radius $ 0.75$ cm that carrying a current of $ 1.6$ mA and resides in a uniform magnetic field of $ 0.25$ T?
\item (2013)  Brielfly explain how you can demonstrate that there are two types of charges in nature.
\item (2013)  A $ 10$ eV proton is circulating in a plane at right angles to a uniform magnetic field of magnetic flux density of $ 1.0 \times 10^{-4}$ Wb$/$m$ ^{2}$ Calculate the cyclotron frequency of a proton.
\item (2013)  A toroid of inner radius $ 25$ cm and an outer radius of $ 28$ cm has $ 4500$ turns of wound around it which passes a Current of $ 12$ A. What will be the induction of the magnetic flux;\begin{itemize}
\item Outside the toroid. 
\item inside the core of the toroid, 
\item in an empty space surrounding the toroid. 
\end{itemize}
\item (2016)  What is meant by the following terms:\begin{itemize}
\item  Phase of alternating e.m.f.
\item  Root mean square (r.m.s.) value of alternating e.m.f.
\end{itemize}
\item (2016)  State the following laws or theorems as applied in magnetism.\begin{itemize}
\item Biot-Savart law
\item Ampere’s theorem
\end{itemize}
\item (2016)  Derive an expression for the magnetic flux density $ B$ at the centre of the circular coil of radius $ r$ and $ N$ turns placed in air carrying a current i.
\item (2016)  The diameter of a $ 40$ turn circular coil is $ 16$ cm and it has a current of $ 5$ A.  Calculate:\begin{itemize}
\item The magnetic induction at the centre of the coil
\item The magnetic moment of the coil.
\item The torque action on the coil if it is suspended in a uniform magnetic field of $ 0.76$ T such that its plane is parallel to the field.
\end{itemize}
\item (2017)  Draw the diagram of the solenoid with certain number of tums placed in the magnetic field and indicate any suitable directions of the flow of current in it.
\item (2017)  Write down the formula for the magnetic field induced at the centre of solenoid. 
\item (2017)  It is desired to design a solenoid that produces a magnetic field of $ 0.1$ T at the centre. If the radius of solenoid is $ 5$ cm, its length is $ 50$ cm and carries a current of $ 10$ A; Calculate:\begin{itemize}
\item The number of turns per unit length of the solenoid. 
\item The total length of a wire required. 
\end{itemize}
\item (2017)  State the Biot-Savart law. 
\item (2017)  In a hydrogen atom, an electron keeps moving around its nucleus with a constant speed of $ 2.18 \times 10^{6}$ m$/$s. Assuming that the orbit is a circular of radius $ 5.3 \times 10^{-11}$ m. determine the magnetic flux density produced at the site of the proton in the nucleus. 
\item (2018)  A circular coil of $ 300$ turns has a radius of $ 10$ cm and carries a current of $ 7.5$ A. Calculate the magnetic field at:\begin{itemize}
\item the centre of the coil. 
\item a point which is at a distance of $ 5$ cm from the centre of the coil. 
\end{itemize}
\item (2019)  Identify four factors that affect the force experienced by a current-carrying conductor in a magnetic field. 
\item (2019)  Write the mathematical expression which define magnetic flux density and use it to deduce its S.I. units. \begin{itemize}
\item Apply an expression obtained above to develop the formula for the force on a conductor carrying current i if the conductor and the magnetic fields are not at night angles.
\end{itemize}
\item (2019)  State the condition which makes the magnetic force on a moving charge in a magnetic field to be maximum. 
\item (2019)  \end{itemize}

\subsection{Magnetic Properties of Materials}
\begin{itemize}
\item (1999)  With the help of clear diagrams, explain briefly how you would convert a sensitive galvanometer into:\begin{itemize}
\item an ammeter
\item a voltmeter
\end{itemize}
\item (2007)  List three $ (3)$ classes of magnetic materials on the basis of magnetic susceptibility and give one example for each class.
\item (2007)  How are the magnetic susceptibility and relative permeability of a magnetic material related to each other?
\item (2007)  State the main differences between.\begin{itemize}
\item diamagnetism and paramagnetism. 
\item ferromagnetism and auntiferromagnetism. 
\item ferromagnetism and ferrielectricity. 
\end{itemize}
\item (2007)  Draw hysteresis loops diagrams for soft iron and hard steel and use them to discuss:\begin{itemize}
\item permanent magnets.
\item electromagnets.
\item transformer cores. 
\end{itemize}
\item (2016)  Draw hysteresis loops diagram for soft iron and hard steel and use them to discuss permanent magnets.
\item (2016)  Define permeability constant.
\item (2018)  Mention the three magnetic materials and briefly explain each one. \begin{itemize}
\item Give the differences between the magnetic materials mentioned above in terms of their magnetic susceptibility. 
\end{itemize}
\item (2018)  Define the following terms:\begin{itemize}
\item Ampere 
\item Hysteresis 
\end{itemize}
\item (2019)  Distinguish the terms magnetically soft and magnetically hard materials.
\end{itemize}

\subsection{Magnetic Forces}
\begin{itemize}
\item (1998)  An electron is projected horizontally with a velocity of $ 2.0 \times 10^{6}$ ms$ ^{-1}$ into a large evacuated enclosure. A magnetic field which has a flux density of $ 15 \times 10^{-4}$ tesla is directed vertically downwards throughout the enclosure. Find\begin{itemize}
\item the radius of curvature of the electron's path.
\item how many complete loops must the electron describe before it falls by $ 1.0$ cm under the influence of gravity?
\item What would be the effect of changing the direction of the magnetic field to upwards?
\end{itemize}
\item (2000)  An electron with charge $ e$ and mass $ m_{e}$ is initially projected with a speed v at right angles to a uniform magnetic field of flux density $ B$ .\begin{itemize}
\item Explain why the path of the electron $ 1$ s circular.
\item Show also that the time to describe one complete circle is independent of the speed of the electron.
\end{itemize}
\item (2000)  Calculate the radius of the path traversed by an electron of energy $ 450$ eV moving at right angles to a uniform magnetic field of flux density $ 1.5\times 10^{-3}$ T.
\item (2009)  Develop an equation for the torque acting on a current carrying coil of dimensions lxb placed in a magnetic field.  How is this effect applied in a moving coil galvanometer?
\item (2009)  A galvanometer coil has $ 50$ turns, each with an area of $ 1.0 $ cm$ ^{2}$ .  If the coil is in a radian field of $ 10^{-2}$ T and suspended by a suspension of torsion constant $ 2 \times 10^{-9}$ Nm per degree, what current is needed to give a deflection of $ 30^{\circ}$ ?
\item (2009)  Give a general form expressing the force exerted on the wire carrying current i if its length $ l$ is inclined at angle angle $ \theta $ to the magnetic field $ B$ .  
\item (2009)  A wire carrying a current of $ 2$ A has a length of $ 100$ mm in a uniform magnetic field of $ 0.8$ Wb$/$m$ ^{2}$ .  Find the force acting on the wire when the field is at $ 60^{\circ}$ to the wire.
\item (2009)  A wire carrying a current of $ 25$ A and $ 8$ m long is placed in a magnetic field of flux density $ 0.42$ T . What is the force on the wire if it is placed:\begin{itemize}
\item At right angles to the field?
\item At $ 45^{\circ}$ to the field?
\item Along the field?
\end{itemize}
\item (2013)  Derive the formula for the torque acting of the rectangular current-carrying coil in a magnetic field
\item (2013)  Give comment on the statement that, an electron suffers no force when it moves parallel to the magnetic field, $ B$ .
\item (2015)  A horizontal straight wire $ 0.05$ m long weighing $ 2.4$ g$/$m is placed perpendicular to a uniform horizontal magnetic field of flux density $ 0.8$ T.  If the resistance of the wire is $ 7.6\Omega /$m, calculate the potential difference that has to be applied between the ends of the wire to make it just self-supporting.
\item (2015)  Two very long wires made of copper and of equal lengths are placed parallel to each other in such a way that they are $ 10$ cm apart.  If the total power dissipated in the two wires is $ 75$ W, find the force between them if the resistivity of the copper wire is $ 1.69	imes 10^{-8}\Omega m$ and of diameter $ 2$ mm.
\item (2017)  State the law of force acting on a conductor of length $ l$ carrying an electric current in a magnetic field. 
\item (2019)  Determine the magnitude of force experienced by a stationary charge in a uniform magnetic field. 
\end{itemize}

\subsection{Electromagnetic Induction}
\begin{itemize}
\item (1998)  Define the term self inductance for a coil.
\item (1998)  Give the S.I units of self inductance.
\item (1998)  Derive an expression for the coefficient of self induction of a uniformly wound solenoid; of length $ 1$ , cross-sectional area A having $ N$ turns in air.
\item (1998)  Two coils $ A$ and $ B$ have $ 200$ and $ 800$ turns respectively. A current of $ 2$ amperes in A produces a magnetic flux of $ 1.8 \times 10^{-4}$ Wb in each turn of $ B$ . Compute:\begin{itemize}
\item the mutual inductance.
\item the magnetic flux through A when there is a current of $ 4.0$ amperes in $ B$ and
\item the emf induced in $ B$ when the current in A changes from $ 3$ amperes to $ 1$ ampere in $ 0.2$ seconds.
\end{itemize}
\item (1999)  State the laws of electromagnetic induction and describe briefly experiments (one in each case) which can be used to demonstrate them.
\item (2007)  State Faraday’s two $ (2)$ laws of electrolysis and calculate the value of Faradays constant given that the e.c.e. of copper is $ 3.30 \times 10^{-7}$ kg$/C$ and the copper is a divalent element. 
\item (2007)  A piece of metal weighing $ 200g$ is to be electroplated with $ 5\%$ of its weight in gold. If the strength of the available current is $ 2$ A, how long would it take to deposit the required amount of gold?
\item (2007)  State Faraday’s law of electromagnetic induction. 
\item (2007)  A coil of cross section area A rotates with an angular velocity $ \omega $ in a uniform. magnetic field, $ B$ . Derive the equation for induced e.m.f. of the system.
\item (2009)  State the laws of electromagnetic induction.
\item (2009)  A coil of $ 100$ turns is rotated at $ 1500$ revolutions per minute in a magnetic field of uniform density $ 0.05$ T.  If the axis of rotation is at right angles to the direction of the flux and the area per turn is $ 4000 $ mm$ ^{2}$ .  Calculate the:\begin{itemize}
\item Frequency
\item Period
\item Maximum induced e.m.f.
\item Maximum value of the induced e.m.f. when the coil has rotated through $ 30^{\circ}$ from the position of zero e.m.f.
\end{itemize}
\item (2013)  State the laws of electromagnetic induction.
\item (2013)  State Lenz’s Jaw of electromagnetic induction.
\item (2015)  Distinguish between self-inductance and mutual inductance.
\item (2018)  Consider a small flat coil which has $ N$ turns of area A and whose plane is perpendicular to a magnetic field of flux density $ B$ . If the search coil is connected to the ballistic galvanometer and the total resistance of the circuit is $ R$ , use the laws of electromagnetic induction to show that the charge delivered to the galvanometer does not depend on how long it takes to remove the search coil from the field. 
\item (2019)  At which position of the rotating coil in the magnetic field, the induced e.m.f. is zero? Give a reason. 
\end{itemize}

\subsection{Magnetic Field of the Earth}
\begin{itemize}
\item (1999)  A flat coil of $ 100$ turns and mean radius $ 5.0$ cm is tying on a horizontal surface and is turned over in $ 0.20$ sec. against the vertical component of the Earth's magnetic field. Calculate the average e.m.f. induced.
\item (2007)  Write short notes on the following terms in relation to changes in the Earth's magnetic field:  long-term (secular) changes, short-period (regular) changes and short-term (irregular) changes.
\item (2013)  An aircraft is flying horizontally at $ 200$ m$/$s through the region where the vertical component of the earth magnetic field is $ 4.0 \times 10^{-5}$ T. If the air craft has a wing span of $ 40$ m, what will be the potential difference (p.d.) produced between the wing tips? 
\item (2015)  List down three sources of earth's magnetism. 
\item (2016)  State any three magnetic components of the earth’s magnetic field.
\item (2016)  The horizontal and vertical components of the Earth’s magnetic field at a certain location are $ 2.7 \times 10^{-5}$ T and $ 2.0 \times 10^{-5}$ T respectively.  Determine the Earth’s magnetic field at the location and its angle of inclination i.
\end{itemize}


\section{Current Electricity}

\subsection{Electric Conduction in Metals}
\begin{itemize}
\item (1999)  State Kirchhoff’s laws of circuit analysis
\item (2000)  State Kirchhoff’s laws of electric circuits.
\item (2000)  What do you understand by the term “drift velocity” as applied to any current carriers in a wire?
\item (2000)  Determine the drift velocity of electrons in a silver wire of a cross—sectional area $ 4.5 \times 10^{-6}$ m$ ^{2}$ when a current of $ 15$ A flows through it. Given: The density of silver $ =1.05 \times 10^{4}$ kg$/$m$ ^{3}$ . The atomic weight of silver $ =108$ .
\item (2000)  An unknown wire of $ 1$ mm diameter is found to carry and passes a total charge of $ 90$ C in $ 1$ hour and $ 15$ min. If the wire has $ 5.8 \times 10^{28}$ free electrons per $ m^{3}$ , find\begin{itemize}
\item  the current in the wire.
\item the drift velocity of the electrons in m s$ ^{-1}$
\end{itemize}
\item (2000)  The current of $ 12$ A is made to pass through an aluminium wire of radius $ 1.5$ mm which is joined in series with a copper wire of radius $ 0.8$ mm. Determine.\begin{itemize}
\item the current density in an aluminium wire.
\item the drift velocity of the electron tn the copper wire, given that the number of free electrons per unit volume in a copper wire is $ 10^{29}$ .
\end{itemize}
\item (2007)  Define the internal resistance (r) of a cell and the terminal potential difference.
\item (2007)  The e.m.f. of a cell is a special terminal potential difference.  Comment.
\item (2007)  State Kirchhoff's laws of electrical network.
\item (2007)  Discuss two $ (2)$ harmful effects of electrolysis. 
\item (2009)  Explain the mechanism of electric conduction in:\begin{itemize}
\item Electrolytes
\end{itemize}
\item (2010)  Define the temperature coefficient of resistance
\item (2013)  What is meant by “power rating" as regards to a resistor?\begin{itemize}
\item Mention two distinct velocities of an electron in a wire.
\end{itemize}
\item (2013)  Explain why it is better to use a small current for a long time to plate a metal with a given thickness of silver than using a larger current for a short time? 
\item (2013)  Give four difference between the passage of electricity through metals and  ionized solution.
\item (2014)  Define the following terms:\begin{itemize}
\item Current density
\item Conductivity 
\end{itemize}
\item (2014)  Under what condition is $ \Omega $ ’s law true?
\item (2014)  Why does the voltage across the terminals of a cell or battery fall when it is delivering a current? 
\item (2014)  Define temperature coefficient of resistance.\begin{itemize}
\item A heating coil of Nichrome wire with cross sectional area of $ 0.1 $ mm$ ^{2}$ operates on a $ 12$ V supply, and has a power of $ 36$ W when immersed in water at $ 373$ K. Calculate the length of the wire.
\end{itemize}
\item (2015)  What is meant by the following terms:\begin{itemize}
\item  Internal resistance of a cell. 
\item  Drift velocity. 
\end{itemize}
\item (2015)  What is a potentiometer. \begin{itemize}
\item Mention two advantages and two disadvantages of potentiometer.
\end{itemize}
\item (2015)  Distinguish between ohmic and non-ohmic conductor. Give one example in each
\item (2016)  What ts the physical significance of Kirchhoff’s first law.
\item (2016)  Why is Kirchhoff’s second law sometimes referred to as the voltage law?
\item (2016)  List down five points to be considered when applying Kirchhoff’s second law in formulating analytical problems or equations.
\item (2017)  What is the advantage of using a greater length of potentiometer wire?
\item (2017)  Why is Wheatstone bridge not suitable for measuring very high resistance?
\item (2017)  List two factors on which the resistivity of a material depends. 
\item (2017)  A wire of resistivity, $ \rho $ , is stretched to double its length. What will be its new resistivity? Give reason for your answer. 
\item (2017)  Why a high voltage supply should have high internal resistance?
\item (2017)  Justify the statement that ‘it is not possible to verify Ohm's law by using a filament lamp’.
\item (2017)  A potential difference of $ 4$ V is connected to $ 4$ uniform resistance wire of length $ 3.0$ m and cross-sectional area $ 9\times 10^{-9}$ , when a current of $ 0.2$ A is flowing in the wire. Find the:\begin{itemize}
\item Resistivity of the wire.
\item Conductivity of the wire. 
\end{itemize}
\item (2018)  Outline three important points which are usually referred as sign convection in  solving Kirchhoff’s second law problems. 
\item (2018)  How is ohmic conductor differ from non-ohmic conductor? Give one example in each case. 
\item (2018)  State a condition that could be employed to make an insulator conduct some electricity. 
\item (2018)  What is meant by the term Ballistic galvanometer? 
\item (2018)  State two conditions to be fulfilled for a galvanometer to be used as a ballistic galvanometer. 
\item (2019)  A researcher has $ 2$ g of gold and wishes to form it into a wire having a resistance of $ 80\Omega $ at $ 0^{\circ}$C . How long should the wire be? 
\end{itemize}

\subsection{Electric Conduction in Gases}
\begin{itemize}
\item (1998)  What is thermionic emission?
\item (2013)  Explain the following observation:\begin{itemize}
\item Light in the bulb comes on once the switch is kept on despite the drift velocity of electrons being very low.
\item The potentiometer is said to be a better device for measuring the potential difference (p.d) than a moving coil voltmeter.
\end{itemize}
\item (2013)  A milliameter connected in series with a hydrogen discharge tube indicates a current of $ 1.0 \times 10^{-3}$ A. If the number of electrons passing the cross section of the tube at a particular point is $ 4.0 \times 10^{15}$ per second, find the number of protons that pass the same cross section per second. 
\item (2015)  Sketch the diagram showing the variation of current with potential difference across the following:\begin{itemize}
\item  Filament electric bulb. 
\item Gas-filled diode. 
\end{itemize}
\item (2018)  Distinguish between ionization energy and excitation energy.
\end{itemize}

\subsection{Alternating Current (ac)}
\begin{itemize}
\item (1999)  What is a resonant frequency of an oscillator?
\item (1999)  An inductance of $ 4$ mH is connected in series with a resistance of $ 20\Omega $ together with a battery:\begin{itemize}
\item Determine how the current will vary with time in this circuit.
\item Sketch the current of above against time
\item Calculate the inductive time constant
\end{itemize}
\item (2000)  What is meant by the terms electrical resistivity and ohmic conductor.
\item (2000)  A $ 4$ m long resistance wire has a cross-sectional area of $ 0.8$ mm? and has a resistance of $ 2.80\Omega $ .  Determine:\begin{itemize}
\item The resistivity of the wire.
\item The length of a similar wire which when joined in parallel will give a total resistance of $ 2.0\Omega $ .
\end{itemize}
\item (2000)  Two cells of emf $ 1.5$ V and $ 2.0$ V and internal resistances of $ 1\Omega $ and $ 2.0\Omega $ respectively are connected in parallel and across them an external resistance of $ 5.0$ Q. Calculate the currents in each of the three branches of the network. 
\item (2000)  What is a rectifier?
\item (2007)  An a.c. generator consists of a coil of $ 50$ turns and an area of $ 2.5$ m$ ^{2}$ , rotates at an angular speed of $ 60$ rad$/$s in a uniform magnetic field of $ 0.30$ T between two fixed pole pieces.  The resistance of the circuit including that of the coil is $ 500\Omega $ .  \begin{itemize}
\item  What is the maximum current that can be drawn from the generator?
\item  What is the magnetic flux through the coil if the current is maximum?
\end{itemize}
\item (2013)  A $ 20$ k$ \Omega$ resistor is to be connected across a potential difference of $ 300$ V Calculate the required power rating.
\item (2013)  Derive an expression for impedance of a series $ R-C$ circuit. 
\item (2013)  Write down two advantages of digital circuits over the analogue circuits.
\item (2014)  What is meant by the following terms:\begin{itemize}
\item Alternating current (a.c.)
\item Effective value of A.C. 
\end{itemize}
\item (2014)  A $ 60$ V, $ 10$ W lamp is to be run on $ 100$ V, $ 60$ Hz A.C mains.\begin{itemize}
\item Calculate the inductance of a choke coil required.
\item If a resistor is used in above instead of choke, what will be value of its resistance.
\end{itemize}
\item (2014)  An LCR circuit with $ R=70\Omega$ in series with a parallel combination of $ L=1.5$ H and\begin{itemize}
\item $ C=30$ $\mu$F is driven by a $ 230$ V supply with angular frequency of $ 300$ rad$/$s.
\item $ (1)$ Find the power in put to the circuit. 
\item  At the frequency $ \omega_{o}=1/(\sqrt{LC})$ , how does the circuit respond?
\end{itemize}
\item (2015)  Explain the statement that, a sinusoidal current, of peak value $ 5$ A passed through an a.c. ammeter reads $ 5/\sqrt{2}$ A.  
\item (2015)  Show that the average power transferred to an a.c. circuit is, in general, given by $ EIR/Z$ , where $ R$ is the resistance in the circuit defined to be the real part of complex impedance and $ Z$ is its impedance.
\item (2015)  A coil which has an inductance of $ 0.2$ H and negligible resistance is in series in a resistor, whose resistance is $ 60\Omega $ . The pair is connected across a $ 50$ V supply alternating at $ 100/\pi$ Hz.  Calculate the toal impedance of the circuit and its power factor.
\item (2016)  An a.c. circuit consists of a pure resistance of $ 10\Omega $ is connected across an a.c. supply of $ 230$ V , $ 50$ Hz.  Calculate the;\begin{itemize}
\item Current flowing in the circuit.
\item Power dissipated
\end{itemize}
\item (2016)  An X-ray tube, operated at a d.c. potential difference of $ 60$ kV , produces heat at the target at the rate of $ 840$ W .  Assuming $ 0.65\%$ of the energy of the incident electrons is converted into X-radiation, calculate:\begin{itemize}
\item The number of electrons per second striking the target.
\item The velocity of the incident electrons.
\item The energy of incident electrons
\end{itemize}
\item (2018)  Calculate the current flowing in the circuit when three similar cells each of emf $ 1.5$ V and internal resistance $ 0.3\Omega $ are connected in parallel across a $ 2\Omega $ resistor. 
\item (2018)  Why choke coil is preferred over resistance to control alternating current?
\item (2018)  Explain what could be done to light a $ 30$ V bulb from a $ 220$ volt A.C. supply?
\item (2019)  A current of $ 3.0$ mA flows in a Television resistor $ R$ when a potential difference of $ 6.0$ V is connected across its terminals. Determine the value of conductance.
\end{itemize}


\section{Electronics}

\subsection{The Band Theory of Solids}
\begin{itemize}
\item (2013)  What is band theory?
\item (2013)  How does the band theory explain electrical properties of solids?
\item (2013)  In an intrinsic semiconductor, the energy gap $ E_{g}=1.2$ eV, and its hole mobility is very much smaller than electron mobility which is Independent of temperature. Assuming that the temperature dependence of intrinsic carrier concentration, $ n_{i}$ is expressed as:\begin{itemize}
\item $ N_{i}=n_{o}$ exp$ (-E_{g}/(K_{B}T))$ , where $ n_{o}$ and $ K_{B}$ are constants, $ T$ is temperature and $ E_{g}$ is an energy equal to $ E_{q}/2$ .  
\item What is the ratio between conductivity at $ 600$ K and that at $ 300$ K?
\item Comment on the result obtained above.
\end{itemize}
\item (2017)  How does the forbidden energy gap of an intrinsic semiconductor vary with increase in temperature? 
\end{itemize}

\subsection{Semiconductors}
\begin{itemize}
\item (1998)  Describe the function of each of;\begin{itemize}
\item the electron gun
\item the deflection system and
\item the display system of the Cathode ray Oscilloscope.
\end{itemize}
\item (1999)  Distinguish between insulators, semi-conductors and metals as far as conduction is concerned.
\item (2000)  Distinguish between metals and semiconductors in terms of energy bands. 
\item (2007)  How does the arrangement of the energy level in a semiconductor differ from that of an insulator?
\item (2013)  Mention one application of LED. What type of a semiconductor is it?
\item (2014)  What is light emitting diode (LED).
\item (2014)  Give three advantages of LED's lamp in radio and other electronic system over filament lamps.
\item (2014)  What is the basic difference between good conductors and semiconductors.
\item (2015)  Mention four important properties of a semiconductor.
\item (2015)  Applying the concept of doping, explain how a free electron and a positive charge can be created in a semiconductor crystal. 
\item (2016)  What is the importance of doping as applied to semiconductors?
\item (2016)  Distinguish between $ n-$ type and $ p-$ type semiconductors.  Give three points.
\item (2016)  Discuss the mode of action of each of the following sensors:\begin{itemize}
\item Thermistor (TH).
\item Light Dependent Resistor (LDR).
\end{itemize}
\item (2017)  Define the term semiconductor.\begin{itemize}
\item Give three examples of semiconductor materials. 
\end{itemize}
\item (2017)  Outline two factors on which electrical conductivity of a pure semiconductor depends. 
\item (2017)  Explain the meaning of the following terms:\begin{itemize}
\item $ P-$ type semiconductor.
\item $ N-$ type semiconductor. 
\end{itemize}
\item (2018)  List two chief properties of semiconductors. 
\item (2018)  Why is it easier to establish the current in a semiconductor than in an insulator?
\item (2018)  Distinguish between conductors and semiconductors on the basis of their energy band structures. 
\end{itemize}

\subsection{Transistors}
\begin{itemize}
\item (1999)  Draw the symbol of $ n-p-n$ transistor.
\item (1999)  With the help of illustrative diagrams explain the action of a choke in a circuit.
\item (1999)  Explain the term “thermal run away” as regards a transistor amplifier.
\item (2000)  Briefly discuss the formation of the potential difference barrier (depletion layer) of a $ p-n$ junction diode.
\item (2000)  Using $ p-n$ junction diodes, draw the arrangement of a full-wave rectifier and briefly explain how it works.\begin{itemize}
\item Define the electron – volt.
\end{itemize}
\item (2000)  Mention any three uses of a transistor
\item (2000)  A certain transistor has a current gain $  \beta =55$ . If it is used in a circuit with common-base configuration, how much change occurs in the collector current if an emitter current is changed by $ 100$ micro A? (Assume the collector potential to be constant and neglect the small collector — current due to the minority current carriers).
\item (2010)  Briefly explain why a $ P-N$ junction is referred as a junction diode.
\item (2013)  What is meant by transistor action?
\item (2013)  Briefly explain why the collector of a transistor is made wider than the emitter and base?
\item (2013)  Derive the closed – loop gain A of an inverting amplifier.  If the input resistor is equal to the feedback resistor, what would be the value of the gain A?
\item (2014)  Mention two types of transistors.\begin{itemize}
\item Which among the transistors mentioned above responds quickly to electrical signal? Give reason for your answer.
\end{itemize}
\item (2015)  A wire of diameter $ 0.1$ mm and resistivity $ 1.69\times10^{-8}\Omega$ m with temperature coefficient\begin{itemize}
\item of resistance of $ 4.3\times10^{-3}$ K$ ^{-1}$ was required to make a resistance,
\item  What length of the wire is required to make a coil with a resistance of $ 0.5\Omega $ ?
\item If on passing a Current of $ 2$ A the temperature of the coil above rises  by $ 10^{\circ}$C, what error would arise in taking the potential drop as $ 1.0$ V 
\end{itemize}
\item (2015)  Why a $ p-n$ junction diode when connected in a circuit and then reversed gives a very small leakage current across the junction? \begin{itemize}
\item How is the size of the current stated in above dependent on the temperature of the diode?
\end{itemize}
\item (2015)  Define closed loop gain. 
\item (2016)  Define the term junction as applied in electrical network.
\item (2016)  Why are transistors mostly used in common emitter arrangement?
\item (2016)  When does a transistor amplifier work as an oscillator?
\item (2017)  List three types of transistor configurations.
\item (2017)  Why is collector of a transistor made wider than emitter and base? 
\item (2018)  What do you understand by the term node as applied to electric circuit?
\item (2018)  Mention four types of energy losses suffered by a transformer.  
\item (2018)  What is meant by depletion layer as used in pn -junction device? 
\item (2018)  Describe the effect of applying a reverse bias to the junction diode. 
\item (2018)  Sketch the graph of transfer characteristic of a transistor. \begin{itemize}
\item State the significance of the slope from the graph above.
\end{itemize}
\item (2018)  What is the basic condition for a transistor to operate properly as an amplifier? 
\item (2018)  Briefly explain how a junction transistor can be connected to act as a current operated device. 
\item (2018)  Why the magnitude of output frequency of a full wave rectifier is twice the input frequency? 
\item (2018)  Draw a simple basic transistor switching circuit diagram. 
\item (2019)  Why transistors can not be used as rectifiers? 
\item (2019)  In NPN transistor circuit the collector current is $ 5$ mA. If $ 95\%$ of the emitted electrons reach the collector region, calculate the base current. 
\item (2019)  What causes damage to transistors? 
\end{itemize}

\subsection{Logic Gates}
\begin{itemize}
\item (2009)  Define the following:\begin{itemize}
\item Logic gate.
\item Integrated circuit.
\item Modulation.
\end{itemize}
\item (2014)  What is meant by the following electronic circuits:\begin{itemize}
\item Logic gates 
\item Integrated circuits
\end{itemize}
\item (2016)  Give symbols, expressions and truth tables for each of the following logic gates: \begin{itemize}
\item NAND gate .
\item Exclusive NOR gate.
\end{itemize}
\item (2016)  Why is NAND gate considered as basic building block for a variety of logic circuits?
\item (2018)  What is meant by a logic gate? 
\item (2018)  List three basic logic gates that make up all digital circuits. 
\end{itemize}

\subsection{Operational Amplifiers}
\begin{itemize}
\item (1998)  Sketch the traces seen on the screen of a cathode ray oscilloscope when two sinusoidal potential differences of the same frequency — and amplitude are applied simultaneously to $ X$ and $ Y$ plates of  a cathode ray oscilloscope, when the phase difference between them is:\begin{itemize}
\item $ 0^{\circ}$ $ 45^{\circ}$ $ 90^{\circ}$ .
\end{itemize}
\item (1999)  Briefly describe the major factors that you would consider when designing a voltage amplifier.
\item (1999)  With the help of clear diagrams, explain how you would overcome thermal run away in a voltage amplifier.
\item (2000)  Mention any three uses of a CRO.
\item (2000)  What is an operational amplifier 
\item (2000)  List three desirable features of an operational amplifier.
\item (2000)  In almost all cases, where an operation amplifier is used as a linear voltage amplifier, negative feedback is employed. State the advantage of negative feedback.
\item (2007)  Make well labelled diagram of the cathode ray oscilloscope and explain briefly how a sinusoidal voltage signal is displayed on its screen.
\item (2007)  Mention three $ (3)$ practical applications of the cathode ray oscilloscope.
\item (2007)  Explain the terms output saturation and negative feedback as applied to op-amplifiers. 
\item (2007)  For an ideal operational amplifier, what are the values of the:\begin{itemize}
\item current into both inputs of the op-amp? 
\item voltage between the inputs if the output is not saturated? 
\end{itemize}
\item (2007)  What is a non-inverting amplifier? 
\item (2009)  Explain the following terms:\begin{itemize}
\item Forward bias.
\item Reverse bias.
\item Inverting and non-inverting amplifier. 
\end{itemize}
\item (2009)  An operational amplifier is to have a voltage gain of $ 100$ .  Calculate the required values for the external resistances $ R_{1}$ and $ R_{2}$ when the following gains are required:\begin{itemize}
\item non-inverting.
\item Inverting.
\end{itemize}
\item (2013)  Briefly explain why Cathode Ray Oscilloscope (C.R.O.) is said to be an excellent instrument for measuring the emf 
\item (2013)  Draw a well labeled circuit diagram of an inverting amplifier.
\item (2014)  What is the purpose of amplifiers in a phone link? 
\item (2015)  List three properties of operational amplifiers.
\item (2015)  What is meant by the term negative feedback? Give four advantages of using it in an op-amp or any type of voltage amplifier.
\item (2015)  Derive an expression of the closed loop gain for an inverting op-amp voltage amplifier with an input resistor $ R$ , and a feedback resistor.
\item (2016)  Explain the use of an op-amp as a summing amplifier.
\item (2016)  Name three electronic circuits in which multivibrators can be constructed.\begin{itemize}
\item List down three types of multivibrators.
\item Briefly explain the applications of multivibrators listed above.
\end{itemize}
\item (2016)  Mention two characteristics of op-amps.
\item (2016)  Briefly explain why op-amps are sometimes called differential amplifiers?
\item (2016)  Describe the structure and the mode of action of a simplified version of the Van de Graaff generator.
\item (2017)  Briefly explain the function of the following:\begin{itemize}
\item Oscilloscope
\item Op-amps
\end{itemize}
\item (2017)  A change of $ 100$ A in the base current produces a change of $ 3$ mA in the collector current. Calculate:\begin{itemize}
\item The current amplification factor, $ \beta$
\item The current gain, $ \alpha $
\end{itemize}
\item (2019)  Distinguish between inverting OP-AMP and non-inverting OP-AMP. \begin{itemize}
\item Give one application of each type of OP-AMP described above.
\end{itemize}
\end{itemize}

\subsection{Telecommunication}
\begin{itemize}
\item (1998)  Give the reason for better reception of radio waves for high Frequency signals at night than during the day time.
\item (1998)  \item (2000)  Explain why Audio amplification is necessary for a practical radio set.
\item (2013)  An electron gun fires electrons at the screen of a TV tube. The electrons start from rest and are accelerated through a potential difference of $ 30$ kV. What is the speed of impact of electrons on the screen of the picture tube?
\item (2013)  Briefly explain why long distance radio broadcasts make use of short wave bands.
\item (2014)  Give the meaning of the following terms:\begin{itemize}
\item Bandwidth
\item  Amplitude modulated carrier wave
\end{itemize}
\item (2014)  Sketch the frequency spectrum for $ 1500$ m radio waves modulated by $ 4$ kHz audio signal.
\item (2014)  List down two advantages of digital signals over analogue signals.
\item (2014)  A carrier of frequency $ 800$ kHz is amplitude modulated by frequencies ranging from $ 1$ kHz to $ 10$ kHz.  What frequency range does each sideband cover?
\item (2015)  Give one advantage of frequency modulation (FM) as compared to amplitude modulation ( AMT).
\item (2015)  Briefly explain the importance of bandwidth of an amplitude modulation (AM) signal.
\item (2015)  State the function of a modulator in radios.
\item (2015)  Sketch a block diagram to show the general plan of any communication system.
\item (2015)  The amplitude modulated (AM) broadcast band ranges from $ 450$ to $ 1200$ kHz. If each station modulates with audio frequencies up to $ 5.5$ kHz, determine the\begin{itemize}
\item  Bandwidth needed for each station.
\item  Total bandwidth available. 
\end{itemize}
\item (2017)  List three basic elements of communication system. 
\item (2018)  Identify two difficulties which would arise when two straight wires are used to transmit electricity direct from the source to the city station. 
\item (2019)  Identify three basic elements of a communication system. 
\item (2019)  Why sky waves are not used for transmission of TV signals? 
\end{itemize}


\section{Atomic Physics}

\subsection{Structure of the Atom}
\begin{itemize}
\item (1999)  State Bohr’s postulates of the atomic model.
\item (1999)  Show that for an electron in a hydrogen atom, the possible radii of an electron orbit are given by:\begin{itemize}
\item $ r_{n}=a_{0}n^{2}$ , $ n=1$ , $ 2$ , $ 3$ , ...
\end{itemize}
\item (2000)  In the Bohr model of the hydrogen atom, an electron circles the nucleus in an orbit of radius $ r$\begin{itemize}
\item Explain what keeps the electron in the orbit and why it does not spiral towards the nucleus.
\item What are the assumptions put forward by Bohr about the orbits of the electron in the hydrogen atom?
\end{itemize}
\item (2007)  Develop an expression for electrical energy spent in the decomposition of water. 
\item (2007)  In a hydrogen atom model an electron of mass $ m$ and charge $ e$ rotates about a heavy nucleus of charge $ e$ in a circular orbit of radius $ r$ . Develop an expression for the angular momentum of the electron in terms of $ m$ , $ e$ , $ r$ , $ \pi$ and $ \epsilon  _{0}-$ the permitting of free space.
\item (2007)  The four lowest energy levels in a mercury atom are $ -10.4$ eV, $ -5.5$ eV, $ -3.7$ eV and $ -1.6$ eV.\begin{itemize}
\item Determine the ionization energy of mercury in joules. 
\item Calculate the wavelength of the radiation emitted when an electron jumps from $ -1.6$ eV to $ -5.5$ eV energy levels. 
\item What will happen if a mercury atom in its excited state is bombarded with electrons having an energy of $ 11$ eV. 
\end{itemize}
\item (2013)  Given that Rydberg’s constant is approximately $ 1.1 \times 10^{7}$ m$ ^{-1}$ Calculate the corresponding range of frequency for emitted radiation in the:\begin{itemize}
\item Lyman series. 
\item Balmer series. 
\end{itemize}
\item (2015)  Why are the energy levels labelled with negative energies?
\item (2016)  The first member of the Balmer series of hydrogen spectrum has wavelength of $ 6563 \times 10^{-10}$ m. Calculate the wavelength of its second member.
\item (2017)  Use the Rydberg constant, $ R_{H}=1.0974 \times 10^{7}$ m$ ^{-1}$ to calculate the shortest wavelength of the Balmer series. 
\item (2017)  Use the Bohr's theory for hydrogen atom to determine the:\begin{itemize}
\item Radius of the first orbit of the hydrogen atom in A units. 
\item Velocity of the electron in the first orbit. 
\end{itemize}
\item (2017)  What is ionization potential of an atom?
\item (2017)  Show that the ionization potential of hydrogen is $ 13.6$ eV. 
\item (2017)  How can you account for the chemical behavior of atoms on the basis of the atomic electrons and shells? 
\item (2017)  \item (2018)  Given: Mass of proton $ =1.0080$ u, Mass of neutron $ =1.0087$ u and Mass of alpha particle $ =4.0026$ u.\begin{itemize}
\item State any three limitations of Bohr’s model of the hydrogen atom.
\end{itemize}
\item (2018)  Why hydrogen spectrum contains a larger number of spectral lines although its  atom has only one electron? 
\item (2018)  State any three limitations of Bohr’s model of the hydrogen atom.
\item (2018)  Distinguish between ionization energy and excitation energy.
\item (2018)  Why hydrogen spectrum contains a larger number of spectral lines although its  atom has only one electron? 
\item (2019)  Based on Balmer series of hydrogen spectra determine the wavelength of the series limit of Paschen series. 
\item (2019)  Why hydrogen atom is stable in the ground state? 
\item (2019)  According to Bohr’s theory, the angular momentum of an electron is an integral multiple of $ h/2\pi$ .  Express this statement. by using a mathematical equation in which angular momentum is represented by the letter Land orbit by the letter $ n$ , 
\end{itemize}

\subsection{Quantum Physics}
\begin{itemize}
\item (1999)  What is the “work function” of a metal?
\item (1999)  The work function of a metal is $ 2.0$ eV. Calculate the stopping potential when the metal is illuminated by light of frequency of $ 6.0 \times 10^{14}$ Hz.
\item (2000)  What is the de Broglie wave equation?
\item (2000)  An electron is accelerated through a potential of $ 400$ V. Determine the de Broglie wavelength of this electron.
\item (2000)  Determine the de Broglie wavelength for the beam of electron whose total energy is\begin{itemize}
\item $ 250$ eV.
\end{itemize}
\item (2000)  What is a photoelectric cell?
\item (2000)  \item (2007)  A certain diatomic gas is contained in a vessel whose inner surface is a small absorber which retains any atoms or molecules of gas which strike it.  Show that if doubling the absolute temperature causes one half of the molecules to dissociate into atoms then the rate at which the absorber is gaining mass increases by a factor $ 1+1/\sqrt{2}$ .
\item (2007)  What is a line spectrum? 
\item (2009)  Write down Bragg’s equation for the study of the atomic structure of the crystals by $ X-$ rays.
\item (2009)  The radiation from an $ X$ — ray tube which operates at $ 50$ kV is diffracted by is diffracted by a cubic KCl crystal of molecular mass $ 74.6$ and density $ 1.99 \times 10^{3}$ kg$/$m$ ^{3}$ .  Calculate:\begin{itemize}
\item The shortest wavelength limit of the spectrum from the tube.
\item The glancing angle for first order reflection from the planes of the crystal for that wavelength and angle of deviation of a diffracted beam.
\end{itemize}
\item (2009)  The radiation emitted by an $ X$ — ray tube consists of continuous spectrum with a line spectrum superimposed on it. Explain how the continuous spectrum and the line spectrum are produced.\begin{itemize}
\item Draw the graph of the spectra stated. ‘
\end{itemize}
\item (2013)  If the energy necessary to cause the ejection of an electron by photoelectric effect from the $ N$ — shell and $ K-$ shell of an atom is $ 10$ eV and $ 20$ eV respectively, calculate the maximum wavelength of radiation for each level.
\item (2015)  Show that the de Broglie hypothesis of matter wave are in agreement with Bohr’s theory.
\item (2015)  Ultraviolet light of wavelength $ 3600 \times 10^{-10}$ m is made to fall on a smooth surface of potassium. Determine:\begin{itemize}
\item The maximum energy of emitted photoelectrons
\item The stopping potential.
\item The velocity of the most energetic photoelectrons given that work function for potassium is $ 2$ eV.
\end{itemize}
\item (2016)  Briefly explain the production of X-rays.
\item (2016)  List down any three uses of X-rays.
\item (2016)  How are the intensity and penetrating power of an X-ray beam controlled?
\item (2018)  Briefly explain what led de-Broglie to think that the material particles may also show wave nature and why the wave nature of matter not noticeable in our daily observations? 
\item (2018)  Prove that de-Broglie wavelength $ \lambda $ , of electrons of kinetic energy $ E$ is given by $ \lambda = h/ \sqrt{2}$ meV  where $ m$ is the mass of the electron, $ e$ is the charge of the electron, $ h$ is the Planck’s constant and v is the accelerating potential difference. 
\item (2018)  Light of wavelength $ 488$ nm is produced by an argon laser which is used in the photoelectric effect. When light from this spectral line is incident on the emitter, the stopping (cut-off) potential of photoelectrons is $ 0.38$ V. Find the work function of the material from which the emitter is made. 
\item (2018)  In a hydrogen atom model, an electron of mass $ m$ and charge $ e$ revolves around the nucleus in a circular orbit of radius $ r$ . Develop an expression for the radius $ 3$ m of the orbit in terms of $ m$ , $ e$ , $ x$ , the quantum number $ n$ , Planck constant $ h$ and the permitting of free space $ \epsilon _{0}$ , and hence, use their values to find the Bohr’s radius. 
\item (2018)  In a hydrogen atom model, an electron of mass $ m$ and charge $ e$ revolves around the nucleus in a circular orbit of radius $ r$ . Develop an expression for the radius $ 3$ m of the orbit in terms of $ m$ , $ e$ , $ x$ , the quantum number $ n$ , Planck constant $ h$ and the permitting of free space $ \epsilon _{0}$ , and hence, use their values to find the Bohr’s radius. 
\item (2019)  Why electrons do not fall into the nucleus due to electrostatic force of attraction?
\item (2019)  Determine the angular momentum of the electron in the orbit of energy level $ -3.4$ eV given that $ E_{n}=-13.6/n^{2}$ eV, where $ E$ is the energy of an electron and $ n$ is the principal quantum number of hydrogen atom. 
\end{itemize}

\subsection{LASER}
\begin{itemize}
\item (2000)  Using an example of your own choice explain the mechanism behind the production of a laser beam.
\item (2000)  Describe two applications of a laser
\item (2007)  Explain breifly the action of a helium-neon laser.
\item (2009)  Define the terms laser and maser. 
\item (2009)  Give three applications of laser. 
\item (2009)  A laser beam has a power of $ 20 \times 10^{9}$ watts and a diameter of $ 2$ mm.  Calculate the peak values of electric field and magnetic fields.
\item (2015)  Give any four uses of LASER lgith.
\end{itemize}

\subsection{Nuclear Physics}
\begin{itemize}
\item (1998)  Explain the terms: atomic mass unit, mass defect, packing fraction and binding energy.
\item (1998)  Discuss carbon dating.
\item (1998)  Find the age at death of an organism, if the ratio of amount of C$ 14$ at death to that of the present time is $ 10^{8}$ and that the half life of Cl$ 4$ is $ 5600$ years.
\item (1999)  What is nuclear fusion 
\item (1999)  What is nuclear fission?
\item (1999)  Define the term “binding energy” of a nuclide.
\item (1999)  Distinguish between:\begin{itemize}
\item $ \beta -$ decay and $ \beta +$ decay.
\item nuclear fission and nuclear fusion
\item activity and half-life of a radioactive material.
\item Taking the half-life of Radium $ -226$ to be $ 1600$ years, what fraction of a given sample remains after $ 4800$ years?
\end{itemize}
\item (2000)  A sample of soil from Olduvai Gorge cave was examined. It was found to contain, among other things, pieces of charcoal. Further investigation on the charcoal revealed that $ 1$ kg of C$ 14$ nuclei decayed each second. It is assumed that this charcoal has resulted from decomposition of the stone-age people who died there (i.e. at the cave) long time ago. Calculate the number of years that have elapsed since these people died.
\item (2007)  It is not possible to separate the different isotopes of an element by chemical means.  Explain.
\item (2007)  Define a mass spectrometer. 
\item (2007)  Ion A of mass $ 24$ and charge $ +e$ and ion $ B$ of mass $ 22$ and charge $ +2e$ both enter the magnetic field of a mass spectrometer with the same speed. If the radius of A is $ 2.5 \times 10^{-1}m$ , calculate the radius of the circular path of $ B$ . 
\item (2007)  If the ratio of mass of lead – $ 206$  to mass of uranium – $ 238$ in a certain rock was found to be $ 0.45$ and that the rock originally contained no lead – $ 206$ , estimate the age of the rock given that the half life of uranium – $ 238$ is $ 4.5 \times 10^{9}$ years.
\item (2007)  Define the following terms:\begin{itemize}
\item Atomic mass unit
\item Binding energy
\item Mass defect.
\end{itemize}
\item (2009)  Explain the following observations:\begin{itemize}
\item A radioactive source is placed in front of a detector which can detect all forms of radioactive emissions. It is found that the activity registered as noticeably reduced when a thin sheet of paper is placed between the source and detector.
\item When a brass plate with a narrow vertical shit is placed in front of the radioactive source (above) and a horizontal: magnetic field normal to the line joining the source and the detector is applied, its found that the activity is further reduced.
\item The magnetic field (above) is removed and a sheet of aluminum is placed in front of the source. The activity recorded is similarly reduced.
\end{itemize}
\item (2009)  \item (2013)  Distinguish between white spectrum and line spectrum. 
\item (2013)  What is the significance of the binding energy per nucleon? 
\item (2013)  Briefly explain why the $ \beta$  — particles emitted from a radioactive source differ from the electrons obtained by thermionic emission? 
\item (2013)  \item (2015)  Define activity and half-life.
\item (2015)  The half-life of radioactive substance is $ 1$ hour.  How long will it take for $ 60\%$ of the substance to decay?
\item (2015)  What is a nuclear reactor?\begin{itemize}
\item Briefly explain any three main components in a nuclear reactor.
\end{itemize}
\item (2015)  Sketch the binding energy curve.\begin{itemize}
\item State any two conclusions that can be drawn from the curve above.
\end{itemize}
\item (2015)  If the mass of deuterium nucleus is $ 2.015$ a.m.u, that of one isotope of helium is $ 3.017$ a.m.u. and that of neutron is $ 1.009$ a.m.u., calculate the energy released by the fusion of $ 1$ kg of deuterium. \begin{itemize}
\item Suppose $ 50\%$ of this energy was used to produce $ 1$ MW of electricity, for how many days would be able to function.
\end{itemize}
\item (2016)  The number of particles $ n$ crossing a unit area perpendicular to $ x-$ axis in a unit time is given as $ n=-D(n_{2}-n_{1})/(x_{2}-x_{1})$ where $ n_{1}$ and $ n_{2}$ are the number of particles per unit volume for the values of $ x_{1}$ and $ x_{2}$ respectively.  What are the dimensions of diffusion constant $ D$ ?
\item (2016)  Differentiate natural radioactivity from artificial radioactivity.
\item (2016)  Name three applications of radioisotopes in medicine.
\item (2016)  State two conditions for stability of nuclides referring to light nuclides and heavy nuclides.
\item (2016)  Derive an expression for the half-life using the radioactive decay law.
\item (2016)  What is carbon $ -14$ ?  Explain its production and how it is used in the dating process.
\item (2016)  \item (2017)  What is meant by the following?\begin{itemize}
\item Atomic Mass Unit (a.m.u.)
\item Binding energy. 
\item Mass defect
\end{itemize}
\item (2017)  Write down the equation for the disintegration.
\item (2018)  Use the concept of radioactive decay and nuclear reactions to define the following terms:\begin{itemize}
\item $ \alpha $ decay
\item $ \beta$ decay
\item $ \gamma $ decay
\item Fission
\item Fusion.
\item For each of the terms above, give one suitable reaction equation. 
\end{itemize}
\item (2018)  A freshly prepared sample of a radioactive isotope $ Y$ contains $ 10^{12}$ atoms. The half-life of the isotope is $ 15$ hours. Calculate;\begin{itemize}
\item the initial activity. 
\item the number of radioactive atoms of $ Y$ remaining after $ 2$ hours, 
\end{itemize}
\item (2018)  Mention any four important features in the design of a nuclear reactor.
\item (2018)  Differentiate binding energy from mass defect.
\item (2018)  Calculate the binding energy per nucleon, in MeV and the packing fraction of an alpha particle.
\item (2018)  A freshly prepared sample of a radioactive isotope $ Y$ contains $ 10^{12}$ atoms. The half-life of the isotope is $ 15$ hours. Calculate;\begin{itemize}
\item the initial activity. 
\item the number of radioactive atoms of $ Y$ remaining after $ 2$ hours, 
\end{itemize}
\item (2018)  Mention any four important features in the design of a nuclear reactor.
\item (2018)  Differentiate binding energy from mass defect.
\item (2018)  Calculate the binding energy per nucleon, in MeV and the packing fraction of an alpha particle.\begin{itemize}
\item Given: Mass of proton $ =1.0080$ u, Mass of neutron $ =1.0087$ u and Mass of alpha particle $ =4.0026$ u.
\end{itemize}
\item (2019)  What is meant by the following terms as used in nuclear Physics?\begin{itemize}
\item Mass defect 
\item Binding energy. 
\end{itemize}
\item (2019)  Elaborate two aspects on which fission reactions differs from fusion reactions.
\item (2019)  Why is high temperature required to cause nuclear fusion? 
\end{itemize}


\section{Environmental Physics}

\subsection{Agricultural Physics}
\begin{itemize}
\item (2013)  \item (2014)  Briefly explain the influence of the following climatic conditions for plant growth and development:\begin{itemize}
\item Rain fall and water
\end{itemize}
\item (2015)  \item (2016)  \item (2017)  Discuss two advantages of windbreaks to plant environment. 
\item (2019)  Give two positive effects of wind on plant growth.
\end{itemize}

\subsection{Energy from the environment}
\begin{itemize}
\item (2016)  Briefly explain three major concepts on solar wind.
\item (2017)  State three sources of heat energy within the interior of the earth. 
\item (2018)  What is meant by solar constant? 
\item (2018)  List two factors on which the solar constant depends. 
\item (2018)  Give two advantages of photovoltaic system. 
\item (2018)  Briefly explain how photovoltaic cells work. 
\item (2018)  \end{itemize}

\subsection{Earthquakes}
\begin{itemize}
\item (1998)  Explain the following terms: Earthquake, Earthquake focus, Epicentre and Body waves.
\item (1998)  List down three $ (3)$ sources of earthquakes.
\item (2000)  With reference to an earthquake on a certain point of the earth explain the terms ‘Focus’ and ‘Epicentre’.
\item (2000)  Describe two ways by which seismic waves may be produced.\begin{itemize}
\item Describe briefly the meaning and application of “seismic prospecting”. 
\end{itemize}
\item (2007)  What are the difference between $ P$ and s waves?
\item (2007)  Explain how the two terms of waves ($ P$ and $ S$ ) can be used in studying the internal structure of the earth. 
\item (2007)  What is geomagnetic micropulsation.
\item (2010)  Explain the following terms Earthquake, Earthquake focus and Epicenter.
\item (2010)  Describe clearly how $ P$ and s waves are used to ascertain that the outer core of the Earth is in liquid form. 
\item (2013)  The main interior of the earth (core) is believed to be in molten form. What seismic evidence supports this belief?
\item (2015)  What is the origin of earthquake?
\item (2015)  A large explosion at the earth's surface creates compressional (P) and shear (S) waves moving with a speed of $ 6.0$ km$/$s and $ 3.5$ km$/$s respectively. If both waves arrive at seismological station with $ 30$ s interval, calculate the distance measured between seismological station and the site of explosion. 
\item (2019)  What 's meant by epicentre and wind belt as used in Geophysics? 
\item (2019)  Identify three types of seismic waves.\begin{itemize}
\item Outline two characteristics of each type of wave described above.
\end{itemize}
\end{itemize}

\subsection{Environmental Pollution}
\begin{itemize}
\item (1998)  Define ionosphere.
\item (1998)  Mention the ionospheric layers that exist during the day time.
\item (2000)  What is the importance of the following layers of the atmosphere?\begin{itemize}
\item The lowest layer
\item The ionosphere
\end{itemize}
\item (2007)  \item (2010)  Define the ionosphere and give one basic use of it.
\item (2010)  \item (2013)  Explain why the small ozone layer on the top of the stratosphere is crucial for human survival
\item (2013)  Electrical properties of the atmosphere are significantly exhibited in the ionosphere.\begin{itemize}
\item  What is the layer composed of and what do you think is the origin of such constituents.
\item  Mention two uses of the ionosphere.
\end{itemize}
\item (2013)  Briefly explain on the following types of environmental pollution:\begin{itemize}
\item  Thermal pollution.
\item  Water pollution.
\end{itemize}
\item (2014)  Describe the sources and effects of the following pollutants on the environment:\begin{itemize}
\item Air pollution. 
\item Radiation pollution.
\end{itemize}
\item (2016)  What is meant by aerial environment?  Give two examples.
\item (2016)  Describe three ways at which the aerial environment is threatened.
\item (2017)  Give two factors which determine whether a planet has an atmosphere or not.
\item (2017)  Briefly explain the major causes of the following types of environmental pollution:\begin{itemize}
\item Water pollution. 
\item  Air pollution.
\end{itemize}

\end{itemize}

\end{document}